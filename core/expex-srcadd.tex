\def\ExPexMessage
   {2013/07/27 v4.1c ExPex linguistics example formatter (JF)}
% -------------------------------------------------------
% jf 2011/12/09  (v4.0a)
%  1) fixed bug in how exceptional gla items are detected
%  2) added everytrailingcitation parameter
%  3) made \glft a long definition to allow \par
%  4) fixed bug in \printrbrack and \printlbrack, now not math mode
% jf 2011/12/09  (v4.0b)
%  1) allow {} gla item
% jf 2012/03/10  (v4.0c)
%  1) fixed problem with IJAL \deftaglabel caused by nonexpansion
%     of \@actualexno
% jf 2012/12/01  (v4.1)
%  1) added auto glskip mode to glosses
%     /abovemoreglskip/ now obsolete
%  2) fixed bug in \@setinckey so that value is expanded when set
% jf 2013/01/13  (v4.1a)
%  1) added \let\pageno=\c@page to LaTex specializations
% jf 2013/06/27  (v4.1b)
%  1) fixed bug in skip over \glft entries
% jf 2013/07/27
%  1) LaTex \include/\includeonly mechanism now works for \excnt
% jf 2013/10/08
%  1) revise the definition of \glw@printcol to use token lists rather
%  than edefs, and to build miniglosses as \vtop{\halign{#\hfil\cr...}
%  structures rather than stacked hboxes. This allows items like
%  {\hfil word}, which results in center alignment.
%  2) removed w from the prefix in \glw@append, \glw@lop, and
%  \glw@lopff since these will be used in other gloss styles
% jf 2013/10/10 fixed belowpreambleskip bug
% --------------------------------------------------------
\edef\resetatcatcode{\catcode`\noexpand\@\the\catcode`\@\relax}
\catcode`\@=11\relax
\ifx\XKeyValLoaded\endinput \else
   \input xkeyval \fi
\ifx\ProvidesFile\@undefined
      \message{\ExPexMessage}
   \else
      \ProvidesFile{expex.tex}[\ExPexMessage]
      \@addtofilelist{expex.tex}
      \let\it=\itshape
      \let\pageno=\c@page
% 2013/07/27
      \g@addto@macro\cl@@ckpt
         {\immediate\write\@partaux{\string\excnt=\the\excnt}}
%
   \fi
%! define eplain primitives, if necessary
\ifx\eplain\@undefined  % eplain stuff
   \def\@futurenonspacelet#1{\def\@cs{#1}%
      \afterassignment\@stepone\let\@nexttoken= }%
   \def\@stepone{\expandafter\futurelet\@cs\@steptwo}%
   \def\@steptwo{\expandafter\ifx\@cs\@sptoken\let\@@next=\@stepthree
      \else\let\@@next=\@nexttoken\fi \@@next}%
   \def\@stepthree{\afterassignment\@stepone\let\@@next= }%
   \def\@getoptionalarg#1{%
      \let\@optionaltemp = #1%
      \let\@optionalnext = \relax
      \@futurenonspacelet\@optionalnext\@bracketcheck
   }
   \def\@bracketcheck{%
      \ifx [\@optionalnext
         \expandafter\@@getoptionalarg
      \else
         \let\@optionalarg = \empty
         \expandafter\@optionaltemp
      \fi }
   \def\@@getoptionalarg[#1]{%
      \def\@optionalarg{#1}%
      \@optionaltemp}
\fi
%----- end of eplain inclusions
\def\ep@identity#1{#1}
\def\ep@expandonce{\expandafter\noexpand}
\def\ep@expandtwice{\expandafter\expandafter\expandafter\noexpand}
\def\ep@expandafterafter#1{\expandafter#1\expandafter}
\def\ep@gobble#1{}
\def\@getoptionaltag#1{%
   \let\@@optionaltemp = #1%
   \let\@optionaltag\empty
   \@ifnextcharacter<\@@gettag\@@optionaltemp
}
\def\@@gettag<#1>{\def\@optionaltag{#1}\@@optionaltemp}

\newif\if@tilde
\def\@tildecheck#1{%
   \@ifnextcharacter~%
      {\@tildetrue\expandafter#1\ep@gobble}%
      {\@tildefalse#1}%
}
%------- XKV parametrization ------
\def\define@lingkey{\define@key{ling}}
\def\define@ling@cmdkeys{\define@cmdkeys{ling}[ling@]}
\def\define@lingcmdkeys{\define@cmdkeys{ling}[ling]}
%
\def\define@linginckey#1#2{%
   \define@key{ling}{#2}{%
      \ep@expandafterafter\@setinckey
         #1\csname ling#2\endcsname ##1\@nil}%
}
\def\define@lingincdimenkeys#1{\XKV@for@n{#1}\@key{%
   \expandafter\define@linginckey\expandafter\dimen\expandafter{\@key}}}
\def\define@lingincskipkeys#1{\XKV@for@n{#1}\@key{%
   \expandafter\define@linginckey\expandafter\skip\expandafter{\@key}}}
\def\@setinckey#1#2#3#4\@nil{%
   \ifx#3!%
         #1 0=#2%
         \advance#1 0 by #4%
      \else
         #1 0=#3#4%
      \fi
      \edef#2{\the #10}%
}
%
\def\lingset#1{\setkeys{ling}{#1}\ignorespaces}
% \Lingset first sets ling keys, if there are non-ling keys
%   remaining, these are then passed to \psset
\def\Lingset#1{\setkeys*{ling}{#1}%
   \ifx\XKV@rm\@empty \else
      \ep@expandafterafter\psset{\XKV@rm}\fi
}
\def\Ling@usearg{%
   \ifx\@optionalarg\empty
      \else \ep@expandafterafter\Lingset{\@optionalarg}\fi
}
\def\ling@usearg{\ep@expandafterafter\lingset{\@optionalarg}}
% styles
\def\e@let#1#2{%
   \ep@expandafterafter\let#1\csname #2\endcsname\ignorespaces
}
\define@lingkey{lingstyle}{%
   \e@let\temp{ling@#1style}
      \ep@expandafterafter\Lingset{\temp}}
\def\definelingstyle#1#2{%
   \expandafter\def\csname ling@#1style\endcsname{#2}}
% if PST available, allow \psset to set ling parameters,
% otherwise cancel \Lingset's ability to set PST parameters
\ifx\PSTricksLoaded\endinput
      \pst@addfams{ling}
   \else
      \let\Lingset=\lingset
   \fi
%------ scratch dimensions -----
\newdimen\epd@a
%\newdimen\epd@b not yet needed
%\newdimen\epd@c not yet needed
%------ \ex -----
\newcount\excnt
\excnt=1
\newbox\numbox
\newdimen\epd@numright
\newif\if@specialexno
\define@lingincskipkeys{aboveexskip,belowexskip}
\define@lingincdimenkeys{numoffset,textoffset}
\define@lingcmdkeys{Everyex,everyex,exbreakfil,exbreakpenalty,splitpartspenalty}
\define@lingkey{exskip}%
   {\edef\lingaboveexskip{#1}\edef\lingbelowexskip{#1}}
\def\ep@settosum#1#2#3{#1=#2\relax \advance#1 by#3}
\define@choicekey{ling}{textanchor}%
   [\scratch\ep@textanchor]{numleft,normal}{}
\def\ex{\bgroup \@tildecheck\ex@a}
\def\ex@a{\def\@optionaltag{}\def\@specialexno{}%
   \@getoptionalarg\ex@b}
\def\ex@b{\@getoptionaltag\ex@c}
\def\ex@c{%
   \ex@setup
   \leavevmode
   \setbox\numbox=\hbox{\hskip\lingnumoffset\ep@actualexnoprint}%
   \epd@numright=\wd\numbox
   \ifcase\ep@textanchor                          % numleft
         \ep@settosum\leftskip\lingnumoffset\lingtextoffset
      \or                                         % normal
         \ep@settosum\leftskip\epd@numright\lingtextoffset
      \fi
   \llap{\hbox to\leftskip{\unhbox\numbox \hss}}%
   \lingeveryex
   \latex@tagex
   \ignorespaces
}
\def\actualexno
   {\if@specialexno {\ep@specialexno}\else \the\excnt \fi}
\def\ep@actualexnoprint{{%
   \if@specialexno
      \ep@specialexnoprint
   \else \ifx\ling@sampleexno\empty
      \exnoprint
   \else
      \hbox to \epd@sampleexnowidth{\exnoprint\hss}%
   \fi\fi
}}
%\define@choicekey{ling}{exnotype}[\ling@exnotype\@N]%
%   {arabic,roman}{%
%      \ifcase\@N
%%%            \let\exnoprint=\ep@arabicexnoprint
%            \def\ep@rawexnoprint{\number\excnt}%
%         \or
%%%            \let\exnoprint=\ep@romanexnoprint
%            \def\ep@rawexnoprint{\romannumeral\the\excnt}%
%         \fi
%}
\define@choicekey{ling}{exnotype}[\ling@exnotype\@N]%
   {arabic,chapter.arabic,roman}{%
      \ifcase\@N
            \def\ep@rawexnoprint{\number\excnt}%
         \or
            \def\ep@rawexnoprint{\thechapter.\number\excnt}%
         \or
            \def\ep@rawexnoprint{\romannumeral\the\excnt}%
         \fi
}
\lingset{exnotype=arabic}
\def\exnoprint{\ep@exnoformat{\ep@rawexnoprint}}
%\def\ep@arabicexnoprint{\ep@exnoformat{\the\excnt}}
%\def\ep@romanexnoprint{\ep@exnoformat{\romannumeral\the\excnt}}
%\def\romanexnumbers{\let\exnoprint=\ep@romanexnoprint}
%\def\ep@specialexnoprint{(\ep@specialexno)}
\def\ep@specialexnoprint{\ep@exnoformat{\ep@specialexno}}  % unformatted
\def\ep@globalstepexcnt{\global\advance\excnt by 1 }
\def\ep@localstepexcnt{\advance\excnt by 1 }
\let\stepexcnt=\ep@globalstepexcnt
\def\keepexcntlocal{\let\stepexcnt=\ep@localstepexcnt}
\def\ex@setup{%             also used by \pex
   \global\@specialexnofalse
   \latex@tagex
   \lingEveryex
   \let\reset@refproofing\@printref
   \let\@printref\ep@identity    % turn off refproofing
   \Ling@usearg
   \let\@printref\reset@refproofing
   \if@specialexno
         \def\@actualexno{\ep@specialexno}%
      \else
         \edef\@actualexno{\ep@rawexnoprint}%
      \fi
   \ifx\@optionaltag\empty
         \let\@localextag=\empty
      \else
         \edef\@localextag{\@optionaltag}%
         \deftag{\@actualexno}{\@optionaltag}
      \fi
   \exbreak
   \if@tilde \else \vskip\lingaboveexskip\fi
   \parindent=0pt
}
\def\noexno{\global\advance\excnt by -1}
\def\exbreak{\endgraf\bgroup\@getoptionalarg\exbreak@a}
\def\exbreak@a{%
   \ifx\@optionalarg\empty
         \skip255=\lingexbreakfil
      \else
         \skip255= 0pt plus\@optionalarg
      \fi
   \vskip\skip255
   \penalty\lingexbreakpenalty
   \vskip-\skip255
   \egroup
}
\def\xe{%
   \expandafter\vskip\lingbelowexskip
   \egroup
   \if@specialexno \else \stepexcnt \fi
   \allowbreak
   \prevdepth\dp\strutbox
   \noindent
}
\def\exdisplay{\bgroup\@tildecheck\exdisplay@a}
\def\exdisplay@a{\@getoptionalarg\exdisplay@b}
\def\exdisplay@b{\let\@optionaltag=\empty \ex@setup}
%------ \pex -----
\newcount\pexcnt
\newdimen\epd@labelleft
\newdimen\epd@labelright
\newdimen\epd@textleft
\newdimen\epd@preambleleft
\newif\if@firstlabel
\define@lingincdimenkeys{preambleoffset,labelwidth,labeloffset}
\define@lingincskipkeys{belowpreambleskip,interpartskip}
\define@choicekey{ling}{preambleanchor}
   [\scratch\ep@preambleanchor]{numright,labelleft,text}{}
\define@boolkey{ling}[ling@]{avoidnumlabelclash}[true]{}
\define@ling@cmdkeys{appendtopexarg}
\newdimen\epd@sampleexnowidth
\define@lingkey{sampleexno}{%
   \setbox0=\hbox{#1}%
   \epd@sampleexnowidth=\wd0
   \def\ling@sampleexno{#1}%
}
\lingset{sampleexno=}
\define@lingcmdkeys{splitexpenalty}
%\define@lingcmdkeys{belowpreambleskip,interpartskip,splitexpenalty}
\define@choicekey{ling}{labelalign}[\ling@labelalign\nr]%
   {left,center,right}{%
      \ifcase\nr
            \def\ep@labelprint{\ep@labelformat\ep@label\hss}%
         \or
            \def\ep@labelprint{\hss \ep@labelformat\ep@label\hss}%
         \or
            \def\ep@labelprint{\hss \ep@labelformat\ep@label}%
         \fi
}
\define@key{ling}{samplelabel}{%
   \setbox0=\hbox{#1}%
   \lingset{labelwidth=\wd0}%
}
\define@boolkey{ling}[ling@]{nopreamble}[true]{}
\def\pex{\bgroup\@tildecheck\pex@a}
%\def\pexns{\bgroup \@tildetrue\pex@a}
\def\pex@a{\def\@optionaltag{}\def\@specialexno{}%
   \@getoptionalarg\pex@b}
\def\pex@b{%
   \ifx\ling@appendtopexarg\empty \else
      \XKV@addtolist@o\@optionalarg{\ling@appendtopexarg}\fi
   \@getoptionaltag\pex@c}
\def\pex@c{\ling@nopreambletrue
   \@futurenonspacelet\temp\pex@d}
\def\pex@d{%
   \ifx\temp\a \let\nextpex@\pex@e
      \else \ifx\temp\label \let\nextpex@\pex@f
      \else \ling@nopreamblefalse \let\nextpex@\pex@e
      \fi\fi
   \ex@setup
   \nextpex@
}
\def\pex@f#1#2{\label{#2}\@futurenonspacelet\temp\pex@g}
\def\pex@g{\ifx\temp\a \let\next\pex@h
   \else \let\next\pex@e \ling@nopreamblefalse \fi \next}
\def\pex@h#1\a{\pex@e\a}
\def\pex@e{\pex@i \lingeveryex }
\def\pex@i{%
   \setbox\numbox=\hbox{\hskip\lingnumoffset\ep@actualexnoprint}%
   \ep@setdimensions
   \ep@pexcntinit
   \@firstlabeltrue
   \let\a\ep@putlabel
   \ifling@nopreamble
         \leftskip=\epd@textleft
      \else
         \ep@setdimpreambleleft
         \leftskip=\epd@preambleleft
      \fi
   \def\next{\llap{\hbox to\leftskip{\unhbox\numbox \hss}}}%
   \ifling@nopreamble
         \ifling@avoidnumlabelclash \let\next\relax \fi\fi
   \leavevmode
   \next
}
\def\ep@setdimensions{%
   \epd@numright=\wd\numbox
   \epd@labelleft=\linglabeloffset
      \advance\epd@labelleft by \ifcase\ep@labelanchor
         \epd@numright\or \lingnumoffset\or  0pt \fi
%   \ep@setdimlabelleft
   \ep@settosum\epd@labelright\epd@labelleft\linglabelwidth
  \ifcase\ep@textanchor   % numleft
         \ep@settosum\epd@textleft\lingnumoffset\lingtextoffset
      \or                 % normal
         \ep@settosum\epd@textleft\epd@labelright\lingtextoffset
      \fi
%   \ep@setdimtextleft
}
%\def\lingnumrightoffset{\the\epd@numright}%
\def\ep@setdimpreambleleft{%
   \epd@preambleleft=\lingpreambleoffset
   \advance\epd@preambleleft by \ifcase\ep@preambleanchor
      \epd@numright\or \epd@labelleft\or \epd@textleft \fi
}
\def\ep@pexcntinit{\ifnum\ep@labelgen=2\else
   \pexcnt=\ling@pexcnt
   \advance\pexcnt by -1 \fi}
\define@key[epx@]{labels}{tag}{\def\@optionaltag{#1}}
\define@key[epx@]{labels}{label}{\def\@specialexno{#1}}
\def\ep@setlabelkeys{\setkeys[epx@]{labels}}
\def\ep@useoptionallabelarg{%
   \expandafter\ep@setlabelkeys\expandafter{\@optionalarg}}
\define@lingkey{tag}{\def\@optionaltag{#1}}
\newtoks\ep@everylabel  % \ep@everylabel is a token list
\define@lingkey{everylabel}{\ep@everylabel{#1}}
%
\def\ep@putlabel{%
   \if@firstlabel
         \ifling@nopreamble \else
            \vskip\lingbelowpreambleskip
            \leftskip=\epd@textleft
            \fi
         \@firstlabelfalse
      \else
         \par\penalty\lingsplitpartspenalty
         \vskip\linginterpartskip
      \fi
   \def\@specialexno{}\def\@optionaltag{}%
   \@getoptionalarg\ep@putlabel@a
}
\def\ep@putlabel@a{%
   \ep@useoptionallabelarg
   \ifx\@specialexno\empty
         \ifcase\ep@labelgen
            \def\ep@label{\the\ep@everylabel \char\the\pexcnt}%
            \advance\pexcnt by 1
         \or
            \def\ep@label{\the\ep@everylabel \number\pexcnt}%
            \advance\pexcnt by 1
         \or
            \ep@popLL
         \or
            \def\ep@label{\the\ep@everylabel \romannumeral\pexcnt}%
            \advance\pexcnt by 1
         \fi
      \else
         \def\ep@label{\the\ep@everylabel\@specialexno}%
      \fi
   \xdef\resumepexcnt{\noexpand\pexcnt\the\pexcnt}%
   \@getoptionaltag
   \ep@putlabel@b
}
\def\ep@putlabel@b{%
   \ifx\@optionaltag\empty \else
      \deftaglabel{\@optionaltag}%
      \fi
   \leavevmode
   \llap{\hbox to\leftskip{\hskip\epd@labelleft
      \hbox to\linglabelwidth{\ep@labelprint}%
   \hfil}}%
   \latex@tagexlabel
   \ignorespaces
}
%
\define@choicekey{ling}{labelanchor}[\scratch\ep@labelanchor]%
   {numright,numleft,margin}[]{}
\define@lingkey{pexcnt}{\edef\ling@pexcnt{#1}}
%-----------------------------------------
%----- judgments -----
\def\judge#1{\rm #1\kern .1em \ignorespaces}
\def\ljudge#1{\llap{\judge{#1}}\ignorespaces}
\define@key{ling}{*}[*]%
   {\setbox0=\hbox{#1}%
   \lingset{textoffset=!\wd0}%
}
%------ table support -----
\define@lingcmdkeys{dima,dimb,dimc}
\lingset{dima=2.4em}
\def\tspace{\@getoptionalarg\ep@tabelspace}
\def\ep@tabelspace{\hskip
   \ifx\@optionalarg\empty
         \lingdima
      \else
         \csname ling\@optionalarg\endcsname
      \fi
}
\def\labels{\@getoptionalarg\ep@labels}
\def\ep@labels{%
   \ifcase\ep@labelgen
         \def\ep@label{\the\ep@everylabel \char\the\pexcnt}%
      \or
         \def\ep@label{\the\ep@everylabel \number\pexcnt}
      \or
      \or
         \def\ep@label{\the\ep@everylabel \romannumeral\pexcnt}
      \fi
   \ling@usearg
   \dimen0=\lingtextoffset
   \advance\dimen0 by \linglabelwidth
   \edef\ling@labelskip{\the\dimen0}%
   \ep@pexcntinit
   \let\tl\ep@inserttabellabel
   \let\nl\ep@omitlabel
   \ignorespaces
}
\def\ep@inserttabellabel{\@getoptionaltag\ep@inserttablelabel@a}
\def\ep@inserttablelabel@a{%
   \global\advance\pexcnt by 1
   \ifx\@optionaltag\empty \else
      \deftaglabel{\@optionaltag}%
      \fi
   \edef\foop{\ep@label.}\foop
}
\def\ep@omitlabel{\omit\hskip\linglabeloffset\hfil}
%\def\endpextable{\egroup\egroup \par \prevdepth=\dp\strutbox}
\def\hwit#1{\hidewidth \it #1\hidewidth}
\define@ling@cmdkeys{crskip}
\lingset{crskip=.6em}
\def\crs{\cr\noalign{\vskip\ling@crskip}}
\def\crnb{\cr\noalign{\par\nobreak}}
% LL is "label list"
\define@lingkey{labellist}{%
   \edef\ling@LL{#1,}%
   \edef\@currLL{#1,}%  current LL
}
\def\ep@popLL{%
   \ifx\@currLL\empty
      \@expexwarn{Not enough labels in labellist}%
      \let\@currLL=\ling@LL  % start over
      \ep@popLL
   \else
      \expandafter\ep@popLL@a\@currLL\@nil
   \fi
}
\def\ep@popLL@a#1,#2\@nil{%
   \def\ep@label{\the\ep@everylabel #1}\def\@currLL{#2}}
\define@choicekey{ling}{labelgen}[\ling@labelgen\ep@labelgen]%
   {char,number,list,romannumeral}{}
\define@choicekey{ling}{labeltype}[\ling@labeltype\@N]%
   {alpha,caps,numeric}{%
      \ifcase\@N
            \lingset{labelgen=char,pexcnt=97,labelformat=A.,
               fullrefformat=XA,labelalign=left}%
         \or
            \lingset{labelgen=char,pexcnt=65,labelformat=A.,
               fullrefformat=XA,labeloffset=!.3em,labelalign=left}%
         \or
            \lingset{labelgen=number,pexcnt=1,labelformat=A.,
               fullrefformat=X.A,labelalign=right}%
         \fi
}
\def\definelabeltype#1#2{%
   \expandafter\def\csname ling@#1labeltype\endcsname{#2}}
\define@lingkey{labeltype}{%
   \e@let\temp{ling@#1labeltype}%
   \ep@expandafterafter\Lingset{\temp}}
\define@lingkey{labelformat}{\ep@omitlabelformat #1\@nil}
\def\ep@omitlabelformat #1A#2\@nil{%
   \def\ep@labelformat##1{#1{##1}#2}}
\define@lingkey{exnoformat}{\ep@mkexnoformater #1\@nil}
\def\ep@mkexnoformater #1X#2\@nil{%
   \def\ep@exnoformat##1{#1{##1}#2}}
\lingset{exnoformat=(X)}
\define@lingkey{fullrefformat}{\@fullrefformat #1\@nil}
\def\@fullrefformat #1X#2A#3\@nil{%
   \def\ep@fullrefformat##1##2{#1##1#2##2#3}}
%------ support for LaTex \label macro -----
\let\latex@tagex\relax
\let\latex@tagexlabel\relax
\ifx\label\relax \else    % else = LaTex is loaded
   \def\latex@tagexlabel{\def\@currentlabel
      {\ep@fullrefformat{{\the\excnt}}{\ep@label}}}%
   \def\latex@tagex{\edef\@currentlabel{\the\excnt}}%
   \fi
%-----------------------------------------
\definelabeltype{alpha}{labelgen=char,pexcnt=`a,labelformat=A.,
   fullrefformat=XA,labelalign=left,labelwidth=.72em}
\definelabeltype{caps}{labelgen=char,pexcnt=`A,labelformat=A.,
   fullrefformat=XA,labelalign=left,labelwidth=.92em}
\definelabeltype{numeric}{labelgen=number,pexcnt=1,labelformat=A.,
   fullrefformat=X.A,labelalign=right,labelwidth=.75em}
\definelabeltype{roman}{labelgen=romannumeral,pexcnt=1,labelformat=(A),
   fullrefformat=XA,labelalign=left,labelwidth=1.5em}
%-- tags and reference -----
%----- local reference to example numbers -----
%\def\nextx{{\@printref{\number\excnt}}}
%\def\anextx{{\@printref{\advance\excnt by 1 \number\excnt}}}
%\def\lastx{{\@printref{\advance\excnt by -1 \number\excnt}}}
%\def\blastx{{\@printref{\advance\excnt by -2 \number\excnt}}}
%\def\bblastx{{\@printref{\advance\excnt by -3 \number\excnt}}}
\def\nextx{{\@printref{\ep@rawexnoprint}}}
\def\anextx{{\@printref{\advance\excnt by 1 \ep@rawexnoprint}}}
\def\lastx{{\@printref{\advance\excnt by -1 \ep@rawexnoprint}}}
\def\blastx{{\@printref{\advance\excnt by -2 \ep@rawexnoprint}}}
\def\bblastx{{\@printref{\advance\excnt by -3 \ep@rawexnoprint}}}
%------ defining tags -----
\def\deftag#1#2{%
   {\let\@printref=\ep@identity
   \expandafter\xdef\csname lingtag@#2\endcsname{#1}%
   \if@g@thertags
      \immediate\write@tags{\noexpand\@fd@f {#2} {{#1}} }%
      \fi}%
   \ignorespaces
}
\def\deftaglabel#1{%
   \expandafter\xdef\csname lingtag@\@localextag.#1\endcsname%
      {{{\ep@label}}%
       {{\ep@fullrefformat{\@actualexno}\ep@label}}%
      }%
   \if@g@thertags
      \immediate\write@tags{%
         \noexpand\@fd@f
         {\@localextag.#1}
         {{{\ep@label}}%
          {{\ep@fullrefformat{\@actualexno}\ep@label}}}%
         }%
      \fi
   \ignorespaces
}
\def\deftagex#1{\edef\@localextag{#1}%
   \expandafter\xdef\csname lingtag@#1\endcsname{{\ep@rawexnoprint}}%
   \if@g@thertags
      \immediate\write@tags{\noexpand\@fd@f {#1} {{\ep@rawexnoprint}}}%
      \fi
   \ignorespaces
}
\def\deftagpage#1{%
   \if@g@thertags
      \write@tags{\noexpand\@fd@f #1 {{\the\pageno}}}%
      \fi
   \ignorespaces
}
\def\lastlabel{{\ep@label}}
\def\@expexwarn#1{\immediate\write16{====> EXPEX WARNING: #1.}}
\def\@expexerror#1{\immediate\write16{====> Fatal EXPEX ERROR: #1.}}
\newif\ifep@highlightref
\ep@highlightreffalse
\def\refproofing{\ep@highlightreftrue}
\def\mathhigh@lightref#1{$\overline{\underline{\hbox{#1}}}$}
\def\psthigh@lightref{\psframebox[boxsep=false,framesep=2pt,linewidth=.2ex]}
\ifx\PSTricksLoaded\endinput
      \let\@highlightprint\psthigh@lightref
   \else
      \let\@highlightprint\mathhigh@lightref
   \fi
\def\@printref#1{%
   \ifep@highlightref \@highlightprint{#1}\else #1\fi}
%%%%
\newbox\exnobox
\define@key{ling}{exno}{%
   \global\@specialexnotrue
   \let\latex@tagexlabel\ep@gobble
   \let\latex@tagex\ep@gobble
   \setbox\exnobox=\hbox{#1}%
   \def\ep@specialexno{\unhcopy\exnobox}%
}
%------ opening the tag file -----
\newif\if@g@thertags
\@g@thertagsfalse
\newwrite\ling@tagsfile
\def\write@tags{\write\ling@tagsfile}
\def\tagfilesuffix#1{\edef\@tagfilesuffix{#1}}
\def\@tagfilesuffix{-tags}
\def\gathertags{%
   \@setupreadtags
   \@g@thertagstrue
   \immediate\openout\ling@tagsfile=\jobname\@tagfilesuffix\relax
   \immediate\write@tags{\noexpand\relax}%
}
%------ reading the tag file and defining the tags it encodes -----
\newif\if@epx@goodtagsfile
\newread\ling@tagsin
\gdef\@fd@f#1 #2 {%
   \expandafter\ifx\csname lingtag@#1\endcsname\relax
      \expandafter\gdef\csname lingtag@#1\endcsname{#2}%
      \fi
}
\newif\if@readtags
\@readtagstrue
\def\@setupreadtags{\if@readtags
   \do@readtags \global\@readtagsfalse \fi}
\def\do@readtags{%
   \immediate\openin\ling@tagsin=\jobname\@tagfilesuffix\relax
   \ifeof\ling@tagsin \else
      \closein\ling@tagsin
      {\catcode`@=11 \input \jobname\@tagfilesuffix\relax}%
   \fi
}
%!
%!------ tagging sections, adapt to your needs -----
%! If \tagsec is used with section macros that do not define
%! counters \secno,\subsecno,\subsubsecno, and \subsubsubsecno,
%! then \currsec must be redefined to whatever is appropriate.
%!\def\chapscurrsec{\ifnum\chapno>0 \the\chapno
%!   \ifnum\secno>0 .\the\secno
%!   \ifnum\subsecno>0 .\the\subsecno
%!   \ifnum\subsubsecno>0 .\the\subsubsecno \fi\fi\fi\fi}
%!\def\nochapscurrsec{\ifnum\secno>0 .\the\secno
%!   \ifnum\subsecno>0 .\the\subsecno
%!   \ifnum\subsubsecno>0 .\the\subsubsecno \fi\fi\fi}
%! choose one of the following twos
%!\let\currsec\nochapscurrsec
%!\let\currsec\chapscurrsec
%!\def\deftagsec#1{\deftag\currsec{#1}}
%!/
%\def\deftaglabel#1{%
%   \expandafter\xdef\csname lingtag@\@localextag.#1\endcsname
%      {%
%      {\ep@expandonce\ep@label}%
%      {\ep@fullrefformat{\@actualexno}\ep@expandonce\ep@label}%
%      }%
%   \ignorespaces
%}
% Uncomment and use the following for debugging if needed
%\def\reporttag#1%
%  {\writeln{\expandafter\meaning\csname lingtag@#1\endcsname}}
\def\getref@aa#1#2{#1}%
\def\getref@ab#1#2{#2}%
\def\getref#1{\getref@a{#1}\getref@aa}
\def\getfullref#1{\getref@a{#1}\getref@ab}
\def\getref@a#1#2{%
   \if@readtags \@setupreadtags \fi
   \expandafter \ifcsname lingtag@#1\endcsname
         \edef\temp{\ep@expandtwice\csname lingtag@#1\endcsname}%
         \ifx\temp\empty
               \@expexwarn{+++tag #1 has no full reference}%
               \@printref{Missing!}%
            \else
               {\@printref{\temp}}%
            \fi
      \else
         \@expexwarn{tag #1 is called but not defined}%
         {\@printref{\tt [#1]}}%
      \fi
}
\newif\ifpartlabel
\newif\iffullref
\def\ep@ispartlabelcheck#1{\ep@ispart@a#1.\@nil}
\def\ep@ispart@a#1.#2\@nil{\def\temp{#2}%
   \ifx\temp\empty \partlabelfalse \else \partlabeltrue\fi}
\def\getref{\fullreffalse \getref@a}
\def\getfullref{\fullreftrue \getref@a}
%\def\getref@a#1{%
%   \if@readtags \@setupreadtags \fi
%   \ep@ispartlabelcheck{#1}%
%   \ifpartlabel
%         \iffullref
%               \let\@chooseref\chooseref@a
%            \else
%               \let\@chooseref\chooseref@g
%            \fi
%      \else
%         \let\@chooseref\relax
%      \fi
%   \expandafter \ifcsname lingtag@#1\endcsname
%   \edef\temp{\ep@expandtwice\csname lingtag@#1\endcsname}%
%   \ifx\temp\empty
%            \@expexwarn{+++tag #1 has no full reference}%
%            \@printref{Missing!}%
%         \else
%            {\@printref{\expandafter\@chooseref\temp}}%
%         \fi
%      \else
%         \@expexwarn{tag #1 is called but not defined}%
%         {\@printref{\tt [#1]}}%
%      \fi
%}
\def\getref@a#1{%
   \if@readtags \@setupreadtags \fi
   \ep@ispartlabelcheck{#1}%
   \ifpartlabel
         \iffullref
               \let\@chooseref\chooseref@a
            \else
               \let\@chooseref\chooseref@g
            \fi
      \else
         \let\@chooseref\relax
      \fi
   \expandafter\ifx\csname lingtag@#1\endcsname \relax
         \@expexwarn{tag #1 is called but not defined}%
         {\@printref{\tt [#1]}}%
      \else
         \expandafter\let\expandafter\temp
            \csname lingtag@#1\endcsname
         \@printref{\expandafter\@chooseref\temp}%
      \fi
}
\def\chooseref@a#1#2{#2}
\def\chooseref@g#1#2{#1}
% ----- glosses -----
\define@choicekey{ling}{glstyle}[\ling@glstyle\gl@stylenum]{wrap,nlevel}{%
   \ifcase\gl@stylenum
      \let\gl@beginglstyle\glw@begingl
      \let\endgl\glw@endgl
      \glw@assignlevels
      \let\glft=\glw@glft
   \else
      \let\gl@beginglstyle\gln@begingl
      \let\endgl\gln@endgl
      \let\glft=\gln@ft
      \fi
}
\def\begingl{\bgroup\@getoptionalarg\gl@begingl}
\def\gl@begingl{%
   \ling@usearg
   \ling@everygl
   \gl@beginglstyle
}
% parameters which are used in both gloss styles
\define@ling@cmdkeys{everygl,everyglpreamble,everygl,everyglft,
   everyglword,glrightskip,glhangindent,glwidth}
\define@lingincskipkeys{glspace,aboveglftskip,belowglpreambleskip}
\define@lingincdimenkeys{extraglskip,gllineskip}
\define@lingkey{abovemoreglskip}{\lingset{gllineskip=#1}} % obsolete
\define@boolkey{ling}[ling@]{autoglskip}{}
\define@boolkey{ling}[ling@]{glstruts}{%
   \ifling@glstruts \let\glstrut=\strut
   \else \let\glstrut=\relax \fi}
\define@choicekey{ling}{glhangstyle}[\temp\ep@glhangstyle]%
   {none,normal,cascade}{}
\lingset{glhangstyle=normal,glrightskip=0pt plus .1\hsize}
% dimensions
\newdimen\epd@gllineskip
% switches
\newif\if@glpreamble
%
% list manipulation macros that are used in both gloss styles
\newtoks\gltoks@a
\newtoks\gltoks@b
\def\gl@append #1\to #2{%
   \gltoks@a={\\{#1}}%
   \gltoks@b=\expandafter{#2}%
   \xdef#2{\the\gltoks@b\the\gltoks@a}%
}
\def\gl@lop#1\to#2{\ifx#1\empty
   \let#2\empty \else\expandafter\gl@lopoff#1\gl@lopoff#1#2\fi}
\long\def\gl@lopoff\\#1#2\gl@lopoff#3#4{\def#4{#1}\def#3{#2}}
\newif\ifgl@loopmore
%----- wrap style glosses -----
%\def\glpreamble #1// {%
%   \@glpreambletrue
% 2013/10/10 bug fix
%   {\hangindent=0pt \ling@everyglpreamble #1\par}%
%   {\hangafter=0 \ling@everyglpreamble #1\vskip\lingbelowpreambleskip}%
%}
%##
\def\glw@begingl{%
   \parindent=0pt
%   \glw@begingl@a
%}
%\def\glw@begingl@a{%
   \ifdim\ling@glwidth=0pt
      \bgroup
   \else
      \vtop\bgroup
         \hsize=\ling@glwidth
         \leftskip=0pt
   \fi
   \parindent=0pt
   \@glspacefalse
   \@glpreamblefalse
   \rightskip=\ling@glrightskip
   \lineskiplimit=0pt
   \ifling@autoglskip
      \let\glstrut=\strut
      \epd@gllineskip=\baselineskip
      \advance\epd@gllineskip by -\dp\strutbox
      \advance\epd@gllineskip by -\ht\strutbox
      \advance\epd@gllineskip by \lingextraglskip
   \else
      \epd@gllineskip=\linggllineskip
   \fi
   \edef\gl@lineskipsave{\noexpand\lineskip=\the\lineskip}%
   \ifcase\ep@glhangstyle
      \or
         \hangindent=\ling@glhangindent
         \hangafter=1
      \or
         \glwcascade@makeshape
         \parshape 9
         \glwcascade@parshapetarget
   \fi
   \let\mainlist=\empty
   \let\aboveskiplist=\empty
   \let\strutlist=\empty
   \let\everylist=\empty
}
% \glw@begingl sets up the environment
% \gla, \glb, etc. fill in the various lists which are initialized in
% the environment
% The input (something like the following)
% \gla x1 x2 x3 //
% \glb y1 y2 y3 //
% \glc z1 z2 z3 //
% gets converted into a list of lists (using Knuth's list macros)
%    {{x1,y1,z1},{x2,y2,z2},{x3,y3,z3}}
% auxiliary lists are also made at the same time
%    list of struts, list of "everygl<name>"
% Then these lists are popped, one position at a time, and vboxes
% are built.  The vboxes are fed into Tex's regular paragraph
% building machinary.
\def\glw@mergerow#1\to #2{%
   \let\alist=#1%
   \let\blist=#2%
   \let\clist=\empty
   \gl@loopmoretrue
   \loop \ifgl@loopmore\glw@mergerow@a \repeat
   \global\let#2=\clist
}
\def\glw@mergerow@a{%
   \gl@lop\blist\to\tempb
   \ifx\tempb\ep@Linebreak
         \expandafter\gl@append\ep@Linebreak\to\clist
      \else\ifx\tempb\ep@Closeup
         \expandafter\gl@append\ep@Closeup\to\clist
      \else\ifx\tempb\ep@Lbrack
         \expandafter\gl@append\ep@Lbrack\to\clist
      \else\ifx\tempb\ep@Rbrack
         \expandafter\gl@append\ep@Rbrack\to\clist
      \else
         \gl@lop\alist\to\tempa
         \expandafter\gl@append \tempa \to \tempb
         \expandafter\gl@append \tempb \to \clist
      \fi\fi\fi\fi
   \ifx\alist\empty \ifx\blist\empty \gl@loopmorefalse \fi\fi
}
% As the list of lists is assembled, heights and depths of each
% item are computed and maximum row heights and maximum row depths
% are computed in each row.  These are used to make struts which
% are used to assemble the vboxes.
\newdimen\glw@maxht
\newdimen\glw@maxdp
\define@ling@cmdkeys{everygla,everyglb,everyglc}
% exceptional items
\def\ep@Closeup{@}\@onelevel@sanitize\ep@Closeup
\def\ep@Linebreak{+}\@onelevel@sanitize\ep@Linebreak
\def\ep@Lbrack{[}\@onelevel@sanitize\ep@Lbrack
\def\ep@Rbrack{]}\@onelevel@sanitize\ep@Rbrack
\def\@alinelabel{a}
\def\glw@gla{%
   \if@glpreamble \vskip\lingbelowglpreambleskip \fi
   \def\lingaboveglaskip{0pt}%
   \def\gl@linelabel{a}%
   \glw@maxht=0pt
   \glw@maxdp=0pt
   \@getoptionalarg
   \glw@gla@a
}
\def\glw@gla@a #1//{\bgroup\ling@usearg
   \glw@gla@b #1 \@nil }
\def\glw@gla@b{\@ifnextchar\@nil\glw@gla@c\glw@gla@d}
\def\glw@gla@c#1{\glw@updatelists \egroup \ignorespaces}
\newif\ifglw@word
\def\glw@gla@d #1 {%
   \glw@wordtrue
   \def\temp{#1}%
   \ifx\temp\empty \else \glw@dodiacritics #1\@nil \fi
   \ifglw@word
%      \gltoks@a={#1}%
%      \gltoks@b=\expandafter{\ling@everygla}%
      \gl@append \\{\glstrut #1}\to\mainlist
      \setbox0=\hbox{\ling@everygla #1}%
      \ifdim\glw@maxht<\ht0 \glw@maxht=\ht0 \fi
      \ifdim\glw@maxdp<\dp0 \glw@maxdp=\dp0 \fi
      \fi
   \glw@gla@b
}
\def\glw@dodiacritics #1#2\@nil{%
   \def\tempa{#2}%
   \ifx\tempa\empty
      \glw@wordfalse
      \if#1\ep@Linebreak
            \expandafter\gl@append\ep@Linebreak\to\mainlist
         \else\if#1\ep@Closeup
            \expandafter\gl@append\ep@Closeup\to\mainlist
         \else\if#1\ep@Lbrack
            \expandafter\gl@append\ep@Lbrack\to\mainlist
         \else\if#1\ep@Rbrack
            \expandafter\gl@append\ep@Rbrack\to\mainlist
         \else \glw@wordtrue
         \fi\fi\fi\fi\fi
}
%
\newdimen\gl@maxdplast
\def\glw@updatelists{%
   \edef\temp{\csname lingabovegl\gl@linelabel skip\endcsname}%
   \expandafter\gl@append\temp \to\aboveskiplist
   \edef\temp{\vrule width0pt height\the\glw@maxht\space depth\the\glw@maxdp\space }%
   \global\gl@maxdplast=\glw@maxdp
   \expandafter\gl@append\temp\to\strutlist
   \expandafter\expandafter\expandafter
      \gl@append\csname ling@everygl\gl@linelabel\endcsname
      \to\everylist
   \gltoks@a=\expandafter{\everylist}%
}
\def\glw@assignlevels{%
   \expandafter\XKV@for@n\expandafter{\glw@levels}\levelname
   {\glw@assign@level\levelname}%
}
\def\glw@assign@level#1{%
   \edef\temp{\noexpand\let
      \expandafter\noexpand\csname gl#1\endcsname
      \expandafter\noexpand\csname glw@gl#1\endcsname}%
   \temp
}
% \glw@gla and \glw@glft get their definitions directly
% \glw@gl<x> gets defined via \defineglwlevels, which adds <x> to the
% list \glw@levels of defined glw levels
\def\glw@levels{a,ft}
\def\define@glw@level#1{%
   \expandafter\ifx\csname glw@gl#1\endcsname\relax
      \define@ling@cmdkeys{everygl#1}%
      \define@linginckey\skip{abovegl#1skip}%
      \XKV@addtolist@o\glw@levels{#1}%
      \lingset{everygl#1=,abovegl#1skip=0pt}
      \expandafter\def\csname glw@gl#1\endcsname{\glw@glx{#1}}%
      \ifx\gl@stylenum\undefined \else
         \ifnum\gl@stylenum=0 \glw@assign@level{#1}\fi\fi
   \else
      \@expexwarn{Level #1 is already defined}
   \fi
}
% It is anticipated that \definegl?levels will be defined in the
% future.
\def\defineglwlevels#1{\XKV@for@n{#1}\thislevel
   {\expandafter\define@glw@level\expandafter{\thislevel}}%
}
\def\glw@glx#1{%   x suggests any label (as argument to \glw@glx)
   \def\worklist{}%
   \glw@maxht=0pt
   \glw@maxdp=0pt
   \def\gl@linelabel{#1}%
   \@getoptionalarg\glw@glx@a
}
\def\glw@glx@a #1// {\bgroup
   \ling@usearg
   \expandafter\let\expandafter\@every
      \csname ling@everygl\gl@linelabel\endcsname
   \glw@glx@b #1 \@nil }
\def\glw@glx@b{\@ifnextchar\@nil\glw@glx@c\glw@glx@d}
\def\glw@glx@c#1{\glw@updatelists \glw@mergerow\worklist\to\mainlist\egroup}
\def\glw@glx@d #1 {%
   \setbox0=\hbox{\@every #1}%
   \ifdim\glw@maxht<\ht0 \glw@maxht=\ht0 \fi
   \ifdim\glw@maxdp<\dp0 \glw@maxdp=\dp0 \fi
   \gl@append {\glstrut #1}\to\worklist
   \glw@glx@b
}
\defineglwlevels{b,c}
\newif\ifglw@spacebefore
\long\def\glpreamble #1// {%
   \@glpreambletrue
   {\hangindent=0pt \hangafter=0 \ling@everyglpreamble #1\vskip\lingbelowpreambleskip}%
}
%##
\def\glw@glft{\@getoptionalarg\glw@glft@a}
% 2011-12-09 make \glw@glft@a a long definition to allow \par's
\long\def\glw@glft@a #1//{%
   \lineskip=\epd@gllineskip
   \glw@print \par
% 2013-06-27  bug fix
%   \prevdepth=\gl@maxdplast
\ifx\glstrut\strut
      \ifdim\gl@maxdplast>\dp\strutbox \prevdepth=\gl@maxdplast \fi
      \fi
   \ling@usearg
   \vskip\lingaboveglftskip
   \ling@everyglft
   \hangindent=0pt
   \let\endgl=\glw@endgl@ii
   #1\par
}
\def\glw@endgl{\glw@endgl@i \glw@endgl@ii}
\def\glw@endgl@i{\lineskip=\epd@gllineskip
   \ifx\mainlist\empty \else \glw@print \par \fi }
\def\glw@endgl@ii{\egroup\egroup}
\def\glw@print{%
   \glw@spacebeforefalse
   \@glpostbrackfalse
   \leavevmode
   \gl@loopmoretrue
   \@glaparsestate=1
   \loop\ifgl@loopmore
      \gl@lop\mainlist\to\tempa\relax       % \tempa is one column
      \ifx\tempa\ep@Linebreak
            \vskip\epd@gllineskip
            \@glaparsestate=1
            \leavevmode
         \else\ifx\tempa\ep@Closeup
            \@glaparsestate=1
         \else\ifx\tempa\ep@Lbrack
            \ifnum\@glaparsestate=0 \hskip\lingglspace
               \else\ifnum\@glaparsestate=2 \hskip\lingglbrackbracksep
               \else\ifnum\@glaparsestate=3 \hskip\lingglspace
               \fi\fi\fi
            \printlbrack\nobreak
            \@glaparsestate=2
         \else\ifx\tempa\ep@Rbrack
            \nobreak
            \ifnum\@glaparsestate=0 \hskip\lingglbrackwordsep
               \else\ifnum\@glaparsestate=2 \hskip\lingglspace
               \else\ifnum\@glaparsestate=3 \hskip\lingglbrackbracksep
               \fi\fi\fi
            \printrbrack
            \@glaparsestate=3
         \else
            \ifnum\@glaparsestate=0 \hskip\lingglspace
               \else\ifnum\@glaparsestate=2 \hskip\lingglbrackwordsep
               \else\ifnum\@glaparsestate=3 \hskip\lingglspace
               \fi\fi\fi
            \@glaparsestate=0
            \glw@printcol
         \fi\fi\fi\fi
      \ifx\mainlist\empty \gl@loopmorefalse \fi
      \repeat
}
% 10/8/13 revise the definition of \glw@printcol to allow items to be
% center aligned in gloss words
\def\gl@lopTL#1\to#2{\ifx#1\empty
   \let#2\empty \else\expandafter\gl@lopoffTL#1\gl@lopoffTL#1#2\fi}
\long\def\gl@lopoffTL\\#1#2\gl@lopoffTL#3#4{#4={#1}\def#3{#2}}
\newtoks\gltoks@sofar
\newtoks\gltoks@every
\newtoks\gltoks@next
\def\glw@printcol{%
   \begingroup
   \gltoks@sofar={}%
   \gl@loopmoretrue
   \loop\ifgl@loopmore
      \gl@lop\aboveskiplist\to\@aboveskip
      \expandafter\ifdim\@aboveskip=0pt \def\@above{}\else
         \edef\@above{\noalign{\vskip\@aboveskip}}\fi
      \gl@lop\strutlist\to\@strut
      \gl@lopTL\everylist\to\gltoks@every
      \gl@lopTL\tempa\to\gltoks@next
      \edef\temp{\the\gltoks@sofar \@above \@strut \the\gltoks@every \the\gltoks@next \cr}%
      \gltoks@sofar=\expandafter{\temp}%
      \ifx\tempa\empty \gl@loopmorefalse \fi
      \repeat
   \vtop{%
      \gl@lineskipsave
      \ling@everyglword
      \halign{##\hfil\cr \the\gltoks@sofar}}%
   \endgroup
}
% ----- brackets -----
\newif\if@glpostbrack
\newcount\@glaparsestate          % 0 normal, 1 post @, 2 post [, 3 post ]
\define@lingincdimenkeys{glbrackbracksep,glbrackwordsep}
\lingset{glbrackbracksep=.05em,glbrackwordsep=.1em}
% 2011-12-09 introduce hook into \printlback and \printrbrack to allow font selection
\define@ling@cmdkeys{everybrack}
\def\printlbrack{{\ling@everybrack [}}
\def\printrbrack{{\ling@everybrack ]}}
\lingset{everybrack=\rm}
%\def\printrbrack{$]$}
%\def\printlbrack{$[$}
% ----- cascading hanging indentation -----
\newdimen\glwcascade@pshapeindent
\newdimen\glwcascade@pshapelinewd
\def\glwcascade@parshapetarget{}
\def\glwcascade@mkshapeaux{%
   \edef\glwcascade@parshapetarget
      {\glwcascade@parshapetarget\space
         \the\glwcascade@pshapeindent\space\the\glwcascade@pshapelinewd}%
   \advance\glwcascade@pshapeindent by \ling@glhangindent
   \advance\glwcascade@pshapelinewd by -\ling@glhangindent
}
\def\glwcascade@mkshapeauxaux{\glwcascade@mkshapeaux\glwcascade@mkshapeaux
   \glwcascade@mkshapeaux}
\def\glwcascade@makeshape{%
   \glwcascade@pshapeindent=0pt
   \glwcascade@pshapelinewd=\hsize
   \glwcascade@mkshapeaux\glwcascade@mkshapeaux
   \glwcascade@mkshapeaux\glwcascade@mkshapeaux
}
\def\glwcascade@makeshape{%
   \glwcascade@pshapeindent=0pt
   \glwcascade@pshapelinewd=\hsize
   \glwcascade@mkshapeauxaux\glwcascade@mkshapeauxaux\glwcascade@mkshapeauxaux
}
% ----- side by side (ss) gloss style -----
\define@choicekey{ling}{glftpos}[\temp\ep@glftpos]%
   {below,right}{%
      \ifcase\ep@glftpos
         \let\gl@beginglstyle\glw@begingl
         \let\endgl\glw@endgl
         \let\glft=\glw@glft
      \or
         \let\gl@beginglstyle=\glw@ss@begingl
         \let\endgl=\glw@ss@endgl
         \let\glft\glw@ss@glft
      \fi
}
\define@lingcmdkeys{sssep,ssratio,ssrightskip}
\lingset{sssep=2em,ssratio=.6,ssrightskip=0pt plus 2em}
\newdimen\ssleftwd
\newdimen\ssrightwd
\def\glw@ss@begingl{%
   \ep@setssdims
   \leavevmode\bgroup\hbox\bgroup
      \hsize=\ssleftwd
      \lingset{glwidth=\ssleftwd}
      \glw@begingl
}
\def\ep@setssdims{%
   \dimen0 =\hsize
   \advance\dimen0 by -\leftskip
   \advance\dimen0 by -\lingsssep
   \ssleftwd=\lingssratio\dimen0
   \ssrightwd=\dimen0
   \advance\ssrightwd by -\ssleftwd
}
\def\glw@ss@endgl{\egroup\egroup}
\def\glw@ss@glft #1//{%
   \glw@endgl
   \hskip\lingsssep
   \vtop{%
      \leftskip=0pt
      \rightskip=\lingssrightskip
      \parindent=0pt
      \hsize=\ssrightwd
      \ling@everyglft
      #1}%
      \ignorespaces
}
% ----- gloss with a side panel
\define@lingcmdkeys{everypanel}
\lingset{everypanel={}}
\def\beginglpanel{\@getoptionalarg\beginglpanel@a}
\def\beginglpanel@a{%
   \bgroup
   \let\endgl=\endgl@panel
   \ling@usearg
   \ep@setssdims
   \leavevmode
      \lingset{glwidth=\ssleftwd}
      \begingl
}
\def\endgl@panel{%
   \glw@endgl
   \hfill
   \vtop\bgroup
   \hsize=\ssrightwd
   \leftskip=0pt
   \rightskip=\lingssrightskip
   \lingeverypanel
}
\def\endpanel{\egroup\egroup\par}
% ----- underfixes -----
\def\gluf/#1/#2/{%
   \vtop{\offinterlineskip\halign{\hfil##\hfil\cr
      \strut #1\cr
      \noalign{\vskip-\ling@glufcloseup}
      \ling@everygluf \strut#2\cr
}}}
\define@ling@cmdkeys{everygluf,glufcloseup}
%\lingset{glufcloseup=.4ex,everygluf=\sc}
% ----- gloss comments and citations -----
\def\rightcomment#1{\leavevmode\rlap{%
   \hbox to\hsize{\hfil \rm #1\hskip\leftskip}}\ignorespaces}
\let\rightcite=\rightcomment % for backwards compatability
\define@ling@cmdkeys{mincitesep}
\lingset{mincitesep=1.5em}
% jf 2011-12-09 introduce everytrailingcitation hook
\define@ling@cmdkeys{everytrailingcitation}
\def\trailingcitation#1{%
   \hskip\ling@mincitesep plus 1fill
   \penalty100\null\nobreak \hskip 0pt plus 1fill
   \hbox{\ling@everytrailingcitation #1}%
}
\lingset{everytrailingcitation=}
%%%%%%%%%%%%%%%%%%%% nlevel %%%%%%%%%%%%%%%%%%%%
\def\gln@begingl{\bgroup
   \@glspacefalse \@getoptionalarg\gln@A}
\def\gln@A{\@ifnextchar\@cs\gln@B\gln@C}
\def\gln@B#1{\gln@C}
\def\gln@C{%
   \ling@usearg
   \ifdim\lingglbaselineskip=0pt \else
      \baselineskip=\lingglbaselineskip \fi
   \lineskip=\lingextraglskip
   \rightskip=\ling@glrightskip
   \ifdim\ling@glhangindent>0pt \hangindent=\ling@glhangindent
      \hangafter=1 \fi
   \leavevmode
   \gln@a
}
\define@key{ling}{glbaselineskip}[0pt]%
   {\@setinckey\skip\lingglbaselineskip#1\@nil}%
\lingset{glbaselineskip}
\newif\if@glspace
\def\gln@a{\futurelet\temp\gln@b}
\def\gln@b{\ifx\temp\endgl \let\next=\relax
   \else \ifx\temp\nogloss \let\next=\gln@bi
   \else \ifx\temp\glft \let\next=\gln@bii
   \else \let\next=\gln@c \fi\fi\fi \next }
\def\gln@bi{\expandafter\gln@c}
\def\gln@bii#1{\vskip\lingaboveglftskip\strut}
\def\gln@c #1[#2]#3 {%
   \def\tempa{\\{#1}}%
   \def\afterword{#3}%
   \gln@d #2/\@nil
   \gln@a
}
\def\nogloss#1{{#1}[]}%
\def\gln@d{\@ifnextchar\@space\gln@di\gln@dii}
\def\gln@di#1#2/{\gl@append #2\to\tempa \gln@e}
\def\gln@dii#1/{\gl@append #1\to\tempa \gln@e}
\def\gln@e{\@ifnextchar\@nil\gln@f\gln@d}
\def\gln@f#1{%
   \gln@printglword
   \ifx\afterword\empty
      \else \expandafter\gln@diacritic\afterword\@nil \fi
}
\def\gln@printglword{%
   \if@glspace \hskip\lingglspace \else \@glspacetrue \fi
   \vtop{%
      \lineskip=0pt
      \ling@everyglword
      \gl@loopmoretrue
      \loop\ifgl@loopmore
         \gl@lop\tempa\to\@tempa
         \gl@lop\gln@aboveskiplist\to\@aboveskip
         \gl@lop\gln@everyline\to\@every
         \ifx\@aboveskip\empty \else
            \expandafter\ifdim\@aboveskip=0pt \else
               \vskip\@aboveskip \fi\fi
         \hbox{\@every\strut\@tempa}%
         \ifx\tempa\empty \gl@loopmorefalse \fi
      \repeat
      }%
}
\def\gln@diacritic#1#2\@nil{%
   \def\temp{#2}\ifx\temp\empty \else \ep@diacriticerror{#1#2}\fi
   \if\noexpand #1\noexpand @\let\next\@glspacefalse
   \else \if\noexpand #1\noexpand +\let\next\gln@after@c
   \else \ep@diacriticerror{#1}\fi\fi
   \next
}
\def\gln@after@c{%
   \endgraf
   \ifdim\ling@glhangindent>0pt
      \hangindent=\ling@glhangindent \hangafter=0 \fi
   \leavevmode
   \@glspacefalse}
\def\ep@diacriticerror#1{%
   \@expexerror{bad gloss diacritic: #1 (only @ and + permitted)}\end}
\def\gln@endgl{\par\egroup\egroup}
\def\gln@ft{\vskip\lingaboveglftskip\strut}
\define@lingkey{glneveryline}{%
   \setlist\gln@everyline{#1}}
\define@lingkey{glnabovelineextraskip}{%
   \setlist\gln@aboveskiplist{#1}}
\def\setlist#1#2{\def#1{}\XKV@for@n{#2}\@this{%
   \expandafter\gl@append\@this\to#1}}
\lingset{glneveryline={},glnabovelineextraskip={}}
%%%%%%%%%%%%%%%% end nlevel %%%%%%%%%%%%%%%%%%%%
%! ----- initial settings -----
\definelingstyle{factorysettings}{%
   aboveexskip=2.7ex plus .8ex minus .8ex,
   belowexskip=2.7ex plus .8ex minus .8ex,
   Everyex=,
   everyex=,
   numoffset=0pt,
   labelanchor=numright,
   labeloffset=1em,
   labelwidth=.78em,
   textanchor=normal,
   textoffset=1em,
   preambleanchor=numright,
   preambleoffset=1em,
   avoidnumlabelclash=false,
   appendtopexarg=,
   labeltype=alpha,
   everylabel=,
   labelalign=left,
   belowpreambleskip=1ex,
   interpartskip=1ex,
   splitexpenalty=200,
   exbreakfil=0pt plus 4ex,
   exbreakpenalty=-50,
   splitpartspenalty=200,
% parameters used in glosses
   glspace=.5em plus.4em minus.15em,
   glrightskip=0pt plus .1\hsize,
   aboveglcskip=0pt,
   aboveglftskip=1ex,
   belowglpreambleskip=1ex,
   everyglpreamble=,
   glhangindent=1em,
   everygla=\it,
   everyglb=,
   everyglc=,
   everyglft=,
   everygl=,
   everyglword=,
   autoglskip=true,
   glwidth=0pt,
   glufcloseup=.4ex,
   everygluf=,
%   everybracket=\rm,
%   glbracketsep=.15em,
   glstyle=wrap,
%   moregloffset=0pt,
   gllineskip=1ex,
   extraglskip=0pt,
   mincitesep=1.5em,
% auxiliary parameters used for building tables
   dima=2.4em,
   crskip=.6em
}
\lingset{lingstyle=factorysettings}
%  restores version 4.0 gloss behaviour
\def\gloldstyle{%
   \lingset{abovemoreglskip=1ex,autoglskip=false,
      glstruts=false,glspace=.6em}}
%
%!
%! addons can be put in expex-add.tex
%! can be used to override factory settings and make
%!   definitions and redefinitions that the user wants to alway be available
%! expex-add.tex can be put in the main Tex tree, or in the local
%! directory, depending on where the user wants it to have force
%!
%\newread\expexadd
%\openin\expexadd = expex-add.tex
%\ifeof\expexadd \else
%   \closein\expexadd \input expex-add \fi

