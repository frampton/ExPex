\documentclass[12pt,a4paper,french]{book}
\usepackage[T1]{fontenc}
%\usepackage[utf8]{inputenc}
\renewcommand{\baselinestretch}{1.3}
\usepackage{babel}
\usepackage{expex}
\newcommand{\pt}{\textsc{2sg.fam}}
\newcommand{\pv}{\textsc{2sg.pol}}
\usepackage{fontspec}
\begin{document}

\lingset{glstyle=nlevel,glossbreaking=true,glrightskip=0pt plus 5em,glspace=1em
plus 3em}

\ex[everyglpreamble=\tt,glhangstyle=none,glspace=1.5em minus .5em,
   glnabovelineskip={,-6pt},glrightskip=0pt plus 10em]
\def\\#1{{\bfseries{#1}}}%
\raggedbottom
\begingl
\glpreamble
ART : dzień dobry \\{pani} \\{mario} artur
górski z tej strony (.) nie nie proszę nie przełączać do ojca ja mam
krótkie pytanie do \\{pani} czy \\{pamięta} \\{pani}
sprawę którą ojciec prowadził przeciwko Tadeuszowi Zawadzkiemu/ czy ten
zawadzki ma firmę konsultingową/ gdyby \\{pani} \\{mogła}
sprawdzić dokładnie (.) tak zaczekam (.) rozumiem (.) i jak to się
skończyło/ wiedziałem że \\{pani} \\{jest} jak najlepszy
komputer \\{Pani} \\{Mario} (.) dziękuję bardzo aha proszę
nie mówić ojcu, że dzwoniłem dobrze/ dobrze dziękuję bardzo do widzenia
\endpreamble
{ART :}[{}]   {dzień dobry}[Bonjour]   \\{pani}[MM-{\sc voc}]
\\{mario}[FN-{\sc voc}]   artur[FN-{\sc nom}]   górski[LN-{\sc nom}]
z[de] tej[ce]   strony[coté]   {(.)}[{}]   nie[non]   nie[non]
proszę[{s'il vous plaît}]   nie[{\sc neg}]
przełączać[{mettre en communication}]   do[chez]
ojca[père] ja[{\sc 1sg}]   mam[avoir-{\sc 1sg}]   krótkie[court]
pytanie[question]   do[pour]   \\{pani}[\pv~]   czy[{\sc q}]
\\{pamięta}[{se souvenir-\pv}]   \\{pani}[\pv~]
sprawę[affaire]   którą[que]   ojciec[père]
prowadził[mener-{\sc 3sg.pst}]   przeciwko[contre]   Tadeuszowi[FN-{\sc dat}]
Zawadzkiemu/[LN-{\sc dat}]   czy[{\sc q}]   ten[ce]   zawadzki[LN-{\sc nom}]
ma[avoir-{\sc 3sg}]   firmę[société]   konsultingową/[{de consulting}]
gdyby[si]   \\{pani}[\pv~]   \\{mogła}[pouvoir-{\sc 3sg}]
sprawdzić[vérifier]   dokładnie[{en détaille}]   {(.)}[{}]   tak[oui]
zaczekam[attendre-{\sc 1sg.fut}]   {(.)}[{}] rozumiem[comprendre-{\sc 1sg}]
{(.)}[{}]   i[et]   jak[comment]   to[ce]   się[{\sc part.refl}]
skończyło/[finir-{\sc 3sg.pst}] wiedziałem[savoir-{\sc 1sg.pst}]   że[que]
\\{pani}[\pv~] \\{jest}[être-{\sc 3sg}] jak[comme]
najlepszy[meilleur] komputer[ordinateur]   \\{Pani}[MM-{\sc voc}]
\\{Mario}[FN-{\sc voc}]   {(.)}[{}]   dziękuję[remercier-{\sc 1sg}]
bardzo[beaucoup]   aha[{\sc pe}]   proszę[{s'il vous plaît}]   nie[{\sc neg}]
mówić[dire]   ojcu,[père] że[que]   dzwoniłem[appeler-{\sc 1sg.pst}]
dobrze/[bien]   dobrze[bien] dziękuję[remercier-{\sc 1sg}] bardzo[beaucoup]
do[au]   widzenia[revoir]
%
\glft Bonjour Madame Marie, c'est Artur Gorski ici. Non, non, ne me passez pas
mon père. J'ai une petit question à vous poser, est-ce que vous vous rappelez
l'affaire que mon père menait contre Tadeusz Zawacki? Est-ce que ce Zawadzki
possède un société de consulting? Si vous pouviez vérifier pour être sûr.
Oui j'attends. Je comprends et comment ça s'est finit? Je savais que vous êtes
comme le meilleur ordinateur Madame Marie. Merci beaucoup. Ah, ne dites au père
que j'ai appelé, d'accord? bien, merci beaucoup. Au revoir.//
\endgl
\xe

\end{document}
