\makeatletter

% ----- reading tags -----
\def\g@tr@f#1#2{%
   \@setupreadtags
   \expandafter\ifx\csname lingtag@#1\endcsname\relax
      \let#2\relax
      \writeln{WARNING: The tag [#1] is not defined.}%
      {\tt $\bullet$#1}%
   \else
      \edef#2{\csname lingtag@#1\endcsname}%
   \fi
}
\def\baregetref#1{\csname lingtag@#1\endcsname}
\def\getref#1{%
   \g@tr@f{#1}\temp
   \ifx\temp\relax
   \else
      \@printref{\temp}%
   \fi
}
\def\getfullref#1{%
   \g@tr@f{#1}\temp
   \ifx\temp\relax \else \get@fullref #1\@nil \fi
}
\def\get@fullref #1.#2\@nil{%
   \g@tr@f{#1}\tempa
   \ifx\tempa\relax
   \else
      \@printref{\tempa\expandafter\@ling@putperiod\temp}%
   \fi
}
\let\@printref\relax
\def\refproofing{%
   \ifx\PSTricksLoaded\endinput
      \let\@printref\@psthighlightref
   \else
      \let\@printref\@highlightref
   \fi\relax
}
\def\@psthighlightref{%
   \leavevmode
   \psframebox[boxsep=false,framesep=4pt,linewidth=.24ex]%
}
\def\@highlightref#1{$\overline{\underline{\hbox{#1}}}$}
\def\@ling@putperiod#1{\ifx#1\char \else .\fi #1}
% ----- opening the tag file -----
\newif\if@g@thertags
\@g@thertagsfalse
\newwrite\ling@tagsfile
\def\write@tags{\write\ling@tagsfile}
%\edef\@@tagfilename{\jobname-tags.tex}
\def\tagfilesuffix#1{\edef\@tagfilesuffix{#1}}
\def\@tagfilesuffix{-tags}
\def\gathertags{%
   \@g@thertagstrue
   \immediate\openout\ling@tagsfile=\jobname\@tagfilesuffix\relax
}
% ----- reading the tag file and defining the tags it encodes -----
\newread\ling@tagsin
\def\@fd@f#1 #2 {\expandafter\xdef\csname lingtag@#1\endcsname{#2}}
\def\@setupreadtags{%
   \immediate\openin\ling@tagsin=\jobname\@tagfilesuffix\relax
   \ifeof\ling@tagsin \else
      \closein\ling@tagsin \input \jobname\@tagfilesuffix\relax
   \fi
   \global\let\@setupreadtags=\relax
}
%\@setupreadtags
% ----- making entries in the tag file -----
%\def\currex{\the\excnt}
\define@key[jXKV]{ling}{tag}{%
   \deftagex{#1}%
   \edef\@localextag{#1}%
}
\def\deftagex#1{\edef\@localextag{#1}\@isextagtrue\deftag{\the\excnt}{#1}}
\def\deftag#1#2{%
   {\let\@printref=\identity
   \expandafter\xdef\csname lingtag@#2\endcsname{#1}%
   \if@g@thertags
      \immediate\write@tags{\noexpand\@fd@f #2 #1 }%
      \fi}%
   \ignorespaces
}
\def\deftagpage#1{%
   \if@g@thertags
      \write@tags{\noexpand\@fd@f #1 {\the\pageno}}%
   \fi
   \ignorespaces
}
\def\deftaglabel#1{%
   \if@isextag
      \expandafter\xdef\csname lingtag@\@localextag.#1\endcsname
         {\@@label}%
      \if@g@thertags
         \immediate\write@tags
            {\noexpand\@fd@f \@localextag.#1 {\@@label}}\fi
      \else
         \writeln{[#1] needs an ex tag}%
      \fi
   \ignorespaces
}
% If \tagsec is used with section macros that do not define
% counters \secno,\subsecno,\subsubsecno, and \subsubsubsecno,
% then \currsec must be redefined to whatever is appropriate.
\def\chapscurrsec{\ifnum\chapno>0 \the\chapno
   \ifnum\secno>0 .\the\secno
   \ifnum\subsecno>0 .\the\subsecno
   \ifnum\subsubsecno>0 .\the\subsubsecno \fi\fi\fi\fi}
\def\nochapscurrsec{\ifnum\secno>0 .\the\secno
   \ifnum\subsecno>0 .\the\subsecno
   \ifnum\subsubsecno>0 .\the\subsubsecno \fi\fi\fi}
\let\currsec\nochapscurrsec
\def\deftagsec#1{\deftag\currsec{#1}}

\newif\iftracingtagrefs
\resetatcatcode

