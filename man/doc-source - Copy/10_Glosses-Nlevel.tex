

%\makeatletter
%\lingset{glnabovelineskip={0,7pt}}
%\gl@lop\gln@abovelineskip\to\temp
%\writeln{====\meaning\gln@abovelineskip}
%\def\goop{\writeln{\meaning\gln@abovelineskip}}
%\def\foop{\writeln{*****\meaning\gln@abovelineskip*****}}
%\resetatcatcode
%\foop
%\ex[glstyle=nlevel,glneveryline={\it,\sc,\sc},
%   glnabovelineskip={,-2pt}]
%\bgroup
%{\lingset{glstyle=nlevel,glneveryline={\it,\sc,\sc},
%   glnabovelineskip={,-2pt}}}
%\setlist\L{1,2}
%\egroup
%\foop




\def\minigloss#1#2{\outlinebox{%
   \vtop{\halign{\ministrut ##\hfil\cr #1\cr #2\cr}}}\hskip1pt}%
\let\ministrut=\relax
\def\miniglossbare#1#2{\vtop{\halign{##\hfil\cr
   #1\cr #2\cr}}}
\def\outlinebox{\psframebox[framesep=0,boxsep=false,linewidth=.4pt]}



\section Nlevel glosses; an alternate coding syntax

\deftagsec{nlevelsec}
\begininventory
\macros* \idx{|\endpreamble|}\endmc
\parameters*
\idx{|glstyle|}& |wrap| or |nlevel|& |wrap|\cr
\endinventory
The last section assumed the default setting, `wrap'.  The `nlevel'
gloss style produces identical displays, but they are coded
differently. Compare the two ways of coding the display (\nextx),
which repeats (\getref{wapm}).

\framedisplay
\ex[glstyle=nlevel]
\begingl
k-[CL/2] wapm[V/see] -a[AGR/\sc 3acc] -s'i[NEG]
-m[AGR/\sc 2pl] -wapunin[TNS/preterit] -uk[AGR/\sc 3pl]
\glft `you (pl) didn't see them'
\endgl
\xe
\endframedisplay

\noindent
\vtop{\hsize=3in \leftskip=0pt
\codedisplay~
\ex[glstyle=nlevel]
\begingl
k-[CL/2]
wapm[V/see]
-a[AGR/\sc 3acc]
-s'i[NEG]
-m[AGR/\sc 2pl]
-wapunin[TNS/preterit]
-uk[AGR/\sc 3pl]
\glft `you (pl) didn't see them'
\endgl
\xe
|endcodedisplay
}
\qquad
\vtop{\hsize=3in \leftskip=0pt
\codedisplay~
\ex[glstyle=wrap]
\begingl
\gla k- wapm -a -s'i -m
   -wapunin -uk //
\glb CL V AGR NEG AGR TNS AGR //
\glc 2 see {\sc 3acc} {}
   {\sc 2pl} preterit {\sc 3pl} //
\glft `you (pl) didn't see them'//
\endgl
\xe
|endcodedisplay
}

The advantage to this style of coding complex glosses is that the
gloss of a word and the word itself are adjacent in the code, just as
they are in the display which is produced; vertically adjacent in the
display, horizontally adjacent in the code. This makes the code much
more readable. In effect, an aspect of WYSIWG is built into the
coding. This is particularly useful if the gloss has many words, the
gloss of a narrative for example.

Putting aside consideration of the preamble and the free translation
for the moment, the
list between |\begingl| and |\endgl| is processed as a space separated
list.  Only top level spaces are separators.  Spaces inside
|[|\dots|]| are effectively hidden from this parsing.  So, for
example, the space in |-a[AGR/\sc 3acc]| does not mislead the parser.
The material inside |[|\dots|]| is processed as a |/| separated list.
Of course, |/| in this material must be hidden from the parser so |/|
cannot appear at the top level.  The same is true of |]|, for obvious
reasons.

The glossed words are written on separate lines above, but this is
only for clarity.  The code below is equivalent.  It saves on virtual
paper, but is not as easily deciphered.

\codedisplay~
\ex[glstyle=nlevel]\begingl k-[CL/2] wapm[V/see] -a[AGR/\sc 3acc]
-s'i[NEG] -m[AGR/\sc 2pl] -wapunin[TNS/preterit] -uk[AGR/\sc 3pl] \glft
`you (pl) didn't see them'\endgl\xe
|endcodedisplay

The preamble and free translation in the wrap style are terminated by
|//|.  In the nlevel style, |\endpreamble| ends the preamble and
|\endgl| terminates the free translation.  There is no special
termination of the free translation.  This is illustrated by the
coding of (\getref{sicher}) in the nlevel style.

\codedisplay
\ex[glstyle=nlevel]
\begingl
\glpreamble Mary ist sicher, dass es den Hans nicht st\"oren w\"urde
seiner Freundin ihr Herz auszusch\"utten.\endpreamble
Mary$_i$[Mary]
ist[is]
sicher,[sure]
dass[that]
es[it]
den[the-\sc acc]
Hans[Hans]
nicht[not]
st\"oren[annoy]
w\"urde[would]
seiner[his-\sc dat]
Freundin[girlfriend-\sc dat]
ihr$_i$[her-\sc acc]
Herz[heart-\sc acc]
auszusch\"utten.[out to throw]
\glft  `Mary is sure that it would not annoy John to reveal her heart to his girlfriend.'
\endgl
\xe
|endcodedisplay


\subsection Parameters which modify particular lines

The various lines in a gloss which is coded in the wrap style are
identified by name, so parameters like |everygla| and |aboveglbskip|
can be used to modify the named line.  |everygla| modifies the |\gla|
line and |aboveglbskip| modifies the |\glb| line.  This method for
modifying lines is not available for coding in the nlevel style
because the lines are not numbered.
Instead, they  must be referred to positionally.

\begininventory
\parameters
\idx{|glneveryline|}& list of token lists& |{\it}|\cr
\idx{|glnabovelineskip|}& list of dimensions& |{}|\cr
\endinventory

\framedisplay
\ex[glstyle=nlevel,glneveryline={\it,\sc,\sc},
   glnabovelineskip={,-2pt}]
\begingl
k-[cl/2]
wapm[v/\rm see]
-a[agr/3acc]
-s'i[neg]
-m[agr/\sc 2pl]
-wapunin[tns/preterit]
-uk[agr/3pl]
\endgl
\xe
\endframedisplay
\codedisplay~
\ex[glstyle=nlevel,glneveryline={\it,\sc,\sc},
   glnabovelineskip={,-2pt}]
\begingl
k-[cl/2]
wapm[v/\rm see]
-a[agr/3acc]
-s'i[neg]
-m[agr/\sc 2pl]
-wapunin[tns/preterit]
-uk[agr/3pl]
\endgl
\xe
|endcodedisplay
Note that a list of dimensions |{,-2pt}| with dimensions missing is
acceptable; the missing dimensions are assumed to be |0pt|.  Missing
entries in the value of |glneveryline|, either because the list is
shorter than the number of lines in the interlinear gloss or because
of null entries on the list, are similarly assumed to be null and
cause no problem.  The material inside the brackets, or material
delimited by |/| can be missing as well, but the brackets are
mandatory.  Spaces after |[| or |/| are ignored, so for example,
|wapm[v/\rm see]| and |wapm[ v/ \rm see]| produce the same output.

There is no parameter |aboveglaskip| for use in the wrap coding style
since vertical skip in the interlinear gloss makes sense only between
lines.  Similarly, the first item in the list specified by
|glnabovelineskip|, which corresponds to |\aboveglaskip|, is ignored.

%\subsection The syntax of coding nlevel glosses
%
%\def\<#1>{$\langle$#1$\rangle$}
%
%Formally, the syntax of nlevel glosses can be given as a system of
%rewriting rules.  Quantities in angle brackets are either specified in
%other rewriting rules or explained in words. \<spaces> consists of one
%or more spaces.  $\vert$ is logical or.
%
%\bigskip
%\begingroup
%\leftskip=\parindent
%\parindent=0pt
%\<gloss display> $\to$ |\begingl| \<interlinear gloss> \<spaces> |\endgl|
%$\vert$\par \qquad |\begingl| \<interlinear gloss> \<spaces> |\glft|
%\<free translation> |\endgl|
%
%\<interlinear gloss> $\to$ \<word gloss> $\vert$
%   \<word gloss> \<spaces> \<interlinear gloss>
%
%\<word gloss> $\to$
%   \<word> |[| \<gloss> |]| $\vert$
%   \<word> |[| \<gloss> |]| \<diacritic>
%
%\<gloss> $\to$ \<gloss item> $\vert$ \<gloss item> |/| \<gloss>
%
%\<diacritic> $\to$ |@| $\vert$ |+|
%\bigskip
%\endgroup
%\noindent
%\<word>, and \<gloss item> must be material
%which can appear in the context |\hbox{|\dots|}|. \<word> cannot
%contain top-level spaces. \<gloss item> cannot contain either
%top-level |/| or top-level |]| (but can contain top-level spaces).

\subsection \ttcs{nogloss} and the diacritics {\tt @} and {\tt +}\thinspace

\def\ge#1{\langle\hbox{\sl #1}\,\rangle}

\noindent A word gloss $\ge{word}|[|\ge{gloss$_1$}|/|\ldots|]|$ can
optionally be immediately followed by one of |@| and |+|.  A |@|
cancels the usual space between the typeset word glosses.  A |+|
introduces a line break.  This should be compared with the
different syntax in the wrap style; see Sections \getref{atsec} and
\getref{plussec}.  In the nlevel style they are diacritics; in the
wrap style they are exceptional words.  For example, the following
code produces the same output as the code which is given for
(\getref{wiye}).

\codedisplay
\ex[glstyle=nlevel]
\begingl wiye[two] kepi[whitemen] e-[\sc 1p:3d-]@ ca[found] \endgl
\xe
|endcodedisplay

|\nogloss| in nlevel style glosses works very much
the same way that it works in wrap style glosses.

%\framedisplay
%\ex
%\begingl[glstyle=nlevel,glneveryline={},glhangindent=0pt]
%Fa'nu'i[show] yu'[me] ni[Obl] {[[\thinspace}[]@
%{\it O\/}[Op] t{\it in}aitai-mu[{\it WH\/}{[obj]}.read-agr]@
%{{\it t}\thinspace]}[] na[L] {lepblu\thinspace ].}[book]
%\endgl
%\xe
%\endframedisplay
%\codedisplay~
%\ex
%\begingl[glstyle=nlevel,glneveryline={},glhangindent=0pt]
%Fa'nu'i[show] yu'[me] ni[Obl] {[[\thinspace}[]@
%{\it O\/}[Op] t{\it in}aitai-mu[{\it WH\/}{[obj]}.read-agr]
%\bare{{\it t}\thinspace]} na[L] {lepblu\thinspace ].}[book]
%\endgl
%\xe
%|endcodedisplay
%
%A term like |{[[\thinspace}[]| is somewhat confusing, because brackets
%are both typeset and used to demarcate the gloss; |[[\thinspace| is a
%word, but |[]| is an empty gloss.  The macro |\nogloss| is provided to
%make the coding of examples with empty glosses more straightforward.

\framedisplay
\ex
\begingl[glstyle=nlevel,glneveryline={}]
Fa'nu'i[show]
yu'[me] ni[Obl]
\nogloss{[[\thinspace}@ {\it O}[Op]
t{\it in\/}aitai-mu[{\it WH\/}{[obj]}.read-agr]
\nogloss{{\it t}\thinspace ]}
na[L]
lepblu[book]@ \nogloss{].}
\endgl
\xe
\endframedisplay
\def\goop{\kern2pt \putfnno}
\codedisplay
\ex
\begingl[glstyle=nlevel,glneveryline={}]
Fa'nu'i[show]
yu'[me]
ni[Obl]
\nogloss{[[\thinspace}@ {\it O}[Op]
t{\it in\/}aitai-mu[{\it WH\/}{[obj]}.read-agr]
\nogloss{{\it t}\thinspace ]}
na[L]
lepblu[book]@ \nogloss{].}
\endgl
\xe|goop
|endcodedisplay
\vfootnote{\the\fnno}{%
Italics are used to highlight an infix in the verb in the
relative clause.  The spacing could be improved.  Horizontal skip should be
used ; |{\it \hskip.8pt in\hskip.4pt}|, for example.}

\subsection Line spacing inside glwords

One spacing problem that is handled automatically in the wrap style
must be handled by the user in the nlevel style.  It is not common,
but is worth mentioning because it might cause a perplexing problem if
it does arise.  Characters on lower levels which are extra tall can
cause misalignment of the baselines.  Suppose, for example, that you
want to use a gloss to give some information about the
morphophonological derivation of a surface form, as is attempted
in (\nextx).

\framedisplay
\ex[glstyle=nlevel,glneveryline={\it}]
\gdef\AccentedBarredW{$\acute{\hbox{$\overline w$}}$}%
\begingl m-[(mo-)] wope[(a\AccentedBarredW ope)] \endgl \xe
\endframedisplay
\codedisplay
\ex[glstyle=nlevel,glneveryline={\it}]
\begingl m-[(mo-)] wope[(a\AccentedBarredW ope)] \endgl \xe
|endcodedisplay
Not only is more space between the lines needed, but there is
misalignment of the baselines on the second level.

One might think to fix the problem in (\lastx) by putting extra
vertical skip between the lines of the gloss, as in the code below.

\codedisplay
\ex[glstyle=nlevel,glneveryline={\it},glnabovelineskip={,.5ex}]
\begingl m-[(mo-)] wope[(a\AccentedBarredW ope)] \endgl \xe
|endcodedisplay

But the result is less than satisfactory:

\framedisplay
\ex[glstyle=nlevel,glneveryline={\it},glnabovelineskip={,.5ex}]
\begingl m-[(mo-)] wope[(a\AccentedBarredW ope)] \endgl \xe
\endframedisplay
There is now vertical space over the second line of the glwords, but
the baselines of the second lines of the glwords are not aligned.  In
wrap style coding, the material for all of the glwords is accumulated
before any of the glwords are constructed, so the presence of an extra
tall line in the second glword is known before the first glword is
constructed.  An adjustment is made to preserve baseline alignment. In
nlevel style coding, the first glword is boxed and contributed to a
paragraph before the material to construct the second glword is read,
so there is no way for any adjustment to be made automatically.

There are two solutions.  One solution for the problem above is to
insert extra tall struts on the second line of every glword.
Normal struts in the 12pt type used in this manual are \textdim{10 pt}
high and \textdim{4 pt} deep.  The following code will produce (\nextx).

\codedisplay
\ex[glstyle=nlevel,glneveryline={\it,\vrule height14pt width0pt}]
\begingl m-[(mo-)] wope[(a\AccentedBarredW ope)] \endgl
\xe
|endcodedisplay
\framedisplay~
\ex[glstyle=nlevel,glneveryline={\it,\vrule height14pt width0pt}]
\begingl m-[(mo-)] wope[(a\AccentedBarredW ope)] \endgl \xe
\endframedisplay

Another solution is to increase the baselineskip inside glwords.

\codedisplay
\ex[glstyle=nlevel,glneveryline={\it},everyglword={\baselineskip=18pt}]
\begingl m-[(mo-)] wope[(a\AccentedBarredW ope)] \endgl \xe
|endcodedisplay
\framedisplay~
\ex[glstyle=nlevel,glneveryline={\it},everyglword={\baselineskip=18pt}]
\begingl m-[(mo-)] wope[(a\AccentedBarredW ope)] \endgl
\xe
\endframedisplay


