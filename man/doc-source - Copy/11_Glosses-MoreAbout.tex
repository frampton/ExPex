
\section More about glosses

\deftagsec{moreaboutsec}
\subsection The parameter {\tt glwidth}

\begininventory
\parameters*
\idx{|glwidth|}& dimension& \textdim{0 pt}\cr
\endinventory
If the parameter |glwidth| is set to a nonzero value, the gloss is
built in a vbox whose width is the setting of |glwidth|. The following
example illustrates the usefulness of the explicit width option.

\ex[glwidth=2.6in]<sicher2>
a.\quad
\begingl
\gla Mary$_i$ ist sicher, dass es den Hans nicht st\"oren w\"urde
seiner Freundin ihr$_i$ Herz auszusch\"utten.//
\glb Mary is sure that it the-{\sc acc} Hans not annoy would
his-{\sc dat} girlfriend-{\sc dat} her-{\sc acc} heart-{\sc acc} {out to
throw}//
\glft  `Mary is sure that it would not annoy John to reveal her
heart to his girlfriend.'//
\endgl
\hfil
b.\quad
\begingl
\gla Mary$_i$ ist sicher, dass seiner Freunden ihr$_i$ Herz
auszuch\"utten dem Hans nicht schaden w\"urde.//
\glb Mary is sure that his-{\sc dat} girlfriend-{\sc dat} her-{\sc acc}
heart-{\sc acc} {out to throw} the-{\sc dat} Hans not damage would//
\glft `Mary is sure that to reveal her heart to his girlfriend
would not damage John.'//
\endgl
\xe

\codedisplay~
\ex[glwidth=2.6in]
a.\quad
\begingl
\gla Mary$_i$ ist sicher, dass es den Hans nicht st\"oren w\"urde
seiner Freundin ihr$_i$ Herz auszusch\"utten.//
\glb Mary is sure that it the-{\sc acc} Hans not annoy would
his-{\sc dat} girlfriend-{\sc dat} her-{\sc acc} heart-{\sc acc} {out to
throw}//
\glft  `Mary is sure that it would not annoy John to reveal her
heart to his girlfriend.'//
\endgl
\hfil
b.\quad
\begingl
\gla Mary$_i$ ist sicher, dass seiner Freunden ihr$_i$ Herz
auszuch\"utten dem Hans nicht schaden w\"urde.//
\glb Mary is sure that his-{\sc dat} girlfriend-{\sc dat} her-{\sc acc}
heart-{\sc acc} {out to throw} the-{\sc dat} Hans not damage would//
\glft `Mary is sure that to reveal her heart to his girlfriend
would not damage John.'//
\endgl
\xe
|endcodedisplay



%
%
%
%\subsection The width of the gloss
%
%In the wrap style (the only gloss style currently implemented),
%the gloss is built in a vbox whose width is determined implicitly
%if the parameter |glwidth| is set to \textdim{0 pt}. The width is
%$h-l-r$, where $h$, $l$, and $r$, are the current values of
%|\hsize|, |\leftskip|, and |\rightskip|, respectively.  This
%implicit determination of the width of the gloss is appropriate
%for use with the \ExPex\ macros which typeset examples because
%they adjust the leftskip appropriately inside examples.
%
%If you want to supply an example number or explicit label, it will not
%work to say something like the following if |glwidth| is set to
%\textdim{0 pt}.
%
%\def\goop{\dots\ }
%\codedisplay
%[A]\quad \begingl |goop \endgl
%|endcodedisplay

%The vbox built by the gloss macro will not fit on the same line with
%the [A].
%
%You must say:
%
%\def\temp{\dots }
%\codedisplay
%\ex[exno=A,exnoformat={[X]}] \begingl |temp \endgl
%|endcodedisplay
%The braces around |[A]| are needed to so that the optional argument
%is correctly delineated.  For example:
%
%\framedisplay
%\ex[exno={(6), p. 14},exnoformat={[X]}]
%\begingl
%\gla Mary$_i$ ist sicher, dass es den Hans nicht st\"oren w\"urde
%seiner Freundin ihr$_i$ Herz auszusch\"utten.//
%\glb Mary is sure that it the-{\sc acc} Hans not annoy would
%his-{\sc dat} girlfriend-{\sc dat} her-{\sc acc} heart-{\sc acc} {out to
%throw}//
%\endgl
%\xe
%\endframedisplay
%\codedisplay~
%\ex[exno={(6), p. 14},exnoformat={[X]}]
%\begingl
%\gla Mary$_i$ ist sicher, dass es den Hans nicht st\"oren w\"urde
%seiner Freundin ihr$_i$ Herz auszusch\"utten.//
%\glb Mary is sure that it the-{\sc acc} Hans not annoy would
%his-{\sc dat} girlfriend-{\sc dat} her-{\sc acc} heart-{\sc acc} {out to
%throw}//
%\endgl
%\xe
%|endcodedisplay
%
%If the parameter |glwidth| is set to a nonzero dimension, the width of
%the vbox that the gloss is constructed in is the specified dimension.
%
%The following example illustrates the usefulness of the explicit
%width option.
%
%\ex[glwidth=2.6in]<sicher2>
%a.\quad
%\begingl
%\gla Mary$_i$ ist sicher, dass es den Hans nicht st\"oren w\"urde
%seiner Freundin ihr$_i$ Herz auszusch\"utten.//
%\glb Mary is sure that it the-{\sc acc} Hans not annoy would
%his-{\sc dat} girlfriend-{\sc dat} her-{\sc acc} heart-{\sc acc} {out to
%throw}//
%\glft  `Mary is sure that it would not annoy John to reveal her
%heart to his girlfriend.'//
%\endgl
%\hfil
%b.\quad
%\begingl
%\gla Mary$_i$ ist sicher, dass seiner Freunden ihr$_i$ Herz
%auszuch\"utten dem Hans nicht schaden w\"urde.//
%\glb Mary is sure that his-{\sc dat} girlfriend-{\sc dat} her-{\sc acc}
%heart-{\sc acc} {out to throw} the-{\sc dat} Hans not damage would//
%\glft `Mary is sure that to reveal her heart to his girlfriend
%would not damage John.'//
%\endgl
%\xe
%
%\codedisplay~
%\ex[glwidth=2.6in]
%a.\quad
%\begingl
%\gla Mary$_i$ ist sicher, dass es den Hans nicht st\"oren w\"urde
%seiner Freundin ihr$_i$ Herz auszusch\"utten.//
%\glb Mary is sure that it the-{\sc acc} Hans not annoy would
%his-{\sc dat} girlfriend-{\sc dat} her-{\sc acc} heart-{\sc acc} {out to
%throw}//
%|exbreak\glft  `Mary is sure that it would not annoy John to reveal her
%heart to his girlfriend.'//
%\endgl
%\hfil
%b.\quad
%\begingl
%\gla Mary$_i$ ist sicher, dass seiner Freunden ihr$_i$ Herz
%auszuch\"utten dem Hans nicht schaden w\"urde.//
%\glb Mary is sure that his-{\sc dat} girlfriend-{\sc dat} her-{\sc acc}
%heart-{\sc acc} {out to throw} the-{\sc dat} Hans not damage would//
%\glft `Mary is sure that to reveal her heart to his girlfriend
%would not damage John.'//
%\endgl
%\xe
%|endcodedisplay


\subsection Comments and citations

\begininventory
\macros
\idx{|\trailingcitation|}, \idx{|\rightcomment|}\endmc
\parameters
\idx{|mincitesep|}& dimension& \textdim{1.5 em}\cr
\idx{|everytrailingcitation|}& token list& empty\cr
\endinventory

The following illustrates two different ways to append a comment or
citation to a gloss.

\framedisplay
\ex
\begingl
\gla \rightcomment{[Potawatami]}k- wapm -a -s'i -m -wapunin -uk //
\glb \rightcomment{category}Cl V Agr$_1$ Neg Agr$_2$ Tns Agr$_3$//
\glb 2 see {\sc 3acc} {} {\sc 2pl} preterit {\sc 3pl} //
\glft `you (pl) didn't see them'\trailingcitation{(Hockett 1948,
   p. 143)}//
\endgl
\xe
\endframedisplay
\codedisplay~
\ex
\begingl
\gla \rightcomment{[Potawatami]}k- wapm -a -s'i -m -wapunin -uk //
\glb \rightcomment{category}Cl V Agr$_1$ Neg Agr$_2$ Tns Agr$_3$//
\glb 2 see {\sc 3acc} {} {\sc 2pl} preterit {\sc 3pl} //
\glft `you (pl) didn't see them'\trailingcitation{(Hockett 1948,
   p. 143)}//
\endgl
\xe
|endcodedisplay

\noindent or

\codedisplay~
\ex
\begingl[glstyle=nlevel,glneveryline={\it}]
{\rightcomment{[Potawatami]}k-}[\rightcomment{category}Cl/2]
wapm[V/see]
-a[Agr$_1$/\sc 3acc]
-s'i[Neg]
-m[Agr$_2$/\sc 2pl]
-wapunin[Tns/peterit]
-uk[Agr$_3$/\sc 3pl]
\glft `you (pl) didn't see them'\trailingcitation{(Hockett 1948,
   p. 143)}
\endgl
\xe
|endcodedisplay
|\trailingcitation| will put the citation on the same line as the last
line of the free translation if there is enough room for it, otherwise at the
end of the  following line.  The parameter |mincitesep| determines the
minimum whitespace between the end of the free translation and the
citation that is tolerated; the default is \textdim{1.5 em}.
|everytrailingcitation| can be used to specify the font for trailing
citations.

|\trailingcitation| is useful outside the context of glosses.

\framedisplay
\pex
\a\relax [Which pilot who shot at it$_1$]$_2$ hit [which
MIG$_2$ that had chased him$_2$]$_1$?\trailingcitation{(Barss, 2000)}

\a\relax [Which pilot who shot at it$_1$]$_2$ hit [which MIG$_2$ that had
chased him$_2$]$_1$?\trailingcitation{(Higgenbotham and May, 1981)}
\xe
\endframedisplay
\codedisplay~
\pex
\a\relax [Which pilot who shot at it$_1$]$_2$ hit [which MIG$_2$ that had
chased him$_2$]$_1$?\trailingcitation{(Barss, 2000)}

\a\relax [Which pilot who shot at it$_1$]$_2$ hit [which MIG$_2$ that had
chased him$_2$]$_1$?\trailingcitation{(Higgenbotham and May, 1981)}
\xe
|endcodedisplay


|\rightcomment{|\dots|}| must come first in the first item on the
line.  The macro is very primitive.  It does not
consider the width of the citation or the amount of whitespace at
the right of the gloss.  The citation will overlap the gloss if
there is not room for it at the right.  For example, if the gloss
(\lastx) is attempted with an hsize of \textdim{3.5 in} and
|glspace=.5em|, the result is (\nextx).

\framedisplay
\hsize=3.5in
\ex
\rightcomment{[Potawatami]}
\begingl
\gla k- wapm -a -s'i -m -wapunin -uk //
\glb \rightcomment{category}Cl V Agr$_1$ Neg Agr$_2$ Tns Agr$_3$//
\glb 2 see {\sc 3acc} {} {\sc 2pl} preterit {\sc 3pl} //
\glft `you (pl) didn't see them'\trailingcitation{(Hockett 1948,
   p. 143)}//
\endgl
\xe
\endframedisplay

In spite of its limitations, |\rightcomment| is occasionally
quite useful.


\subsection  Line spacing in wrapped glosses

Users might prefer to increase \parvalue{extraglskip}.  Compare
(\nextx a), which uses the default setting of |extraglskip|
(\textdim{.5 ex}), to (\nextx b), which increases this by another
\textdim{.5 ex}.

\framedisplay
\pex[extraglskip=2pt]
\a \begingl
\gla Um-\"asudda' h\"am yan \nogloss{$[\,$} @ i taotao \nogloss{$[\,$} @ {\it O\/}
ni si Juan ilek-\~na nu guahu \nogloss{$[\,$} @ mal\"agu' gui
\nogloss{$[\,$} @
asudd\"a'-\~na \nogloss{{\it t\/}$\,]]]]$.}//
\glb agr-meet we with the person Op Comp the Juan say-agr Obl me
agr.want he {\it WH\/}[obl].meet-agr//
\endgl
\a \begingl[extraglskip=!.5ex]
\gla Um-\"asudda' h\"am yan \nogloss{$[\,$} @ i taotao \nogloss{$[\,$} @ {\it O\/}
ni si Juan ilek-\~na nu guahu \nogloss{$[\,$} @ mal\"agu' gui
\nogloss{$[\,$} @
asudd\"a'-\~na \nogloss{{\it t\/}$\,]]]]$.}//
\glb agr-meet we with the person Op Comp the Juan say-agr Obl me
agr.want he {\it WH\/}[obl].meet-agr//
\endgl
\xe
\endframedisplay
\def\temp{\quad \dots\ {\rm ({\it same as the gloss above\/})} }%
\codedisplay~
\pex[everygla=]
\a \begingl
\gla Um-\"asudda' h\"am yan \nogloss{$[\,$} @ i taotao \nogloss{$[\,$}
@ {\it O\/} ni si Juan ilek-\~na nu guahu \nogloss{$[\,$} @ mal\"agu'
gui \nogloss{$[\,$} @ asudd\"a'-\~na \nogloss{{\it t\/}$\,]]]]$.}//
\glb agr-meet we with the person Op Comp the Juan say-agr Obl me
agr.want he {\it WH\/}[obl].meet-agr//
\endgl
\a \begingl[extraglskip=!.5ex] |temp \endgl
\xe
|endcodedisplay


\subsubsection How to avoid overfull boxes in glosses
\deftagsec{overfullsec}

\begininventory
\parameters
%\idx{|glspace|}& skip& \textdim{.5 em} plus \textdim{.4 em} minus
%   \textdim{.15 em}\hfil\cr
\idx{|glrightskip|}& skip& \textdim{0 pt} plus \textdim{.1 hsize}\hfil \cr
\endinventory
%
Hopefully, you will never encounter the problem which the
parameterization in this section is designed to solve, the
``overfull line'' error message.  The default setting puts
\textdim{.5 em} of glue between the glwords on a line, plus
\textdim{.4 em} of stretchability and \textdim{.15 em} of
shrinkage.  The interglword spacing on a line can therefore vary
from \textdim{.35 em} up to \textdim{.9 em} to accommodate the
needs of the Tex linebreaking algorithm.\footnote{%
This is a much larger range than is typical in running text, but
it is appropriate in glosses, which typically have many large
whitespace gaps on one line or the other.}
Further, the default settings allow up to 10\% of the hsize of
whitespace to appear at the end of the line of glwords.  The
possible extra space at the right edge and
stretchability/shrinkability of the space between glwords means
that line breaking in glosses will rarely encounter problems with
overfull lines.

In unusual cases, there may be a problem.  Your publisher (or
you) may be particularly fussy and demand a particular spacing
and/or better right alignment in glosses.  Or you might want to make
an unusually narrow gloss, which increases the chances that there
might not be enough flexibility in the spacing.

If you do run into the problem of overfull lines in a gloss, two
parameters allow for a great deal of flexibility; |glspace| and
|glrightskip|. |glspace| is an incremental parameter, so you could say,
for example, |\lingset{glspace=!0pt plus .2em}|, increasing the
stretchability of the interglword space by \textdim{.2 em}. Or,
you might want to increase the stretchability of the rightskip
in a particular troublesome gloss to allow more whitespace at the
right edge.  An acceptable solution will depend on your
particular typesetting aesthetics. Solving line breaking problems
is often troublesome and requires experimentation.


