
\section Tables in examples

\begininventory
\macros* \idx{|\hwit|}\endmc
\endinventory
Most of the difficulty of formatting example displays which
contain tables comes from formatting the table itself. This
manual will not teach the reader how to use \Tex\ to format
tables. It will be assumed that the reader knows how to use the
\Tex\ primitive |\halign| or the \LaTex\ macros based on
|\halign|.  But this section might contribute something to
understanding how to use the table making tools in linguistics
examples.

Many tabular examples have the form

\def\goop{\enspace\dots\enspace}
\codedisplay
\ex \vtop{\halign{|goop}}\xe
|endcodedisplay

For example:

\framedisplay
\ex
\vtop{\halign{#\hfil&& \qquad #\hfil\cr
baudh& bu-baudh& know, wake\cr
smai& si-smai& smile\cr
suap& su-suap& sleep\cr
miaks& mi-miaks& glitter\cr
auc& u-auc& please\cr
}}
\xe
\endframedisplay

\codedisplay
\ex \vtop{\halign{%
#\hfil&& \qquad #\hfil\cr
baudh& bu-baudh& know, wake\cr
smai& si-smai& smile\cr
suap& su-suap& sleep\cr
miaks& mi-miaks& glitter\cr
auc& u-auc& please\cr
}}\xe|endcodedisplay

A more elaborate version of (\lastx), with a title and labeled
columns, is given below. |\hwit| is described below.

\framedisplay
\ex  The perfect stems of some roots with a
high vowel in their nucleus\par\nobreak\medskip
\quad\vbox{\halign{%
#\hfil&& \hskip3em #\hfil\cr
\hfil\hwit{root}& \hfil\hwit{perfect stem}&
   \hfil\hwit{gloss}\cr
\noalign{\smallskip}
baudh& bu-baudh& `know, wake'\cr
smai& si-smai& `smile'\cr
suap& su-suap& `sleep'\cr
miaks& mi-miaks& `glitter'\cr
auc& u-auc& `please'\cr
}}\xe
\endframedisplay

\codedisplay~
\ex  The perfect stems of some roots with a
high vowel in their nucleus\par\nobreak\medskip
\quad\vbox{\halign{%
#\hfil&& \hskip3em #\hfil\cr
\hfil\hwit{root}& \hfil\hwit{perfect stem}&
   \hfil\hwit{gloss}\cr
\noalign{\smallskip}
baudh& bu-baudh& `know, wake'\cr
smai& si-smai& `smile'\cr
suap& su-suap& `sleep'\cr
miaks& mi-miaks& `glitter'\cr
auc& u-auc& `please'\cr
}}\xe |endcodedisplay

\noindent \idx{|\hwit|} (hidewidth italics) inserts the italicized
label into the alignment in such a way that it is centered over
the nonwhite portion of the column it heads.  It hangs over
equally on both sides if necessary. Hiding the width of column
labels is often important so that the column labels do not affect
the column widths.

|\par\nobreak| appears after the title so that page breaking does
not detach the title from the table that follows.

\subsection Tables with labeled lines

\begininventory
\macros \idx{|\labels[]|}, \idx{|\tl|}, \idx{|\nl|}\endmc
\endinventory
If reference must be made to particular lines in (\lastx), the
lines need labels of some sort.  One approach is to explicitly
enter the line labels a--e:

\framedisplay
\ex  The perfect stems of some roots with a
high vowel in their nucleus\par\nobreak\medskip
\quad\vbox{\halign{%
#\hfil& \quad #\hfil&& \hskip3em #\hfil\cr
& \hfil\hwit{root}& \hfil\hwit{perfect stem}&
   \hfil\hwit{gloss}\cr
\noalign{\smallskip}
a.& baudh& bu-baudh& `know, wake\cr
b.& smai& si-smai& `smile'\cr
c.& suap& su-suap& `sleep'\cr
d.& miaks& mi-miaks& `glitter'\cr
e.& auc& u-auc& `please'\cr
}}\xe
\endframedisplay

\codedisplay
\ex  The perfect stems of some roots with a
high vowel in their nucleus\par\nobreak\medskip
\quad\vbox{\halign{%
#\hfil& \quad #\hfil&& \hskip3em #\hfil\cr
& \hfil\hwit{root}& \hfil\hwit{perfect stem}&
   \hfil\hwit{gloss}\cr
\noalign{\smallskip}
a.& baudh& bu-baudh& `know, wake'\cr
b.& smai& si-smai& `smile'\cr
c.& suap& su-suap& `sleep'\cr
d.& miaks& mi-miaks& `glitter'\cr
e.& auc& u-auc& `please'\cr
}}\xe |endcodedisplay

\noindent If a line is deleted or added, or if lines are
interchanged for some reason, considerable relabeling may be
required.

\ExPex\  provides some macros which simplify this code and make
it easier to manipulate.  They are used in the alternate code for
(\lastx) below and described below. \idx{|\labels|} initializes
the counter |\pexcnt|, which is then used to generate the labels.
It also activates the macros \idx{|\tl|} (table label) and
\idx{|\nl|} (no label). |\tl| inserts the appropriate label, with
following period, and increments the counter. |\nl| abbreviates
|\omit\hfil|, so that it can be used to prevent the appearance of
a label in a cell. |\labels| takes parameters, so you can say
things like |\labels[labeltype=caps,everylabel=\it]|.  Of course,
these parameters can also be set at the |\ex| level, if desired, or
even globally.

\codedisplay
\ex  The perfect stems of some roots with a
high vowel in their nucleus\par\nobreak\medskip
\quad\vbox{\labels\halign{%
\tl #\hfil& #\hfil& \quad #\hfil&& \hskip3em #\hfil\cr
\nl & \hfil\hwit{root}& \hfil\hwit{perfect stem}&
   \hfil\hwit{gloss}\cr
\noalign{\smallskip}
& baudh& bu-baudh& `know, wake'\cr
& smai& si-smai& `smile'\cr
& suap& su-suap& `sleep'\cr
& miaks& mi-miaks& `glitter'\cr
& auc& u-auc& `please'\cr
}}\xe |endcodedisplay

The advantages of implicit line label insertion should be
obvious.  One often decides to insert another entry, or to delete
an entry.  If the labels are inserted explicitly, this usually
requires changing multiple labels.  If the table has many lines,
this is particularly onerous.  References in the text to
particular lines also need to changed to match the new labeling.
We will see below that lines in tables can be named and reference
made by name, using implicit line numbering.

\exbreak
\subsection Tagging implicit labels in tables

\sidx{|\deftaglabel| inside tables}
|\deftaglabel| can be used to associate a tag with a label that is
introduced into a table by |\tl|~(``table label'').  |\tl| can
also read an optional tag using the |<|\dots|>| mechanism.

\codedisplay
\ex<Washo>
\vtop{\labels\halign{\tl #\hfil&& \quad #\hfil\cr
\nl & \hwit{Root}& \hwit{Plural}& \hwit{Gloss}\cr
& baloxat& baloxaxat& bows\cr
& moya& moyaya& shoulder\cr
<A>& nent'us& net'unt'us& old women\cr
<B>& mokgo& mogokgo& shoes\cr
}}
\xe
Examples (\getfullref{Washo.A}) and (\getfullref{Washo.B}) are the most
complex, and therefore the most revealing. Examples
(\getref{Washo}\getref{Washo.A},\getref{Washo.B}) are the most complex,
and therefore the most revealing.
|endcodedisplay

\framedisplay~[fdcorrect=top]
\ex<Washo>
\vtop{\labels\halign{\tl #\hfil&& \quad #\hfil\cr
\nl & \hwit{Root}& \hwit{Plural}& \hwit{Gloss}\cr
& baloxat& baloxaxat& bows\cr
& moya& moyaya& shoulder\cr
\deftaglabel{A}& nent'us& net'unt'us& old women\cr
\deftaglabel{B}& mokgo& mogokgo& shoes\cr
}}\xe
Examples (\getfullref{Washo.A}) and (\getfullref{Washo.B}) are
the most complex, and therefore the most revealing.
Examples (\getref{Washo}\getref{Washo.A},\getref{Washo.B}) are
the most complex, and therefore the most revealing.
\endframedisplay
\medskip
\noindent

\subsection Some useful table making tools

\begininventory
\macros* % \idx{|\lingdima|}, \idx{|\lingdimb|}, \idx{|\lingdimc|},
   \idx{|\tspace|}|[]|\endmc
\parameters
\idx{|dima|\user}& dimension& |2.4em|\cr
\idx{|dimb|\user}& dimension& (not set)\cr
\idx{|dimc|\user}& dimension& (not set)\cr
\idx{|crskip|}& skip& |1.2ex|\cr
\endinventory
%
Tables often need considerable adjustment in order to
balance the needs of readability, space limitations, and matching
the typographic structure to the conceptual structure.
Sometimes this requires a delicate balancing act. \ExPex\
provides three parameters (scratch dimensions) |dima|, |dimb|,
and |dimc|, and corresponding macros which expand to their
settings, which can be used for this.  Additionally,
|\tspace[�key�]| expands to |\hskip\ling�tag�|, so that
horizontal skip like |\tspace[dima]| or |\tspace[textoffset]| can
be easily used in table construction.  The code for (\lastx)
could be written:

\codedisplay
\ex[textoffset=1em,dima=1em,dimb=3em]
The perfect stems of some roots with a high vowel in their
nucleus\par\nobreak\medskip
\tspace[dima]\vbox{\labels\halign{%
\tl #\hfil& \tspace[textoffset]#\hfil&& \tspace[dimb]#\hfil\cr
\nl & \hfil\hwit{root}& \hfil\hwit{perfect stem}&
   \hfil\hwit{gloss}\cr |endcodedisplay
\removelastskip\vskip-1ex
$\vdots$
\bigskip

\noindent Localizing all the parameters which might need
adjustment in the optional argument of |\ex| helps organize the
adjustment process.

If no optional argument is supplied to |\tspace|, it expands to
|\hskip\lingdima|.

The macro \idx{|\crs|} and command key \idx{|crskip|} are
provided to assist in fine tuning the vertical spacing inside an
alignment. |\ling@crskip| expands to the setting of |crskip|, and

\exfrag |\crs|\quad $\to$\quad |\cr \noalign{\vskip\ling@crskip}|
\xe

\exbreak
\subsection Tables that can break between pages

\begininventory
\macros \idx{|\exdisplay|}, \idx{|\noexno|}, \idx{|\exnoprint|},
   \idx{|\crnb|}\endmc
\endinventory
Up to this point, only tables that are typeset in a vbox have
been considered.  The \Tex\ page breaking algorithm does not
split a box between pages.  Sometimes, an especially tall table
is needed and one has the choice of floating the table to the top
of the next page (using \Tex's |\topinsert|) or constructing a
breakable table.  The latter choice is often preferable and can
be implemented using \Tex's primitive |\halign|.

The example number must be part of the |\halign|, not inserted by
|\ex|.  |\exdisplay| is designed to accommodate an unboxed
|\halign|. Like |\ex|, |\exdisplay| must be closed by |\xe|.
|\exdisplay| \dots|\xe| is just like |\ex| \dots|\xe| except that
an example number is not printed and the horizontal dimensions
|\numoffset| and |\textoffset| are irrelevant. |aboveexskip| and
|belowexskip| play the same role, paragraph indentation is
cancelled, |\lingeveryex| and |\lingEveryex| are executed, and
|\excnt| is advanced.

|\noexno| is provided in case the user wants to use |\exdisplay|
for something other than a numbered example.  It cancels the
automatic advancement of |\excnt|.  You can say
|\exdisplay|\idx{|\noexno|}\ or |\exdisplay[� \dots�]\noexno| to
cancel |\excnt| advancement.

Now consider the `tall display' (\nextx).  It is sufficiently
tall that it either must be put in an insertion or typeset so
that it can be split by a page break.

\exdisplay[labeloffset=2em,dima=2em,dimb=1em,dimc=-.8em]
\def\\#1-{{\bf #1}-}%
\labels
\openup1pt
\halign{%
   #\tspace[labeloffset]\hfil&    % example number
   #\tl\tspace[textoffset]\hfil&  % line label
   #\tspace \hfil&                % root
   #\tspace \hfil&                % perfect stem
   `#'\hfil\cr                    % gloss
(\the\excnt)& \omit\tspace[dimc] High vowel in the nucleus of the
   root\hidewidth\crnb
& \nl & \hwit{root}& \hwit{perfect stem}&
   \omit\tspace\hwit{gloss}\crnb
&& baudh& \\bu-baudh& know, wake\cr
&& stau& \\tu-stau& praise\cr
&& smai& \\si-smai& smile\cr
\noalign{\exbreak}
&& suap& \\su-suap& sleep\cr
&& miaks& \\mi-miaks& glitter\cr
&& auc& \\u-auc& please\cr
\noalign{\exbreak\smallskip}
& \omit\tspace[dimc] No high vowel in the nucleus of the
   root\hidewidth\crnb
&& suaj& \\sa-suaj& embrace\cr
&& krand& \\ka-krand& cry out\cr
\noalign{\exbreak}
&& skand& \\ka-skand& leap\cr
&& mard& \\ma-mard& rub, crush\cr
&& mnaa& \\ma-mnaa& note\cr
}\xe

The example number in the code below is inserted by
\idx{|\exnoprint|}, which is how |\ex| inserts example numbers.
If, by the way, you don't like the way example numbers are
inserted (surrounded by parentheses), you can redefine
|\exnoprint|. The code uses \idx{|\crnb|}, which expands to
|\cr\noalign{\par\nobreak}| to prevent a page
break between a heading and the remainder of the table which
follows. Page breaks are encouraged by |\exbreak| at a few points
where it is judged that the logic of the table can tolerate it. A
negative horizontal skip |\tspace[dimc]| is used to highlight the
subheadings (and to show the reader that it is a possibility, if
you need it).\exbreak[.2\vsize]

\vbox{\codedisplay
\exdisplay[labeloffset=2em,dima=2em,dimb=1em,dimc=-.8em]
\def\\#1-{{\bf #1}-}%
\labels
\openup1pt
\halign{%
   #\tspace[labeloffset]\hfil&    % example number
   #\tl\tspace[textoffset]\hfil&  % line label
   #\tspace \hfil&                % root
   #\tspace \hfil&                % perfect stem
   `#'\hfil\cr                    % gloss
(\the\excnt)& \omit\tspace[dimc] High vowel in the nucleus of the
   root\hidewidth\crnb
& \nl & \hwit{root}& \hwit{perfect stem}&
   \omit\tspace\hwit{gloss}\crnb
&& baudh& \\bu-baudh& know, wake\cr
&& stau& \\tu-stau& praise\cr
&& smai& \\si-smai& smile\cr
\noalign{\exbreak}
&& suap& \\su-suap& sleep\cr
&& miaks& \\mi-miaks& glitter\cr
&& auc& \\u-auc& please\cr
\noalign{\exbreak\smallskip}
& \omit\tspace[dimc] No high vowel in the nucleus of the
   root\hidewidth\crnb
&& suaj& \\sa-suaj& embrace\cr
&& krand& \\ka-krand& cry out\cr|par|nobreak
\noalign{\exbreak}|par|nobreak
&& skand& \\ka-skand& leap\cr|par|nobreak
&& mard& \\ma-mard& rub, crush\cr|par|nobreak
&& mnaa& \\ma-mnaa& note\cr|par|nobreak
}\xe |endcodedisplay
\vskip-2ex}

\subsection Squeezing tables into tight places

Foregoing |\ex| and directly using |\halign| inside |\exdisplay|
\dots|\xe| has other uses besides constructing tables in numbered
examples which can be broken between pages.  The technique is
also sometimes useful in fitting a table in a numbered example
into a narrow page width. The table below was constructed to fit
on a page of width $4.3\,\rm in$, with no room to spare. It gives
the present indicative conjunction of the Sanskrit verb root {\sl
dve\.s\kern.6pt/dvi\.s\/} `hate'.

\framedisplay
\hsize=4.3in
\exdisplay[dima=.5em,dimb=.4em,textoffset=.5em]
\def\\#1{$\acute{\hbox{\=#1}}$}%
\tabskip=0pt
\openup.4ex
\halign to \hsize{\tspace[dima]#\tspace[textoffset]\hfil&
   #\hfil\tabskip=0pt plus 1fil&
   #\hfil& #\hfil& \tspace[dimb]#\hfil&
   #\hfil &  #\hfil\tabskip=0pt\cr
\omit\exnoprint\hidewidth&
   \multispan6 \hwit{Present Indicative}\crnb
&\multispan3 \hwit{active}& \multispan3 \hwit{middle}\cr
& \hwit{sg}& \hwit{du}& \hwit{pl}&
   \hwit{sg}& \hwit{du}& \hwit{pl}\cr
\it 1& {\bf dv\'e\.s}-mi& dvi\.s-v\'as& dvi\.s-m\'as&
   dvi\.s-\'e& dvi\.s-v\'ahe& dvi\.s-m\'ahe\cr
\it 2& {\bf dv\'ek}-\.si& dvi\.s-\.th\'as& dvi\.s-\.th\'a&
   dvik\.s-\'e& dvi\.s-\\athe& dvi\.d-\.dhv\'e\cr
\it 3& {\bf dv\'e\.s}-\.ti& dvi\.s-\.t\'as& dvi\.s-\'anti&
   dvi\.s-\.t\'e& dvi\.s-\\ate& dvi\.s-\'ate\cr
}
\xe
\endframedisplay

Assuming that the page width has been set to $4.3\,\rm in$
(|\hsize=4.3in|), the following code produces (\lastx), which is
precisely $4.3\,\rm in$ wide.

\codedisplay
\exdisplay[dima=.5em,dimb=.4em,textoffset=.5em]
\def\\#1{$\acute{\hbox{\=#1}}$}%
\tabskip=0pt
\openup.4ex
\halign to \hsize{\tspace[dima]#\tspace[textoffset]\hfil&
   #\hfil\tabskip=0pt plus 1fil&
   #\hfil& #\hfil& \tspace[dimb]#\hfil&
   #\hfil &  #\hfil\tabskip=0pt\cr
\omit\exnoprint\hidewidth&
   \multispan6 \hwit{Present Indicative}\crnb
&\multispan3 \hwit{active}& \multispan3 \hwit{middle}\cr
& \hwit{sg}& \hwit{du}& \hwit{pl}&
   \hwit{sg}& \hwit{du}& \hwit{pl}\cr
\it 1& {\bf dv\'e\.s}-mi& dvi\.s-v\'as& dvi\.s-m\'as&
   dvi\.s-\'e& dvi\.s-v\'ahe& dvi\.s-m\'ahe\cr
\it 2& {\bf dv\'ek}-\.si& dvi\.s-\.th\'as& dvi\.s-\.th\'a&
   dvik\.s-\'e& dvi\.s-\\athe& dvi\.d-\.dhv\'e\cr
\it 3& {\bf dv\'e\.s}-\.ti& dvi\.s-\.t\'as& dvi\.s-\'anti&
   dvi\.s-\.t\'e& dvi\.s-\\ate& dvi\.s-\'ate\cr
}
\xe
|endcodedisplay



