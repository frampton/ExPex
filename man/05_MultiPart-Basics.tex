%\makeatletter
%\def\goop#1{\psline(#1,2ex)(#1,-3ex)\ignorespaces}
%\def\lines{\llap{\hbox to\leftskip{%
%   \goop 0
%   \goop\epd@numright
%   \goop\epd@labelleft
%   \goop\epd@labelright
%   \goop\epd@textleft
%   \hfil}}}
%\resetatcatcode
%
%\lingset{interpartskip=0pt,exskip=0pt}
%
%\pex[labeltype=alpha,labelalign=left,labelwidth=1em]
%\a \lines First part.
%\a Second part.
%\xe
%\pex~[labeltype=alpha,labelalign=right,labelwidth=1em]
%\a \lines First part.
%\a Second part.
%\xe
%\pex~[labeltype=alpha,labelalign=center,labelwidth=1em]
%\a \lines First part.
%\a Second part.
%\xe
%\pex~[labeltype=numeric,labelalign=left]
%\a First part.
%\a Second part.
%\xe
%\pex~[labeltype=numeric,labelalign=right]
%\a First part.
%\a Second part.
%\xe
%\pex~[labeltype=caps,labelalign=left]
%\a First part.
%\a Second part.
%\xe
%\pex~[labeltype=caps,labelalign=right]
%\a First part.
%\a Second part.
%\xe
%
%\endinput

%%%%%%%%%%%%%%%%%%%%%%%%%%%%%%%%%%%%%%%%%%%%%%%%%%%%%%%%%%%%%%
\section Examples with labeled parts: Basics

%\begingroup
%\parindent=0pt
%Macros:\hskip2.5em |\pex~[]|, |\a|, |\linglabeloffset|\par
%\parinventory
%& \idx{|numoffset|}& incrementable dimension& |0pt|\cr
%& \idx{|labeltype|}& name of list of parameter settings&
%   |alpha|\cr
%& \idx{|labeloffset|}& incrementable dimension& |1em|\cr
%& \idx{|labelwidth|}& incrementable dimension& |.78em|\cr
%& \idx{|textoffset|}& incrementable dimension& |1em|\cr
%& \idx{|interpartskip|}& skip& |1ex plus .2ex minus .2ex|\cr
%\idx{|nopreamble|}& |true|, |false|\cr
%& \idx{|samplelabel|}\cr
%}}
%\endparinventory
%\endgroup
\bigskip
%
\noindent Typical examples are given below, with the default
settings of the parameters.\medskip

\framedisplay
\pex
\a This is the first example.
\a This is the second example.
\xe

\pex~<Pre> Multipart examples often have a title or preamble of some
kind.
\a This is the first example.
\a This is the second example.
\xe
\endframedisplay

\codedisplay~
\pex
\a This is the first example.
\a This is the second example.
\xe

\pex~ Multipart examples often have a title or preamble of some kind.
\a This is the first example.
\a This is the second example.
\xe |endcodedisplay

\noindent
Just like |\ex|, |\pex| must be closed by |\xe|, can be
modified by a tilde diacritic to suppress adding vertical space
above the example, and accepts parameters. The macro \idx{|\a|},
which introduces each labeled part, is defined only
within \hbox{|\pex| \dots|\xe|}.  It accepts certain parameters.
Extra vertical skip (set by |interpartskip|) is inserted
between the parts; and extra vertical skip (determined by
|belowpreambleskip|) is inserted between
the {\it preamble\/} and the first part.  The preamble is the
visible material, if any, that appears after the example number
and before the first part.

The horizontal dimensions are parametrized as pictured
in~(\nextx), assuming the default settings of the parameters
|labelanchor| and |textanchor|.  (The effects of changing the
settings of the anchoring parameters will be considered in
Section \getref{anchors}.)  Note that the parameters |numoffset|
and |textoffset| are also used in formatting examples without
parts.

\ex \makeatletter \quad
\edef\resetexcnt{\noexpand\global\noexpand\excnt=\the\excnt}%
\vrule height1.6em depth3.5em width0pt
\psscalebox{1.5}{%
\parindent=0pt
\leavevmode
\lingset{numoffset=4.5em,preambleoffset=4em,labeloffset=3em,
   textoffset=4em,labelwidth=.8em}
\pnode(0,0){A}
\lower5ex\vbox{\hsize=3.8in
\excnt=23
\psset{arrows=<->}
\pex
\a \pnode{E6}%
This is an example.%
\SpecialCoor
\rput(A|E6){\pnode{E1}}
\rput(E1){\pnode(\lingnumoffset,0){E2}}
\rput(E6){\pnode(-\lingtextoffset,0){E5}}
\rput(E5){\pnode(-\linglabelwidth,0){E4}}
\rput(E4){\pnode(-\linglabeloffset,0){E3}}
\psset{nodesep=0,labelsep=0}
\pcline[offset=2.5ex](E1)(E2)
\Aput{\strut\eighttt numoffset}
\pcline[offset=2.5ex](E3)(E4)
\Aput{\strut\eighttt labeloffset}
\pcline[offset=2.5ex](E5)(E6)
\Aput{\strut\eighttt textoffset}
%\pcline[offset=-1.5ex,arrows=>-<,nodesep=-.65ex](E4)(E5)
\pcline[offset=-1.5ex](E4)(E5)
\Bput{\strut\eighttt labelwidth}
\psset{offset=0,angle=90,linestyle=dotted,arrows=-}
\XKV@for@n{1,2,3,4,5,6}\which{%
   \pcline(E\which)([nodesep=3ex]E\which)}
\pcline([nodesep=1.5ex]E4)([nodesep=-2ex]E4)
\pcline([nodesep=1.5ex]E5)([nodesep=-2ex]E5)
\resetexcnt
\xe}}%
\resetatcatcode
\xe
Adjustment for the width of the example number is automatic, but
the width of the label slot is a parameter setting, not adjusted
to the width of the particular label which appears in the label
slot. The initial setting of |labelwidth| is the width of ``a.''
at the point that the default setting is established.  This does
not automatically change if the font is changed, in a footnote
for example.  It can be set by the user explicitly by setting
|labelwidth| to the desired dimension, or as an incremental
change to its previous value (|labelwidth=!3pt|, for example,
increases the width of the label slot by 3pt).  It can also be
set indirectly by giving a sample label.  |labelwidth| is then
set to the current width of that sample.  {\it ExPex\/} sets the
default label width by |samplelabel=a.|.

{\sl ExPex\/} comes with three label types predefined: |alpha|,
|caps|, and |numeric|.

\beginss
\pex[labeltype=alpha]
\a First part.
\a Second part.
\xe |midss
\pex[labeltype=alpha]
\a First part.
\a Second part.
\xe
\endss

\beginss
\pex[labeltype=caps]
\a First part.
\a Second part.
\xe |midss
\pex[labeltype=caps]
\a First part.
\a Second part.
\xe
\endss

\beginss
\pex[labeltype=numeric]
\a First part.
\a Second part.
\xe |midss
\pex[labeltype=numeric]
\a First part.
\a Second part.
\xe
\endss

Section \getref{advanced} will detail all the parameters relevant
to label types and how additional label types can be defined by
the user.


Later we will see how to define custom
label types as needed.  {\it ExPex\/} sets the defaults by
saying |labeltype=alpha|.  It contains the lines:




If you look closely, the effect a fixed labelwidth can be seen in
(\blastx).  The label and text are too close together because the
width of capital letters makes them spill out of the right side
of the label slot.  This is rectified in (\nextx).

\beginss
\pex[labeltype=caps,samplelabel=A.]
\a First part.
\a Second part.
\xe |midss
\pex[labeltype=caps,samplelabel=A.]
\a First part.
\a Second part.
\xe
\endss
The effect of |samplewidth=A.| is to set the label width to be
the width of the sample, in the current font.  So there are three
ways to adjust the labelwidth: 1) setting it to the width of a
sample label; 2) setting |labelwidth| to an explicit dimension;
or 3) incrementing the contextual setting of
|labelwidth| by a specified dimension.

\subsection Formatting the preamble

\parinventory
& \idx{|preambleoffset|}& incrementable dimension& |1em|\cr
& \idx{|belowpreambleskip|}& skip& |1ex|\cr
& \idx{|nopreamble|}&& (default only)\cr
\endparinventory

\noindent Visible material which occurs before the first labeled
entry, as in (\getref{Pre}) for example, is called the preamble.
Although the initial settings produce the format in (\blastx),
the offset of the preamble can be set independently of the offset
of the labels and the extra vertical skip between the preamble
and the first part can also be set independently of the extra
vertical skip between the various parts.

ExPex sets the label offset and the preamble offset to be equal,
so that the left edge of the preamble aligns with the left edge
of the labels.  But this is under the control of the user.

\framedisplay
\pex[labeltype=caps,labeloffset=!.8em]
{\it Principles of the Theory of Binding}
\a An anaphor is bound in its governing category.
\a A pronomial is free in its governing category.
\a An R-expression is free
\xe
\endframedisplay

\codedisplay
\pex[labeltype=caps,labeloffset=!.8em]
{\it Principles of the Theory of Binding}
\a An anaphor is bound in its governing category.
\a A pronomial is free in its governing category.
\a An R-expression is free
\xe |endcodedisplay


The different effects of |interpartskip| and |belowpreambleskip|
are illustrated below.

\framedisplay
\pex[labeltype=caps,belowpreambleskip=.75ex,interpartskip=.25ex]
{\it Principles of the Theory of Binding}
\a An anaphor is bound in its governing category.
\a A pronomial is free in its governing category.
\a An R-expression is free.
\xe
\endframedisplay

\goodbreak
\codedisplay
\pex[labeltype=caps,belowpreambleskip=.75ex,interpartskip=.25ex]
{\it Principles of the Theory of Binding}
\a An anaphor is bound in its governing category.
\a A pronomial is free in its governing category.
\a An R-expression is free.
\xe
|endcodedisplay

If |\a| or |\label| (see Section \getref{labelsec}) directly follows
|\pex| (and a possible tilde diacritic
and parameter settings), |\pex| assumes that
there is no preamble, otherwise it assumes that there is.
This poses a problem if you want to have nonprinting material
other than a label specification before the first part.  For example,
suppose you want to increase the baselineskip by 2pt.  You might
try (\nextx), but it fails to achieve what you want.

\beginss
\pex \openup2pt
\a
\a
\a
\xe|midss
\pex \openup2pt
\a
\a
\a
\xe
\endss
\noindent {\sl ExPex\/} provides the parameter |nopreamble| to
solve the problem.  Setting it to ``true'' tells |\pex| that it
should assume that there is no preamble, in spite of superficial
appearances to the contrary.  So:

\beginss
\pex[nopreamble] \openup2pt
\a
\a
\a
\xe|midss
\pex[nopreamble] \openup2pt
\a
\a
\a
\xe
\endss
\noindent Invoking |nopreamble| with no specified value sets it
to the stipulated default value ``true''.  Setting
|nopreamble=false| has no effect since |\pex| always starts out
assuming that there is a preamble (i.e. that |nopreamble| has
been set to |false|) and this is always overridden
if a following |\a| or |\label| is detected, regardless of the
setting of |nopreamble|.

\font\titlett=txtt at 13.3pt

\subsection Stipulated labels

|\a| takes an optional argument, which is inserted as the label,
ignoring automatic label generation.  This can be useful if only
some parts of a multipart example are being repeated.  See
for example (\getref{explicit}) in Section~\getref{explicit-sec}.

%???? labelanchor and labeloffset should be removed and
% example suitably modified
\beginss
\pex
\pex[exno={7, partially repeated},
   labelanchor=margin,
   labeloffset=1.5em]\par
\a[label=b]
\a[label=d]
\a[label=g]
\xe|midss
\pex[exno={7, partially repeated},labelanchor=margin,
labeloffset=1.5em]\par
\a[label=b]
\a[label=d]
\a[label=g]
\xe
\deftagpage{partialrepeat}
\endss

%\definelingstyle{UBC}{%
%   numlabelclash=false,
%   textanchor=numleft,textoffset=50pt,
%   labelanchor=numleft,labeloffset=25pt,
%   everylabel=,
%   preambleanchor=text,preambleoffset=0pt,
%   appendtopexarg=
%}

