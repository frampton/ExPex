
%I am trying to write a macro \convert which will take a dimension
%register and define a macro, say \goop, which will expand to the
%the dimension in the register expressed in terms of em units.
%
%So, for example, if the current em unit is 12pts, and \dimen0 is
%set to 18pt, then \convert{\dimen0} should result in defining
%\goop so that it expands to 1.5em and be used in expressions like
%\dimen1=\goop.  If this assignment takes place when the current
%value of em is 6pt, then \dimen1 should be set to 9pts.
%
%The following code is probably well-known by experts to fail.
%I'm not sure why.
%
%\def\trim #1pt{#1}
%\dimen0=13pt
%\expandafter\trim\the\dimen0
%
%Any help will be appreciated.
%
%
%\endinput


%%%%%%%%%%%%%%%%%%%%%%%%%%%%%%%%%%%%%%%%%%%%%%%%%%%%%%%%%%%%%%
\section Examples with labeled parts: Basics

\lingset{dima=1.3em}
\noindent\begininventory
\omit Macros:\hidewidth\cr
\idx{|\pex|}, \idx{|\a|}\cr
\omit Parameters:\hidewidth\cr
\hfil\it key& \hfil\it value& \hfil\it initial value\cr
\idx{|labeltype|}& name of a key-value list&
   |alpha|\cr
\idx{|labeloffset|}& incrementable dimension& |1em|\cr
\idx{|labelwidth|}& incrementable dimension& |.72em|\cr
\idx{|preambleoffset|}& incrementable dimension& |1em|\cr
\idx{|interpartskip|}& incrementable skip&
   |1ex plus .2ex minus .2ex|\cr
\idx{|belowpreambleskip|}& incrementable skip&
   |1ex plus .2ex minus .2ex|\cr
\idx{|nopreamble|}& boolean& (not relevant)\cr
\idx{|samplelabel|}& token list& (pseudo-parameter)\cr
\endinventory

\medskip
\noindent Typical examples are given below, with the default
settings of the parameters.\medskip

\framedisplay
\pex
\a This is the first example.
\a This is the second example.
\xe

\pex~<Pre> Multipart examples often have a title or preamble of some
kind.
\a This is the first example.
\a This is the second example.
\xe
\endframedisplay

\codedisplay~
\pex
\a This is the first example.
\a This is the second example.
\xe

\pex~ Multipart examples often have a title or preamble of some kind.
\a This is the first example.
\a This is the second example.
\xe |endcodedisplay

\noindent
Just like |\ex|, |\pex| must be closed by |\xe|, can be
modified by a tilde diacritic to suppress adding vertical space
above the example, and accepts parameters. The macro \idx{|\a|},
which introduces each labeled part, is defined only
within \hbox{|\pex| \dots|\xe|}.  It accepts certain parameters.
Extra vertical skip (set by |interpartskip|) is inserted
between the parts; and extra vertical skip (determined by
|belowpreambleskip|) is inserted between
the {\it preamble\/} and the first part.  The preamble is the
visible material, if any, that appears after the example number
and before the first part.

The horizontal dimensions are parametrized as pictured
in~(\nextx), assuming the default settings of the parameters
|labelanchor| and |textanchor|.  (The effects of changing the
settings of the anchoring parameters will be considered in
Section \getref{anchors}.)  Note that the parameters |numoffset|
and |textoffset| are also used in formatting examples without
parts.

\ex \makeatletter \quad
\edef\resetexcnt{\noexpand\global\noexpand\excnt=\the\excnt}%
\vrule height1.6em depth3.5em width0pt
\psscalebox{1.5}{%
\parindent=0pt
\leavevmode
\lingset{numoffset=4.5em,preambleoffset=4em,labeloffset=3em,
   textoffset=4em,labelwidth=.8em}
\pnode(0,0){A}
\lower5ex\vbox{\hsize=3.8in
\excnt=23
\psset{arrows=<->}
\pex
\a \pnode{E6}%
This is an example.%
\SpecialCoor
\rput(A|E6){\pnode{E1}}
\rput(E1){\pnode(\lingnumoffset,0){E2}}
\rput(E6){\pnode(-\lingtextoffset,0){E5}}
\rput(E5){\pnode(-\linglabelwidth,0){E4}}
\rput(E4){\pnode(-\linglabeloffset,0){E3}}
\psset{nodesep=0,labelsep=0}
\pcline[offset=2.5ex](E1)(E2)
\Aput{\strut\eighttt numoffset}
\pcline[offset=2.5ex](E3)(E4)
\Aput{\strut\eighttt labeloffset}
\pcline[offset=2.5ex](E5)(E6)
\Aput{\strut\eighttt textoffset}
%\pcline[offset=-1.5ex,arrows=>-<,nodesep=-.65ex](E4)(E5)
\pcline[offset=-1.5ex](E4)(E5)
\Bput{\strut\eighttt labelwidth}
\psset{offset=0,angle=90,linestyle=dotted,arrows=-}
\XKV@for@n{1,2,3,4,5,6}\which{%
   \pcline(E\which)([nodesep=3ex]E\which)}
\pcline([nodesep=1.5ex]E4)([nodesep=-2ex]E4)
\pcline([nodesep=1.5ex]E5)([nodesep=-2ex]E5)
\resetexcnt
\xe}}%
\resetatcatcode
\xe
Adjustment for the width of the example number is automatic, but
the width of the label slot is a parameter setting, not adjusted
to the width of the particular label which appears in the label
slot. The initial setting of |labelwidth| is the width of ``a.''
at the point that the default setting is established.  This does
not automatically change if the font is changed, in a footnote
for example.  It can be set by the user explicitly by setting
|labelwidth| to the desired dimension, or as an incremental
change to its previous value (|labelwidth=!3pt|, for example,
increases the width of the label slot by 3pt).  It can also be
set indirectly by giving a sample label.  |labelwidth| is then
set to the current width of that sample.  \expex\/ sets the
default label width by |samplelabel=a.|.

\expex\/ comes with three label types predefined: |alpha|,
|caps|, and |numeric|.

\beginss
\pex[labeltype=alpha]
\a First part.
\a Second part.
\xe |midss
\pex[labeltype=alpha]
\a First part.
\a Second part.
\xe
\endss

% 07/26/11 modified \@setinckey so that it does not convert a
%   relative dimension into a fixed dimension if an increment is
%   not specified
\beginss
\pex[labeltype=caps]
\a First part.
\a Second part.
\xe |midss
\pex[labeltype=caps]
\a First part.
\a Second part.
\xe
\endss

\beginss
\pex[labeltype=numeric]
\a First part.
\a Second part.
\xe |midss
\pex[labeltype=numeric]
\a First part.
\a Second part.
\xe
\endss

Section \getref{advanced} will detail all the parameters relevant
to the label types and how additional label types can be defined
by the user.

%\begingroup
%\parindent=0pt
%Macros:\hskip2.5em |\pex~[]|, |\a|, |\linglabeloffset|\par
%\parinventory
\ex Basic parameters for |\pex| formatting\smallskip
\leftskip=0pt
\hfil\vtop{\halign{#\hfil&& \qquad #\hfil\cr
\hfil\it key& \hfil\it value& \hfil\it initial value\cr
\noalign{\vskip.12ex}
%\idx{|numoffset|}& incrementable dimension& |0pt|\cr
\idx{|labeloffset|}& incrementable dimension& |1em|\cr
\idx{|labelwidth|}& incrementable dimension& |.72em| (set by
|labeltype=alpha|)\cr
%\idx{|textoffset|}& incrementable dimension& |1em|\cr
\idx{|interpartskip|}& skip& |1ex plus .2ex minus .2ex|\cr
\idx{|belowpreambleskip|}& skip& |1ex plus .2ex minus .2ex|\cr
\idx{|labeltype|}& name of a key-value list&
   |alpha|\cr
\idx{|nopreamble|}& |true|, |false|\cr
\idx{|samplelabel|}& token list& \it pseudo-parameter\cr
}}\xe

\subsection {\tt nopreamble}

In order to properly format examples with parts, there are
various reasons that |\pex| must be able to figure out whether
or not there is a preamble.  One reason is fairly obvious.  If
there is a preamble, every part introduced by |\a| must start on
a new line, otherwise only parts after the first part start a new
line.  |\pex| tries to figure it out without help.  If |\a|
directly follows |\pex|(|~|)(|[|\dots|]|) and perhaps a following
space, |\pex| knows there is no preamble and acts accordingly.
There are a few other following tokens, considered later in this
manual, that |\pex| also knows are not signals that there is a
preamble.  But |\pex|'s preamble detection abilities are
primitive.  The following, for example, will confuse |\pex|.

\beginss
\pex \it
\a first
\a second
\xe|midss
\pex \it
\a first
\a second
\xe
\endss

|\pex| needs some guidance in this case.  It is provided by the
parameter |nopreamble|.

\beginss
\pex[nopreamble=true] \it
\a first
\a second
\xe|midss
\pex[nopreamble=true] \it
\a first
\a second
\xe
\endss

In fact, it is a little simpler than this.  XKV parametrization
allows one to stipulate a default value for each key that is
defined.  If the key is given to the parameter setting machinary
with no value, then the key is set to the default value.
|nopreamble| has the default value |true|.  So the following is
sufficient.

\beginss
\pex[nopreamble] \it
\a first
\a second
\xe|midss
\pex[nopreamble] \it
\a first
\a second
\xe
\endss

The global setting of |nopreamble| is irrelevant to |\pex|, which
always assumes there is a preamble unless it sees a directly
following |\a| or |\pex| is told directly that there is no
preamble.

\beginss
\lingset{nopreamble=true}
\pex \it
\a first
\a second
\xe|midss
\lingset{nopreamble=true}
\pex \it
\a first
\a second
\xe
\endss

%\font\titlett=txtt at 13.3pt

\subsection Stipulated labels

|\a| takes an optional argument, which is inserted as the label,
ignoring automatic label generation.  This can be useful if only
some parts of a multipart example are being repeated.  See
for example (\getref{explicit}) in Section~\getref{explicit-sec}.

\beginss
\pex[exno=47]
\a[label=b]
\a[label=d]
\a[label=g]
\xe|midss
\pex[exno=47]
\a[label=b]
\a[label=d]
\a[label=g]
\xe
\deftagpage{partialrepeat}
\endss

\noindent |\a| recognizes only two keys; |label| and |tag|.  See
??? for the meaning and use of |tag|.

%\definelingstyle{UBC}{%
%   avoidnumlabelclash=false,
%   textanchor=numleft,textoffset=50pt,
%   labelanchor=numleft,labeloffset=25pt,
%   everylabel=,
%   preambleanchor=text,preambleoffset=0pt,
%   appendtopexarg=
%}
\endinput
%%%%%%%%%%%%%%%%%%%%%%%%%%%%%%%%%%%%%%%%%%%%%%%%%%%%%%%%
%%%%%%%%%%%%%%%%%%%%%%%%%%%%%%%%%%%%%%%%%%%%%%%%%%%%%%%%
Later we will see how to define custom
label types as needed.  \expex\/ sets the defaults by
saying |labeltype=alpha|.  It contains the lines:




If you look closely, the effect a fixed labelwidth can be seen in
(\blastx).  The label and text are too close together because the
width of capital letters makes them spill out of the right side
of the label slot.  This is rectified in (\nextx).

\beginss
\pex[labeltype=caps,samplelabel=A.]
\a First part.
\a Second part.
\xe |midss
\pex[labeltype=caps,samplelabel=A.]
\a First part.
\a Second part.
\xe
\endss
The effect of |samplewidth=A.| is to set the label width to be
the width of the sample, in the current font.  So there are three
ways to adjust the labelwidth: 1) setting it to the width of a
sample label; 2) setting |labelwidth| to an explicit dimension;
or 3) incrementing the contextual setting of
|labelwidth| by a specified dimension.

\subsection Formatting the preamble

\parinventory
& \idx{|preambleoffset|}& incrementable dimension& |1em|\cr
& \idx{|belowpreambleskip|}& skip& |1ex|\cr
& \idx{|nopreamble|}&& (default only)\cr
\endparinventory

\noindent Visible material which occurs before the first labeled
entry, as in (\getref{Pre}) for example, is called the preamble.
Although the initial settings produce the format in (\blastx),
the offset of the preamble can be set independently of the offset
of the labels and the extra vertical skip between the preamble
and the first part can also be set independently of the extra
vertical skip between the various parts.

\expex\ sets the label offset and the preamble offset to be equal,
so that the left edge of the preamble aligns with the left edge
of the labels.  But this is under the control of the user.

\framedisplay
\pex[labeltype=caps,labeloffset=!.8em]
{\it Principles of the Theory of Binding}
\a An anaphor is bound in its governing category.
\a A pronomial is free in its governing category.
\a An R-expression is free
\xe
\endframedisplay

\codedisplay
\pex[labeltype=caps,labeloffset=!.8em]
{\it Principles of the Theory of Binding}
\a An anaphor is bound in its governing category.
\a A pronomial is free in its governing category.
\a An R-expression is free
\xe |endcodedisplay


The different effects of |interpartskip| and |belowpreambleskip|
are illustrated below.

\framedisplay
\pex[labeltype=caps,belowpreambleskip=.75ex,interpartskip=.25ex]
{\it Principles of the Theory of Binding}
\a An anaphor is bound in its governing category.
\a A pronomial is free in its governing category.
\a An R-expression is free.
\xe
\endframedisplay

\goodbreak
\codedisplay
\pex[labeltype=caps,belowpreambleskip=.75ex,interpartskip=.25ex]
{\it Principles of the Theory of Binding}
\a An anaphor is bound in its governing category.
\a A pronomial is free in its governing category.
\a An R-expression is free.
\xe
|endcodedisplay

If |\a| or |\label| (see Section \getref{labelsec}) directly follows
|\pex| (and a possible tilde diacritic
and parameter settings), |\pex| assumes that
there is no preamble, otherwise it assumes that there is.
This poses a problem if you want to have nonprinting material
other than a label specification before the first part.  For example,
suppose you want to increase the baselineskip by 2pt.  You might
try (\nextx), but it fails to achieve what you want.

\beginss
\pex \openup2pt
\a
\a
\a
\xe|midss
\pex \openup2pt
\a
\a
\a
\xe
\endss
\noindent \expex\/ provides the parameter |nopreamble| to
solve the problem.  Setting it to ``true'' tells |\pex| that it
should assume that there is no preamble, in spite of superficial
appearances to the contrary.  So:

\beginss
\pex[nopreamble] \openup2pt
\a
\a
\a
\xe|midss
\pex[nopreamble] \openup2pt
\a
\a
\a
\xe
\endss
\noindent Invoking |nopreamble| with no specified value sets it
to the stipulated default value ``true''.  Setting
|nopreamble=false| has no effect since |\pex| always starts out
assuming that there is a preamble (i.e. that |nopreamble| has
been set to |false|) and this is always overridden
if a following |\a| or |\label| is detected, regardless of the
setting of |nopreamble|.

