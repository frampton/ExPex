
%\def\typedim#1{\typed@m#1\@nil}
%\def\typed@m#1 #2\@nil{$#1\,\rm #2$}
\def\!{\kern-1.6pt}

\section XKV parameterization

\begininventory
\macros* \idx{|\lingset|}\endmc
\endinventory
%
(Here and in following sections and subsections, an inventory of
the macros, parameters, and count registers which are described
in what follows appears at the beginning of the section. ``What
follows'' refers to the text up to the next section or subsection
heading.)

The key-value approach to parameter setting in \Tex, which
originated with David Carlisle's |keyval| package, is illustrated
by the key |textoffset|. \ExPex\ makes the distance from the example
number to the text equal to value associated with this key.
Executing the command


\sidx{|\lingset|}
\exfrag
|\lingset{textoffset=1.3em}|
\xe

\noindent results in the definition (or redefinition) of the macro
|\lingtextoffset| so that it expands to the value \textdim{1.3 em}.
The macro |\ex|, which is used to typeset examples without labeled
parts, uses |\lingtextoffset|. But \ExPex\ users never have to
concern themselves with the macro |\lingtextoffset|.  If they are not
satisfied with the default spacing, they simply have to know that
|textoffset| is the key for setting the distance from example number
to the text.

The argument of |\lingset| can be a comma separated sequence of
key/value pairs. The syntax is:

\exfrag \everymath={}%
|\lingset{�key$_1$�=�value$_1$�,�\dots�,�key$_n$�=�value$_n$�}|
\xe
%
The comma separated key/value pairs are processed sequentially,
from left to right. If a value contains a comma, it must be
hidden from the mechanism which parses the list by putting the
value in braces. The braces are removed by the parser.

Many \ExPex\ macros take an optional argument, delimited by
brackets, which is passed to |\lingset|.  |\ex|, for example,
takes an optional argument.  You might say, for
example,

\exfrag
|\ex[textoffset=1.4em,aboveexskip=0pt]|
\xe

\noindent The argument of |\ex| will be passed to |\lingset| and the result
evaluated, so that the example will be typeset with these parameter
settings.  This is carried out inside a group, so the global settings
of the parameters are not affected. As we will see later,
|aboveexskip=0pt| will cause the example to be typeset with no
vertical skip above it.  This is sometimes useful in avoiding
exaggerated spacing when an example directly follows another one, with
no intervening text.

\ExPex\ has various kinds of keys.  The distinctions depend on
the effect of executing (\nextx) and the restrictions on the
possible values which can appear.

\ex
|\lingset{�key�=�value�}|
\xe

\noindent {\it Command key\/}:\enspace
After (\lastx) is executed, the macro |\ling�key�| or |\ling@�key�|
expands to {\sl value}.  It will be made clear when the key is
introduced whether it is |\ling�key�| or |\ling@�key�| that is
defined.\footnote{If there was a significant chance that some users
might want easy access to the macro value, |\ling�key�|, with no |@|
in the macro name, was used.}

\smallskip
\noindent {\it Incremental dimension parameter\/}:\enspace {\sl
value\/} must be a dimension or a dimension prefixed by |!|. If {\sl
value\/} is a dimension, it is stored in |\ling�key�|. If it is a
!-prefixed dimension, the prefixed dimension is added to its former
dimension and the result stored in |\ling�key�|. There must be a
dimension to increment, so a fatal error results if |\ling�key�| does
not expand to a dimension. Incremental parameters are very useful if a
minor adjustment to the format in a particular example is desired.
There are {\it incremental skip parameters\/} as well, which operate
in an entirely parallel manner.

In the case of command and incremental keys, the notation Val({\sl
key}) is used to indicate the value of the key.  So, for example,
\parvalue{textoffset} is the value associated with the key
|textoffset|.

\smallskip
\noindent {\it Pseudo parameter\/}: (\lastx) is executed for its
side effects.  A {\sl value\/} is not stored. The key |samplelabel|
illustrates this.  When |\lingset{samplelabel=A.}| is executed, no
\ExPex\ macro which expands to {\sl value\/} is defined.
Instead, the parameter |labelwidth| is set to the width of
``A.'' in the current font.

\smallskip
\noindent {\it Choice parameter\/}:\enspace Choice parameters are a
kind of pseudo parameter.  The {\sl value\/} which is
assigned must be drawn from a prescribed list. \ExPex\ choice
parameters do not store {\sl value\/} when (\lastx) is executed; the
purpose of executing (\lastx) is the side effects which are coded into
the definition of the key.

The parameter |labelalign| illustrates this.
|\lingset{labelalign=�value�}| is valid only if {\sl value\/} is one
of |left|, |center|, or |right|.  A fatal error results otherwise. The
effect is to appropriately define the macro |\@labelprint| which is
used to typeset the labels of subparts in multipart examples.
\smallskip

\medskip
When XKV keys are defined, they can be given default values, as part
of their definitions.  If a key |foo|, for example, is given the
default value |2pt|, then executing |\lingset{foo}| is equivalent to
executing |\lingset{foo=2pt}|.  Only a few \ExPex\ keys have default
values in the XKV sense, but most are set to an initial value in {\sl
expex.tex}.  This initial value will sometimes be called the default
setting of the key, even though the key does not have a default
setting in the XKV sense.







