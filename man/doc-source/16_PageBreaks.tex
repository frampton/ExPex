\def\TEX{T\kern-.1em\lower.5ex\hbox{E}X}

\section Control over page breaking inside examples

Page breaking is generally much more disruptive if it occurs
inside an example than if it occurs in a paragraph of text. Steps
can be taken to minimize the possibility that examples are
disrupted by page breaks.  Avoiding disruption can mean either
that a page break does not occur, or if it does, it happens at a
``good place''.  If the parts of a multi-part example naturally
form subgroups, it is desirable to encourage a break between
subgroups and discourage a break inside a subgroup.

\ExPex\ attempts to provide tools to help avoid breaking examples
between pages and, if they must be split, making it easy to
specify where in the example splitting should be encouraged or
discouraged. This must be balanced with limitations on how much
white space at the bottom of pages is allowable.

The amount of whitespace at the bottom of pages and how much
variability in the vertical skip between examples and the
surrounding text is acceptable will probably vary between
different intended purposes: handout, draft, paper, camera-ready
paper, lecture notes, camera-ready book manuscript, etc.  It will
also vary between different stages of production.  Different
editors may have different views of the tradeoff between keeping
examples intact and allowing whitespace at the bottom of pages.
There is no sense in spending time on formatting while you are
still working on a draft.  \ExPex\ therefore provides a number of
parameters which can be adjusted to control various aspects of
page breaking.

In spite of efforts to guide \Tex\ in making good decisions about
page breaking, the results are not always satisfactory.
\medskip
\begingroup \leftskip=3em \rightskip=3em
\noindent
\noindent Every once and a while,
\TEX\ will produce a really awful looking page and you will
wonder what happened.  For example, you might get just one
paragraph and a whole lot of white space, when some of the text
on the following page would easily fit into the white space.  The
reason for such apparently anomalous behavior is almost always
that no good page break is possible; even the alternative that
looks better to you
is quite terrible as far as \TEX\ is concerned!\par
\hfill
(from Knuth's
\TEX book, p. 115)\par
\endgroup
\medskip
\noindent In situations like this, one has to give up on
encouraging \Tex\ to do the right thing and give explicit orders
to break the page by inserting |\eject| or |\vfil\eject| in the
appropriate place.  The drawback to an explicit order is that it
remains in force after revisions are made to the document.  If it
causes page breaking that is obviously bad, the problem will
usually be seen when editing.  The biggest danger is that
the output is not terrible, but it is less than ideal and the
discrepancy is missed in less than meticulous copy editing.

\subsection Discouraging page breaks in examples

\deftagpage{exbreakpage}%
\begininventory
\macros* \idx{|\exbreak|}\endmc
\parameters
\idx{|exbreakfil|}& skip& 0\dimskip pt plus 4\dimskip ex\cr
\idx{|exbreakpenalty|}& integer& |-500|\cr
\idx{|splitpartspenalty|}& integer& |200|\cr
\endinventory

\noindent \Tex\ carries out page breaking by assigning a cost
(via penalties) to each possible page break and choosing the
lowest cost option.  Plain \Tex\ defines a macro |\goodbreak|
which ends the current paragraph and gives a negative penalty
($-500$) if a page break is taken after the paragraph.  A
negative penalty is a reward, which encourages page breaking at
that point.  The \ExPex\ macro |\exbreak| is a variant of
|\goodbreak|.  It not only ends the current paragraph and awards a
penalty (a reward if the penalty is negative) if a page break is
taken at that point, but makes it easier for a satisfactory page
break to be taken by allowing the page to end with some vertical
fill, allowing some whitespace to appear at the bottom of the
broken page.

|\exbreak| is inserted at the beginning of every |\ex| or |\pex|
block.  The penalty and amount of fill are parameterized.  The
default values are suitable for camera-ready copy, assuming that
the publisher is willing to tolerate a small amount of whitespace
at the bottom of a page in return for keeping an example intact.
$4\,\rm\,ex$ is about $1.5$ times the distance between baselines
in most fonts.  For drafts, I set |exbreakfil=.3\vsize|, which
keeps almost all examples intact at the cost of allowing substantial
whitespace to appear at the bottom of a page.

The setting of |exbreakfil| can be overridden by supplying a
value directly as an optional argument of |\exbreak|.  For
example, |\exbreak[5ex]| acts like |\exbreak| (with no argument)
with |exbreakfil| is set to \textdim{5 ex}.

Page breaks right before an example part in a |\pex| construction
introduced by |\a| are discouraged by imposing a penalty
determined by the setting of |splitpartspenalty|.
If |splitpartspenalty| is set to 10000, such breaks will
be completely disallowed.  I use this setting for writing drafts.
Coupled with a generous setting of |\exbreakfil|, almost all page
breaks in examples are eliminated, which is what I prefer in
writing a draft.  The default value is more suitable for finished
documents.

In older versions of \ExPex\relax, |\exbreak| was called |\goodpar|.
The present name is more consistent with \Tex's naming habits.


\subsection Controlling where page breaks occur in examples (if
they must)

If a page break within an example cannot be avoided
because it would require too much vertical fill to be inserted,
it is often important to control where within the example the
page break is made.  Consider for example a display like
(\nextx).

\framedisplay
\pex[exbreakpenalty=-10,interpartskip=.25ex]
\a example A
\a contrast with example A\exbreak
\a example B
\a variation on B
\a another variation on B
\a a third variation on B\exbreak
\a example C
\a contrast with example C
\xe
\endframedisplay

Suppose the logic of the examples pairs a and b, c--f, and g and
h.  How do you avoid a page break which interrupts the logic of
the collection of examples?  One option is the standard \Tex\
method of using |\nobreak| to prevent a page break, as in the
code below:

 \codedisplay
\pex[interpartskip=.25ex]
\a example A\par\nobreak
\a contrast with example A
\a example B\par\nobreak
\a variation on B\par\nobreak
\a another variation on B\par\nobreak
\a a third variation on B
\a example C\par\nobreak
\a contrast with example C
\xe |endcodedisplay

\noindent  Note that |\par| must precede |\nobreak|.  We are
interested in preventing a page break in the process of adding
lines to the current page, not in preventing a break in the
process of building the lines which are added to the current
page.  |\par| breaks the line, which takes \Tex\  out of line
building mode and puts it into page building mode.

|\exbreak| offers an alternative.

\codedisplay
\pex[interpartskip=.25ex]
\a example A
\a contrast with example A\exbreak
\a example B
\a variation on B
\a another variation on B
\a a third variation on B\exbreak
\a example C
\a contrast with example C
\xe |endcodedisplay




