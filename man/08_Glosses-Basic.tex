
\def\small{\tenrm}
\lingset{glstyle=wrap,abovemoreglskip=1ex}
\def\suff#1{{-\small #1}}%
\everymath={}

\section
Basic Glosses

\lingset{dima=1.3em}
\noindent\begininventory
\omit Macros:\enspace
\idx{|\begingl|}|[]|, \idx{|\endgl|}, \idx{|\gla|}|[]|,
   \idx{|\glb|}|[]|, \idx{|\glft|}|[]|\cr
\endinventory


\noindent Before I discuss the parameters that can modify gloss
displays, here are a few examples.

\framedisplay
\ex<wapm>
\begingl
\gla k- wapm -a -s'i -m -wapunin -uk //
\glb Cl V Agr Neg Agr Tns Agr //
\glb 2 see {3\sc ACC} {} 2{\sc PL} preterit 3{\sc PL} //
\glft `you (pl) didn't see them'//
\endgl
\xe
\endframedisplay
\codedisplay~
\ex
\begingl
\gla k- wapm -a -s'i -m -wapunin -uk //
\glb Cl V Agr Neg Agr Tns Agr //
\glb 2 see {3\sc ACC} {} 2{\sc PL} preterit 3{\sc PL} //
\glft `you (pl) didn't see them'//
\endgl
\xe
|endcodedisplay

\framedisplay
\ex
\begingl
\gla Mary$_i$ ist sicher, dass es den Hans nicht st\"oren w\"urde
seiner Freundin ihr$_i$ Herz auszusch\"utten.//
\glb Mary is sure that it the-{\sc ACC} Hans not annoy would
his-{\sc DAT} girlfriend-{\sc DAT} her-{\sc ACC} heart-{\sc ACC} {out to
throw}//
\glft  `Mary is sure that it would not annoy John to reveal her
heart to his girlfriend.'//
\endgl
\xe
\endframedisplay
\codedisplay~
\ex
\begingl
\gla Mary$_i$ ist sicher, dass es den Hans nicht st\"oren w\"urde
seiner Freundin ihr$_i$ Herz auszusch\"utten.//
\glb Mary is sure that it the-{\sc ACC} Hans not annoy would his-{\sc
DAT} girlfriend-{\sc DAT} her-{\sc ACC} heart-{\sc ACC} {out to
throw}//
\glft  `Mary is sure that it would not annoy John to reveal her
heart to his girlfriend.'//
\endgl
\xe
|endcodedisplay

The |\gla| and |\glb| lines are parsed as a sequence of space
separated items terminating in~|//|.  The parser only looks for spaces
at the top-level.  Consequently, in (\lastx), for example, it is not
sensitive to the space in items like |the-{\sc ACC}| since the space
is inside a group, therefore not at the top level.  Spaces that
directly preceed terminating |//| are disregarded.  The |\glft| (free
translation) line must also be terminated by |//|.

Only one |\gla| line is permitted.  It is obligatory and must come
first. Only one |\glft| line is permitted and it must come last.  It
is optional.  There can be multiple |\glb| lines, as there are in
(\blastx).  User defined lines are also allowed. They are discussed in
the next section.

The next example gives a small taste of how parameter settings can be
used to adapt gloss displays to your needs.

\font\ips=xipasl10 at 12pt
\font\ipss=xipasl10 at 7pt
\def\mroot{$\surd$}
\def\L{\char'354}
\def\v#1{{\accent"07 #1}}
\def\C{{\accent"07 c}}
\def\W{$^{\hbox{\ipss w}}\mskip-2mu$}

\framedisplay
\ex
\begingl
\gla[everygla=\ips] hoi Ekn {\L}E {x\W}ElEP t{g\W}El' st{\'\i}m {hE\L}
   {kuPEcx\W ist} {\L a} Pa{\v c}sEtqEt//
\glb[everyglb=\ips] hoi {\mroot}PEkn {\L}E {x\W}ElEP t{g\W}El' {s +
   \mroot t\'\i m} {hE\L} ku-PEc-\mroot{x\W}ist {\L}a Pa{\v c}sEtqEt//
\glb[aboveglbskip=3ex] then {\mroot}say det Meadowlark why {nomlz + \mroot what} conn
   2nom-cust-{\mroot}one.travels det {day time}//
\glb then {she said} det Meadowlark why {what is it} conn {you travel
   about} det {day time}//
\glft Then Meadowlark said, ``Why do you travel about in the day
   time?''//
\endgl
\xe
\endframedisplay
\def\goop{\thinspace\putfnno}%
\codedisplay~
\ex|goop
\begingl
\gla[everygla=\ips] hoi Ekn {\L}E {x\W}ElEP t{g\W}El' st{\'\i}m {hE\L}
   {kuPEcx\W ist} {\L a} Pa{\v c}sEtqEt//
\glb[everyglb=\ips] hoi {\mroot}PEkn {\L}E {x\W}ElEP t{g\W}El' {s +
   \mroot t\'\i m} {hE\L} ku-PEc-\mroot{x\W}ist {\L}a Pa{\v c}sEtqEt//
\glb then {\mroot}say det Meadowlark why {nomlz + \mroot what} conn
   2nom-cust-{\mroot}one.travels det {day time}//
\glb then {she said} det Meadowlark why {what is it} conn {you travel
   about} det {day time}//
\glft Then Meadowlark said, ``Why do you travel about in the day
   time?''//
\endgl
\xe
|endcodedisplay
\vfootnote{\the\fnno}{In my Tex setup, |\ips| invokes one of the slant tipa IPA fonts, |\L|
produces {\ips \L}, |\W| produces {\ips \W}, and |\mroot|
(morphological root) produces {\mroot}.}

\subsection The basic parameters

\hfill\noindent\begininventory
\hfil\it key& \hfil\it value& \hfil\it initial value\cr
\idx{|glspace|}& skip& |1em|\cr
\idx{|everygl|}& token list& \it empty\cr
\idx{|everygla|}& token list& |\it|\cr
\idx{|everyglb|}& token list& \it empty\cr
\idx{|everyglft|}& token list& \it empty\cr
\idx{|everyglword|}& token list& \it empty\cr
\idx{|aboveglbskip|}& skip& |0pt|\cr
\idx{|aboveglftskip|}& skip& |1ex|\cr
\idx{|abovemoreglskip|}& skip& |1ex|\cr
\idx{|glhangstyle|}& choice (|normal|, |none|, |cascade|)& |normal|\cr
\idx{|glhangindent|}& skip& |1em|\cr
\idx{|glwidth|}& skip& |0pt|\cr
\endinventory

Suppose you would like certain glosses to look like (\nextx).
Compared with (\getref{wapm}): the first line is set in roman type,
not italics; the second line has been set in a smaller
font; and the gap between the first and second lines has been
decreased.

\ex
\begingl
\gla[everygla=] k- wapm -a -s'i -m -wapunin -uk //
\glb[everyglb=\tenrm,aboveglbskip=-.5ex] Cl V Agr Neg Agr Tns Agr //
\glb 2 see {3\sc ACC} {} 2{\sc PL} preterit 3{\sc PL} //
\glft `you (pl) didn't see them'//
\endgl
\xe

Here is one way to accomplish this.

\codedisplay
\ex
\begingl
\gla[everygla=] k- wapm -a -s'i -m -wapunin -uk //
\glb[everyglb=\tenrm,aboveglbskip=-.5ex] Cl V Agr Neg Agr Tns Agr //
\glb 2 see {3\sc ACC} {} 2{\sc PL} preterit 3{\sc PL} //
\glft `you (pl) didn't see them'//
\endgl
\xe
|endcodedisplay

|everygl| can be used to open up the baselines in glosses.

\subsection User defined levels

Here is a better way to do it, assuming that you have a number of
glosses that you want to do in this way.  First, you use
|\definemwlevels{cat}| to define a new level.  This makes |\glcat|
available, just like |\glb|.  It also make parameters |everyglcat| and
|aboveglcatskip| available for adjustment.  They are initialized to
the empty token string and to $0\,\rm pt$.  On the face of it, this
does not seem to accomplish anything.  It seems that you still need to
say:
\defineglwlevels{cat}

\codedisplay
\ex
\begingl
\gla[everygla=] k- wapm -a -s'i -m -wapunin -uk //
\glcat[everyglcat=\tenrm,aboveglcatskip=-.5ex] Cl V Agr Neg Agr Tns Agr //
\glb 2 see {3\sc ACC} {} 2{\sc PL} preterit 3{\sc PL} //
\glft `you (pl) didn't see them'//
\endgl
\xe
|endcodedisplay

But defining the `cat' level allows you to say:

\codedisplay
\definelingstyle{Potawatomi}
   {everygla=,everyglcat=\tenrm,aboveglcatskip=-.5ex}
|endcodedisplay
\definelingstyle{Potawatomi}{everygla=,everyglcat=\tenrm,aboveglcatskip=-.5ex}

Then, you get (\lastx) by saying:

\codedisplay
\ex[lingstyle=Potawatomi]
\begingl
\gla k- wapm -a -s'i -m -wapunin -uk //
\glcat Cl V Agr Neg Agr Tns Agr //
\glb 2 see {3\sc ACC} {} 2{\sc PL} preterit 3{\sc PL} //
\glft `you (pl) didn't see them'//
\endgl
\xe
|endcodedisplay

\ex[lingstyle=Potawatomi]
\begingl
\gla k- wapm -a -s'i -m -wapunin -uk //
\glcat Cl V Agr Neg Agr Tns Agr //
\glb 2 see {3\sc ACC} {} 2{\sc PL} preterit 3{\sc PL} //
\glft `you (pl) didn't see them'//
\endgl
\xe


%\defineglwlevels{aa}
%\lingset{everygla=\ips,everyglaa=\ips}

\framedisplay
\defineglwlevels{aa}
\ex
\begingl[everygla=\ips,everyglaa=\ips]
\gla hoi Ekn {\L}E {x\W}ElEP t{g\W}El' st{\'\i}m {hE\L}
   {kuPEcx\W ist} {\L a} Pa{\v c}sEtqEt//
\glaa hoi {\mroot}PEkn {\L}E {x\W}ElEP t{g\W}El' {s +
   \mroot t\'\i m} {hE\L} ku-PEc-\mroot{x\W}ist {\L}a Pa{\v c}sEtqEt//
\glb then {\mroot}say det Meadowlark why {nomlz + \mroot what} conn
   2nom-cust-{\mroot}one.travels det {day time}//
\glb then {she said} det Meadowlark why {what is it} conn {you travel
   about} det {day time}//
\glft Then Meadowlark said, ``Why do you travel about in the day
   time?''//
\endgl
\xe
\endframedisplay
\codedisplay~
\defineglwlevels{aa}

\ex
\begingl[everygla=\ips,everyglaa=\ips]
\gla hoi Ekn {\L}E {x\W}ElEP t{g\W}El' st{\'\i}m {hE\L}
   {kuPEcx\W ist} {\L a} Pa{\v c}sEtqEt//
\glaa hoi {\mroot}PEkn {\L}E {x\W}ElEP t{g\W}El' {s +
   \mroot t\'\i m} {hE\L} ku-PEc-\mroot{x\W}ist {\L}a Pa{\v c}sEtqEt//
\glb then {\mroot}say det Meadowlark why {nomlz + \mroot what} conn
   2nom-cust-{\mroot}one.travels det {day time}//
\glb then {she said} det Meadowlark why {what is it} conn {you travel
   about} det {day time}//
\glft Then Meadowlark said, ``Why do you travel about in the day
   time?''//
\endgl
\xe
|endcodedisplay

%(The example and the later related one are from an article by Idan
%Landau.)

\subsection The diacritics |+| and |@|

Sometimes it is desirable to override natural wrapping and
break up the gloss so that the syntax is emphasized, as in the
following.

\framedisplay
\ex
\begingl
\gla Mary$_i$ ist sicher, + dass es den Hans nicht st\"oren w\"urde
+ seiner Freundin ihr$_i$ Herz auszusch\"utten.//
\glb Mary is sure that it the\suff{ACC} Hans not annoy would
his\suff{DAT} girlfriend\suff{DAT} her\suff{ACC} heart\suff{ACC} {out to
throw}//
\glft  `Mary is sure that it would not annoy John to reveal her
heart to his girlfriend.'//
\endgl
\xe
\endframedisplay

\bigskip
This is accomplished by inserting `|+|'
appropriately, as shown in the code below.

\codedisplay
\ex
\begingl
\gla Mary$_i$ ist sicher, + dass es den Hans nicht st\"oren w\"urde
+ seiner Freundin ihr$_i$ Herz auszusch\"utten.//
\glb Mary is sure that it the\suff{ACC} Hans not annoy would
his\suff{DAT} girlfriend\suff{DAT} her\suff{ACC} heart\suff{ACC} {out to
throw}//
\glft  `Mary is sure that it would not annoy John to reveal her
heart to his girlfriend.'//
\endgl
\xe
|endcodedisplay

Sometimes it is desirable to omit the space between two entries.

\framedisplay
\ex
\begingl
\gla wiye kepi e- @ ca//
\glb two whitemen \sc1P:3D- found//
\endgl
\xe
\endframedisplay

This is accomplished by inserting `|@|' appropriately, as shown
in the code below.

\codedisplay
\ex
\begingl
\gla wiye kepi e- @ ca//
\glb two whitemen \sc1P:3D- found//
\endgl
\xe
|endcodedisplay
In the unlikely event that you need an entry which
would normally be entered as |@|, enter it as |{{@}}|
so that it is not interpreted as a directive to omit a space.

\subsection The width of the gloss

In the wrap style, the gloss is built in a vbox whose
width is determined implicitly if the parameter |glwidth| is set to
$0\,\mathrm pt$.  The width is $h-l-r$, where $h$, $l$, and $r$, are
the current values of |\hsize|, |\leftskip|, and |\rightskip|,
respectively.  This implicit determination of the width of the gloss
is appropriate for use with the ExPex macros which typeset examples
because they adjust the leftskip appropriately inside examples.

If you want to supply an example number or explicit label, it will not
work to say something like the following if |glwidth| is set to
$0\,\mathrm pt$.
$$
\hbox{|[A]\quad \begingl |\dots| \endgl|}
$$
The vbox built by the gloss macro will not fit on the same line with
the [A].

You must say:
$$
\hbox{|\ex[exno={[A]}] \begingl |\dots| \endgl|}$$
The braces around |[A]| are needed to so that the optional argument
is correctly delineated.  The mechanism looks for the first right
brace {\it at the top level}.  For example:

\framedisplay
\ex[exno={[(6), p. 14]}]
\begingl
\gla Mary$_i$ ist sicher, dass es den Hans nicht st\"oren w\"urde
seiner Freundin ihr$_i$ Herz auszusch\"utten.//
\glb Mary is sure that it the\suff{ACC} Hans not annoy would
his\suff{DAT} girfriend\suff{DAT} her\suff{ACC} heart\suff{ACC} {out to
throw}//
\endgl
\xe
\endframedisplay
\codedisplay~
\ex[exno={[(6), p. 14]}]
\begingl
\gla Mary$_i$ ist sicher, dass es den Hans nicht st\"oren w\"urde
seiner Freundin ihr$_i$ Herz auszusch\"utten.//
\glb Mary is sure that it the\suff{ACC} Hans not annoy would
his\suff{DAT} girfriend\suff{DAT} her\suff{ACC} heart\suff{ACC} {out to
throw}//
\endgl
\xe
|endcodedisplay

If the parameter |glwidth| is set to a nonzero dimension, the width of
the vbox the gloss is constructed in is the specified dimension.

The following example illustrates the usefullness of the explicit
width option.

\ex
a.\quad
\begingl[glwidth=2.6in]
\gla Mary$_i$ ist sicher, dass es den Hans nicht st\"oren w\"urde
seiner Freundin ihr$_i$ Herz auszusch\"utten.//
\glb Mary is sure that it the\suff{ACC} Hans not annoy would
his\suff{DAT} girfriend\suff{DAT} her\suff{ACC} heart\suff{ACC} {out to
throw}//
\glft  `Mary is sure that it would not annoy John to reveal her
heart to his girlfriend.'//
\endgl
\hfil
b.\quad
\begingl[glwidth=2.6in]
\gla Mary$_i$ ist sicher, dass seiner Freunden ihr$_i$ Herz
auszuch\"utten dem Hans nicht schaden w\"urde.//
\glb Mary is sure that his\suff{DAT} girlfriend\suff{DAT} her\suff{ACC}
heart\suff{ACC} {out to throw} the\suff{DAT} Hans not damage would//
\glft `Mary is sure that to reveal her heart to his girlfriend
would not damage John.'//
\endgl
\xe
\codedisplay~
\ex
a.\quad
\begingl[glwidth=2.6in]
\gla Mary$_i$ ist sicher, dass es den Hans nicht st\"oren w\"urde
seiner Freundin ihr$_i$ Herz auszusch\"utten.//
\glb Mary is sure that it the\suff{ACC} Hans not annoy would
his\suff{DAT} girfriend\suff{DAT} her\suff{ACC} heart\suff{ACC} {out to
throw}//
\glft  `Mary is sure that it would not annoy John to reveal her
heart to his girlfriend.'//
\endgl
\hfil
b.\quad
\begingl[glwidth=2.6in]
\gla Mary$_i$ ist sicher, dass seiner Freunden ihr$_i$ Herz
auszuch\"utten dem Hans nicht schaden w\"urde.//
\glb Mary is sure that his\suff{DAT} girlfriend\suff{DAT} her\suff{ACC}
heart\suff{ACC} {out to throw} the\suff{DAT} Hans not damage would//
\glft `Mary is sure that to reveal her heart to his girlfriend
would not damage John.'//
\endgl
\xe
|endcodedisplay


\subsection Inline citations

\noindent Macros:\quad \idx{|\pushrightcite|}
\smallskip
\noindent Parameters:\quad |mincitesep|
\medskip

Inline citations are possible in two ways.


\framedisplay
\hsize=5in
\pex[interpartskip=2ex,glstyle=wrap,everygla=,everyglb=,everyglc=,
   aboveglaskip=0pt,aboveglbskip=0pt,aboveglftskip=0ex]
\a \begingl
\gla k- wapm -a -s'i -m -wapunin -uk //
\glb Cl V Agr$_1$ Neg Agr$_2$ Tns Agr$_3$//
\glb 2 see {3\sc ACC} {} 2{\sc PL} preterit 3{\sc PL} //
\glft `you (pl) didn't see them'//
\endgl \hfill\llap{(Hockett 1948, p. 143)}
\a \begingl
\gla k- wapm -a -s'i -m -wapunin -uk //
\glb Cl V Agr$_1$ Neg Agr$_2$ Tns Agr$_3$//
\glb 2 see {3\sc ACC} {} 2{\sc PL} preterit 3{\sc PL} //
\glft    `you (pl) didn't see them'\pushciteright{(Hockett 1948, p.
143)}//
\endgl
\xe
\endframedisplay
\codedisplay~
\hsize=5in
\pex[interpartskip=2ex,glstyle=wrap,everygla=,everyglb=,everyglc=,
   aboveglaskip=0pt,aboveglbskip=0pt,aboveglftskip=0ex]
\a \begingl
\gla k- wapm -a -s'i -m -wapunin -uk //
\glb Cl V Agr$_1$ Neg Agr$_2$ Tns Agr$_3$//
\glb 2 see {3\sc ACC} {} 2{\sc PL} preterit 3{\sc PL} //
\glft `you (pl) didn't see them'//
\endgl \hfill\llap{(Hockett 1948, p. 143)}
\a \begingl
\gla k- wapm -a -s'i -m -wapunin -uk //
\glb Cl V Agr$_1$ Neg Agr$_2$ Tns Agr$_3$//
\glb 2 see {3\sc ACC} {} 2{\sc PL} preterit 3{\sc PL} //
\glft    `you (pl) didn't see them'\pushciteright{(Hockett 1948, p.
143)}//
\endgl
\xe
|endcodedisplay

\def\mathrm{\fam0}

|\rightcomment| is primitive, but sometimes useful.  It does not check
to see that there is enough room for the citation.  If there is not,
it will overwrite the gloss material.

|\pushciteright| is more sophisticated.  If there is not enough room
on the line for the citation, it will bump the citation to the next
line, right-aligned.  The minium separation between the citation and
the free translation can be set by the parameter |mincitesep|.  The
default is $1.5\,\mathrm em$.


\lingset{glstyle=wrap}

\subsection Bracketing in glosses

\framedisplay[doubleline=true]
\bigskip\noindent
\parinventory
& \idx{|glbrackbracksep|}& dimension& |.1em|\cr
& \idx{|glbrackwordsep|}& dimension& |.2em|\cr
\endparinventory
\endframedisplay
\medskip

\noindent Sometimes it is desirable to displays brackets which
delimit grammatical constituents.  The display below is easy to
create, but is not as attractive as it might be.

\smallskip
\framedisplay
\ex[glstyle=wrap]
\begingl
\gla Mary$_i$ ist sicher, $[$dass es den Hans nicht st\"oren
w\"urde $[$seiner Freundin ihr$_i$ Herz auszusch\"utten$]]$//
\glb Mary is sure that it the Hans not annoy would
his girlfriend her heart {out to throw}//
\glft `Mary is sure that to reveal her heart to his girlfriend
would not damage John.'//
\endgl
\xe
\endframedisplay
\smallskip

\codedisplay
\ex[glstyle=wrap]
\begingl
\gla Mary$_i$ ist sicher, $[$dass es den Hans nicht st\"oren
w\"urde $[$seiner Freundin ihr$_i$ Herz auszusch\"utten$]]$//
\glb Mary is sure that it the Hans not annoy would
his girlfriend her heart {out to throw}//
\glft `Mary is sure that to reveal her heart to his girlfriend
would not damage John.'//
\endgl
\xe
|endcodedisplay

\framedisplay \makeatletter
\ex[glstyle=wrap,glbrackwordsep=.1em,glbrackbracksep=.04em]
\begingl
\gla Mary$_i$ ist sicher, $[\,$ @ dass es den Hans nicht st\"oren
w\"urde $[\,$ @ seiner Freundin ihr$_i$ Herz auszusch\"utten
@ $\,]].$//
\glb Mary is sure {} that it the Hans not annoy would
{} his girlfriend her heart {out to throw}//
\glft `Mary is sure that to reveal her heart to his girlfriend
would not damage John.'//
\endgl
\xe
\endframedisplay

\smallskip
\framedisplay \makeatletter
\ex[glstyle=wrap,glbrackwordsep=.1em,glbrackbracksep=.0em]
\begingl
\gla Mary$_i$ ist sicher, [ dass es den Hans nicht st\"oren
w\"urde [ seiner Freundin ihr$_i$ Herz auszusch\"utten ] @ $].$//
\glb Mary is sure that it the Hans not annoy would
his girlfriend her heart {out to throw}//
\glft `Mary is sure that to reveal her heart to his girlfriend
would not damage John.'//
\endgl
\xe
\endframedisplay
\smallskip

\framedisplay
\ex[glstyle=wrap]
\begingl
\gla Mary$_i$ ist sicher, $[\,$ @ dass es den Hans nicht st\"oren
w\"urde $[\,$ @ seiner Freundin ihr$_i$ Herz auszusch\"utten$\,]]\,.$//
\glb Mary is sure {} that it the Hans not annoy would
{} his girlfriend her heart {out to throw}//
\glft `Mary is sure that to reveal her heart to his girlfriend
would not damage John.'//
\endgl
\xe
\endframedisplay
\smallskip

\noindent Note that the gloss of words following brackets is
aligned with the word, not the bracket.  Note also that some
extra space is inserted between the brackets and the words, and
between adjacent brackets.

One relatively easy way to produce a display like this is to use
the |@| mark to close up the space where needed and math mode to
insert the brackets.  Placeholders (|{}|) must be inserted in the
|\glb| line because the left delimiters are treated as items in
the |\gla| line.  The code used for the above was.

\codedisplay
\ex[glstyle=wrap]
\begingl
\gla Mary$_i$ ist sicher, $[\,$ @ dass es den Hans nicht st\"oren
w\"urde $[\,$ @ seiner Freundin ihr$_i$ Herz
{auszusch\"utten$\,]]\,.$} //
\glb Mary is sure {} that it the Hans not annoy would {} his
girlfriend her heart {out to throw} //
\glft `Mary is sure that to reveal her heart to his girlfriend
would not damage John.'//
\endgl
\xe
|endcodedisplay

\noindent Recall that |\,| inserts a math `thin space' in math
mode. It is not necessary to enclose the last item on the |\gla|
line in brackets, but doing so makes the logic of the code
somewhat clearer.

{\it ExPex\/} provides a way to automate this process without the
need to insert placeholders on the |\glb| line. The code below
produces the display which follows.

\codedisplay
\pex[glstyle=wrap,everygla=,nopreamble]
\a \begingl       % 59a, p. 237
\gla Fa'nu'i yu' ni [ [ {\it O} t{\it in\/}aitai-mu {\it t\/} ] na
lepblu ] @ .//
\glb show me Obl Op {\it WH\/}[obj].read-agr {} L book//
\glft ``Show me the book that you read.''//
\endgl
\a \begingl       % 82a, p. 247
\gla Um-\"asudda' h\"am yan [ i taotao [ {\it O\/} ni si Juan
ilek-\~na nu guahu + [ mal\"agu' gui [ asudd\"a'-\~na {\it
t\/} ] ] ] ] @ .//
\glb agr-meet we with the person Op Comp the Juan say-agr Obl me
agr.want he {\it WH\/}[obl].meet-agr//
\glft ``I met the person who Juan told me he wanted to meet.''//
\endgl
\xe
|endcodedisplay
\framedisplay
\pex[glstyle=wrap,everygla=,nopreamble]
\a \begingl       % 59a, p. 237
\gla Fa'nu'i yu' ni [ [ {\it O} t{\it in\/}aitai-mu {\it t\/} ] na
lepblu ] @ .//
\glb show me Obl Op {\it WH\/}[obj].read-agr {} L book//
\glft ``Show me the book that you read.''//
\endgl
\a \begingl       % 82a, p. 247
\gla Um-\"asudda' h\"am yan [ i taotao [ {\it O\/} ni si Juan
ilek-\~na nu guahu + [ mal\"agu' gui [ asudd\"a'-\~na {\it
t\/} ] ] ] ] @ .//
\glb agr-meet we with the person Op Comp the Juan say-agr Obl me
agr.want he {\it WH\/}[obl].meet-agr//
\glft ``I met the person who Juan told me he wanted to meet.''//
\endgl
\xe
\endframedisplay

\noindent The examples are adapted from {\it The Design of
Agreement\/} by Sandra Chung.  See (59a) on page 237 and (82a) on
page 247.


