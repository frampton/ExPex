%\lingset{abovemoreglskip=1ex}
\everymath={}

\section
Basic Glosses

\noindent
Macros: \idx{|\begingl[]|},
\idx{|\glpreamble[]|},
\idx{|\gla[]|},
\idx{|\glb[]|},
\idx{|\glc[]|},
\idx{|\glft[]|},
\idx{|\endgl|}

\medskip
\noindent Before I introduce the parameters that control the
characteristics of gloss displays, here are a few examples done with
the initial settings of the parameters.

\medskip
\framedisplay
\ex<wapm>
\begingl
\gla k- wapm -a -s'i -m -wapunin -uk //
\glb Cl V Agr Neg Agr Tns Agr //
\glb 2 see {3\sc ACC} {} 2{\sc PL} preterit 3{\sc PL} //
\glft `you (pl) didn't see them'//
\endgl
\xe
\endframedisplay
\codedisplay~
\ex
\begingl
\gla k- wapm -a -s'i -m -wapunin -uk //
\glb Cl V Agr Neg Agr Tns Agr //
\glb 2 see {3\sc ACC} {} 2{\sc PL} preterit 3{\sc PL} //
\glft `you (pl) didn't see them'//
\endgl
\xe
|endcodedisplay

\framedisplay
\ex<sicher>
\begingl
\gla Mary$_i$ ist sicher, dass es den Hans nicht st\"oren w\"urde
seiner Freundin ihr$_i$ Herz auszusch\"utten.//
\glb Mary is sure that it the-{\sc ACC} Hans not annoy would
his-{\sc DAT} girlfriend-{\sc DAT} her-{\sc ACC} heart-{\sc ACC} {out to
throw}//
\glft  `Mary is sure that it would not annoy John to reveal her
heart to his girlfriend.'//
\endgl
\xe
\endframedisplay
\codedisplay~
\ex
\begingl
\gla Mary$_i$ ist sicher, dass es den Hans nicht st\"oren w\"urde
seiner Freundin ihr$_i$ Herz auszusch\"utten.//
\glb Mary is sure that it the-{\sc ACC} Hans not annoy would his-{\sc
DAT} girlfriend-{\sc DAT} her-{\sc ACC} heart-{\sc ACC} {out to
throw}//
\glft  `Mary is sure that it would not annoy John to reveal her
heart to his girlfriend.'//
\endgl
\xe
|endcodedisplay


\framedisplay
\ex<um>
\begingl
\glpreamble Um-\"asudda' h\"am yan i taotao ni si Juan
ilek-\~na nu guahu mal\"agu' gui
asudd\"a'-\~na.//
\gla Um-\"asudda' h\"am yan [ i taotao [ {\it O\/} ni si Juan
ilek-\~na nu guahu [ mal\"agu' gui [ asudd\"a'-\~na {\it
t\/} ] ] ] ] @ .//
\glb agr-meet we with the person Op Comp the Juan say-agr Obl me
agr.want he {\it WH\/}[obl].meet-agr//
\glft ``I met the person who Juan told me he wanted to
meet.''//
\endgl
\xe
\endframedisplay
\codedisplay~
\ex
\begingl
\glpreamble Um-\"asudda' h\"am yan i taotao ni si Juan
   ilek-\~na nu guahu mal\"agu' gui asudd\"a'-\~na.//
\gla Um-\"asudda' h\"am yan [ i taotao [ {\it O\/} ni si Juan
   ilek-\~na nu guahu [ mal\"agu' gui [ asudd\"a'-\~na {\it
   t\/} ] ] ] ] @ .//
\glb agr-meet we with the person Op Comp the Juan say-agr Obl me
   agr.want he {\it WH\/}[obl].meet-agr//
\glft ``I met the person who Juan told me he wanted to
   meet.''//
\endgl
\xe
|endcodedisplay

Glosses (|\begingl| \dots\ |\endgl|) have up to three
parts.  First, an optional preamble (|\glpreamble| \dots\ |//|).
Second, an interlinear gloss, which is also optional provided there is
a preamble.\footnote{%
Contrary to first impressions, there are good reasons to allow the
interlinear gloss to be omitted in certain situations.}  Third, a free
translation (|\glft| \dots\ |//|), which can be omitted if there is an
interlinear gloss. They are illustrated below, which repeats (\lastx)
with a narrower hsize.

\begingroup
\input pstricks-add
\ex
\def\TOP{\pnode(3.4in,1.5ex)}%
\def\BOT{\pnode(0,-.5ex)}%
\begingl[glwidth=3.2in]
\glpreamble \TOP{A1}Um-\"asudda' h\"am yan i taotao ni si Juan
ilek-\~na nu guahu mal\"agu' gui
asudd\"a'-\~na.\BOT{B1}//
\gla \TOP{A2}Um-\"asudda' h\"am yan [ i taotao [ {\it O\/} ni si Juan
ilek-\~na nu guahu [ mal\"agu' gui [ asudd\"a'-\~na {\it
t\/} ] ] ] ] @ .//
\glb agr-meet we with the person Op Comp the Juan say-agr Obl me
agr.want he {\it WH\/}[obl].meet-agr\BOT{B2}//
\glft \TOP{A3}``I met the person who Juan told me he wanted to
meet.''\BOT{B3}//
\endgl
\SpecialCoor
\psbrace[ref=lC](A1|B1)(A1){ preamble}
\psbrace[ref=lC](A1|B2)(A2){ interlinear gloss}
\psbrace[ref=lC](A1|B3)(A3){ free translation}
\xe
\endgroup

The interlinear gloss consists of a sequence of lines of the form
$$\hbox{|\gl|{\it levelname\/} \dots\ |//|}$$
where {\it levelname\/} is |a|, |b|, or |c|.\footnote{%
This will be extended later to allow the user to define new level
names.}  There must be one and only one |\gla| line, which must come
first in the interlinear gloss. |\glb| and |\glc| lines can come in
any order and can be repeated arbitrarily.

The material delineated by |\gl|{\it levelname\/} and |//| is parsed
as a sequence of space separated items.  The parser only looks for
spaces at the top-level.  Consequently, in (\blastx), for example, it
is not sensitive to the space in items like |the-{\sc ACC}| since the
space is inside a group, therefore not at the top level.  Spaces that
directly preceed terminating |//| are disregarded.  A line in the
interliear gloss can have fewer items on it than the line above it. It
acts as if it ended with empty |{}| items.  But {\it no line in the
interlinear gloss can have more items than the line above it}.  If
there are ``excess'' items on a line, they will act as if they are on
the higher line (recursively).

%\bigskip
%\framedisplay[doubleline=true]
%\bigskip\noindent
%Macros:\par
%\noindent\quad
%|\begingl|, |\endgl|, |\glpreamble|, |\gla|, |\glb|, |\glc|,
%|\glft|
%\parinventory
%& \idx{|glbrackbracksep|}& dimension& |.1em|\cr
%& \idx{|glbrackwordsep|}& dimension& |.2em|\cr
%\endparinventory
%\endframedisplay
%\bigskip
%Now is the time for all good men to come to the aid of the party.
%Now is the time for all good men to come to the aid of the party.
%Now is the time for all good men to come to the aid of the party.
%Now is the time for all good men to come to the aid of the party.
%Now is the time for all good men to come to the aid of the party.
%
%
%\endinput


\subsection Parameters

\def\\#1{\noalign{\smallskip}\quad #1\cr}%
\parinventory
\\{Horizontal spacing\hidewidth}
\quad & \idx{|glspace|}& dimension& \textdim{1 em}\cr
\\{Font changes, struts, etc.\hidewidth}
& \idx{|everygl|}& token list& |{}|\cr
& \idx{|everyglpreamble|}& token list& |{}|\cr
& \idx{|everygla|}& token list& |\it|\cr
& \idx{|everyglb|}& token list& |{}|\cr
& \idx{|everyglc|}& token list& |{}|\cr
& \idx{|everyglft|}& token list& |{}|\cr
\\{Vertical spacing\hidewidth}
& \idx{|belowpreambleskip|}& dimension& \textdim{1 ex}\cr
& \idx{|aboveglbskip|}& dimension& \textdim{0 pt}\cr
& \idx{|aboveglcskip|}& dimension& \textdim{0 pt}\cr
& \idx{|aboveglftskip|}& dimension& \textdim{0 pt}\cr
\endparinventory

Suppose for example that rather than the gloss in (\getref{wapm}) the
following style is desired.

\framedisplay
\ex
\begingl[glspace=2em]
\gla[everygla=] k- wapm -a -s'i -m -wapunin -uk //
\glb[everyglb=\tenpoint,aboveglbskip=-.4ex]
   Cl V Agr Neg Agr Tns Agr //
\glb 2 see {3\sc ACC} {} 2{\sc PL} preterit 3{\sc PL} //
\glft[everyglft=\it,aboveglftskip=0pt] `you (pl) didn't see them'//
\endgl
\xe
\endframedisplay

There are several differences with (\getref{wapm}).  The words are
more widely separated; the font in the |\gla| line is not italic;  the
font in the free translation is italic; the first |\glb| line is set
in a smaller font and is moved up closer to the |\gla| line; and there
is no extra vertical skip between the free translation and the last
line of the interlinear gloss, as there is in (\getref{wapm}).

Here is one way to write the code:

\codedisplay
\ex
\begingl[glspace=2em]
\gla[everygla=] k- wapm -a -s'i -m -wapunin -uk //
\glb[everyglb=\tenpoint,aboveglbskip=-.4ex]
   Cl V Agr Neg Agr Tns Agr //
\glb 2 see {3\sc ACC} {} 2{\sc PL} preterit 3{\sc PL} //
\glft[everyglft=\it,aboveglftskip=0pt] `you (pl) didn't see them'//
\endgl
\xe
|endcodedisplay

Here is an alternative which produces the same output.  It has the
advantage that all of the parameter settings are gathered in one
place, making it easier to modify.

\codedisplay
\ex
\begingl[glspace=2em,everygla=,everyglb=\tenpoint,
   aboveglbskip=-.4ex,everyglft=\it,aboveglftskip=0pt]
\gla k- wapm -a -s'i -m -wapunin -uk //
\glb Cl V Agr Neg Agr Tns Agr //
\glc 2 see {3\sc ACC} {} 2{\sc PL} preterit 3{\sc PL} //
\glft `you (pl) didn't see them'//
\endgl
\xe
|endcodedisplay

If you use multiple instances of the same gloss format, a style should
probably be defined

\codedisplay
\definelingstyle{Potawatami}{glspace=2em,everygla=,everyglb=\tenpoint,
   aboveglbskip=-.4ex,everyglft=\it,aboveglftskip=0pt}
|endcodedisplay

Then, typesetting a gloss in the format is done simply by using that
style, as follows:

\codedisplay
\ex
\begingl[lingstyle=Potawatami]
\gla k- wapm -a -s'i -m -wapunin -uk //
\glb Cl V Agr Neg Agr Tns Agr //
\glc 2 see {3\sc ACC} {} 2{\sc PL} preterit 3{\sc PL} //
\glft `you (pl) didn't see them'//
\endgl
\xe
|endcodedisplay

If you want a sequence of glosses to all be done in this style, you can say:

\def\goop{\quad $\vdots$}
\codedisplay
\begingroup
\lingset{lingstyle=Potawatami}
|goop
\endgroup
|endcodedisplay

\subsection The width of the gloss

In the wrap style, the gloss is built in a vbox whose
width is determined implicitly if the parameter |glwidth| is set to
\textdim{0 pt}. The width is $h-l-r$, where $h$, $l$, and $r$, are
the current values of |\hsize|, |\leftskip|, and |\rightskip|,
respectively.  This implicit determination of the width of the gloss
is appropriate for use with the \expex\ macros which typeset examples
because they adjust the leftskip appropriately inside examples.

If you want to supply an example number or explicit label, it will not
work to say something like the following if |glwidth| is set to
\textdim{0 pt}.
$$
\hbox{|[A]\quad \begingl |\dots| \endgl|}
$$
The vbox built by the gloss macro will not fit on the same line with
the [A].

You must say:
$$
\hbox{|\ex[exno={[A]}] \begingl |\dots| \endgl|}$$
The braces around |[A]| are needed to so that the optional argument
is correctly delineated.  The mechanism looks for the first right
brace {\it at the top level}.  For example:

\def\suff#1{{-\small #1}}%
\framedisplay
\ex[exno={[(6), p. 14]}]
\begingl
\gla Mary$_i$ ist sicher, dass es den Hans nicht st\"oren w\"urde
seiner Freundin ihr$_i$ Herz auszusch\"utten.//
\glb Mary is sure that it the\suff{ACC} Hans not annoy would
his\suff{DAT} girfriend\suff{DAT} her\suff{ACC} heart\suff{ACC} {out to
throw}//
\endgl
\xe
\endframedisplay
\codedisplay~
\ex[exno={[(6), p. 14]}]
\begingl
\gla Mary$_i$ ist sicher, dass es den Hans nicht st\"oren w\"urde
seiner Freundin ihr$_i$ Herz auszusch\"utten.//
\glb Mary is sure that it the\suff{ACC} Hans not annoy would
his\suff{DAT} girfriend\suff{DAT} her\suff{ACC} heart\suff{ACC} {out to
throw}//
\endgl
\xe
|endcodedisplay

If the parameter |glwidth| is set to a nonzero dimension, the width of
the vbox that the gloss is constructed in is the specified dimension.

The following example illustrates the usefullness of the explicit
width option.

\ex<sicher2>
a.\quad
\begingl[glwidth=2.6in]
\gla Mary$_i$ ist sicher, dass es den Hans nicht st\"oren w\"urde
seiner Freundin ihr$_i$ Herz auszusch\"utten.//
\glb Mary is sure that it the\suff{ACC} Hans not annoy would
his\suff{DAT} girfriend\suff{DAT} her\suff{ACC} heart\suff{ACC} {out to
throw}//
\glft  `Mary is sure that it would not annoy John to reveal her
heart to his girlfriend.'//
\endgl
\hfil
b.\quad
\begingl[glwidth=2.6in]
\gla Mary$_i$ ist sicher, dass seiner Freunden ihr$_i$ Herz
auszuch\"utten dem Hans nicht schaden w\"urde.//
\glb Mary is sure that his\suff{DAT} girlfriend\suff{DAT} her\suff{ACC}
heart\suff{ACC} {out to throw} the\suff{DAT} Hans not damage would//
\glft `Mary is sure that to reveal her heart to his girlfriend
would not damage John.'//
\endgl
\xe
\codedisplay~
\ex
a.\quad
\begingl[glwidth=2.6in]
\gla Mary$_i$ ist sicher, dass es den Hans nicht st\"oren w\"urde
seiner Freundin ihr$_i$ Herz auszusch\"utten.//
\glb Mary is sure that it the\suff{ACC} Hans not annoy would
his\suff{DAT} girfriend\suff{DAT} her\suff{ACC} heart\suff{ACC} {out to
throw}//
\glft  `Mary is sure that it would not annoy John to reveal her
heart to his girlfriend.'//
\endgl
\hfil
b.\quad
\begingl[glwidth=2.6in]
\gla Mary$_i$ ist sicher, dass seiner Freunden ihr$_i$ Herz
auszuch\"utten dem Hans nicht schaden w\"urde.//
\glb Mary is sure that his\suff{DAT} girlfriend\suff{DAT} her\suff{ACC}
heart\suff{ACC} {out to throw} the\suff{DAT} Hans not damage would//
\glft `Mary is sure that to reveal her heart to his girlfriend
would not damage John.'//
\endgl
\xe
|endcodedisplay


\subsection Comments and citations

\noindent Macros:\quad \idx{|\trailingcitation|},
\idx{|\rightcomment|}
\smallskip
\noindent Parameters:\quad |mincitesep|
\medskip

The following illustrates different ways to append a comment or
citation to a gloss.

\framedisplay
\ex
\begingl
\gla \rightcomment{[Potawatami]}k- wapm -a -s'i -m -wapunin -uk //
\glb \rightcomment{category}Cl V Agr$_1$ Neg Agr$_2$ Tns Agr$_3$//
\glb 2 see {3\sc ACC} {} 2{\sc PL} preterit 3{\sc PL} //
\glft `you (pl) didn't see them'\trailingcitation{(Hockett 1948,
   p. 143)}//
\endgl
\xe
\endframedisplay
\codedisplay~
\ex
\begingl
\gla \rightcomment{[Potawatami]}k- wapm -a -s'i -m -wapunin -uk //
\glb \rightcomment{category}Cl V Agr$_1$ Neg Agr$_2$ Tns Agr$_3$//
\glb 2 see {3\sc ACC} {} 2{\sc PL} preterit 3{\sc PL} //
\glft `you (pl) didn't see them'\trailingcitation{(Hockett 1948,
   p. 143)}//
\endgl
\xe
|endcodedisplay
|\trailingcitation| will put the citation on the same line as the last
line of the free translation if there is enough room for it, otherwise at the
end of the  following line.  The parameter |mincitesep| determines the
minimum whitespace between the end of the free translation and the
citation that is tolerated; the default is \textdim{1.5 em}.

The macro |\rightcomment| is very primitive.  It does not consider the
width of the citation or the amount of whitespace at the right of the
gloss.  The citation will overlap the gloss if there is not room for
it at the right.  For example, if the gloss (\lastx) is attempted with
an hsize of \textdim{3.5 in} and |glspace=.5em|, the result is (\nextx).

\framedisplay
\hsize=3.5in
\ex
\rightcomment{[Potawatami]}
\begingl
\gla k- wapm -a -s'i -m -wapunin -uk //
\glb \rightcomment{category}Cl V Agr$_1$ Neg Agr$_2$ Tns Agr$_3$//
\glb 2 see {3\sc ACC} {} 2{\sc PL} preterit 3{\sc PL} //
\glft `you (pl) didn't see them'\trailingcitation{(Hockett 1948,
   p. 143)}//
\endgl
\xe
\endframedisplay

Nevertheless, |\rightcomment| is occasionally
useful.

\subsection  Exceptional {\tt \char"5C gla} items

Items on the |\gla| line are generally associated with items on the
other lines of the interlinear gloss.  There are however a few items,
called here {\it exceptional items}, which are interpreted in another
fashion. There are four exceptional items, all consisting of a single character;
|+|, |@|, |[|, or |]|.

\subsubsection {\tt +}

Sometimes it is desirable to override natural wrapping and
break up the gloss so that the syntax is emphasized, as in the
following.

\framedisplay
\ex
\begingl
\gla Mary$_i$ ist sicher, + dass es den Hans nicht st\"oren w\"urde
+ seiner Freundin ihr$_i$ Herz auszusch\"utten.//
\glb Mary is sure that it the\suff{ACC} Hans not annoy would
his\suff{DAT} girlfriend\suff{DAT} her\suff{ACC} heart\suff{ACC} {out to
throw}//
\glft  `Mary is sure that it would not annoy John to reveal her
heart to his girlfriend.'//
\endgl
\xe
\endframedisplay

\bigskip
This is accomplished by inserting `|+|' appropriately, as shown in the
code below.  When |+| is encountered, the line is broken and a new
line started, ignoring any hanging indentation.

\codedisplay
\ex
\begingl
\gla Mary$_i$ ist sicher, + dass es den Hans nicht st\"oren w\"urde
+ seiner Freundin ihr$_i$ Herz auszusch\"utten.//
\glb Mary is sure that it the\suff{ACC} Hans not annoy would
his\suff{DAT} girlfriend\suff{DAT} her\suff{ACC} heart\suff{ACC} {out to
throw}//
\glft  `Mary is sure that it would not annoy John to reveal her
heart to his girlfriend.'//
\endgl
\xe
|endcodedisplay

\subsubsection {\tt @}

Sometimes it is desirable to omit the space between two entries.

\framedisplay
\ex<wiye>
\begingl
\gla wiye kepi e- @ ca//
\glb two whitemen \sc1P:3D- found//
\endgl
\xe
\endframedisplay

This is accomplished by inserting `|@|' appropriately, as shown
in the code below.

\codedisplay
\ex
\begingl
\gla wiye kepi e- @ ca//
\glb two whitemen \sc1P:3D- found//
\endgl
\xe
|endcodedisplay
In the unlikely event that you need an entry which
would normally be entered as |@|, enter it as |{{@}}|
so that it is not interpreted as a directive to omit a space.
(\getref{@period}) below shows another use for the |@| diacritic.

\subsection Bracketing in glosses: {\tt [} and {\tt ]}

\medskip
\parinventory
& \idx{|glbrackbracksep|}& dimension& \textdim{.1 em}\cr
& \idx{|glbrackwordsep|}& dimension& \textdim{.2 em}\cr
\endparinventory
\bigskip

Suppose you want to produce a gloss display like the one below.

\framedisplay
\pex[everygla=,glhangstyle=normal]<@period>
\a
\begingl
\gla Fa'nu'i yu' ni [ [ {\it O} t{\it in\/}aitai-mu {\it t\/} ] na
lepblu ] @ .//
\glb show me Obl Op {\it WH\/}[obj].read-agr {} L book//
\glft ``Show me the book that you read.''//
\endgl
\a \begingl
\gla Um-\"asudda' h\"am yan [ i taotao [ {\it O\/} ni si Juan
ilek-\~na nu guahu [ mal\"agu' gui [ asudd\"a'-\~na {\it
t\/} ] ] ] ] @ .//
\glb agr-meet we with the person Op Comp the Juan say-agr Obl me
agr.want he {\it WH\/}[obl].meet-agr//
\glft ``I met the person who Juan told me he wanted to meet.''//
\endgl
\xe
\endframedisplay
\noindent The brackets are not glossed; there is a little extra space
between brackets; and there is a little extra space between words and
brackets.

(\lastx) is produced by:

\codedisplay
\pex[everygla=]
\a \begingl
\gla Fa'nu'i yu' ni [ [ {\it O} t{\it in\/}aitai-mu {\it t\/} ] na
lepblu ] @ .//
\glb show me Obl Op {\it WH\/}[obj].read-agr {} L book//
\glft ``Show me the book that you read.''//
\endgl
\a \begingl
\gla Um-\"asudda' h\"am yan [ i taotao [ {\it O\/} ni si Juan
ilek-\~na nu guahu [ mal\"agu' gui [ asudd\"a'-\~na {\it
t\/} ] ] ] ] @ .//
\glb agr-meet we with the person Op Comp the Juan say-agr Obl me
agr.want he {\it WH\/}[obl].meet-agr//
\glft ``I met the person who Juan told me he wanted to meet.''//
\endgl
\xe
|endcodedisplay

\noindent Note that |@| is used to eliminate an unwanted space before
the period on the gla-line.

The Chamorro examples in (\getref{@period}) are adapted from Sandy
Chung's {\it The Design of Agreement}.  See (59a) on page 237 and
(82a) on page 247.  The Potawatami example (\getref{wapm}) is from
Halle and Marantz's ``Distributed Morphology'' article. The German
examples in (\getref{sicher2}) are from an article by Idan Landau. The
Kiowa example (\getref{wiye}) was contributed by Daniel Harbor.



