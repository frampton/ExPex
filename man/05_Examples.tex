\medskip\hrule\medskip
\makeatletter
\def\marginoffset#1{\the\csname epd@#1\endcsname \ignorespaces}
\pex
\a
\begingroup
\vskip\ling@interpartskip
\leftskip=\marginoffset{labelleft}
Note.
\par
\endgroup
\a
\a
\xe
\resetatcatcode
\medskip\hrule\medskip
\begingroup
\lingset{%
   labelanchor=numleft,
   labeloffset=0pt,
   avoidnumlabelclash,
   interpartskip=0pt,
   labelgen=char,
   pexcnt=`a,
   labelalign=left,
   labelformat=(A),
   labelwidth=1.5em,
   nopreamble,
   exskip=1ex,
}
7.\enspace Previous footnote.
The footnotes are generally quite extensive, many running over
many lines of text.

8.\enspace
Introductory text to an extensive footnote.
The footnote contains five related examples, which are labeled
with a,b,\dots\ which run consequtively.
\pex
\a first example
\a second example
{\advance\pexcnt by 1 \xdef\resetpexcnt{\the\pexcnt}}%
\xe
Some explanatory text after the first two examples..
\pex[pexcnt=\resetpexcnt]
\a third example
{\advance\pexcnt by 1 \xdef\resetpexcnt{\the\pexcnt}}%
\xe
Still more explanatory text after the examples above.
\pex[pexcnt=\resetpexcnt]
\a fourth example
\a fifth example
\xe
Finally, the concluding text.

9. Next footnote

\endgroup
\medskip\hrule\medskip

\begingroup
\lingset{%
   labelanchor=numleft,
   labeloffset=0pt,
   avoidnumlabelclash,
   interpartskip=0pt,
   labelgen=char,
   pexcnt=`a,
   labelalign=left,
   labelformat=(A),
   labelwidth=1.5em,
   nopreamble,
   exskip=1ex,
}
7.\enspace Previous footnote.
The footnotes are generally quite extensive, many running over
many lines of text.

8.\enspace
Introductory text to an extensive footnote.
The footnote contains five related examples, which are labeled
with a,b,\dots\ which run consequtively.
\pex
\a first example
\a second example
\vskip1ex\begingroup \parindent=0pt \leftskip=0pt
Some explanatory text after the first two examples.
\vskip1ex\endgroup
\a third example\par
\vskip1ex\begingroup \parindent=0pt \leftskip=0pt
Still more explanatory text after the examples above.
\vskip1ex\endgroup
\a fourth example
\a fifth example
\xe
Finally, the concluding text.

9. Next footnote

\endgroup
\medskip\hrule\medskip

\begingroup

% example from Barss in Step by Step
Here is another reason to anchor the label at the left edge of the
example number rather than the right edge.
\lingset{labelanchor=numleft,labeloffset=2em,interpartskip=0pt,
   textoffset=1.5em,exskip=1ex}
\pex[exno=i]
\a $[_{\rm S}[_{\rm NP}\,$his$_2$ mother$\,]\,[_{\rm
VP}\,$everyone$\,]_2$ saw
{\it e}$_2\,]]]$
\a \ljudge* His mother saw everyone.
\xe

\pex~[exno=ii]
\a \ljudge*$[_{\rm S'/CP}[\,$which pilot who shot at
it$\,]_2\,[_{\rm S}\,${\it e}$_2\,[_{\rm VP}[\,$every MIG that
chased him$_2\,]_3\,[_{\rm VP}\,$hit {\it e}$_3\,]]]]$
\a $[\,$Which pilot who shot at it$_3\,]_2\,$hit$\,[\,$every MIG
that chased him$_2\,]_3$?
\xe

\lingset{labelanchor=numright,labeloffset=1.2em}
\pex[exno=i]
\a $[_{\rm S}[_{\rm NP}\,$his$_2$ mother$\,]\,[_{\rm
VP}\,$everyone$\,]_2$ saw
{\it e}$_2\,]]]$
\a \ljudge* His mother saw everyone.
\xe

\pex~[exno=ii]
\a \ljudge*$[_{\rm S'/CP}[\,$which pilot who shot at
it$\,]_2\,[_{\rm S}\,${\it e}$_2\,[_{\rm VP}[\,$every MIG that
chased him$_2\,]_3\,[_{\rm VP}\,$hit {\it e}$_3\,]]]]$
\a $[\,$Which pilot who shot at it$_3\,]_2\,$hit$\,[\,$every MIG
that chased him$_2\,]_3$?
\xe

\endgroup

\endinput


