
\subsection The parameters {\tt exnotype}, {\tt minexnolabelsep}, and
{\tt sampleexno}

Macro:\quad |\setmaxexnowidth|\par
\parinventory
& \idx{|exnotype|}& choice (|arabic|, |roman|)& |arabic|\cr
& \idx{|minexnolabelsep|}& dim& \textdim{.5 em}\cr
& \idx{|sampleexno|}& pseudo parameter& (not relevant)\cr
\endparinventory


In footnotes, example numbers are commonly given as lowercase roman
numerals.  This can be accomplished by setting |exnotype| to |roman|.
Furthermore, in some footnote styles the label offset is measured from
the left edge of the example number (i.e. |labelanchor| is set to
|numleft|) and is the same for all the examples in a footnote.  Since
the widths of the displayed example numbers vary when given as roman
numerals, this means that the separation between the displayed example
number and the text will be different for different examples in a
footnote. The labeloffset is chosen so that the smallest such
separation is equal to a fixed value (|minexnolabelsep|) which is the
same for all footnotes.  This is tricky to implement, since the
setting of |labeloffset| in the first example in a footnote, for
example, can depend on what other example numbers appear in the
footnote.  The pseudo-parameter |sampleexno| is designed to solve this
problem.   |\lingset{sampleexno=(iv)}|, for example, has the effect of
setting |labeloffset| to be the width of (iv) plus the value of
|minexnolabelsep|.

\excnt=34
\input eptexfn

\subsection Footnotes and endnotes

Footnotes and endnotes pose a somewhat thorny problem many different
\Tex\ and \Latex\ macros are in use.  Further, there are different
ways of assigning example numbers and labels in multipart examples.
The footnote referenced at the end of this sentence is
formatted in one fairly common style. It is an
abbreviated version of footnote 17 in Chapter 2 of Diesing's {\sl
Indefinites}, MIT Press.%
\footnote{%
\setmaxexnowidth{(ii)}
The existential reading does not seem to be available for subjects of
small clause complements of {\it consider\/}:
\pex
\a I consider firemen available. (generic only)
\a I consider firemen intelligent. (generic only)
\xe
Exceptional case marking (ECM) verbs seem more or less to allow both
existential and generic interpretations of complement subjects:
\pex
\a I believe firemen to be available. (both generic and existential)
\a I believe violists to be intelligent. (generic only)
%\a \ljudge{??}I believe opera singers to know Hittite.
\xe
}

The example numbers are roman numerals, starting at ``i'' in each
footnote.  As opposed to examples in the running text, the labeloffset
is measured from the left margin, not the right edge of the typeset
example number.  Although the value of |labeloffset| is the same for
all the examples in a footnote, it depends on the width of the widest
example number.  It is chosen so that the label is \textdim{.5 em} to
the right of the widest example number.

%How is this formatting achieved?  We could try\footnote{


We put off for a moment the issue of adjusting
the labeloffset for tto take into account example numbers wider than the
width of (\romannumeral1).  There are two things that have to be done.
One is to provide some ``hook'' into the footnote macro so that the
appropriate formating commands can be inserted whenever a footnote is
processed.  We employ the method Knuth gave in the {\sl TexBook\/}.
A token list |\everyfootnote| is defined which contains all of the
commands needed to format footnotes appropriately.  We then modify the
``usual'' footnoting macros appropriately so that |\everyfootnote| is
evaluated veryearly in the process of typesetting footnotes.
The problem is that different users employ different ``usual''
footnoting macros.



by modifying


Below, I will describe how a command |\everyfootnote| can
be properly inserted into

\codedisplay
\renewcommand{\@makefntext}[1]{%
   \the\everyfootnote \@thefnmark.\enspace ##1\@finalstrut\strutbox}}

\everyfootnote={%
   \keepexcntlocal
   \excnt=1
   \lingset{exskip=1ex,exnotype=roman,labeltype=alpha,
      labelanchor=numleft,minexnolabelsep=.5em,
      sampleexno=(i),textoffset=.5em}
   \noindent

\newcommand{adjusttoexno}[1]{\lingset{sampleexno=#1}}
|endcodedisplay


\endinput



Some relevant examples follow:
\ex example one\xe

Taking into account (45), we get:
\pex
\a example two, first part
\a example two, second part
\xe

There are other possible interactions.}

Others prefer this way of doing footnotes.%
\mkeveryfootnote{\keepexcntlocal \excnt=1 \tenpoint
   \lingset{exskip=1ex,labeltype=roman}\noindent}
%\setfootnoteenvironment{
%   \keepexcntlocal
%   \excnt=1
%   \tenpoint
%   \lingset{exskip=1ex minus .3ex,labeltype=roman}
%}%
\footnote{%
%\lingeveryfootnote
%\keepexcntlocal \excnt=1
%\tenpoint
%\lingset{exskip=1ex,labeltype=roman}
%\romanexnumbers
Text.

\ex example one\xe

More text.

\pex
\a example two, first part
\a example two, second part
\xe

Some more text.}

How is this accomplished?  There are two steps.  The first step is
relatively easy.  To get , you first execute the following code
somewhere in the introduction to your document

\codedisplay
\setfootnoteenvironment{
   \keepexcntlocal
   \excnt=1
   \tenpoint
   \lingset{exskip=1ex minus .3ex,labeltype=roman}
}
|endcodedisplay





The idea is to arrange things so that
certain commands are executed when

