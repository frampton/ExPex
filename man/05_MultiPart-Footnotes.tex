
\ifnum\secno<1 \secno=5 \fi

\subsection The parameter {\tt sampleexno}

\parinventory
& \idx{|sampleexno|}& token list& {\it empty}\cr
\endparinventory

In some publications, if two examples are close together in the
running text and the widths of the typeset example numbers are
different, the offsets are modified so that the texts in the two
examples are aligned.  The following is considered, under this
aesthetic, to be less than ideal.

\framedisplay
\keepexcntlocal \excnt=9
\pex
\a I consider firemen available. (generic only)
\a I consider firemen intelligent. (generic only)
\xe
Exceptional case marking (ECM) verbs seem more or less to allow both
existential and generic interpretations of complement subjects:
\pex
\a I believe firemen to be available. (both generic and existential)
\a I believe violists to be intelligent. (generic only)
\xe
\endframedisplay

\expex\ provides the parameter |sampleexno| to handle this formatting
problem.  If the parameter is set to the empty token list, it has no
effect of the formatting.  If it is set to a nonempty token list, that
token list is put in an hbox and its width is taken to be the
effective width typeset example numbers.

\framedisplay
\keepexcntlocal \excnt=9
\pex[sampleexno=(10)]
\a I consider firemen available. (generic only)
\a I consider firemen intelligent. (generic only)
\xe
Exceptional case marking (ECM) verbs seem more or less to allow both
existential and generic interpretations of complement subjects:
\pex
\a I believe firemen to be available. (both generic and existential)
\a I believe violists to be intelligent. (generic only)
\xe
\endframedisplay
\codedisplay~
\pex[sampleexno=(10)]
\a I consider firemen available. (generic only)
\a I consider firemen intelligent. (generic only)
\xe
Exceptional case marking (ECM) verbs seem more or less to allow both
existential and generic interpretations of complement subjects:
\pex
\a I believe firemen to be available. (both generic and existential)
\a I believe violists to be intelligent. (generic only)
\xe
|endcodedisplay

Clearly, fine tuning of this sort in the running text should only be
undertaken as a last step in typesetting a document since it depends
on knowing the final typeset example numbers.

It is much more common to do fine tuning of this sort in footnotes.
There are two reasons for this. First, lowercase roman numerals are
commonly used and the widths vary noticably. Second, only a few
examples are involved, assuming that examples numbers in a footnote
start at (i).

\excnt=34

\subsection Footnotes and endnotes

Footnotes and endnotes pose a somewhat thorny problem since many different
\Tex\ and \Latex\ macros are used to typeset footnotes and endnotes.
Further, there are different ways of assigning example numbers and
labels in multipart examples. The footnote referenced at the end of
this sentence is formatted is an
abbreviated version of footnote 17 in Chapter 2 of Diesing's {\sl
Indefinites}, MIT Press.%
\footnote{%
\lingset{sampleexno=(ii)}
The existential reading does not seem to be available for subjects of
small clause complements of {\it consider\/}:
\pex
\a I consider firemen available. (generic only)
\a I consider firemen intelligent. (generic only)
\xe
Exceptional case marking (ECM) verbs seem more or less to allow both
existential and generic interpretations of complement subjects:
\pex
\a I believe firemen to be available. (both generic and existential)
\a I believe violists to be intelligent. (generic only)
%\a \ljudge{??}I believe opera singers to know Hittite.
\xe
}
This footnote style is fairly common.

The \expex2\/ distribution contains two files, {\it eptexfn.tex\/} and
{\it exltxfn.sty\/} which may be helpful in producing footnotes in
this style.  They do two things.  First, they each define a macro
(|\everyfootnote|) which, if evaluated at the start of processing a
footnote, ensures that examples are correctly formatted.  Second, they
each perform minor surgery on a standard |\footnote| so that (among
other things) |\everyfootnote| is inserted in the appropriate place.

I intend these macro files both to be used directly or to serve as
models for the creation of variations which satisfy user needs. To the
latter end, I will go through {\it expex.sty\/} and explain how it
works.  The file listing is:

\CLnum
\makeatletter
\renewcommand{\@makefntext}[1]{%
   \everyfootnote
   \parindent=1em
   \noindent
   \@thefnmark.\enspace #1%
}
\def\everyfootnote{%
   \keepexcntlocal
   \excnt=1
   \lingset{exskip=1ex,exnotype=roman,sampleexno=,
      labeltype=alpha,labelanchor=numright,labeloffset=.6em,
      textoffset=.6em}
}
\resetatcatcode|endCLnum

|\footnote| uses |\@makefntext| to typeset the footnote.  It is
defined in the {\it .cls\/} file which is used.  The redefinition in
lines 2--7 is a modification of the |\@makefntext| macro defined in
{\it article.cls}.  It first executes |\everyfootnote|, which allows
the user to introduce formatting commands into the footnote
typesetting instructions.  The number is set with no indentation and is
set as text, not a superscript.  |\everyfootnote| first ensures that changes in
|excnt| that are made in the footnote are kept local to the footnote.
|\keepexcntlocal| is an \expex\/ macro. Otherwise, changes in |excnt|
would visible outside the footnote group in which the changes occur.
|excnt| is then initialized.  Finally, parameters are set which
control the formatting of examples in the footnote.

Assuming that |\usepackage{epltxfn}|, the folloing code
produces fn. \the\fnno.
\codedisplay
\footnote{%
\lingset{sampleexno=(ii)}
The existential reading does not seem to be available for subjects of
small clause complements of {\it consider\/}:
\pex
\a I consider firemen available. (generic only)
\a I consider firemen intelligent. (generic only)
\xe
Exceptional case marking (ECM) verbs seem more or less to allow both
existential and generic interpretations of complement subjects:
\pex
\a I believe firemen to be available. (both generic and existential)
\a I believe violists to be intelligent. (generic only)
\xe
}
|endcodedisplay

The code in {\sl eptexfn.tex\/} is somewhat more complicated because
\Tex\ does not number footnotes and it uses the same font for running
text and footnotes.  |\everyfootnote| is identical.  If it is compared
with the footnote macros in the {\sl TexBook\/} (p. 363) it is easy to
see the significance of the various modifications. The user must
define the macro |\footnotesize| before {\sl eptexfn.tex\/} is loaded.
In my own work, |\twelvepoint| and |\tenpoint| are defined modeled on
Knuth's macros on pages 414--15 in the TexBook.  Running text is done
in the context of |\twelvepoint| and |\let\footnotesize=\tenpoint|
is executed before typesetting starts.

I anticipate that although the footnote macros in {\sl epltxfn.sty\/}
and {\sl eptexfn.tex\/} will be useful to some readers without
modification, other users will need to modify them for one reason or
the other.  Endnotes, in particular, will require work.  I hope that
these two files will serve as useful models.  Of course, trivial
modifications of the various dimensions can be done easily.  More
extensive modification require appropriate levels of expertise.  Users
should feel free to write to me directly ({\sl j.frampton@neu.edu\/}) or
post questions to the {\sl Ling-Tex\/} discussion group, or to send me
modifications that might prove useful to others.  A file {\it
epltxfn-endnotes.sty}, for example, would be useful.


\endinput



Some relevant examples follow:
\ex example one\xe

Taking into account (45), we get:
\pex
\a example two, first part
\a example two, second part
\xe

There are other possible interactions.}

Others prefer this way of doing footnotes.%
\mkeveryfootnote{\keepexcntlocal \excnt=1 \tenpoint
   \lingset{exskip=1ex,labeltype=roman}\noindent}
%\setfootnoteenvironment{
%   \keepexcntlocal
%   \excnt=1
%   \tenpoint
%   \lingset{exskip=1ex minus .3ex,labeltype=roman}
%}%
\footnote{%
%\lingeveryfootnote
%\keepexcntlocal \excnt=1
%\tenpoint
%\lingset{exskip=1ex,labeltype=roman}
%\romanexnumbers
Text.

\ex example one\xe

More text.

\pex
\a example two, first part
\a example two, second part
\xe

Some more text.}

How is this accomplished?  There are two steps.  The first step is
relatively easy.  To get , you first execute the following code
somewhere in the introduction to your document

\codedisplay
\setfootnoteenvironment{
   \keepexcntlocal
   \excnt=1
   \tenpoint
   \lingset{exskip=1ex minus .3ex,labeltype=roman}
}
|endcodedisplay





The idea is to arrange things so that
certain commands are executed when

