\everymath={}
\lingset{glwidth=2.4in,everygl=,glhangstyle=none}
\psset{framesep=1ex}

\def\minigloss#1#2{\outlinebox{%
   \vtop{\halign{##\hfil\cr #1\cr #2\cr}}}\hskip1pt}%
\def\miniglossbare#1#2{\vtop{\halign{##\hfil\cr
   #1\cr #2\cr}}}
\psset{linewidth=.4pt}
\def\outlinebox{\psframebox[framesep=0,boxsep=false,linewidth=.4pt]}

\subsection  Line spacing in wrapped glosses

\begininventory
\parameters*
\idx{|extraglwrapskip|}& skip& \hfil \textdim{0 pt}\cr
\endinventory

\noindent ExPex initializes the glossing algorithm so that it
produces glosses like~(\nextx). |glwidth| is set to \textdim{2.4
in} in all of the examples in this section in the interests of
``economy of space''.  The distances between the baselines of
adjacent lines are all equal to the current baselineskip.


\setss .52 .48

\beginss
\ex
\begingl
\gla Ti ma'a'\~nao hao kumuentusi
   ni h\'ayiyi ha'.//
\glb not agr.afraid you
   Infin.speak.to not anyone Emp//
\endgl
\xe|midss
\ex<Ti>
\begingl
\gla Ti ma'a'\~nao hao kumuentusi ni h\'ayiyi ha'.//
\glb not agr.afraid you Infin.speak.to not anyone Emp//
\endgl
\xe
\endss

It is easy to produce glosses like (\nextx) by adjusting the
parameter |extraglwrapskip|.  In (\nextx), the distance between
the baselines of the 2nd and 3rd lines is \textdim{1.2 ex} larger
than the baselineskip.

\beginss
\ex
\begingl[extraglwrapskip=1.2ex]
\gla Ti ma'a'\~nao hao kumuentusi
   ni h\'ayiyi ha'.//
\glb not agr.afraid you
   Infin.speak.to not anyone Emp//
\endgl
\xe|midss
\ex
\begingl[extraglwrapskip=1.2ex]
\gla Ti ma'a'\~nao hao kumuentusi ni h\'ayiyi ha'.//
\glb not agr.afraid you Infin.speak.to not anyone Emp//
\endgl
\xe
\endss

Most users will never need to know anything about the details of
the complicated (software) plumbing required to produce the line
spacing above.  But some, at some point, might face some unusual
situation requiring more control over the line spacing than the
limited control which |extraglwrapskip| allows.  The next section
is for such users only.

\subsection The plumbing which determines wrap spacing in glosses

\begininventory
\parameters
\idx{|autoglwrapskip|}& boolean& \hfil true\cr
\idx{|glwrapskip|}& skip& \hfil\textdim{1 ex}\cr
\idx{|extraglwrapskip|}& skip& \hfil \textdim{0 pt}\cr
\idx{|glstruts|}& boolean& \hfil true\cr
\idx{|abovemoreglskip|}& skip& \hfil \textdim{1 ex}& (obsolete)\cr
\endinventory

\noindent Wrapping is carried out by first forming a sequence of
boxes, as in (\nextx) and then feeding this sequence to Tex's
apparatus for breaking paragraphs into lines and stacking
sequences of lines into pages.  The baseline of these boxes is
the baseline of the top line.  As far as Tex is concerned,
the sequence of boxes is a sequence of words; a paragraph.  We
use (\getref{Ti}) as an illustration.

\ex<seq>
\let\\=\minigloss
\spaceskip=\lingglspace
\\{Ti}{not}, \\{ma'a'\~nao}{agr.afraid}, \\{hao}{you},
\\{kumuentusi}{Infin.speak.to}, \\{ni}{not}, \\{h\'ayiyi}{anyone},
\\{ha'}{Emp}
\xe

The outcome will depend on many Tex parameters.  ExPex
manipulates the Tex dimension |\lineskip| to produce (\nextx).
The Tex dimension |\baselineskip| is left unchanged.

\ex[textoffset=1.2in]
\def\outlinebox{\psframebox[framesep=0,boxsep=false,linewidth=.15pt]}
\let\\=\minigloss
\psscalebox{2}{\vtop{\leftskip=0pt \raggedright \hsize=2.4in
\lineskip=2ex
\leavevmode
\\{\pnode{Y1}Ti}{\pnode{Y3}not}\
\\{ma'a'\~nao}{agr.afraid}\
\\{hao}{\rnode[b]{B1}{y}ou}\
\\{kumuentusi}{Infin.speak.to}\
\\{\pnode(-.2em,0){Y2}ni}{not}\
\\{\rnode[t]{T1}{h}\'ayiyi}{anyone}\
\\{ha'}{Emp}%
}}
\pcline[linewidth=1pt]{<->}(B1)(B1|T1)
\naput[ref=l]{interline skip}
\pcline[linewidth=1pt]{<->}(Y2|Y1)(Y2|Y3)
\pcline[linewidth=1pt]{|*-|*}(Y2|Y1)(Y2|Y3)
\nbput[ref=r]{baselineskip}
\pcline[linewidth=1pt]{<->}(Y2|Y3)(Y2|Y2)
\pcline[linewidth=1pt]{<->|*}(Y2|Y3)(Y2|Y2)
\nbput[ref=r]{$\mit x$\ }
%\pcline[linewidth=1pt]{<->}(B1)(B1|T1)
%\naput[ref=l]{interline skip}
%\pcline[linewidth=1pt]{|<->|}(Y2|Y1)(Y2|Y3)
%\nbput[ref=r]{baselineskip}
%\pcline[linewidth=1pt]{<->|}(Y2|Y3)(Y2|Y2)
%\nbput[ref=r]{$\mit x$\ }
\xe

There are two different modes for determining the interline
skip.  If |autoglwrapskip| is set to true, the default setting,
then struts are inserted before every word in every glword and
the Tex dimension |\lineskip| is set so that $x$ is equal to the
baselineskip plus the value of |extraglwrapskip|.  If the value
of |extraglwrapskip| is \textdim{0 pt}, then $x$ will be equal to
the baselineskip. Note that strut insertion means that in the
auto wrapskip mode it is not the sequence (\getref{seq}) which
undergoes line wrapping, but the sequence~(\nextx).

\ex
\def\\#1#2{\outlinebox{%
   \vtop{\halign{\strut ##\hfil\cr #1\cr #2\cr}}}\hskip1pt}%
\spaceskip=\lingglspace
\\{Ti}{not}, \\{ma'a'\~nao}{agr.afraid}, \\{hao}{you},
\\{kumuentusi}{Infin.speak.to}, \\{ni}{not}, \\{h\'ayiyi}{anyone},
\\{ha'}{Emp}
\xe
With struts, $x$ is therefore the strut depth plus the
interlineskip plus the strut height.

If |autoglwrapskip| is set to false, then |\lineskip| is set so
that is the interlineskip is the  value of |glwrapskip|.  $x$
will therefore be the depth of the deepest box on the line above
plus the value of |\glwrapskip| plus the height of the tallest
box in the line below.  If the parameter |glstruts| is set to
true, then struts are inserted in every word of every glword,
just as in the ``auto mode''.

If auto wrapskip is turned on, only the ExPex parameter
|extraglwrapskip| is relevant; the settings of |glwrapskip| and
|glstruts| are not. If auto wrapskip is turned off, then the
settings of |glwrapskip| and |glstruts| are relevant; the setting
of |extraglwrapskip| is not.

The adjustment of |\lineskip| is made inside the group that
|\begingl| establishes, immediately after its parameters take
effect.  This fact means that if the auto mode is turned on, and
the extra glwrapskip is expressed in font dependent units, the
format for glosses will not need adjustment for glosses in
footnotes which are set in smaller type with tighter baselines.
This provided only that the footnote environment adjusts the
strut size appropriately, as it should.  Generally, the height
plus depth of struts is made equal to the baselineskip.

This text is set in a \textdim{12 pt} font with \textdim{14 pt}
baselines.  Struts are \textdim{10 pt} high and 4pt deep. Suppose
|\fnenvironment| switches to a \textdim{10 pt} font with
\textdim{12 pt} baselines, with struts that are 8.5pt high and
3.5pt deep.  The two examples below are in the scope of
\smallskip\quad
|\lingset{autoglwrapskip=true,extraglwrapskip=1ex}|
\smallskip

\beginss
\ex
\begingl
\gla Ti ma'a'\~nao hao kumuentusi
   ni h\'ayiyi ha'.//
\glb not agr.afraid you
   Infin.speak.to not anyone Emp//
\endgl
\xe|midss
\ex
\lingset{autoglwrapskip=true,extraglwrapskip=1ex}
\begingl
\gla Ti ma'a'\~nao hao kumuentusi ni h\'ayiyi ha'.//
\glb not agr.afraid you Infin.speak.to not anyone Emp//
\endgl
\xe
\endss
\vskip-2ex
\beginss
\fnenvironment
\ex
\begingl
\gla Ti ma'a'\~nao hao kumuentusi
   ni h\'ayiyi ha'.//
\glb not agr.afraid you
   Infin.speak.to not anyone Emp//
\endgl
\xe|midss
\lingset{autoglwrapskip=true,extraglwrapskip=1ex}
\baselineskip=12pt
\setbox\strutbox=\hbox{\vrule height 8.5pt depth 3.5pt width 0pt}
\footnotesize
\ex
\begingl
\gla Ti ma'a'\~nao hao kumuentusi ni h\'ayiyi ha'.//
\glb not agr.afraid you Infin.speak.to not anyone Emp//
\endgl
\xe
\endss

\subsection Differences with version 4.0; the parameter {\tt abovemoreglskip}

\noindent Version 4.0 relied on only one parameter,
|abovemoreglwrapskip|, to control line spacing in gloss wrapping.
That parameter should now be considered obsolete.  It is still
defined in version 4.1, but it is now an alias for |glwrapskip|.
|glwrapskip| should now be used.  The macro |\gloldstyle| will
reinstate the initial settings in version 4.0.

\endinput



construct the gloss.  for breaking para  building apparatus typeset that
sequence of boxes exactly as it typesets a sequence of words. The
gloss display (\nextx a), for example, is assembled by first
generating the sequence of boxes (called here glwords) in
(\nextx b). Then Tex's paragraph building apparatus organizes
these glwords into a paragraph.  The lineskip isis assembles the
boxes in (\nextx c).  Finally, these two boxes are stacked
\pex[interpartskip=2ex]
\a
\begingl
\gla Ti ma'a'\~nao hao kumuentusi ni h\'ayiyi ha'.//
\glb not agr.afraid you Infin.speak.to not anyone Emp//
\endgl
%
\a\let\\=\minigloss
\spaceskip=\lingglspace
\\{Ti}{not}, \\{ma'a'\~nao}{agr.afraid}, \\{hao}{you},
\\{kumuentusi}{Infin.speak.to}, \\{ni}{not}, \\{h\'ayiyi}{anyone},
\\{ha'}{Emp}
%
\a\let\\=\minigloss
\vtop{\hsize=2.4in \leftskip=0pt \raggedright
\def\minigloss#1#2{\outlinebox{%
   \vtop{\halign{##\hfil\cr #1\cr #2\cr}}}}%
\lineskip=2ex
\spaceskip=\lingglspace
\\{\rnode[l]{A}{Ti}}{not}\ \\{ma'a'\~nao}{agr.afraid}\ \\{hao}{you}\
\\{kumuentusi}{Infin.speak.t\rnode[r]{C}{o}}\par
\nointerlineskip\pnode{B}%\hrule
\vskip2ex \hrule
\\{ni}{not}\
\\{h\'ayiyi}{anyone}\ \\{ha'}{Emp}
}
%\psdot(B)
%\psdot(C)
\psline[linewidth=1pt,linestyle=dotted](B)(C|B)
\xe
The baseline of a glword is the baseline of its top word. The
vertical spacing in each glword is determined as in page
building; the vertical spacing is determined by the baselineskip,
provided that there are no overly tall or overly deep characters
in the words.

Tex's line building apparatus then assembles the two boxes below.

\pex[interpartskip=2ex]<hboxes>
%
\a \outlinebox{\hbox{\let\\=\miniglossbare  \spaceskip=.6em
\\{Ti}{not} \\{ma'a'\~nao}{agr.afraid} \\{hao}{you}
\\{kumuentusi}{Infin.speak.to}}}

\a \outlinebox{\hbox{\let\\=\miniglossbare  \spaceskip=.6em
\\{ni}{not} \\{h\'ayiyi}{anyone} \\{ha'}{Emp}}}
\xe
The baselines of these boxes is the baseline of the top line.

The boxes (\lastx) must then be stacked by Tex's page building
algorithm. Since the depth of the top box plus the height of the
first box is larger than the baselineskip, no interline glue is
inserted by the algorithm. Remember that the depth and height of
these boxes is determined by baseline of the top line in the
glwords;\footnote{I am not the first to wish Tex boxes had the
provision for two baselines} the baseline of the bottom words is
irrelevant. Without glue between the boxes, (\nextx) would
result.

\ex
\vtop{\hbox{\outlinebox{\hbox{\let\\=\miniglossbare  \spaceskip=.6em
\\{Ti}{not} \\{ma'a'\~nao}{agr.afraid} \\{hao}{you}
\\{kumuentusi}{Infin.speak.to}}}}
%
\hbox{\outlinebox{\hbox{\let\\=\miniglossbare  \spaceskip=.6em
\\{ni}{not} \\{h\'ayiyi}{anyone} \\{ha'}{Emp}\pnode{B}}}}}
\xe

Since the page building algorithm does not insert vertical glue
between the boxes, ExPex must do it.  The amount is determined by
the parameter |glwrapskip|.\footnote{This parameter was called
{\tt abovemoreglskip} in version 2.7 .  For compatibility, the
old name will still work as before, but should be considered
obsolete.}
The relevant dimensions are shown in (\lastx). What I call here
the ``wrap separation'' has only a descriptive role; there is no
parameter or other Tex internal entity corresponding to this
dimension.  It is impossible to keep careful control of the
dimensions unless the words have a fixed known height and depth,
so I assume that a strut has been inserted before every word in
every glword.

\psset{framesep=0,boxsep=false,linewidth=.15pt}

\exdisplay \let\\=\miniglossbare
\def\miniglossbare#1#2{\vtop{\halign{##\hfil\cr
   #1\cr #2\cr}}}%
\hskip1.4in
\psscalebox{2}{\vtop{\hbox{\psframebox{\rnode[br]{B1b}{\hbox{\let\\=\miniglossbare  \spaceskip=.6em
\\{\pnode{A1}\rnode[t]{R33}{\strut Ti}}{{\pnode{A2}\strut not}} \\{ma'a'\~nao}{agr.afraid} \\{\dots}{\dots}
}}}\kern-60pt}
\vskip1ex
\hbox{\psframebox{\rnode[tr]{B2t}{\hbox{\let\\=\miniglossbare  \spaceskip=.6em
\enspace\\{\pnode{A3}\strut ni}{\pnode{A4}\strut not} \\{h\'ayiyi}{anyone} \\{ha'}{Emp}}}}}}}
\rput(B1b){\rput(0,14pt){\pnode{R1}}}
\psline[linewidth=4pt,linecolor=white](B1b)(B1b|R33)
\rput(A1|B1b){\pnode(-1em,0){Q1}}
\psset{linewidth=1pt,labelsep=.5em}
\pcline{|<->|}(Q1)(Q1|B2t)
\nbput[ref=r]{glwrapskip}
\pcline{|<->|}(Q1|A1)(Q1|A2)
\nbput[ref=r]{baselineskip}
\pcline{|<->|}(Q1|A3)(Q1|A4)
\nbput[ref=r]{baselineskip}
\rput(B1b|A2){\pnode(1em,0){Q2}}
\pcline{|<->|}(Q2)(Q2|A3)
\naput[ref=l]{\it ``wrap separation''}
\xe

A common way to establish a gloss format is to choose the
glwrapskip so that $x$ is equal to the baselineskip, as in
(\nextx a), or to choose glwrapskip so that $x$ is the
baselineskip plus some increment, as in (\nextx b). ExPex
provides the macro |\normalglwrapskips| which does two things.
First, it turns on automatic strut insertion so that struts are
inserted before every word in every glword. Second, the
glwrapskip is adjusted so that the interline wrap skip is equal
to the baselineskip.  The boolean parameter |glstruts| is also
provided to control automatic strut insertion independently of
|normalglwrapskips|, if desired.  Summarising,
|\normalglwrapskip| sets both |glstruts| and |glwrapskip|, but
these parameters can be independently set.

Here are some examples.

\psset{framesep=1.5ex}
\setss .55 .45

\beginss
\ex
\baselineskip=20pt
\normalglwrapskips
\begingl
\gla Ti ma'a'\~nao hao kumuentusi
   ni h\'ayiyi ha'.//
\glb not agr.afraid you
   Infin.speak.to not anyone Emp//
\endgl
\xe|midss
\ex
\baselineskip=20pt
\normalglwrapskips
\begingl
\gla Ti ma'a'\~nao hao kumuentusi
   ni h\'ayiyi ha'.//
\glb not agr.afraid you
   Infin.speak.to not anyone Emp//
\endgl
\xe
\endss

|glwrapskip| is an incremental parameter, so we can use it to
modify the value set by |normalglwrapskips|, as shown below.

\beginss
\ex
\baselineskip=20pt
\normalglwrapskips
\begingl[glwrapskip=!5pt]
\gla Ti ma'a'\~nao hao kumuentusi
   ni h\'ayiyi ha'.//
\glb not agr.afraid you
   Infin.speak.to not anyone Emp//
\endgl
\xe|midss
\ex
\baselineskip=20pt
\normalglwrapskips
\lingset{glwrapskip=!5pt}
\begingl
\gla Ti ma'a'\~nao hao kumuentusi
   ni h\'ayiyi ha'.//
\glb not agr.afraid you
   Infin.speak.to not anyone Emp//
\endgl
\xe
\endss

The result above can be obtained in a different way.
|\normalglbaselines| takes an optional parameter.
|\normalglbaselines[�$\mit\mskip-2mu x$�]| adjusts the glwrapskip
so that the interline wrap skip is the baselineskip {\it plus
$x$}.  Compare (\nextx) with (\lastx).

\beginss
\ex
\baselineskip=20pt
\normalglwrapskips[5pt]
\begingl
\gla Ti ma'a'\~nao hao kumuentusi
   ni h\'ayiyi ha'.//
\glb not agr.afraid you
   Infin.speak.to not anyone Emp//
\endgl
\xe|midss
\ex
\baselineskip=20pt
\normalglwrapskips[5pt]
\begingl
\gla Ti ma'a'\~nao hao kumuentusi
   ni h\'ayiyi ha'.//
\glb not agr.afraid you
   Infin.speak.to not anyone Emp//
\endgl
\xe
\endss





\beginss
\ex
\baselineskip=14pt
\normalglwrapskips
\begingl
\gla Ti ma'a'\~nao hao kumuentusi
   ni h\'ayiyi ha'.//
\glb not agr.afraid you
   Infin.speak.to not anyone Emp//
\endgl
\xe|midss
\ex
\baselineskip=14pt
\normalglwrapskips
\begingl
%\begingl[normalglwrapskips]
\gla Ti ma'a'\~nao hao kumuentusi
   ni h\'ayiyi ha'.//
\glb not agr.afraid you
   Infin.speak.to not anyone Emp//
\endgl
\xe
\endss
Struts in this series of examples have height \textdim{10 pt} and
depth \textdim{4 pt}.

\beginss
\baselineskip=18pt
\begingl[glstruts=true,
   glwrapskip=4pt]
\gla Ti ma'a'\~nao hao kumuentusi
   ni h\'ayiyi ha'.//
\glb not agr.afraid you
   Infin.speak.to not anyone Emp//
\endgl
\xe|midss
\ex
\baselineskip=18pt
\begingl[glstruts=true,
   glwrapskip=4pt]
\gla Ti ma'a'\~nao hao kumuentusi
   ni h\'ayiyi ha'.//
\glb not agr.afraid you
   Infin.speak.to not anyone Emp//
\endgl
\xe
\endss

\beginss
\ex
\baselineskip=14pt
\begingl[glstruts=false,
   glwrapskip=1ex]
\gla Ti ma'a'\~nao hao kumuentusi
   ni h\'ayiyi ha'.//
\glb not agr.afraid you
   Infin.speak.to not anyone Emp//
\endgl
\xe
|midss
\ex
\baselineskip=14pt
\begingl[glstruts=false,
   glwrapskip=1ex]
\gla Ti ma'a'\~nao hao kumuentusi
   ni h\'ayiyi ha'.//
\glb not agr.afraid you
   Infin.speak.to not anyone Emp//
\endgl
\xe
\endss
This gloss style was the ExPex default in version 2.7.


When ExPex is loaded, it executes |\normalglwrapskips|, which
sets |glstruts| to true and adjusts the glwrapskip appropriately.
{\it This is a change from version 2.7}.  The macro
\hbox{|\normalglwrapskips|} and the parameter |glstruts| are new.
If required for compatibility, version 2.7 initial behaviour can
be achieved by |\lingset{glstruts=false,glwrapskip=1ex}|.  Note
carefully that |\normalglwrapskips| uses the value of the
baselineskip in whatever environment it is executed in.  If the
baselineskip is changed by the user or by files that are loaded
after ExPex, it will be necessary to renormalize the glwrapskip by
executing |\normalglwrapskips| again after the baselineskip has
been changed.

Many Tex formats change the font, baselineskip, and strut size in
footnotes.  Suppose that the text is in \textdim{12 pt}, with a
baselineskip of\textdim{14 pt}, and struts which are \textdim{10
pt} high and \textdim{4 pt}s deep, like the one I am using here.
Suppose also that the footnote goes to a \textdim{10 pt} font,
with a baselineskip of \textdim{12 pt} and struts which are
\textdim{8.5 pt} high and \textdim{3.5 pt} deep, as is the
footnote style in this document.
Suppose |lingset{everygl=\normalglwrapskips[.5ex]}| somewhere
earlier in the document.  Then we have:

\beginss
\ex
\begingl
\gla Ti ma'a'\~nao hao kumuentusi
   ni h\'ayiyi ha'.//
\glb not agr.afraid you
   Infin.speak.to not anyone Emp//
\endgl
\xe|midss
\ex<topgloss>
\begingl
\gla Ti ma'a'\~nao hao kumuentusi
   ni h\'ayiyi ha'.//
\glb not agr.afraid you
   Infin.speak.to not anyone Emp//
\endgl
\xe
\endss
In footnotes, {\it the identical code\/} produces the footnote
below.%
\footnote{In footnotes in this document, {\tt numoffset} is
\textdim{0 pt} and example numbering is roman.
\ex
\begingl
\gla Ti ma'a'\~nao hao kumuentusi
   ni h\'ayiyi ha'.//
\glb not agr.afraid you
   Infin.speak.to not anyone Emp//
\endgl
\xe
The gloss breaks in a different place than the one in
(\getref{topgloss}) because the gloss width has
remained unchanged, \textdim{2.4 in},
so more glwords fit on a line.}



\subsection Line breaking in glosses:
How to avoid the dreaded ``overfull box''

\begininventory
\parameters
\idx{|glspace|}& skip& |.5em plus .25em minus .16667 em|\hfil\cr
\idx{|glrightskip|}& skip& |0pt plus .1\hsize|\hfil \cr
\endinventory
%
Hopefully, you will never encounter the problem which the
parameterization in this section is designed to solve, the
``overfull line'' error message.  The default setting puts
\textdim{.5 em} of glue between the glwords a line, plus
\textdim{.25 em} of stretchability and \textdim{.16667 em} of
shrinkage.  The interglword spacing on a line can therefore vary
from \textdim{.33333 em} to \textdim{.75 em} to accommodate the
needs of the Tex linebreaking algorithm.  Further, the default settings
allow up to 10\% of the hsize of whitespace to appear at the end
of the line of glwords.  The possible extra space at the right
edge and stretchability/shrinkability of the space between
glwords means that line breaking in glosses will rarely encounter
problems with overfull lines.

In unusual cases, there may be a problem.  Your publisher (or
you) may be particularly fussy and demand a particular spacing
and/or better right alignment in glosses.  Or you might want to make a
particularly narrow gloss, which increases the chances that there
might not be enough flexibility in the spacing.

If you do run into the problem of overfull lines in a gloss, two
parameters allow for a great deal of flexibility; |glspace| and
|glrightskip|. |glspace| is an "inc parameter", so you could say,
for example, |glspace=!0pt plus .2em|, increasing the
stretchability of the interglword space by \textdim{2 em}). Or,
you might want to increase the stretchability of the right skip
to allow more whitespace at the right edge.  An acceptable
solution will depend on your particular typesetting aesthetics.
Solving line breaking problems is often troublesome and requires
experimentation.


\endinput






In order to do this, it


Here are the various dimensions in (\lastx) which are relevant to
line spacing in the gloss.
\lingset{framesep=0,boxsep=false}

\exdisplay \let\\=\miniglossbare
\hskip1.3in
\psscalebox{2.5}{\vtop{\hbox{\psframebox{\rnode[br]{B1b}{\hbox{\let\\=\miniglossbare  \spaceskip=.6em
\\{\pnode{A1}\rnode[t]{R33}{Ti}}{{\pnode{A2}not}} \\{ma'a'\~nao}{agr.afraid} \\{\dots}{\dots}
}}}\kern-60pt}
\vskip1ex
\hbox{\psframebox{\rnode[tr]{B2t}{\hbox{\let\\=\miniglossbare  \spaceskip=.6em
\\{\pnode{A3}ni}{\pnode{A4}not} \\{h\'ayiyi}{anyone} \\{ha'}{Emp}}}}}}}
\rput(B1b){\rput(0,14pt){\pnode{R1}}}
\psline[linewidth=4pt,linecolor=white](B1b)(B1b|R33)
\rput(A1|B1b){\pnode(-1em,0){Q1}}
\psset{linewidth=1pt,labelsep=.5em}
\pcline{|<->|}(Q1)(Q1|B2t)
\nbput[ref=r]{glwrapskip}
\pcline{|<->|}(Q1|A1)(Q1|A2)
\nbput[ref=r]{baselineskip}
\pcline{|<->|}(Q1|A3)(Q1|A4)
\nbput[ref=r]{baselineskip}
\rput(B1b|A2){\pnode(1em,0){Q2}}
\pcline{|<->|}(Q2)(Q2|A3)
\naput[ref=l]{glwrapbaselineskip}
\xe

