\makeatletter
\def\showpex{%
   \edef\resetexcnt{\noexpand\global\noexpand\excnt=\the\excnt}%
   \quad
   \psscalebox{1.5}{%
   \vrule height.5em depth4.6em width0pt
   \parindent=0pt
   \leavevmode
   \Lingset{numoffset=4.5em,preambleoffset=4em,labeloffset=3em,
      textoffset=4em,labelwidth=.8em,arrows=<->}
   \pnode(0,0){A}
   \lower10ex\vbox{\hsize=3.8in
   \excnt=23
   \pex
   \pnode(-\ling@preambleoffset,0){B3}\pnode{B4}
   This is the preamble.
   \a \pnode{E4}%
   This is an example.%
   \SpecialCoor
   \rput(A|B3){\pnode{B1}}
   \rput(B1){\pnode(\lingnumoffset,0){B2}}
   \rput(E4){\pnode(-\lingtextoffset,0){E3}}
   \rput(E3){\pnode(-\linglabelwidth,0){E2}}
   \rput(E2){\pnode(-\linglabeloffset,0){E1}}
   \psset{nodesep=0,labelsep=0}
   \pcline[offset=2.5ex](B1)(B2)
   \Aput{\strut\eighttt numoffset}
   \pcline[offset=2.5ex](B3)(B4)
   \Aput{\strut\eighttt preambleoffset}
   \pcline[offset=-1.5ex](E1)(E2)
   \Bput{\strut\eighttt labeloffset}
   \pcline[offset=-1.5ex](E3)(E4)
   \Bput{\strut\eighttt textoffset}
   \psset{offset=0,angle=90,linestyle=dotted,arrows=-}
   \XKV@for@n{1,2,3,4}\which{%
      \pcline(B\which)([nodesep=3ex]B\which)}%
   \XKV@for@n{1,2,3,4}\which{%
      \pcline([nodesep=1.5ex]E\which)([nodesep=-2ex]E\which)}%
   \xe}}%
   \resetexcnt
}
\resetatcatcode


%\lingset{lingstyle=fdabovecd}
%%%%%%%%%%%%%%%%%%%%%%%%%%%%%%%%%%%%%%%%%%%%%%%%%%%%%%%%%%%%%%
\section Examples with labeled parts

\begingroup
\parindent=0pt
Macros:\hskip2.5em |\pex~[]|, |\a|, |\linglabeloffset|\par
\parinventory
& \idx{|labeltype|}& \idx{|alpha|}, \idx{|caps|}, or \idx{|numeric|}&
   |alpha|\cr
& \idx{|labeloffset|}& incrementable dimension& |1em|\cr
& \idx{|labelwidth|}& incrementable dimension& |.78em|\cr
& \idx{|interpartskip|}& skip& |1ex plus .2ex minus .2ex|\cr
%\idx{|nopreamble|}& |true|, |false|\cr
& \idx{|samplelabel|}\cr
%}}
\endparinventory
\endgroup
%\bigskip

\noindent Typical examples are given below, with the default
settings of the parameters.

\framedisplay
\pex
\a This is the first example.
\a This is the second example.
\xe

\pex~<Pre> Multipart examples often have a title or preamble of some
kind.
\a This is the first example.
\a This is the second example.
\xe
\endframedisplay

\codedisplay~
\pex
\a This is the first example.
\a This is the second example.
\xe

\pex~ Multipart examples often have a title or preamble of some kind.
\a This is the first example.
\a This is the second example.
\xe |endcodedisplay

\noindent
Just like |\ex|, |\pex| must be closed by |\xe|, can be
modified by a tilde diacritic to suppress adding vertical space
above the example, and accepts parameters.
The macro \idx{|\a|}, which introduces each
labeled part, is defined only
within \hbox{|\pex| \dots|\xe|}.  Extra vertical skip is inserted
between the parts; the amount is determined by |interpartskip|.

The various dimensions are pictured in~(\nextx), assuming the
default settings of the parameters |labelanchor| and
|textanchor|.  (The effects of changing the settings of the
anchoring parameters will be considered in Section
\getref{anchors}.)  Note that the parameters |numoffset| and
|textoffset| are also used in formatting examples without parts.

\ex \makeatletter \quad
\edef\resetexcnt{\noexpand\global\noexpand\excnt=\the\excnt}%
\vrule height1.6em depth3.5em width0pt
\psscalebox{1.5}{%
\parindent=0pt
\leavevmode
\lingset{numoffset=4.5em,preambleoffset=4em,labeloffset=3em,
   textoffset=4em,labelwidth=.8em}
\pnode(0,0){A}
\lower5ex\vbox{\hsize=3.8in
\excnt=23
\psset{arrows=<->}
\pex
\a \pnode{E6}%
This is an example.%
\SpecialCoor
\rput(A|E6){\pnode{E1}}
\rput(E1){\pnode(\lingnumoffset,0){E2}}
\rput(E6){\pnode(-\lingtextoffset,0){E5}}
\rput(E5){\pnode(-\linglabelwidth,0){E4}}
\rput(E4){\pnode(-\linglabeloffset,0){E3}}
\psset{nodesep=0,labelsep=0}
\pcline[offset=2.5ex](E1)(E2)
\Aput{\strut\eighttt numoffset}
\pcline[offset=2.5ex](E3)(E4)
\Aput{\strut\eighttt labeloffset}
\pcline[offset=2.5ex](E5)(E6)
\Aput{\strut\eighttt textoffset}
%\pcline[offset=-1.5ex,arrows=>-<,nodesep=-.65ex](E4)(E5)
\pcline[offset=-1.5ex](E4)(E5)
\Bput{\strut\eighttt labelwidth}
\psset{offset=0,angle=90,linestyle=dotted,arrows=-}
\XKV@for@n{1,2,3,4,5,6}\which{%
   \pcline(E\which)([nodesep=3ex]E\which)}
\pcline([nodesep=1.5ex]E4)([nodesep=-2ex]E4)
\pcline([nodesep=1.5ex]E5)([nodesep=-2ex]E5)
\resetexcnt
\xe}}%
\resetatcatcode
\xe
Adjustment for the width of the example number is
automatic, but the width of the label slot is a parameter
setting, not adjusted to the width of the particular label which
appears in the label slot. The initial setting should a good
approximation to the width of ``a.'' and ``1.'' in the
current font, unless your font is unusual.  If the labels are
alphabetic, wide characters like ``w'' will spill over the label
noticeably and the label width might need adjustment, depending on
the width of the slot it spills into. If the labels are numeric
and numbers greater than 9 are needed, the labels will spill
over substantially and |labelwidth| will almost certainly need
adjustment if high quality output is desired.  We will return to
this issue later.

{\sl ExPex\/} comes with three label types predefined: |alpha|,
|caps|, and |numeric|.  Later we will see how to define custom
label types as needed.

\beginss
\pex[labeltype=alpha]
\a First part.
\a Second part.
\xe |midss
\pex[labeltype=alpha]
\a First part.
\a Second part.
\xe
\endss

\beginss
\pex[labeltype=caps]
\a First part.
\a Second part.
\xe |midss
\pex[labeltype=caps]
\a First part.
\a Second part.
\xe
\endss

\beginss
\pex[labeltype=numeric]
\a First part.
\a Second part.
\xe |midss
\pex[labeltype=numeric]
\a First part.
\a Second part.
\xe
\endss

If you look closely, the effect a fixed labelwidth can be seen in
(\blastx).  The label and text are too close together because the
width of capital letters makes them spill out of the right side
of the label slot.  This is rectified in (\nextx).

\beginss
\pex[labeltype=caps,samplelabel=A.]
\a First part.
\a Second part.
\xe |midss
\pex[labeltype=caps,samplelabel=A.]
\a First part.
\a Second part.
\xe
\endss
The effect of |samplewidth=A.| is to set the label width to be
the width of the sample, in the current font.  So there are three
ways to adjust the labelwidth: 1) setting it to the width of a
sample label; 2) setting |labelwidth| to an explicit dimension;
or 3) incrementing the contextual setting of
|labelwidth| by a specified dimension.

\subsection Formatting the preamble

\parinventory
& \idx{|preambleoffset|}& incrementable dimension& |1em|\cr
& \idx{|belowpreambleskip|}& skip& |1ex|\cr
& \idx{|nopreamble|}&& (default only)\cr
\endparinventory

\noindent Visible material which occurs before the first labeled
entry, as in (\getref{Pre}) for example, is called the preamble.
Although the initial settings produce the format in (\blastx),
the offset of the preamble can be set independently of the offset
of the labels and the extra vertical skip between the preamble
and the first part can also be set independently of the extra
vertical skip between the various parts.

ExPex sets the label offset and the preamble offset to be equal,
so that the left edge of the preamble aligns with the left edge
of the labels.  But this is under the control of the user.

\framedisplay
\pex[labeltype=caps,labeloffset=!.8em]
{\it Principles of the Theory of Binding}
\a An anaphor is bound in its governing category.
\a A pronomial is free in its governing category.
\a An R-expression is free
\xe
\endframedisplay

\codedisplay
\pex[labeltype=caps,labeloffset=!.8em]
{\it Principles of the Theory of Binding}
\a An anaphor is bound in its governing category.
\a A pronomial is free in its governing category.
\a An R-expression is free
\xe |endcodedisplay


The different effects of |interpartskip| and |belowpreambleskip|
are illustrated below.

\framedisplay
\pex[labeltype=caps,belowpreambleskip=.75ex,interpartskip=.25ex]
{\it Principles of the Theory of Binding}
\a An anaphor is bound in its governing category.
\a A pronomial is free in its governing category.
\a An R-expression is free.
\xe
\endframedisplay

\goodbreak
\codedisplay
\pex[labeltype=caps,belowpreambleskip=.75ex,interpartskip=.25ex]
{\it Principles of the Theory of Binding}
\a An anaphor is bound in its governing category.
\a A pronomial is free in its governing category.
\a An R-expression is free.
\xe
|endcodedisplay

If |\a| or |\label| (see Section \getref{labelsec}) directly follows
|\pex| (and a possible tilde diacritic
and parameter settings), |\pex| assumes that
there is no preamble, otherwise it assumes that there is.
This poses a problem if you want to have nonprinting material
other than a label specification before the first part.  For example,
suppose you want to increase the baselineskip by 2pt.  You might
try (\nextx), but it fails to achieve what you want.

\beginss
\pex \openup2pt
\a
\a
\a
\xe|midss
\pex \openup2pt
\a
\a
\a
\xe
\endss
\noindent {\sl ExPex\/} provides the parameter |nopreamble| to
solve the problem.  Setting it to ``true'' tells |\pex| that it
should assume that there is no preamble, in spite of superficial
appearances to the contrary.  So:

\beginss
\pex[nopreamble] \openup2pt
\a
\a
\a
\xe|midss
\pex[nopreamble] \openup2pt
\a
\a
\a
\xe
\endss
\noindent Invoking |nopreamble| with no specified value sets it
to the stipulated default value ``true''.  Setting
|nopreamble=false| has no effect since |\pex| always starts out
assuming that there is a preamble (i.e. that |nopreamble| has
been set to |false|) and this is always overridden
if a following |\a| or |\label| is detected, regardless of the
setting of |nopreamble|.

\font\titlett=txtt at 13.3pt

\subsection Formatting the labels

\parinventory
& \idx{|everylabel|}& token list\qquad& |{}|\cr
& \idx{|labelformat|}& formatting string (see below)\hidewidth\cr
\endparinventory

The value of |everylabel| is inserted just before labels are
typeset.  It is grouped so that it affects only the label.  It is
commonly used to set the font used for the labels if it differs
from the font in the running text.  For example:

\beginss
\pex[everylabel=\it]
\a one
\a two
\xe|midss
\pex[everylabel=\it]
\a one
\a two
\xe
\endss

There are other uses aside from setting the label font.

\beginss
\pex[everylabel=A,labeltype=numeric,
   samplelabel=A1.]
\a An example
\a An example
\a An example
\xe|midss
\pex[everylabel=A,labeltype=numeric,
   samplelabel=A1.]
\a An example
\a An example
\a An example
\xe
\endss

The effect of the value of |labelformat| is illustrated in
(\nextx).

\beginss
\pex[labelformat=$\langle A\rangle$,
   labelwidth=!1em]
\a first
\a second
\xe|midss
\pex[labelformat=$\langle A\rangle$,
   labelwidth=!1em]
\a first
\a second
\xe
\endss

\noindent The default is |labelformat=A.|.  Most users will never
need anything different.

The label formatting mechanism is primitive.
|labelformat| must
be of the form\medskip
\leftline{\quad$\rm \langle balanced\ text\rangle A
\langle balanced\ text\rangle$}
\medskip
\noindent The pre-A text is inserted before the label (including
the material specified by |everypar|) and the post-A text is
inserted after the label. In Knuth's terminology, {\it balanced
text\/} is a string of tokens with properly nested (explicit)
braces. No error checking is done to ensure that the format
specification has the required form, so errors can lead to
obscure error messages.

\subsection Aligning the labels

\parinventory*
& \idx{|labelalign|}& \idx{|left|}, \idx{|right|}, \idx{|margin|}&
   \idx{|right|}\cr
\endparinventory
\bigskip

\noindent There is a choice of left, right, or center alignment of the
labels in the label slot.  This is chosen by the parameter
\idx{|labelalign|}, which can be set to |left|, |center|, or |right|.
The part labels can be uppercase letters, lowercase letters, or
integers.  The choice is made via by the parameter
\idx{|labeltype|}, which can be set to |caps|, |alpha|, or
|numeric|.  The setting of \idx{|interpartskip|} determines the
vertical skip which is inserted between the labeled entries.

For |\pex| constructions which use the letters or numbers which
have roughly the same width, label alignment is not a significant
concern.  But if labels include, for example, the narrow letter
``i'' and the wide letter ``m'', as below, label alignment has a
significant effect on the appearance.  Individual tastes (and
publisher's demands) may differ, but I prefer center alignment in
these cases.

\line{%
\lingset{belowpreambleskip=.4ex,interpartskip=.3ex,
   everyex=\advance\pexcnt by 8,samplelabel=m.}%
\hsize=.33\hsize
\vbox{%
\pex[labelalign=left]<triplea1>
{\it left aligned labels}
\a A typical example.
\a A typical example.
\a A typical example.
\a A typical example.
\a A typical example.
\a A typical example.
\xe}%
\vbox{%
\pex[labelalign=center]
{\it center aligned labels}
\a A typical example.
\a A typical example.
\a A typical example.
\a A typical example.
\a A typical example.
\a A typical example.
\xe}
\vbox{%
\pex[labelalign=right]<triplea3>
{\it right aligned labels}
\a A typical example.
\a A typical example.
\a A typical example.
\a A typical example.
\a A typical example.
\a A typical example.
\xe}}

If the labels are numeric, label alignment can have an even bigger
effect.  Again, individual tastes and publishers' demands may
differ.  My preference is right alignment in this case.

\exbreak[8ex]
\line{%
\lingset{interpartskip=.3ex,labeltype=numeric,
   everyex=\advance\pexcnt by 6,samplelabel=10.}%
\hsize=.33\hsize
\vbox{%
\pex[labelalign=left]<tripleb1>
{\it left aligned labels}
\a A typical example.
\a A typical example.
\a A typical example.
\a A typical example.
\a A typical example.
\a A typical example.
\xe}\hfil
\vbox{%
\pex[labelalign=center]
{\it center aligned labels}
\a A typical example.
\a A typical example.
\a A typical example.
\a A typical example.
\a A typical example.
\a A typical example.
\xe}\hfil
\vbox{%
\pex[labelalign=right]<tripleb3>
{\it right aligned labels}
\a A typical example.
\a A typical example.
\a A typical example.
\a A typical example.
\a A typical example.
\a A typical example.
\xe}}

The default is left alignment for letters (either uppercase or
lowercase) and right alignment for numbers. Unless numbers or
letters of significantly different widths appear as labels, most
users will not notice the difference and can safely ignore the
issue.

Particularly wide labels create another problem which needs
attention.  Recall that the slot in which labels are typeset does
not automatically adjust its width to fit the widest label.  It
is fixed in advance by the setting of |labelwidth|. Wide labels
will spill out of the label slot; to the right, left, or on both
sides, depending on the setting of |labelalign|. Particularly in
the case of double digit numeric labels, the width of the label
slot needs to be changed from the default value. One can do this
by saying something like |labelwidth=!1ex|, which will increase
the label width by $1\,ex$.  This gives more or less
satisfactory results. The pseudo parameter \idx{|samplelabel|} is
provided to help in adjusting the label width.  If you say, for
example,

\codedisplay
\lingset{samplelabel=10.}
|endcodedisplay

\noindent then |labelwidth| is set to the width of ``10.'' in the
current font. Examples (\getref{triplea1}--\getref{triplea3}) used
|\lingset{samplelabel=m.}| and
(\getref{tripleb1}--\getref{tripleb3}) used
|\lingset{samplelabel=10.}|

It is a side issue, but the reader may have wondered how
(\getref{triplea1}--\getref{triplea3}) and
(\getref{tripleb1}--\getref{tripleb3}) were typeset. The idea is
simple.  You say:

\codedisplay
\line{\divide\hsize by 3
   \vbox{\pex|verbdots \xe}\hss
   \vbox{\pex|verbdots\xe}\hss
   \vbox{\pex|verbdots \xe}}
|endcodedisplay
|\hss| is used to give a little stretch or shrink so
that dimensional rounding does not lead to an under or overfull
|\line{| \dots|}|.

Variations are useful.  One can easily imagine a situation in
which something like the following is appropriate for two side by
side examples.

\codedisplay
\line{%
   \vbox{\hsize=.55\hsize \pex|verbdots \xe}\hss
   \vbox{\hsize=.45\hsize \pex|verbdots \xe}}
|endcodedisplay


\subsection  Anchoring the labels, text, and preamble%
\deftag{\the\secno.\the\subsecno}{anchors}

%Choice parameter:\quad |labelanchor|\qquad set to |numright| or
%|margin|
\parinventory*
& \idx{|labelanchor|}& \idx{|numright|}, \idx{|numleft|},
   \idx{|margin|}& \idx{|numright|}\cr
& \idx{|textanchor|}& \idx{|normal|}, \idx{|numleft|}&
   \idx{|normal|}\cr
& \idx{|preambleanchor|}& \idx{|numright|}, \idx{|labelleft|}&
   \idx{|text|}\cr
\endparinventory

\noindent Ordinarily, the label offset is measured from the right
edge of the example number.  There is little choice if a label
occurs on the same line as the example number (i.e. there is no
preamble).  In example displays with a preamble, however, it is
sometimes desirable to measure the labeloffset from the left
margin, as shown below.

\ex[labelanchor=margin] \showpex \xe

One application of anchoring the label offset at the
left margin is in examples like the following.  If it is
impossible or undesirable to break a line (or graphic display of
some kind), (\anextx) is an alternative to (\nextx).  There is a
different kind of example on page \getref{partialrepeat}.

\framedisplay
\pex
\a A long example that will not fit in one line and
should not be broken if that is at all possible.
\a A short line might be more interesting linguistically, but less
interesting typographically.
\xe

\pex~[labelanchor=margin,labeloffset=1em] Make up a title.
Typographically, almost anything looks better than nothing.
\a A long example that will not fit in one line and
should not be broken if that is at all possible.
\a A short line might be more interesting linguistically, but less
interesting typographically.
\xe
\endframedisplay

It is produced by:

\vbox{\codedisplay
\pex[labelanchor=margin,labeloffset=1em]\par
\a A long example that will not fit in one line and
should not be broken if that is at all possible.
\a A short line might be more interesting linguistically, but less
interesting typographically.
\xe
|endcodedisplay}


\noindent Note that the preamble in the
second |\pex| is |\par|.  |\relax| would accomplish the same
thing.  Something must come before the first |\a|, otherwise
|\pex| will think there is no preamble.

Another way to handle situations like this is to use |\ex| and
insert explicit labels.  Something like the following works well.


\codedisplay
\ex \smallskip \leftskip=1em
a.\enspace This is a very long example that will not fit in one
line and cannot be broken for some reason.
\smallskip
b.\enspace A short line might be more interesting linguistically,
but less interesting typographically.
\xe |endcodedisplay

\framedisplay~
\ex \smallskip \leftskip=1em
a.\enspace
A long example that will not fit in one line and
should not be broken if that is at all possible.
\smallskip
b.\enspace A short line might be more interesting linguistically,
but less interesting typographically.
\xe
\endframedisplay

\noindent If you can think up a title, so that the top line
contains more than the example number, the appearance is
improved.

Anchoring the label offset at the left edge of the example number
(setting |labelanchor| to |numleft|) is unlikely to be useful.
The possibility is included mostly for the sake of completeness.

\exbreak[25ex]
\subsection User designed labeling

\begingroup \exbreak \parindent=0pt
Macros:\quad \idx{|\definelabeltype|}, \idx{|\a[]|}\par\nobreak
Counter:\quad \idx{|\pexcnt|}\par\nobreak
\parinventory
& \idx{|labelgen|}& \idx{|char|}, \idx{|number|}, or \idx{|list|}&
   |char|\cr
& \idx{|pexcnt|}& integer& |97|\cr
& \idx{|labellist|}& comma separated list& |{}|\cr
\endparinventory
\endgroup

\noindent
If |labeltype| is set to |alpha|, the counter |\pexcnt| is set to
97, the character code of lowercase a in standard roman font
sets, and |labelgen| is set to |char|.  The successive labels are
generated by taking the character corresponding to |\pexcnt| and
stepping the counter by 1.  This relies on the fact that the
alphabetical sequence of characters corresponds to the numerical
order of the character codes of the characters.  Setting
|labeltype| to |caps| is almost the same, except that |\pexcnt|
is initialized to 65, the character code of uppercase A.  If
|labeltype| is set to |numeric|, |labelgen| is set to |number|
and |\pexcnt| is initialized to 1.  The labels are generated by
taking the number corresponding to value of |\pexcnt|.

\let\mit=\twelvei
Something like the following might be useful.  In the format I am
using, |\mit| selects a math italics font which has the lowercase
greek letters starting with the position 11.

\beginss
\pex[labelgen=char,pexcnt=11,
   everylabel=\mit]
\a
\a
\a
\a
\xe|midss
\pex[labelgen=char,pexcnt=11,
   everylabel=\mit]
\a
\a
\a
\a
\xe
\endss
\noindent This scheme has a quirk if labels above $\xi$ are
needed because the correspondence between numerical order and
alphabetical order breaks down at this point.  {\it o} will be
missing, with $\pi$ following $\xi$.

This labeling scheme can be defined by:
\medskip
\noindent
|\definelabeltype{greekmath}{labelgen=char,pexcnt=11,everylabel=\mit,|\par
\noindent
|   labelformat=A.}|
\medskip
\noindent Then |\pex[labeltype=greekmath]| is sufficient to
invoke this labeling style.

If a sequence of character labels is needed which does not appear
in sequence in a font, it is necessary to generate the labels
from a list.  Documents in Greek, for example, face the following
problem.  Roman letters were largely borrowed from the Greek
alphabet, but Greek alphabetical order was not.  Common greek
fonts place letters in the position of the borrowing, not in
their natural order in the Greek alphabet.  Documents written in
Greek, therefore, will have to generate the labels in multipart
examples from a list. The solution is to set |labelgen| to
|list|, and |labellist| to the desired list of labels.

For example, suppose the label type |greek| is defined by
\medskip
\noindent |\definelabeltype{greek}{labelgen=list,|\par
\noindent |   labellist={a,b,g,d,e,z,h,j,i,k,l,m,n,x,o,p,r,sv,t,u,f,q,y,w}}|
\medskip
\noindent and that |\gr| selects one of the grmn series of fonts
in the cb family of Greek fonts.

\begingroup
\gr

\definelabeltype{greek}{labelgen=list,
   labellist={a,b,g,d,e,z,h,j,i,k,l,m,n,x,o,p,r,sv,t,u,f,q,y,w}}
\lingset{labeltype=greek}

\beginss
\gr
\pex[labeltype=greek]
\a {\rm a,b,g,d,e,z}
\a a,b,g,d,e,z
\a {\rm a,b,c,d,e,f}
\a a,b,c,d,e,f
\a {\rm A,B,G,D,E,Z}
\a A,B,G,D,E,Z
\xe|midss
\pex[labeltype=greek]
\a {\rm a,b,g,d,e,z}
\a a,b,g,d,e,z
\a {\rm a,b,c,d,e,f}
\a a,b,c,d,e,f
\a {\rm A,B,G,D,E,Z}
\a A,B,G,D,E,Z
\xe
\endss
\endgroup

\noindent ``sv'' appears in the list rather than ``v'' so that a
nonfinal sigma is produced rather than a final sigma.  They
differ.  ``v'' produces what amounts to a vertical strut of zero
width, making the sigma nonfinal, but contributing no visible
material. (Thanks to Christos Vlachos for this idea.)

Of course, if the entire document is written in Greek, then
\medskip
\noindent |\lingset{labeltype=greek}|
\medskip
\noindent should have global scope, so that parameter
setting is not necessary in each |\pex| construction.

The computation is faster if the label list is shorter, so
the label list should not be much longer than the longest list
you anticipate using.  If more labels are called for than the
list provides, labeling starts over again at the beginning and a
warning is issued.

\subsection Stipulated labels

|\a| takes an optional argument, which is inserted as the label,
ignoring automatic label generation.  This can be useful if only
some parts of a multipart example are being repeated.  See
for example (\getref{explicit}) in Section~\getref{explicit-sec}.

\beginss
\pex
\pex[exno={7, partially repeated},
   labelanchor=margin,
   labeloffset=1.5em]\par
\a[label=b]
\a[label=d]
\a[label=g]
\xe|midss
\pex[exno={7, partially repeated},labelanchor=margin,
labeloffset=1.5em]\par
\a[label=b]
\a[label=d]
\a[label=g]
\xe
\deftagpage{partialrepeat}
\endss

%\definelingstyle{UBC}{%
%   numlabelclash=false,
%   textanchor=numleft,textoffset=50pt,
%   labelanchor=numleft,labeloffset=25pt,
%   everylabel=,
%   preambleanchor=text,preambleoffset=0pt,
%   appendtopexarg=
%}
\subsection Tabular format in multiline examples

\parinventory
& \idx{|textanchor|}&  choice\enspace
   (|numleft|, |numright|, or |labelleft|)& |numright|\cr
& \idx{|preambleanchor|}& choice\enspace
   (|numright|, |labelleft|, or |text|)& |numright|\cr
\endparinventory

Some publications specify that the text and label offset are
measured from the left margin.  ExPex 3.3 measures the label
offset from the right edge of the example number and the
textoffset from the right edge of the ``label gutter'' (the slot
which labels are typeset in). ExPex already had a parameter
|labelanchor| (with values |margin|, |numleft|, and |numright|)
which determined how the value of |labeloffset| was used to set
the position of the labels. A parameter has |textanchor| been
introduced, with obvious meaning.  In multipart examples in ExPex
3.3, the position of a preamble in multipart examples was
measured with respect to the right edge of the typeset example
number.  The preamble is the stuff (if any) which appears before
the first labeled part of a multipart example.  ExPex introduces
the parameter |preambleanchor| (with possible values |numright|,
|numleft|, and |text|), with obvious meaning.

For example,
\begingroup
\codedisplay
\lingset{%
   textanchor=numleft,textoffset=50pt,
   labelanchor=numleft,labeloffset=25pt,
   preambleanchor=text,preambleoffset=0pt,
}
\pex[exno=137]
\a first
\a second
\xe

\pex~[labelwidth=2in]
Preamble
\a first
\a[label=bb] second
\xe
|endcodedisplay

\noindent produces

\lingset{%
   textanchor=numleft,textoffset=5em,
   labelanchor=numleft,labeloffset=2.5em,
   preambleanchor=text,preambleoffset=0pt,
}
\pex[exno=137]
\a first
\a second
\xe

\pex~[labelwidth=2in]
Preamble
\a first
\a[label=bb] second
\xe
The position of the labels and text is unaffected by the
width of the typeset example numer, the specified labelwidth, or
the actual label width.
\endgroup

\subsection IJAL style format of multiline examples

\definelingstyle{IJAL}{%
   numlabelclash=true,
   labelanchor=numleft,labeloffset=0pt,
   textanchor=normal,textoffset=1.5em,
   preambleanchor=text,preambleoffset=0pt,
   labelformat=(A),
   everylabel=\actualexno,
   appendtopexarg={samplelabel=(\actualexno a)}
   }

\parinventory
& \idx{|appendtopexarg|}&  token list& |{}|\cr
& \idx{|numlabelclash|}& boolean (|true| or |false|)& |false|\cr
\endparinventory

\noindent The formatting demands of the International Journal of
American Linguistics (IJAL) poses unsurmountable problems for
ExPex 3.3. Examples of the desired style are given below.

\begingroup
\lingset{lingstyle=IJAL}
\pex<nonum>
\a first
\a second
\xe

\pex~[exno=65]
Preamble
\a first
\a second
\xe

\pex~[exno=1026]
Preamble
\a first
\a second
\xe

Clearly, we must set |labelanchor| to |numleft| and |labeloffset|
to |0pt| (or |0in|, or \dots).

The first issue is that the labelwidth depends
on the example number.
ExPex can deal with this by |\pex[samplelabel=(\actualexno
a)]|.  But it is considerably less than elegant to
have to write this in every |\pex| example.  It will not produce
the desired result to put |\lingset{samplelabel=(\actualexno a)}|
at the beginning of the document.  |labelwidth| will be set to
whatever the current width of |(\actualexno a)| is, which is
not likely to be what you want for the entire document even if
|\actualexno| happens to be defined (which is unlikely) at the
point the |\lingset| is executed.  To circumvent the global/local
problem, ExPex now provides the parameter |appendtopexarg|.  Its
value is appended to whatever other arguments are given to
|\pex| and evaluated locally.  So\smallskip

|\lingset{appendtopexarg=(labelwidth=\actualexno a)}|\smallskip
\noindent accomplishes what we
want.

There is another issue.  The example number does not appear in
(\getref{nonum}).  ExPex now provides the boolean (true or false
values only) parameter |numlabelclash|.  If it is set to |true|,
the example number is not typeset in |\pex| examples with no
preamble.

The code used for the IJAL examples above was:
\codedisplay
\definelingstyle{IJAL}{%
   numlabelclash=true,
   labelanchor=numleft,labeloffset=0pt,
   preambleanchor=text,preambleoffset=0pt,
   labelformat=(A),
   everylabel=\actualexno,
   appendtopexarg={samplelabel=(\actualexno a)}
   }

\pex
\a first
\a second
\xe

\pex~[exno=65]
Preamble
\a first
\a second
\xe

\pex~[exno=1026]
Preamble
\a first
\a second
\xe
|endcodedisplay

\endgroup

\subsection Multiline preambles

Macro: |\multilinepreamble|
\bigskip

\def\\{\noindent$\bullet$\enspace\ignorespaces}

\noindent Preambles were not properly formatted in ExPex 3.3 if they
extended over more than one line.  For example:

\codedisplay
\pex[preambleoffset=1.5ex,preambleanchor=numright]
This paper must be at least 10 pages long,
so it is useful to this end to write long wordy introductions to
all examples.  Please note that the first example comes before
the second example.  It could be done otherwise, but this
convention is adopted for the benefit of the reader.
\a first
\a second
\xe
|endcodedisplay
gives
\framedisplay
\pex[preambleoffset=1.5ex,preambleanchor=numright]
This paper must be at least 10 pages long,
so it is useful to this end to write long wordy introductions to
all examples.  Please note that the first example comes before
the second example.  It could be done otherwise, but this
convention is adopted for the benefit of the reader.
\a first
\a second
\xe
\endframedisplay

The same problem extends to Expex 3.4, except in the case that
|preambleanchor=text| and |preambleoffset=0pt| (and other
accidental situations which turn out to be equivalent).  The use
of |\multilinepreamble| circumvents the problem by putting the
preamble into an appropriate vbox.

\codedisplay
\pex[preambleoffset=0pt,preambleanchor=labelleft]
\multilinepreamble{This paper must be at least 10 pages long,
so it is useful to this end to write long wordy introductions to
all examples.  Please note that the first example comes before
the second example.  It could be done otherwise, but this
convention is adopted for the benefit of the reader.}
\a first
\a second
\xe
|endcodedisplay
produces
\framedisplay
\pex[preambleoffset=0pt,preambleanchor=labelleft]
\multilinepreamble{This paper must be at least 10 pages long,
so it is useful to this end to write long wordy introductions to
all examples.  Please note that the first example comes before
the second example.  It could be done otherwise, but this
convention is adopted for the benefit of the reader.}
\a first
\a second
\xe
\endframedisplay


