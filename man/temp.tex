
\lingset{everygla=,extraglskip=0pt}
\def\X{x{\vrule height6pt depth2pt}}
\def\A{x{\vrule height6pt depth2pt}}
\def\Z{.\vrule width2pt height8.5pt depth3.5pt .}




\hsize=2in
\lingset{everyglilg={\baselineskip=20pt}}
\lingset{belowglpreambleskip=-8pt}
\lingset{glhangstyle=none}
\lingset{extraglskip=8pt}
\lingset{autoglskip=true}
\lingset{belowglpreambleskip=0pt}
\lingset{aboveglftskip=0pt}

\ex[glstyle=wrap]
\begingl
\glpreamble preamble preamble preamble preamble //
\gla
{\X} {\X} {\X} {\X} {\X} {\X} {\X} {\X} {\X} {\X} {\X} {\X} {\X} {\X}
{\X} {\X} {\X} {\X}
//
\glb
{\A} {\A} {\A} {\A} {\A} {\A} {\A} {\A} {\A} {\A} {\A} {\A} {\A} {\A}
{\A} {\A} {\A} {\A}
//
\glft free
free free free free free free free free free free
//
\endgl
\xe

\endinput
\parindent=0pt

\begingl[glstyle=nlevel]
\glpreamble preamble preamble preamble preamble \endpreamble
\X[\A] \X[\A] \X[\A] \X[\A] \X[\A] \X[\A] \X[\A] \X[\A] \X[\A] \X[\A]
\X[\A] \X[\A] \X[\A] \X[\A] \X[\A] \X[\A] \X[\A] \X[\A] \X[\A]
\glft
free free free free free free free free free free
\endgl

\bigskip

\setbox0=\vbox{\begingl[glstyle=nlevel]
\glpreamble preamble\endpreamble
\X[\A]
\endgl}

%\showbox0
\copy0

\endinput



\makeatletter
\define@key{ling}{glrightskip}[]{%
   \skip0=0pt plus .1\hsize
   \def\temp{#1}\ifx\temp\empty \else \skip0=#1\fi
   \edef\ling@glrightskip{\the\skip0}}
\lingset{glrightskip=100pt plus 2em}
%\writeln{++\ling@glrightskip}
%\hsize=100pt
%\lingset{glrightskip}
%\writeln{++\ling@glrightskip}

\define@cmdkeys{ling}[ling@]{glrightskip}
\lingset{glrightskip=0pt plus .1\hsize}
\hsize=100pt
\skip0=\ling@glrightskip
\writeln{+++\the\skip0}

\resetatcatcode


\endinput

It is easiest to understand what the parameters do by examining in
some detail how a gloss display is constructed.  Consider (\nextx),
for example, in which box outlines have been added to facilitate
discussion.

\framedisplay
\boxglwords
\ex[glspace=1.2em,extraglskip=.5ex,glhangindent=1.5em]
\hsize=4in
\begingl
\glpreamble Mary ist sicher, dass es den Hans nicht st\"oren w\"urde
seiner Freundin ihr Herz auszusch\"utten.//
\gla Mary$_i$ ist sicher, dass es den Hans nicht st\"oren w\"urde
seiner Freundin ihr$_i$ Herz auszusch\"utten.//
\glb Mary is sure that it the-{\sc acc} Hans not annoy would
his-{\sc dat} girlfriend-{\sc dat} her-{\sc acc} heart-{\sc acc} {out to
throw}//
\glft  `Mary is sure that it would not annoy John to reveal her
heart to his girlfriend.'//
\endgl
\xe
\endframedisplay
\codedisplay~
\ex[glspace=1.2em,extraglskip=.5ex,glhangindent=1.5em]
\hsize=4in
\begingl
\glpreamble Mary ist sicher, dass es den Hans nicht st\"oren w\"urde
seiner Freundin ihr Herz auszusch\"utten.//
\gla Mary$_i$ ist sicher, dass es den Hans nicht st\"oren w\"urde
seiner Freundin ihr$_i$ Herz auszusch\"utten.//
\glb Mary is sure that it the-{\sc acc} Hans not annoy would
his-{\sc dat} girlfriend-{\sc dat} her-{\sc acc} heart-{\sc acc} {out to
throw}//
\glft  `Mary is sure that it would not annoy John to reveal her
heart to his girlfriend.'//
\endgl
\xe
|endcodedisplay

\def\parvalue#1{{\tt \char"7C #1\char"7C }}


\noindent 1. {\it Formating the preamble}\enspace The preamble is set
as ordinary running text, using whatever \Tex\ settings hold in the
environment, with the tokens \parvalue{everyglpreamble} inserted at
the beginning.  This is grouped so that any changes made in the
preamble do not leak out.  This is followed by inserting vertical skip
determined by \parvalue{afterpreambleskip}.

\smallskip

\noindent 2. {\it Formating the interlinear gloss}: \enspace All of
the items in the various lines in the interlinear gloss are
accumulated before the boxes which make up the interlinear gloss are
typeset.  These vboxes, outlined in (\lastx), are called glwords.
These boxes are fed to \Tex's standard paragraph building machinery.
There are two separate issues; parameters which affect how the boxes
are built and parameters that affect paragraph building.

\smallskip

\begingroup
\leftskip 1.5em
\parindent=0pt
{\it Formating the glwords}:\enspace \parvalue{everyglword} is
inserted at the start.  Each line of the glword starts with a strut
and an appropriate hook; \parvalue{everygla} for a gla line, etc.
Vertical skip \parvalue{aboveglbskip} is inserted above the glb line,
and so forth.

\smallskip

{\it Forming the glwords into a paragraph}:\enspace Horizontal skip
\parvalue{glskip} is inserted between the glwords and vertical skip
\parvalue{extraglskip} is inserted between the lines of glwords.   The
rightskip is set to \parvalue{glrightskip}.  Hanging indentation is
determined by \parvalue{glhangstyle} and \parvalue{glhangindent}.  The
interlinear gloss is grouped so that \Tex\ parameter changes are kept
local.

\endgroup
\smallskip
\noindent 3. {\it Formating the free translation}:\enspace Finally,
the free translation is typeset as standard text, separated by
vertical skip \parvalue{aboveglftskip} from the interlinear skip above
it.  \parvalue{everyglft} is inserted at the start of processing the
free translation.

\resetatcatcode

\endinput


\hsize=5in

\immediate\openout0=tempfile
\newtoks\stoks
\obeylines
\stoks={\begingl
\glpreamble Mary ist sicher, dass es den Hans nicht st\"oren w\"urde
seiner Freundin ihr Herz auszusch\"utten.//
\gla Mary$_i$ ist sicher, dass es den Hans nicht st\"oren w\"urde
seiner Freundin ihr$_i$ Herz auszusch\"utten.//
\glb Mary is sure that it the-{\sc acc} Hans not annoy would
his-{\sc dat} girlfriend-{\sc dat} her-{\sc acc} heart-{\sc acc} {out to
throw}//
\glft  `Mary is sure that it would not annoy John to reveal her
heart to his girlfriend.'//
\endgl}%
\immediate\write0{\the\stoks}
\immediate\closeout0

\expandafter\codedisplay\input tempfile
|endcodedisplay








\endinput
\makeatletter

\newdimen\ep@cascadeindent
\newdimen\ep@hangindentamount
\newcount\ep@cascadecount

\ep@cascadecount=10
\ep@cascadeindent=0pt
\ep@hangindentamount=\ling@glhangindent

\parindent=0pt
\def\makeshape#1#2{{%
   \count0=#1
   \ep@dima=#2
   \ep@dimb=\hsize
   \advance\ep@dimb by -\ep@dima
   \edef\cascadeshape{}
   \loop\ifnum\count0 >0
      \edef\cascadeshape{\cascadeshape \the\ep@dima\space\the\ep@dimb\space}%
      \advance\ep@dima by \ling@glhangindent
      \advance\ep@dimb by -\ling@glhangindent
      \advance\count0 by -1
      \repeat
   \edef\next{#1\space\cascadeshape}%
   \xdef\next{\noexpand\expandafter\noexpand\parshape#1\space\cascadeshape}}%
   \next
   \ignorespaces
}
\def\@glhangcarry{%
   \par
   \edef\next{\the\prevgraf}%
   \advance\ep@cascadecount by -\next
   \advance\ep@cascadeindent by \next\ep@hangindentamount
   \makeshape{\ep@cascadecount}{\ep@cascadeindent}
}
\
%\begingroup
%\parindent=0pt
%\hsize=4in
%\makeshape{\ep@cascadecount}{\ep@cascadeindent}
%Now is the time for all good men to come to the aid of the party.
%Now is the time for all good men to come to the aid of the party.
%Now is the time for all good men to come to the aid of the party.
%\@glhangcarry
%Now is the time for all good men to come to the aid of the party.
%\@glhangcarry
%Now is the time for all good men to come to the aid of the party.
%\par
%\endgroup
%\endinput



%\ex[glstyle=nlevel]
%\begingl
%k-[CL/2] wapm[V/see] -a[AGR/\sc 3acc] -s'i[NEG]
%-m[AGR/\sc 2pl] -wapunin[TNS/preterit] -uk[AGR/\sc 3pl]
%\glft `you (pl) didn't see them'
%\endgl
%\xe

\makeatletter
\newtoks\mytoks
\def\@showtoks{\@getoptionalarg\@showtoks@i}
\def\@showtoks@i #1{\mytoks=\expandafter{#1}\writeln{%
   \@optionalarg\the\mytoks}}
\let\showtoks=\@showtoks

%When autoglskip is set to false, extraglskip is irrelevant.  The skip
%between lines of the glwords is determined by gllineskip.
%?? when autoglskip=true, it shouldn't make any difference whether
%?? glstruts is true or false.

\ex[everygla=,glwidth=2in,autoglskip=true,glstruts=true,extraglskip=0pt,
   gllineskip=0pt]
\begingl
\gla
LLLL LLLL LLLL
LLLL LLLL LLLL
//
\glb
LLLL LLLL LLLL
LLLL LLLL LLLL
//
\endgl
\xe

\endinput

\ex
\hsize=5in
\begingl[everyglc=\tenrm,aboveglcskip=-2pt,glspace=!.5em,
   extraglskip=1ex,aboveglftskip=0pt]
\gla Mary$_i$ ist sicher, dass es den Hans nicht st\"oren w\"urde
seiner Freundin ihr$_i$ Herz auszusch\"utten.//
\glb Mary is sure that it the Hans not annoy would
his girlfriend her heart {out to throw}//
\glc {} {} {} {} {} DAT {} {} {} {} DAT DAT ACC ACC //
\glft  `Mary is sure that it would not annoy John to reveal her
heart to his girlfriend.'//
\endgl
\xe



\ex[glstyle=nlevel,glhangstyle=normal,glneveryline={\it,,\footnotesize},
   glnabovelineextraskip={,,-3pt}]
\begingl
Mary$_i$[Mary]
ist[is]
sicher,[sure]
dass[that]
es[it]
den[the/ACC]
Hans[Hans]
nicht[not]
st\"oren[annoy]
w\"urde[would]
seiner[his/DAT]
Freundin[girlfriend/DAT]
ihr$_i$[her/ACC]
Herz[heart/ACC]
auszusch\"utten[out to throw]
\endgl
\xe

\ex[glstyle=nlevel,glhangstyle=normal,glneveryline={\it,,\footnotesize},
   glnabovelineextraskip={,,-3pt},glwordalign=center]
\begingl
Mary$_i$[Mary]
ist[is]
sicher,[sure]
dass[that]
es[it]
den[the/ACC]
Hans[Hans]
nicht[not]
st\"oren[annoy]
w\"urde[would]
seiner[his/DAT]
Freundin[girlfriend/DAT]
ihr$_i$[her/ACC]
Herz[heart/ACC]
auszusch\"utten[out to throw]
\endgl
\xe

\ex[glhangstyle=normal,glufcloseup=.4ex,everygluf=\footnotesize]
\begingl
\gla Mary$_i$ ist sicher, dass es den Hans nicht st\"oren
   w\"urde seiner Freundin ihr$_i$ Herz auszusch\"utten.//
\glb Mary is sure that it \gluf/the/ACC/ Hans not annoy would
   \gluf/his/DAT/ \gluf/girlfriend/DAT/ \gluf/her/ACC/
   \gluf/heart/ACC/ {out to throw}//
\glft `Mary is sure that to reveal her heart to his girlfriend
would not damage John.'//
\endgl
\xe

\endinput











`Mary is sure that to reveal her heart to his girlfriend
would not damage John.'//
\endgl
\xe

\endinput

%\makeatletter
%
%
%\def\AccentedBarredW{$\acute{\hbox{$\overline w$}}$}
%
%\framedisplay
%\ex
%\begingl[glstyle=nlevel,glneveryline={,\varstrut{4pt}}]
%m-[(mo-)] wope[(a\AccentedBarredW ope)] \endgl \xe
%\endframedisplay
%
%
%\ex[glstyle=nlevel,glneveryline={\it}]
%\begingl
%\glpreamble Independent order verb structure with its three agreement
%suffixes, labeled Agr$_1$, Agr$_2$, and Agr$_3$.
%\glilg k-[CL/2] wapm[V/see] -a[Agr$_1$/\sc 3acc] -s'i[NEG]
%-m[Agr$_2$/\sc 2pl] -wapunin[TNS/preterit] -uk[Agr$_3$/\sc 3pl]
%\glft `you (pl) didn't see them'
%\endgl
%\xe
%
%\endinput

\ex[belowpreambleskip=0pt,aboveglftskip=0pt,aboveglbskip=0pt,
   aboveglcskip=0pt,everygla=,everyglb=,everyglc=,
   everyglft=]<wapm>
\begingl
\glpreamble (Independent order verb structure with its three
agreement suffixes, labeled {\sc AGR$_1$}, {\sc AGR$_2$}, and {\sc
AGR$_3$}.)//
\gla k- wapm -a -s'i -m -wapunin -uk //
\glb CL V AGR$_1$ NEG AGR$_2$ TNS AGR$_3$ //
\glb 2 see {\sc 3acc} {} {\sc 2pl} preterit {\sc 3pl} //
\glft `you (pl) didn't see them'//
\endgl
\xe


\ex[glstyle=nlevel]
\begingl
k-[CL/2] wapm[V/see] -a[AGR/\sc 3acc] -s'i[NEG]
-m[AGR/\sc 2pl] -wapunin[TNS/preterit] -uk[AGR/\sc 3pl]
\glft `you (pl) didn't see them'
\endgl
\xe

\endinput

\ex[glstyle=nlevel,belowpreambleskip=0pt,aboveglftskip=0pt,
   aboveglbskip=0pt,   aboveglcskip=0pt,everygla=,everyglb=,
   everyglc=,everyglft=]<gln-wapm>
\begingl
\glpreamble
(Independent order verb structure with its three
agreement suffixes, labeled {\sc AGR$_1$}, {\sc AGR$_2$}, and {\sc
AGR$_3$}.)
\glilg
k-[CL/2] wapm[V/see] -a[AGR/\sc 3acc] -s'i[NEG]
-m[AGR/\sc 2pl] -wapunin[TNS/preterit] -uk[AGR/\sc 3pl]
\glft
`you (pl) didn't see them'
\endgl
\xe

\endinput


\makeatletter
\let\mergerow=\glw@mergerow
\resetatcatcode

\ex[everygla=]
\begingl
\gla a1 {\hfil b1} //
\glb a2 bbbb2 //
\showtoks\mainlist
\endgl\xe



\endinput





\newcount\@itemtype
% 1=+, 2=@, 3=[, 4=], 5=\nogloss, 0=other
\def\@getitemtype #1{\expandafter\@getitemtype@i #1\@nil}
\def\@getitemtype@i #1#2\@nil{%
   \def\temp{#2}%
   \ifx\temp\empty
      \if\ep@samecharcode#1+\relax
         \@itemtype=1
      \else\if\ep@samecharcode#1@\relax
         \@itemtype=2
      \else\if\ep@samecharcode #1[\relax
         \@itemtype=3
      \else\if\ep@samecharcode #1]
         \@itemtype=4
      \else \@itemtype=0
         \fi\fi\fi\fi
   \else
      \ifx#1\nogloss \@itemtype=5
      \else \@itemtype=0 \fi
   \fi
}

\lingset{glhangindent=0pt,glstyle=wrap,glrightskip=0pt plus 1in,
   extraglskip=1ex,everygla=}

%\def\nogloss#1{#1}

\ex\begingl
\gla a1 b1 @ r r//
\glb a2 b2 e e e//
\endgl\xe

\endinput

\ex[glwidth=2.4in]
a.\quad
\begingl
\gla
aaa bbb ccc
aaa bbb ccc
aaa bbb ccc
aaa bbb ccc
aaa bbb ccc
aaa bbb ccc//
\glb
aaa bbb ccc
aaa bbb ccc
aaa bbb ccc
aaa bbb ccc
aaa bbb ccc
aaa bbb ccc//
\endgl
\hfil
b.\quad
\begingl
\gla
aaa bbb ccc
aaa bbb ccc
aaa bbb ccc
aaa bbb ccc
aaa bbb ccc
aaa bbb ccc//
\glb
aaa bbb ccc
aaa bbb ccc
aaa bbb ccc
aaa bbb ccc
aaa bbb ccc
aaa bbb ccc//
\endgl
\xe

\ex[glwidth=2.4in,glhangstyle=cascade,glhangindent=1em,extraglskip=0pt]
%\ex[glwidth=2.4in,glhangstyle=normal,glhangindent=1em,extraglskip=0pt]
%\ex[glwidth=2.4in,glhangstyle=none,glhangindent=1em]
a.\quad
\begingl
\glpreamble this is one way of doing it, although opinions on this
may differ//
\gla
aaa bbb ccc
aaa bbb ccc
aaa bbb ccc
aaa bbb ccc
aaa bbb ccc
aaa bbb ccc//
\glb
aaa bbb ccc
aaa bbb ccc
aaa bbb ccc
aaa bbb ccc
aaa bbb ccc
aaa bbb ccc//
\glft this is one way of doing it, although opinions on this
may differ//
\endgl
\hfil
b.\quad
\begingl
\gla
aaa bbb ccc
aaa bbb ccc
aaa bbb ccc
aaa bbb ccc
aaa bbb ccc
aaa bbb ccc//
\glb
aaa bbb ccc
aaa bbb ccc
aaa bbb ccc
aaa bbb ccc
aaa bbb ccc
aaa bbb ccc//
\endgl
\xe

\hsize=2in
\ex[glwidth=0pt,glhangstyle=cascade,glhangindent=1em]
\begingl
\gla
aaa bbb ccc
aaa bbb ccc
aaa bbb ccc
aaa bbb ccc
aaa bbb ccc
aaa bbb ccc//
\glb
aaa bbb ccc
aaa bbb ccc
aaa bbb ccc
aaa bbb ccc
aaa bbb ccc
aaa bbb ccc//
\endgl
\xe

\endinput

\ex
\begingl[glstyle=wrap,glwidth=0pt,glbreaking=false]
\gla
aaa bbb ccc
aaa bbb ccc
aaa bbb ccc
aaa bbb ccc
aaa bbb ccc
aaa bbb ccc
aaa bbb ccc
aaa bbb ccc
aaa bbb ccc
aaa bbb ccc
//
\glb
aaa bbb ccc
aaa bbb ccc
aaa bbb ccc
aaa bbb ccc
aaa bbb ccc
aaa bbb ccc
aaa bbb ccc
aaa bbb ccc
aaa bbb ccc
aaa bbb ccc
//
\endgl
\xe
