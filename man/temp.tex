

\ex
\vtop{\hsize=.6\hsize \leftskip=0pt
\begingl
\gla
Hmao sa co po tha  nu nao nga hmua. Nu dja ga, nu dja cog nu laih
gui reo ne. Todang$^2$ bboi rok jolan nu nao hma, nu bboh sa droi mra
do bboi gah, a, hruh nu.//
\glb
exist one {clf}$^1$ person old 3s go do field 3s hold machete 3s hold
hoe 3s pst carry.on.back back.basket 3s while at {along} trail 3s
go field 3s see one clf peacock stay at drct {?} nest 3s//
\endgl
}\hfil
\vtop{\hsize=.25\hsize \rightskip=0pt plus 4em \leftskip=0pt
\footnotesize
1.\kern.5em clf here denotes something.  The reader can determine what by
consulting any good grammar.
\smallskip
2.\kern.5em This should also be noted.  There is a long history of
comments on words like this, particularly among scholars writing
their dissertations.
}
\bigskip
`There was an old person who went to work in the field. He took
along his machete, he took along his hoe, and he carried his
basket on his back. While he was on his way to the farm, he saw a
peacock beside its nest.'
\xe

\endinput
















\ex
\begingl
\gla Mary$_i$ ist sicher, dass es den Hans nicht st\"oren
w\"urde seiner Freundin ihr$_i$ Herz auszusch\"utten.//
\glb Mary is sure that it the-ACC Hans not annoy would
his-DAT girlfriend-DAT her-ACC heart-ACC {out to throw}//
\glft  `Mary is sure that it would not annoy John to reveal her
heart to his girlfriend.'//
\endgl
\xe
\endinput



\begingroup
\lingset{exskip=2pt,exnotype=roman,textanchor=numleft,
   textoffset=5em}
\ex they expected each other to get Bill to like them\xe
\ex they expected each other to like each other\xe
\ex they expected each other to hurt themselves\xe
\endgroup
\bigskip
\pex[interpartskip=4pt,exnotype=roman,labelformat=(A),
   textoffset=4em]
\a they expected each other to get Bill to like them
\a they expected each other to like each other
\a they expected each other to hurt themselves
\xe

\endinput
they expected each other to get Bill to like them
they expected each other to like each other
they expected each other to hurt themselves


\lingset{exnoformat=[X],exno=43}

\excnt=45
\ex[exno=22] hello\xe

\endinput


\pex[textanchor=numleft,
   labelanchor=numleft,
   labeloffset=.35in,
   textoffset=.7in]
\a first
\a[label=aa] second
\xe

\lingset{textanchor=numleft,
   labelanchor=numleft,
   labeloffset=.35in,
   textoffset=.7in}
\a first
\a[label=aa] second
\xe


\lingset{sampleexno=(ix)}
\excnt=9
\lingset{exnotype=roman}

\ex one\xe
\ex two\xe
The existential reading does not seem to be available for subjects of
small clause complements of {\it consider\/}:
\pex
\a I consider firemen available. (generic only)
\a I consider firemen intelligent. (generic only)
\xe
Exceptional case marking (ECM) verbs seem more or less to allow both
existential and generic interpretations of complement subjects:
\pex
\a I believe firemen to be available. (both generic and existential)
\a I believe violists to be intelligent. (generic only)
%\a \ljudge{??}I believe opera singers to know Hittite.
\xe



\endinput




\ex[glhangstyle=cascade]
\def\\#1{{\small\uppercase{#1}}}%
\begingl
\gla
Hom\^{a}o sa \v{c}\^{o} p\^{o} tha  \~{n}u nao ng\u{a} hmua. \~{N}u
dj\u{a} g\u{a}, \~{n}u dj\u{a} \v{c}\u{o}ng \~{n}u, laih gui r\^{e}o
\~{n}u. Todang bboi r\^{o}k jolan \~{n}u nao hma, \~{n}u bb\^{o}h sa
droi mr\u{a} d\u{o} bboi gah, a, hruh \~{n}u.//
\glb
\\{exist} one \\{clf} person old \\{3s} go do field \\{3s} hold
machete \\{3s} hold hoe \\{3s} and carry.on.back back.basket \\{3s}
while at along trail \\{3s} go field \\{3s} see one \\{clf} peacock
stay at \\{drct} -- nest \\{3s}//
\glft
`There was an old person who went to work in the field. He took
along his machete, he took along his hoe, and he carried his
basket on his back. While he was on his way to the farm, he saw a
peacock beside its nest.'//
\endgl
\xe


\endinput

Hom\^{a}o sa \v{c}\^{o} p\^{o} tha  \~{n}u nao ng\u{a} hmua. \~{N}u
dj\u{a} g\u{a}, \~{n}u dj\u{a} \v{c}\u{o}ng \~{n}u, laih gui r\^{e}o
\~{n}u. Todang bboi r\^{o}k jolan \~{n}u nao hma, \~{n}u bb\^{o}h sa
droi mr\u{a} d\u{o} bboi gah, a, hruh \~{n}u.//

{\sc exist} one {\sc clf} person old {\sc 3s} go do field {\sc 3s} hold
machete {\sc 3s} hold hoe {\sc 3s} and carry.on.back back.basket \sc 3s
while at along trail {\sc 3s} go field {\sc 3s} see one {\sc clf} peacock
stay at {\sc drct} -- nest {\sc 3s} //

\secno=6

\section section

\subsection subsection

\subsection subsection\deftagsec{subsection}
\deftag{\currsec}{AA}
\writeln{\meaning\currsec}
\writeln{currsec: \currsec}

\subsection subsection

\getref{subsection}

%\makeatletter
%\writeln{\lingtag@subsection}
%\writeln{\lingtag@AA}
\endinput


\ex
\begingl
\glpreamble ggg ggg//
%\gla
%xxx xx yyy zzz//
%\glb
%xx ppp zzz//
\glft This is a free translation.//
\endgl
\xe



