\definelingstyle{IJAL}{%
   numlabelclash=true,
   labelanchor=numleft,labeloffset=0pt,
   textanchor=normal,textoffset=1.5em,
   preambleanchor=text,preambleoffset=0pt,
   labelformat=(A),
   everylabel=\actualexno,
   appendtopexarg={samplelabel=(\actualexno a)}
   }
\definelingstyle{UBC}{%
   numlabelclash=false,
   textanchor=numleft,textoffset=50pt,
   labelanchor=numleft,labeloffset=25pt,
   everylabel=,
   preambleanchor=text,preambleoffset=0pt,
   appendtopexarg=
}
\definelingstyle{frampton}{%
   numlabelclash=false,
   textanchor=normal,textoffset=1.4em,
   labelanchor=numright,labeloffset=1.4em,
   everylabel=,
   preambleanchor=labelleft,preambleoffset=0pt,
   appendtopexarg=,
}

\centerline{\titlefont{main} User's Manual, Supplement}
\hfil

\noindent This document explains the new features incorporated in
ExPex version 3.4.  The major additions were made in order to
make it possible to typeset examples in formats rather different
that ``standard ExPex''.  For example, the University of British
Columbia Working Papers specify that the text and label offset
are measured from the left margin.  ExPex 3.3 measures the label
offset from the right edge of the example number and the
textoffset from the right edge of the ``label gutter'' (the slot
which labels are typeset in). ExPex already had a parameter
|labelanchor| (with values |margin|, |numleft|, and |numright|)
which determined how the value of |labeloffset| was used to set
the position of the labels. A parameter has |textanchor| been
introduced, with obvious meaning.  In multipart examples in ExPex
3.3, the position of a preamble in multipart examples was
measured with respect to the right edge of the typeset example
number.  The preamble is the stuff (if any) which appears before
the first labeled part of a multipart example.  ExPex introduces
the parameter |preambleanchor| (with possible values |numright|,
|numleft|, and |text|), with obvious meaning.

For example,
\begingroup
\codedisplay
\lingset{%
   textanchor=numleft,textoffset=50pt,
   labelanchor=numleft,labeloffset=25pt,
   preambleanchor=text,preambleoffset=0pt,
}
\pex[exno=137]
\a first
\a second
\xe

\pex~[labelwidth=2in]
Preamble
\a first
\a[label=bb] second
\xe
|endcodedisplay

\noindent produces

\lingset{%
   textanchor=numleft,textoffset=5em,
   labelanchor=numleft,labeloffset=2.5em,
   preambleanchor=text,preambleoffset=0pt,
}
\pex[exno=137]
\a first
\a second
\xe

\pex~[labelwidth=2in]
Preamble
\a first
\a[label=bb] second
\xe
The position of the labels and text is unaffected by the
width of the typeset example numer, the specified labelwidth, or
the actual label width.
\endgroup

The formatted demanded by the International Journal of American
Linguistics (IJAL) poses even more problems for ExPex.  Examples
are given below.

\begingroup
\lingset{lingstyle=IJAL}
\pex<nonum>
\a first
\a second
\xe

\pex~[exno=65]
Preamble
\a first
\a second
\xe

\pex~[exno=1026]
Preamble
\a first
\a second
\xe

Clearly, we must set |labelanchor| to |numleft| and |labeloffset|
to |0pt| (or |0in|, or \dots).

The first issue is that the labelwidth depends
on the example number.
ExPex can deal with this by |\pex[samplelabel=(\actualexno
a)]|.  But it is considerably less than elegant to
have to write this in every |\pex| example.  It will not produce
the desired result to put |\lingset{samplelabel=(\actualexno a)}|
at the beginning of the document.  |labelwidth| will be set to
whatever the current width of |(\actualexno a)| is, which is
not likely to be what you want for the entire document even if
|\actualexno| happens to be defined (which is unlikely) at the
point the |\lingset| is executed.  To circumvent the global/local
problem, ExPex now provides the parameter |appendtopexarg|.  Its
value is appended to whatever other arguments are given to
|\pex| and evaluated locally.  So\smallskip

|\lingset{appendtopexarg=(labelwidth=\actualexno a)}|\smallskip
\noindent accomplishes what we
want.

There is another issue.  The example number does not appear in
(\getref{nonum}).  ExPex now provides the boolean (true or false
values only) parameter |numlabelclash|.  If it is set to |true|,
the example number is not typeset in |\pex| examples with no
preamble.

The code used for the IJAL examples above was:
\codedisplay
\definelingstyle{IJAL}{%
   numlabelclash=true,
   labelanchor=numleft,labeloffset=0pt,
   preambleanchor=text,preambleoffset=0pt,
   labelformat=(A),
   everylabel=\actualexno,
   appendtopexarg={samplelabel=(\actualexno a)}
   }

\pex
\a first
\a second
\xe

\pex~[exno=65]
Preamble
\a first
\a second
\xe

\pex~[exno=1026]
Preamble
\a first
\a second
\xe
|endcodedisplay

\endgroup

\subsection Other additions to version 3.4

\def\\{\noindent$\bullet$\enspace\ignorespaces}

\\ |\multilinepreamble|\smallskip

\noindent Preambles were not properly formatted in ExPex 3.3 if they
extended over more than one line.  For example:

\codedisplay
\pex[preambleoffset=1.5ex,preambleanchor=numright]
This paper must be at least 10 pages long,
so it is useful to this end to write long wordy introductions to
all examples.  Please note that the first example comes before
the second example.  It could be done otherwise, but this
convention is adopted for the benefit of the reader.
\a first
\a second
\xe
|endcodedisplay
gives
\framedisplay
\pex[preambleoffset=1.5ex,preambleanchor=numright]
This paper must be at least 10 pages long,
so it is useful to this end to write long wordy introductions to
all examples.  Please note that the first example comes before
the second example.  It could be done otherwise, but this
convention is adopted for the benefit of the reader.
\a first
\a second
\xe
\endframedisplay

The same problem extends to Expex 3.4, except in the case that
|preambleanchor=text| and |preambleoffset=0pt| (and other
accidental situations which turn out to be equivalent).  The use
of |\multilinepreamble| circumvents the problem by putting the
preamble into an appropriate vbox.

\codedisplay
\pex[preambleoffset=0pt,preambleanchor=labelleft]
\multilinepreamble{This paper must be at least 10 pages long,
so it is useful to this end to write long wordy introductions to
all examples.  Please note that the first example comes before
the second example.  It could be done otherwise, but this
convention is adopted for the benefit of the reader.}
\a first
\a second
\xe
|endcodedisplay
produces
\framedisplay
\pex[preambleoffset=0pt,preambleanchor=labelleft]
\multilinepreamble{This paper must be at least 10 pages long,
so it is useful to this end to write long wordy introductions to
all examples.  Please note that the first example comes before
the second example.  It could be done otherwise, but this
convention is adopted for the benefit of the reader.}
\a first
\a second
\xe
\endframedisplay


\bigskip
\\ |\rightcomment and \leftcomment|\medskip

You might want to produce a display like the following.

\framedisplay
\ex[glstyle=multilevel,textoffset=1in]
\begingl
\gla \rightcomment{\rm [Potawatami]}k- wapm -a -s'i -m -wapunin -uk //
\glb \leftcomment{\hfill\rm (negation)\quad}%
   Cl V Agr$_1$ Neg Agr$_2$ Tns Agr$_3$//
\glb 2 see {3\sc ACC} {} 2{\sc PL} preterit 3{\sc PL} //
\glft \leavevmode\rightcomment{(Hockett 1948, p. 143)}
   `you (pl) didn't see them'//
\endgl
\xe
\endframedisplay
The code is:
\codedisplay
\ex[glstyle=multilevel,textoffset=1.5in]
\begingl
\gla \rightcomment{\rm [Potawatami]}k- wapm -a -s'i -m -wapunin -uk //
\glb \leftcomment{\hfill\rm (negation)\qquad}%
   Cl V Agr$_1$ Neg Agr$_2$ Tns Agr$_3$//
\glb 2 see {3\sc ACC} {} 2{\sc PL} preterit 3{\sc PL} //
\glft \leavevmode\rightcomment{(Hockett 1948, p. 143)}
   `you (pl) didn't see them'//
\endgl
\xe
|endcodedisplay

These macros are very primitive, but sometimes very useful.  They
do not check to see that there is enough space for the material
they place at the margins.

\bigskip
\\ |\pushciteright|\medskip

\ex~[glstyle=multilevel]
\begingl
\gla xxxx xxxxx xxxxxx//
\glb aaa aa a//
\glft this is a very wordy free translation
of a very very very
very very very
simple sentence\hfill (Anonymous)//
\endgl
\xe

\lingset{glstyle=multilevel,mincitesep=1.5em}

\ex
\begingl
\gla xxxx xxxxx xxxxxx//
\glb aaa aa a//
\glft this is a very wordy free translation
of a simple sentence\pushciteright (Anonymous)//
\endgl
\xe

\ex~[mincitesep=3em]
\begingl
\gla xxxx xxxxx xxxxxx//
\glb aaa aa a//
\glft this is a very wordy free translation
of a simple sentence\pushciteright (Anonymous)//
\endgl
\xe

\ex~
\begingl
\gla xxxx xxxxx xxxxxx//
\glb aaa aa a//
\glft this is a very wordy free translation
of a very simple sentence\pushciteright (Anonymous)//
\endgl
\xe

\ex~
\begingl
\gla xxxx xxxxx xxxxxx//
\glb aaa aa a//
\glft this is a very wordy free translation
of a very very very
very very very
simple sentence\pushciteright (Anonymous)//
\endgl
\xe









\bye



\bullsec |lingstyle=frampton|
\lingset{lingstyle=frampton}


\pex[exno=65,preambleanchor=text,preambleoffset=0pt]
This paper must be at least 10 pages long,
so it is useful to this end to write long wordy introductions to
all examples.  Please note that the first example comes before
the second example.  It could be done otherwise, but this
convention is adopted for the benefit of the reader.
\a first
\a second
\xe

\pex[exno=65,preambleanchor=labelleft,preambleoffset=0pt]
This paper must be at least 10 pages long,
so it is useful to this end to write long wordy introductions to
all examples.  Please note that the first example comes before
the second example.  It could be done otherwise, but this
convention is adopted for the benefit of the reader.
\a first
\a second
\xe

\pex[exno=65,preambleanchor=labelleft,preambleoffset=0pt]
\multilinepreamble{This paper must be at least 10 pages long,
so it is useful to this end to write long wordy introductions to
all examples.  Please note that the first example comes before
the second example.  It could be done otherwise, but this
convention is adopted for the benefit of the reader.}
\a first
\a second
\xe

\bye

