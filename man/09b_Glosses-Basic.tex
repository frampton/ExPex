
\psset{framesep=1ex}
\def\minigloss#1#2{\outlinebox{%
   \vtop{\halign{##\hfil\cr #1\cr #2\cr}}}\hskip1pt}%
\def\miniglossbare#1#2{\vtop{\halign{##\hfil\cr
   #1\cr #2\cr}}}
\psset{linewidth=.4pt}
\lingset{glwidth=2.4in}
\def\outlinebox{\psframebox[framesep=0,boxsep=false,linewidth=.4pt]}


\subsection  Line spacing in wrapped glosses

\begininventory
\macros*
\idx{|\normalglwrapskips|}\endmc
\parameters
\idx{|glwrapskip|}& skip& \hfil $-$& (depends on the baselineskip)\cr
\idx{|abovemoreglskip|}& skip& \hfil |1ex|& (obsolete)\cr
\idx{|glstruts|}& boolean& \hfil |true|\cr
\endinventory
%
Wrapping is achieved by constructing the gloss as a sequence of
boxes and letting the Tex line building apparatus typeset that
sequence of boxes exactly as it typesets a sequence of words. The
gloss display (\nextx a), for example, is assembled by first
generating the sequence of boxes (called here glwords) in
(\nextx b).  |glwidth| is set to \textdim{2.4 in} in all of the
examples in this section in the interests of ``economy of space''.

\pex[interpartskip=2ex]
\a
\begingl
\gla Ti ma'a'\~nao hao kumuentusi ni h\'ayiyi ha'.//
\glb not agr.afraid you Infin.speak.to not anyone Emp//
\endgl
%
\a\let\\=\minigloss
\spaceskip=\lingglspace
\\{Ti}{not}, \\{ma'a'\~nao}{agr.afraid}, \\{hao}{you},
\\{kumuentusi}{Infin.speak.to}, \\{ni}{not}, \\{h\'ayiyi}{anyone},
\\{ha'}{Emp}
\xe
The baseline of a glword is the baseline of its top word. The
vertical spacing in each glword is determined as in page
building; the vertical spacing is determined by the baselineskip,
provided that there are no overly tall or overly deep characters
in the words.

Tex's line building apparatus then assembles the two boxes below.

\pex[interpartskip=2ex]<hboxes>
%
\a \outlinebox{\hbox{\let\\=\miniglossbare  \spaceskip=.6em
\\{Ti}{not} \\{ma'a'\~nao}{agr.afraid} \\{hao}{you}
\\{kumuentusi}{Infin.speak.to}}}

\a \outlinebox{\hbox{\let\\=\miniglossbare  \spaceskip=.6em
\\{ni}{not} \\{h\'ayiyi}{anyone} \\{ha'}{Emp}}}
\xe
The baselines of these boxes is the baseline of the top line.

The boxes (\lastx) must then be stacked by Tex's page building
algorithm. Since the depth of the top box plus the height of the
first box is larger than the baselineskip, no interline glue is
inserted by the algorithm. Remember that the depth and height of
these boxes is determined by baseline of the top line in the
glwords;\footnote{I am not the first to wish Tex boxes had the
provision for two baselines} the baseline of the bottom words is
irrelevant. Without glue between the boxes, (\nextx) would
result.

\ex
\vtop{\hbox{\outlinebox{\hbox{\let\\=\miniglossbare  \spaceskip=.6em
\\{Ti}{not} \\{ma'a'\~nao}{agr.afraid} \\{hao}{you}
\\{kumuentusi}{Infin.speak.to}}}}
%
\hbox{\outlinebox{\hbox{\let\\=\miniglossbare  \spaceskip=.6em
\\{ni}{not} \\{h\'ayiyi}{anyone} \\{ha'}{Emp}\pnode{B}}}}}
\xe

Since the page building algorithm does not insert vertical glue
between the boxes, ExPex must do it.  The amount is determined by
the parameter |glwrapskip|.\footnote{This parameter was called
{\tt abovemoreglskip} in version 2.7 .  For compatability, the
old name will still work as before, but should be considered
obsolete.} One common way to design glosses is to make the
distance between the baseline of the bottom line in the top box
in (\lastx) and the baseline of the bottom box the same as the
baselineskip. ExPex provides the macro |\normalglwrapskips| to
accomplish this. It does two things.  First, it turns on
automatic strut insertion so that struts are inserted before
every word in every glword. Second, the glwrapskip is adjusted
appropriately.  The boolean parameter |glstruts| can be used to
control automatic strut insertion independently of
|normalglwrapskips|.

Here are some examples.

\setss .55 .45
\beginss
\ex
\baselineskip=20pt
\normalglwrapskips
\begingl
\gla Ti ma'a'\~nao hao kumuentusi
   ni h\'ayiyi ha'.//
\glb not agr.afraid you
   Infin.speak.to not anyone Emp//
\endgl
\xe|midss
\ex
\baselineskip=20pt
\normalglwrapskips
\begingl
\gla Ti ma'a'\~nao hao kumuentusi
   ni h\'ayiyi ha'.//
\glb not agr.afraid you
   Infin.speak.to not anyone Emp//
\endgl
\xe
\endss

\beginss
\ex
\baselineskip=20pt
\normalglwrapskips
\begingl[glwrapskip=!1ex]
\gla Ti ma'a'\~nao hao kumuentusi
   ni h\'ayiyi ha'.//
\glb not agr.afraid you
   Infin.speak.to not anyone Emp//
\endgl
\xe|midss
\ex
\baselineskip=20pt
\normalglwrapskips
\begingl[glwrapskip=!1ex]
%\begingl[glwrapskip=!1ex]
\gla Ti ma'a'\~nao hao kumuentusi
   ni h\'ayiyi ha'.//
\glb not agr.afraid you
   Infin.speak.to not anyone Emp//
\endgl
\xe
\endss
Above, |glwrapskip| is an incremental parameter, so |glwrapskip=!1ex|
increases the glwrapskip by \textdim{1 ex}.

\beginss
\ex
\baselineskip=14pt
\normalglwrapskips
\begingl
\gla Ti ma'a'\~nao hao kumuentusi
   ni h\'ayiyi ha'.//
\glb not agr.afraid you
   Infin.speak.to not anyone Emp//
\endgl
\xe|midss
\ex
\baselineskip=14pt
\normalglwrapskips
\begingl
%\begingl[normalglwrapskips]
\gla Ti ma'a'\~nao hao kumuentusi
   ni h\'ayiyi ha'.//
\glb not agr.afraid you
   Infin.speak.to not anyone Emp//
\endgl
\xe
\endss
Struts in this series of examples have height \textdim{10 pt} and
depth \textdim{4 pt}.

\beginss
\baselineskip=18pt
\begingl[glstruts=true,
   glwrapskip=4pt]
\gla Ti ma'a'\~nao hao kumuentusi
   ni h\'ayiyi ha'.//
\glb not agr.afraid you
   Infin.speak.to not anyone Emp//
\endgl
\xe|midss
\ex
\baselineskip=18pt
\begingl[glstruts=true,
   glwrapskip=4pt]
\gla Ti ma'a'\~nao hao kumuentusi
   ni h\'ayiyi ha'.//
\glb not agr.afraid you
   Infin.speak.to not anyone Emp//
\endgl
\xe
\endss

\beginss
\ex
\baselineskip=14pt
\begingl[glstruts=false,
   glwrapskip=1ex]
\gla Ti ma'a'\~nao hao kumuentusi
   ni h\'ayiyi ha'.//
\glb not agr.afraid you
   Infin.speak.to not anyone Emp//
\endgl
\xe
|midss
\ex
\baselineskip=14pt
\begingl[glstruts=false,
   glwrapskip=1ex]
\gla Ti ma'a'\~nao hao kumuentusi
   ni h\'ayiyi ha'.//
\glb not agr.afraid you
   Infin.speak.to not anyone Emp//
\endgl
\xe
\endss
This gloss style was the ExPex default in version 2.7.


When ExPex is loaded, it executes |\normalglwrapskips|, which
sets |glstruts| to true and adjusts the glwrapskip appropriately.
{\it This is a change from version 2.7}.  The macro
\hbox{|\normalglwrapskips|} and the parameter |glstruts| are new.
If required for compatibility, version 2.7 initial behaviour can
be achieved by |\lingset{glstruts=false,glwrapskip=1ex}|.  Note
carefully that |\normalglwrapskips| uses the value of the
baselineskip in whatever environment it is executed in.  If the
baselineskip is changed by the user or by files that are loaded
after ExPex, it will be necessary to renormalize the glwrapskip by
executing |\normalglwrapskips| again after the baselineskip has
been changed.

\subsection Line breaking in glosses:
How to avoid the dreaded ``overfull box''

\begininventory
\parameters
\idx{|glspace|}& skip& |.5em plus .25em minus .16667 em|\hfil\cr
\idx{|glrightskip|}& skip& |0pt plus .1\hsize|\hfil \cr
\endinventory
%
Hopefully, you will never encounter the problem which the
parameterization in this section is designed to solve, the
``overfull line'' error message.  The default setting puts
\textdim{.5 em} of glue between the glwords a line, plus
\textdim{.25 em} of stretchability and \textdim{.16667 em} of
shrinkage.  The interglword spacing on a line can therefore vary
from \textdim{.33333 em} to \textdim{.75 em} to accomodate the
needs of the Tex linebreaking algorithm.  Further, the default settings
allow up to 10\% of the hsize of whitespace to appear at the end
of the line of glwords.  The possible extra space at the right
edge and stretchability/shrinkability of the space between
glwords means that line breaking in glosses will rarely encounter
problems with overfull lines.

In unusual cases, there may be a problem.  Your publisher (or
you) may be particularly fussy and demand a particular spacing
and/or better right alignment in glosses.  Or you might want to make a
particularly narrow gloss, which increases the chances that there
might not be enough flexability in the spacing.

If you do run into the problem of overfull lines in a gloss, two
parameters allow for a great deal of flexibility; |glspace| and
|glrightskip|.  If you must have right alignment, |glrightskip|
should be set to \textdim{0 pt}.  Or you might want to increase
the glrightskip to ease the burden of line breaking in the gloss.
|glspace| is an "inc parameter", so you could say, for example,
|glspace=!0pt plus .2em| (increasing the stretchability of the
interglword space by \textdim{2 em}) to increase the
stretchability of the interglword gl space. Solving this type of
problem will generally require some experimentation.  An
acceptable solution will depend on your particular typesetting
aesthetics.



\endinput






In order to do this, it


Here are the various dimensions in (\lastx) which are relevant to
line spacing in the gloss.
\lingset{framesep=0,boxsep=false}

\exdisplay \let\\=\miniglossbare
\hskip1.3in
\psscalebox{2.5}{\vtop{\hbox{\psframebox{\rnode[br]{B1b}{\hbox{\let\\=\miniglossbare  \spaceskip=.6em
\\{\pnode{A1}\rnode[t]{R33}{Ti}}{{\pnode{A2}not}} \\{ma'a'\~nao}{agr.afraid} \\{\dots}{\dots}
}}}\kern-60pt}
\vskip1ex
\hbox{\psframebox{\rnode[tr]{B2t}{\hbox{\let\\=\miniglossbare  \spaceskip=.6em
\\{\pnode{A3}ni}{\pnode{A4}not} \\{h\'ayiyi}{anyone} \\{ha'}{Emp}}}}}}}
\rput(B1b){\rput(0,14pt){\pnode{R1}}}
\psline[linewidth=4pt,linecolor=white](B1b)(B1b|R33)
\rput(A1|B1b){\pnode(-1em,0){Q1}}
\psset{linewidth=1pt,labelsep=.5em}
\pcline{|<->|}(Q1)(Q1|B2t)
\nbput[ref=r]{glwrapskip}
\pcline{|<->|}(Q1|A1)(Q1|A2)
\nbput[ref=r]{baselineskip}
\pcline{|<->|}(Q1|A3)(Q1|A4)
\nbput[ref=r]{baselineskip}
\rput(B1b|A2){\pnode(1em,0){Q2}}
\pcline{|<->|}(Q2)(Q2|A3)
\naput[ref=l]{glwrapbaselineskip}
\xe

