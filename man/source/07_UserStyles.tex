
\section User defined styles

\begininventory
\macros* \idx{|\definelingstyle|}\endmc
\parameters
\idx{|everyex|\user}& token list& |{}|\cr
\idx{|Everyex|\user}& token list& |{}|\cr
\endinventory
%
Aside from all the \ExPex\ parameters introduced to this point,
there are numerous \Tex\ parameters that affect the appearance
of examples: line spacing, hsize, font selection, etc.  These are
all quantities that \Tex\ itself provides mechanisms for setting
to suit the user.  You may want to make adjustments of these
parameters every time a particular kind of example is typeset, so
that special settings hold inside these examples independently of
the contextual settings.  For example, suppose you have many
examples that, for some reason, you want to format like~(\nextx),
with a narrow width, italic font, an oversized gap between the
example number and the text, and a somewhat greater than normal
separation between baselines.

\framedisplay
\leftline{\vbox{%
\ex[textoffset=3em]
\hsize=3in \rightskip=0pt plus 2em \it \advance\baselineskip by 2pt
Und hier k\"onnen wir sehen was f\"ur Unfug wird gemacht
wenn er einen ganz langen Satz binnen kriegt.
\xe
}\hfil}
\endframedisplay

One way to do this is:

\codedisplay~
\ex[textoffset=3em]
\hsize=3in \rightskip=0pt plus 2em \it \advance\baselineskip by 2pt
Und hier k\"onnen wir sehen was f\"ur Unfug wird gemacht
wenn er einen ganz langen Satz binnen kriegt.
\xe
|endcodedisplay
Some rightskip is included because with such a narrow width it is
otherwise difficult to avoid overfull lines.

If you have many examples that you want to typeset in this way
that are scattered throughout a document, it is awkward to have
to remember all of these settings and enter them each time an
example of this kind needs to be formatted.  Furthermore, if you
change your mind about some detail of this special formatting,
you need to go through the document and change each instance of
this formatting.  \ExPex\ allows you to package all the
formatting changes into one named unit, called a ``style''.  If
this style is called ``narrow italic'', then you can write the following
to achieve the format of (\lastx).

\codedisplay
\ex[lingstyle=narrow italic]
|endcodedisplay

The various format specifications are packaged into a style
by saying:

\codedisplay
\definelingstyle{narrow italic}{textoffset=3em,
   everyex={\hsize=3in \rightskip=0pt plus 2em
      \it \advance\baselineskip by 2pt}}
|endcodedisplay
Thereafter, setting the parameter |lingstyle| to |narrow italic|
has the effect of setting all the parameters as specified in the
style definition (the two parameters |textoffset| and |everyex|
in this example). The parameter |everyex| works in the following
way.  If |everyex| is set to {\sl value\/}, a macro |\lingeveryex|
is defined whose expansion is {\sl value}.  When
an |\ex| or |\pex| construction is initiated, {\it after\/} any
parameter settings take effect, |\lingeveryex| is executed.

\ExPex\ provides a second parameter, |Everyex|, which is similar to
|expex|.  If |Expex| is set to value, a macro |ling@Everypex| is
defined. When
an |\ex| or |\pex| construction is initiated, {\it before\/} after any
local parameter changes take effect, |\lingEveryex| is executed.
The reasons for providing both |everyex| and |Everyex| are
subtle. Different stages of the writing/rewriting/editing process
may call for different treatments of example formatting,
particularly if the final aim is a camera ready product.  Suppose
you normally make double spaced drafts with an hsize of
\textdim{6.5 in} and that your final aim is to produce camera ready copy with
an hsize of \textdim{4.375 in}.  Suppose also that you are at
editing stage where you want to see exactly how the examples
will be formatted in the finished product, but still want full width
double spaced text.  One way to accomplish this is to say

\codedisplay
\lingset{everyex={\hsize=4.375in \normalbaselines}}
|endcodedisplay
at the beginning of your document.  |\normalbaselines|
is a standard \Tex\ macro which establishes the normal line
spacing for the current font.  This makes the hsize and line
spacing inside examples independent of the hsize and line spacing
you select for the text (outside examples).

This will accomplish what you want.  But it makes the further use
of |everyex| in your document awkward.  If you want a particular
example to be set in italics, for example, you might think to
use:

\codedisplay
\ex[everyex=\it]
|endcodedisplay

\noindent This will certainly produce an italicized example
because |\it| will be evaluated early on in typesetting the
example.  But it has the unfortunate consequence of overwriting
the initial setting of |everyex|, removing the special line
spacing within examples.

|Everyex| is provided to accommodate situations like this. The
intention is that special settings that should hold inside
examples throughout the document are assigned to |Everyex|, with
|expex| reserved for local variations.  Since |\lingEveryex| is
evaluated before local parameter changes take effect, parameter
settings specified by |Expex| will be overridden by any local
parameter settings.

