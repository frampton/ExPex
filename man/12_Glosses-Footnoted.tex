\ifnum\secno < 2 \secno=12 \fi
\ifnum\subsecno < 2 \subsecno=2 \fi
\ifnum\fnno < 5 \fnno=12 \fi
\everymath={}%

\subsection Footnoted items in glosses

\subsubsection Notes at the foot of the gloss

Suppose you want to produce an example like the one below, where certain items
in the gloss are footnoted, with the notes appearing {\it at the foot of the
gloss}.

\framedisplay
\ex
\begingl
\gla {k\'\i} dh\`an {g\`aj\`a g\`aj\`a g\`aj\`a} y\`a-\`a c\'e'd-k{\'\i} bw\`a
m\`o \'a c\'esh m\'un 'k\=a'b\=us$^b$ //
\glb {\sc comp} great {\sc idph$^a$} go.{\sc sing-lnk}
break-{\sc ad1} stomach.{\sc part} {\sc mo} {\sc loc.cl2} ground
{\sc dem.loc.anph} shattered //
\glft `and he flailed through the air, breaking into pieces there on the
ground.' //
\endgl
\medskip
\footnotesize
$a.\,${\sl g\`aj\`a g\`aj\`a g\`aj\`a\/} is an ideophone ({\sc idph}) for a long (thin)
thing going through the air, such as a spear with a wobbling shaft, or a tall
person waving their arms about flailingly.\par
$b.\,${\sl 'k\=a'b\=us\/} denotes something broken into pieces, or something
which has become fragile after a heavy impact of some kind.
\xe
\endframedisplay
\noindent Letters $a$, $b$, \dots\ are used rather than numbers so that there is
no confusion with footnotes at the bottom of the page.  The example, from the
Uduk language, was provided by Don Killian.

The code is fairly straighforward.

\codedisplay
\ex[glstyle=wrap]
\begingl
\gla {k\'\i} dh\`an {g\`aj\`a g\`aj\`a g\`aj\`a} y\`a-\`a c\'e'd-k{\'\i}
bw\`a m\`o \'a c\'esh m\'un 'k\=a'b\=us$^b$ //
\glb {\sc comp} great {\sc idph$^a$} go.{\sc sing-lnk} break-{\sc ad1}
stomach.{\sc part} {\sc mo} {\sc loc.cl2} ground {\sc dem.loc.anph}
shattered //
\glft `and he flailed through the air, breaking into pieces there on the ground.' //
\endgl
\medskip
\footnotesize
$a.\,${\sl g\`aj\`a g\`aj\`a g\`aj\`a\/} is an ideophone ({\sc idph}) for
a long (thin) thing going through the air, such as a spear with a wobbling
shaft, or a tall person waving their arms about flailingly.\par
$b.\,${\sl 'k\=a'b\=us\/} denotes something broken into pieces, or
something which has become fragile after a heavy impact of some kind.
\xe
|endcodedisplay
\noindent The only subtlety here is that the $\mit b$ superscript appears in the
code before that $\mit a$ superscript.  This is a consequence of the way that a
gloss is typeset, with the word and its gloss put into a box.  The box with the
$a$-superscripted word is typeset before the box with the $b$-superscripted
gloss.

There is no such order question for `nlevel' style glosses, hence less
possibility of a typesetting error creeping into document preparation. There is
no ordering question in the nlevel style because the various items in the gloss
appear in the Tex code in the same order that they appear on the printed
page.\footnote{This assumes that all the items in a glossed word are considered
to appear before any item in a following glossed word.} The example above can
be coded as follows.

\codedisplay
\ex[glstyle=nlevel]
\begingl
{k\'\i}[\sc comp]
dh\`an[great]
{g\`aj\`a g\`aj\`a g\`aj\`a}[\sc idph$^a$]
y\`a-\`a[go.\sc sing-lnk]
c\'e'd-k{\'\i}[break-\sc ad1]
bw\`a[stomach.\sc part]
m\`o[\sc mo]
\'a[\sc loc.cl2]
c\'esh[ground]
m\'un[\sc dem.loc.anph]
'k\=a'b\=us$^b$[shattered]
\glft `and he flailed through the air, breaking into pieces there on the ground.' //
\endgl
\medskip
\footnotesize
$a.\,${\sl g\`aj\`a g\`aj\`a g\`aj\`a\/} is an ideophone ({\sc idph}) for
a long (thin) thing going through the air, such as a spear with a wobbling
shaft, or a tall person waving their arms about flailingly.\par
$b.\,${\sl 'k\=a'b\=us\/} denotes something broken into pieces, or
something which has become fragile after a heavy impact of some kind.
\xe
|endcodedisplay

\subsubsection Notes at the foot of the page

\begininventory
\macros \idx{|\innerfnno|}, \idx{|\outerfntext|},
\idx{|\glfnmark|}\endmc
\endinventory
Some \ExPex\ users have been faced with the problem of editors who insisted that
all footnotes appear at the bottom of the page. Suppose you want to produce an
example like the one below, where certain items in the gloss are footnoted, with
the footnotes appearing in the regular sequence of footnotes at the bottom of
the page.

\vskip 0pt plus 1.5in \penalty-120 \vskip 0pt minus 1.5in

\framedisplay
\ex
\begingl
\gla {k\'\i} dh\`an {g\`aj\`a g\`aj\`a g\`aj\`a} y\`a-\`a c\'e'd-k{\'\i}
bw\`a m\`o \'a c\'esh m\'un {'k\=a'b\=us\innerfnno{2}} //
\glb
{\sc comp}  great  {\sc idph\innerfnno{1}} go.  {\sc sing-lnk}  break-{\sc ad1}
stomach.{\sc part}  {\sc mo}  {\sc loc.cl2}  ground  {\sc dem.loc.anph}
shattered //
\glft `and he flailed through the air, breaking into pieces there on the
ground.' //
\endgl
\xe
\endframedisplay
\outerfntext{{\sl g\`aj\`a g\`aj\`a g\`aj\`a\/} is an ideophone for a long (thin)
thing going through the air, such as a spear with a wobbling shaft, or a tall
person waving their arms about flailingly.}
\outerfntext{{\sl 'k\=a'b\=us\/} denotes something broken into pieces, or something
which has become fragile after a heavy impact of some kind.}


In addition to the problem of ordering the footnotes correctly noted above,
there is a second problem, which is caused by the fact that the footnotes are
associated with positions inside boxes.  Each word-gloss pair above is typeset
in a box and footnote text in Tex/LaTex cannot be assigned inside boxes.
There is a standard way of getting around this limitation; only the footnote
number is typeset in the box, with the text typeset outside the box (using
|\vfootnote| in Tex or |\footnotetext| in LaTex).  This is straightforward if
there is only one footnoted item is a gloss, but if there are multiple
footnoted items in a single gloss, there is a problem.  {\sl eptexfn.tex\/} and
{\sl epltxfn.sty\/} provide a solution, albeit fairly primitive. Code for the
example above can be written in either of the two ways below:

\codedisplay
\ex
\begingl
\gla {k\'\i} dh\`an {g\`aj\`a g\`aj\`a g\`aj\`a} y\`a-\`a c\'e'd-k{\'\i}
bw\`a m\`o \'a c\'esh m\'un {'k\=a'b\=us\innerfnno{2}} //
\glb
{\sc comp}  great  {\sc idph\innerfnno{1}} go.  {\sc sing-lnk}  break-{\sc ad1}
stomach.{\sc part}  {\sc mo}  {\sc loc.cl2}  ground  {\sc dem.loc.anph}
shattered //
\glft `and he flailed through the air, breaking into pieces there on the
ground.' //
\endgl
\xe
\outerfntext{{\sl g\`aj\`a g\`aj\`a g\`aj\`a\/} is an ideophone for a long
(thin) thing going through the air, such as a spear with a wobbling shaft,
or a tall person waving their arms about flailingly.}
\outerfntext{{\sl 'k\=a'b\=us\/} denotes something broken into pieces, or
something which has become fragile after a heavy impact of some kind.}
|endcodedisplay

\codedisplay
\ex
\begingl
\gla {k\'\i} dh\`an {g\`aj\`a g\`aj\`a g\`aj\`a} y\`a-\`a c\'e'd-k{\'\i}
bw\`a m\`o \'a c\'esh m\'un {'k\=a'b\=us\innerfnno{2}} //
\glb
{\sc comp}  great  {\sc idph\innerfnno{1}} go.  {\sc sing-lnk}  break-{\sc ad1}
stomach.{\sc part}  {\sc mo}  {\sc loc.cl2}  ground  {\sc dem.loc.anph}
shattered //
\glft `and he flailed through the air, breaking into pieces there on the
ground.' //
\endgl
\xe
\outerfntext{{\sl g\`aj\`a g\`aj\`a g\`aj\`a\/} is an ideophone for a long
(thin) thing going through the air, such as a spear with a wobbling shaft,
or a tall person waving their arms about flailingly.}
\outerfntext{{\sl 'k\=a'b\=us\/} denotes something broken into pieces, or
something which has become fragile after a heavy impact of some kind.}
|endcodedisplay

\noindent Alternatively,

\codedisplay
\ex[glstyle=nlevel]
\begingl
{k\'\i}[\sc comp]
dh\`an[great]
{g\`aj\`a g\`aj\`a g\`aj\`a}[\sc idph\innerfnno1]
y\`a-\`a[go.\sc sing-lnk]
c\'e'd-k{\'\i}[break-\sc ad1]
bw\`a[stomach.\sc part]
m\`o[\sc mo]
\'a[\sc loc.cl2]
c\'esh[ground]
m\'un[\sc dem.loc.anph]
'k\=a'b\=us\innerfnno2[shattered]
\glft `and he flailed through the air, breaking into pieces there on the
ground.'
\endgl
\outerfntext{{\sl g\`aj\`a g\`aj\`a g\`aj\`a\/} is an ideophone for a long
(thin) thing going through the air, such as a spear with a wobbling shaft,
or a tall person waving their arms about flailingly.}
\outerfntext{{\sl 'k\=a'b\=us\/} denotes something broken into pieces, or
something which has become fragile after a heavy impact of some kind.}
\xe
|endcodedisplay

The mechanism is primitive.  It is the user's responsibility to ensure that
sequencing of the footnotes is what is desired and that each occurence of
|\innerfnno| matches an occurence of |\outerfntext|.  If the last footnote was
numbered $n$, |\innerfnno{|$k$|}| will assign number $n+k$ and associate with
the $k$-th outer footnote text.  Proof reading should be painstaking.  Remember
that in if footnote numbering is reordered, the outer footnote texts must be
reordered.  As observed above, the nlevel style is safer because the order of
footnote numbering in the Tex file corresponds to the order of footnotes in the
typeset display.




\endinput

\subsection* User defined levels

\begininventory
\macros* \idx{|\defineglwlevels|}\endmc
\endinventory
%
|\glb| and |\glc| are given definitions in {\it expex.tex\/}
by the command |\defineglwlevels{b,c}|.  The command also creates the
parameters |everyglb|, |everyglc|, |aboveglbskip|, and |aboveglcskip|.
|everyglb| and |everyglc| are initialized to empty token lists and
|aboveglbskip| and |aboveglcskip| to \textdim{0 pt}. The user may want
to use |\defineglwlevels| to create and name new gloss
levels.

For example, suppose more suggestive level names are
desired.

\codedisplay
\defineglwlevels{cat,gloss}
\lingset{everyglcat=\footnotesize,aboveglcatskip=-.5ex}|exbreak

\ex
\begingl
\gla k- wapm -a -s'i -m -wapunin -uk //
\glcat Cl V Agr Neg Agr Tns Agr //|par|allowbreak
\glgloss 2 see 3{\sc acc} {} {2\sc pl} preterit {3\sc pl} //
\glft `you (pl) didn't see them'//
\endgl
\xe
|endcodedisplay
produces

\framedisplay
\defineglwlevels{cat,gloss}
\lingset{everyglcat=\footnotesize,aboveglcatskip=-.5ex}

\ex
\begingl
\gla k- wapm -a -s'i -m -wapunin -uk //
\glcat Cl V Agr Neg Agr Tns Agr //
\glgloss 2 see {3\sc acc} {} {2\sc pl} preterit {3\sc pl} //
\glft `you (pl) didn't see them'//
\endgl
\xe
\endframedisplay

Another instance in which the user might want to define a new gloss level
or levels is if more than 3 lines of interlinear gloss are needed and
the desired flexibility cannot be obtained by repeated use of |\glb|
or |\glc|.


%%%%%%%%%%%%%%%%%%%%%%%%%%%%%%%%%%%%%%%%%%%%%%%%%
\subsection Positioning the free translation to the right of the
interlinear gloss

\begininventory
\parameters
\idx{|glftpos|}& choice (|below| or |right|)& |below|\cr
\idx{|sssep|}& dimension& 3\dimskip em\cr
\idx{|ssratio|}& decimal& |.6 |\cr
\idx{|ssrightskip|}& skip& 0\dimskip pt plus 2\dimskip em\cr
\endinventory

\lingset{everygl=\openup.5ex,
   everyglword=\normalbaselines,everyglft=\normalbaselines,
   glhangindent=2em}

\framedisplay
\ex[glftpos=right,glhangstyle=none]
\let\\=\textsc
\begingl
\gla
Hom\^{a}o sa \v{c}\^{o} p\^{o} tha  \~{n}u nao ng\u{a} hmua. \~{N}u
dj\u{a} g\u{a}, \~{n}u dj\u{a} \v{c}\u{o}ng \~{n}u, laih gui r\^{e}o
\~{n}u. Todang bboi r\^{o}k jolan \~{n}u nao hma, \~{n}u bb\^{o}h sa
droi mr\u{a} d\u{o} bboi gah, a, hruh \~{n}u.//
\glb
\\{exist} one \\{clf} person old \\{3s} go do field \\{3s} hold
machete \\{3s} hold hoe \\{3s} and carry.on.back back.basket \\{3s}
while at along trail \\{3s} go field \\{3s} see one \\{clf} peacock
stay at \\{drct} -- nest \\{3s}//
\glft
`There was an old person who went to work in the field. He took
along his machete, he took along his hoe, and he carried his
basket on his back. While he was on his way to the farm, he saw a
peacock beside its nest.'//
\endgl
\xe
\endframedisplay

\noindent is achieved by
\codedisplay~
\ex[glftpos=right,glhangstyle=none]
\let\\=\textsc
\begingl
\gla
Hom\^{a}o sa \v{c}\^{o} p\^{o} tha  \~{n}u nao ng\u{a} hmua. \~{N}u
dj\u{a} g\u{a}, \~{n}u dj\u{a} \v{c}\u{o}ng \~{n}u, laih gui r\^{e}o
\~{n}u. Todang bboi r\^{o}k jolan \~{n}u nao hma, \~{n}u bb\^{o}h sa
droi mr\u{a} d\u{o} bboi gah, a, hruh \~{n}u.//
\glb
\\{exist} one \\{clf} person old \\{3s} go do field \\{3s} hold
machete \\{3s} hold hoe \\{3s} and carry.on.back back.basket \\{3s}
while at along trail \\{3s} go field \\{3s} see one \\{clf} peacock
stay at \\{drct} -- nest \\{3s}//
\glft
`There was an old person who went to work in the field. He took
along his machete, he took along his hoe, and he carried his
basket on his back. While he was on his way to the farm, he saw a
peacock beside its nest.'//
\endgl
\xe
|endcodedisplay
(This example, as well as (\getref{panelex}), was contributed by
Joshua Jensen.  It is from Jarai, an Austronesian language.  The
story teller was Hyech Ksor.  The orthography here is somewhat
simplified in order to keep the font requirements for the
examples in this documentation elementary.)

|ss| stands for ``side-by-side''. |sssep| gives the separation of the
gloss and the free translation. |ssratio| gives the proportion of the
available width that the gloss occupies. The point of hanging
indentation is to visually separate the free translation and the
gloss, so |glhangstyle=none| is completely satisfactory if the
tree translation is on the right. But \ExPex\ will happily use
hanging indentation with the free translation on the right.

Line breaking in the free translation is delicate because it will
generally set in a narrow column. The default setting of |ssrightskip|
allows up to \textdim{2 em} departure from right alignment. This usually avoids
overfull lines and awkward hyphenation.  |ssrightskip| can be
increased (all the way to \textdim{0 pt} plus \textdim{1 fil}) if there is a
problem, at the cost of a more ragged appearance.  This can be done
globally, or simply in troublesome examples.

\subsection Glosses with a side panel

\begininventory
\macros
\idx{|\beginglpanel[]|}, \idx{|\endpanel|}\endmc
\parameters*
\idx{|everypanel|\user}& token list& |{}|\cr
\endinventory
The mechanism for positioning the free translation to the right of the
interlinear gloss can be adapted to create a side panel for
glosses which can be used for other purposes, as illustrated
below.

\framedisplay
\ex[everypanel=\footnotesize]<panelex>
\let\\=\textsc
\beginglpanel[ssratio=.5,glhangstyle=none]
\gla Hom\^{a}o$^1$ sa \v{c}\^{o} p\^{o} tha  \~{n}u nao ng\u{a}
hmua. \~{N}u dj\u{a} g\u{a}, \~{n}u dj\u{a} \v{c}\u{o}ng \~{n}u,
laih gui r\^{e}o \~{n}u. Todang bboi r\^{o}k jolan \~{n}u nao
hma, \~{n}u bb\^{o}h sa droi mr\u{a} d\u{o}$\,^4$ bboi gah, a, hruh
\~{n}u.//
\glb \\{exist} one \\{clf} person old \\{3s} go$^2$ do field
\\{3s} hold machete \\{3s} hold hoe \\{3s} and$^3$ carry.on.back
back.basket \\{3s} while at along trail \\{3s} go field \\{3s}
see one \\{clf} peacock stay at \\{drct} -- nest \\{3s}
//
\endgl
1.\enspace {\it hom\^{a}o} also means `have', reflecting the
strong tendency across languages to use the same word for
possession and the existential. {\it hom\^{a}o} is clause-initial
in existential clauses, but it comes after the subject in
possession clauses.

2.\enspace All verbs are glossed with a bare form, as Jarai has
no inflectional morphology. Although Jarai has lexical items that
encode tense, they are relatively infrequent in text.

3.\enspace The word {\it laih} is literally `after; finish', but
that is clearly not the meaning here. Probably {\it laih} here is
an abbreviation for {\it laih an\u{u}n}, `after that; and', hence
the gloss `and'.

4.\enspace {\it d\u{o}} `sit, stay' is used like a copula in
locative clauses, which is what I assume here (`a~peacock
[which was] beside its nest'); however, this could just as well
mean `a peacock sitting beside its nest', retaining the posture
semantics.
\endpanel
\bigskip
`There was an old person who went to work in the field. He took
along his machete, he took along his hoe, and he carried his
basket on his back. While he was on his way to the farm, he saw a
peacock beside its nest.'
\xe
\endframedisplay

The syntax is:
\codedisplay
\beginglpanel |dots \endgl |dots \endpanel
|endcodedisplay
The first part is the gloss, with the usual syntax.  The second
part is put in a vbox and set alongside the gloss. The tokens
|lingeverypanel| are inserted when the vbox begins. All of the
parameters which are special to positioning the free translation
to the right of the gloss apply here as well, with the obvious
meanings.

The complete code for the example above is:

\codedisplay
\ex[everypanel=\footnotesize]<panelex>
\let\\=\textsc
\beginglpanel[ssratio=.5,glhangstyle=none]
\gla Hom\^{a}o$^1$ sa \v{c}\^{o} p\^{o} tha  \~{n}u nao ng\u{a}
hmua. \~{N}u dj\u{a} g\u{a}, \~{n}u dj\u{a} \v{c}\u{o}ng \~{n}u,
laih gui r\^{e}o \~{n}u. Todang bboi r\^{o}k jolan \~{n}u nao
hma, \~{n}u bb\^{o}h sa droi mr\u{a} d\u{o}$\,^4$ bboi gah, a, hruh
\~{n}u.//
\glb \\{exist} one \\{clf} person old \\{3s} go$^2$ do field
\\{3s} hold machete \\{3s} hold hoe \\{3s} and$^3$ carry.on.back
back.basket \\{3s} while at along trail \\{3s} go field \\{3s}
see one \\{clf} peacock stay at \\{drct} -- nest \\{3s}//
\endgl
1.\enspace {\it hom\^{a}o} also means `have', reflecting the
strong tendency across languages to use the same word for
possession and the existential. {\it hom\^{a}o} is clause-initial
in existential clauses, but it comes after the subject in
possession clauses.

2.\enspace All verbs are glossed with a bare form, as Jarai has
no inflectional morphology. Although Jarai has lexical items that
encode tense, they are relatively infrequent in text.

3.\enspace The word {\it laih} is literally `after; finish', but
that is clearly not the meaning here. Probably {\it laih} here is
an abbreviation for {\it laih an\u{u}n}, `after that; and', hence
the gloss `and'.

4.\enspace {\it d\u{o}} `sit, stay' is used like a copula in
locative clauses, which is what I assume here (`a~peacock [which
was] beside its nest'); however, this could just as well mean `a
peacock sitting beside its nest', retaining the posture
semantics.
\endpanel
\bigskip
`There was an old person who went to work in the field. He took
along his machete, he took along his hoe, and he carried his
basket on his back. While he was on his way to the farm, he saw a
peacock beside its nest.'
\xe
|endcodedisplay
Note that the free translation here comes after |\endpanel| and is
typeset the way any material inside an |\ex| construction is
typeset.  This allows it to have full width, spanning both the
gloss and notes.  It could have been part of the gloss, with a
different result.

No support is given to side note numbering.  It must be done ``by
hand''.  If the construction turns out to be sufficiently useful
and hand numbering is sufficiently tedious, a more automatic
scheme might be possible.  It would not be trivial, because the
order in which the notes appear inside the gloss before it is
typeset is not necessarily the same as the order in which they
appear after it is typeset.

\subsection Cascading hanging indentation in glosses

\deftagsec{cascadingsec}
\raggedbottom
\lingset{glhangindent=.25in,everygl=\openup.5ex,
   everyglword=\normalbaselines,everyglft=\normalbaselines,
   extraglskip=1ex}

\lingset{glhangstyle=none}
\framedisplay
\ex[glhangstyle=cascade]
\let\\=\textsc
\begingl
\gla
Hom\^{a}o sa \v{c}\^{o} p\^{o} tha  \~{n}u nao ng\u{a} hmua. \~{N}u
dj\u{a} g\u{a}, \~{n}u dj\u{a} \v{c}\u{o}ng \~{n}u, laih gui r\^{e}o
\~{n}u. Todang bboi r\^{o}k jolan \~{n}u nao hma, \~{n}u bb\^{o}h sa
droi mr\u{a} d\u{o} bboi gah, a, hruh \~{n}u.//
\glb
\\{exist} one \\{clf} person old \\{3s} go do field \\{3s} hold
machete \\{3s} hold hoe \\{3s} and carry.on.back back.basket \\{3s}
while at along trail \\{3s} go field \\{3s} see one \\{clf} peacock
stay at \\{drct} -- nest \\{3s}//
\glft
`There was an old person who went to work in the field. He took
along his machete, he took along his hoe, and he carried his
basket on his back. While he was on his way to the farm, he saw a
peacock beside its nest.'//
\endgl
\xe
\endframedisplay

is produced by
\codedisplay~
\ex[glhangstyle=cascade]
\let\\=\textsc
\begingl
\gla
Hom\^{a}o sa \v{c}\^{o} p\^{o} tha  \~{n}u nao ng\u{a} hmua. \~{N}u
dj\u{a} g\u{a}, \~{n}u dj\u{a} \v{c}\u{o}ng \~{n}u, laih gui r\^{e}o
\~{n}u. Todang bboi r\^{o}k jolan \~{n}u nao hma, \~{n}u bb\^{o}h sa
droi mr\u{a} d\u{o} bboi gah, a, hruh \~{n}u.//
\glb
\\{exist} one \\{clf} person old \\{3s} go do field \\{3s} hold
machete \\{3s} hold hoe \\{3s} and carry.on.back back.basket \\{3s}
while at along trail \\{3s} go field \\{3s} see one \\{clf} peacock
stay at \\{drct} -- nest \\{3s}//
\glft
`There was an old person who went to work in the field. He took
along his machete, he took along his hoe, and he carried his
basket on his back. While he was on his way to the farm, he saw a
peacock beside its nest.'//
\endgl
\xe
|endcodedisplay

\subsection  Gloss underfixes

\begininventory
\macros* \idx{|\gluf|}\endmc
\parameters
\idx{|glufcloseup|}& dimension& \textdim{.4 ex}\cr
\idx{|everygluf|}& token list& |{}|\cr
\endinventory

\noindent Sometimes, gloss displays like the following are
desired, with grammatical markings written below the gloss.

\framedisplay
\ex[glufcloseup=.4ex,everygluf=\footnotesize]
\begingl
\gla Mary$_i$ ist sicher, dass es den Hans nicht st\"oren
   w\"urde seiner Freundin ihr$_i$ Herz auszusch\"utten.//
\glb Mary is sure that it \gluf/the/ACC/ Hans not annoy would
   \gluf/his/DAT/ \gluf/girlfriend/DAT/ \gluf/her/ACC/
   \gluf/heart/ACC/ {out to throw}//
\glft `Mary is sure that to reveal her heart to his girlfriend
would not damage John.'//
\endgl
\xe
\endframedisplay

\ExPex\ provides the macro |\gluf| which can be used to
construct such a display.

\codedisplay
\ex[glufcloseup=.4ex,everygluf=\footnotesize]
\begingl
\gla Mary$_i$ ist sicher, dass es den Hans nicht st\"oren
   w\"urde seiner Freundin ihr$_i$ Herz auszusch\"utten.//
\glb Mary is sure that it \gluf/the/ACC/ Hans not annoy would
   \gluf/his/DAT/ \gluf/girlfriend/DAT/ \gluf/her/ACC/
   \gluf/heart/ACC/ {out to throw}//
\glft `Mary is sure that to reveal her heart to his girlfriend
would not damage John.'//
\endgl
\xe
|endcodedisplay

The grammatical markings are essentially ``underfixes'' (rather than
prefixes or suffixes), hence the name ``gluf'' (gl underfix). When the
underfix is typeset, the value of |everygluf| is first inserted.  It
is provided so that the user has control of the font used to typeset
the underfixes. The value of |glufcloseup| determines how much the
baselineskip between the underfix and the underfixed word is closed
up.  Without some closeup, the underfixes are not positioned close
enough to the glosses they modify (in my opinion). The macro |\gluf|
centers the underfix below the word it annotates.  Its syntax should
be clear from the example above.

The introduction of the nlevel gloss style makes |\gluf| unnecessary.
Its use should be avoided in the future because it will be removed
from \ExPex\ in some future version.  The code below shows (\lastx)
would be coded in the nlevel style.

\codedisplay
\ex[glstyle=nlevel,glneveryline={\it,,\footnotesize},
   glnabovelineskip={,,-.4ex},extraglskip=0pt]
\begingl
Mary$_i$[Mary]
ist[is]
sicher,[sure]
dass[that]
es[it]
den[the/ACC]
Hans[Hans]
nicht[not]
st\"oren[annoy]
w\"urde[would]
seiner[his/DAT]
Freundin[girlfriend/DAT]
ihr$_i$[her/ACC]
Herz[heart/ACC]
auszusch\"utten.[out to throw]
\glft `Mary is sure that to reveal her heart to his girlfriend
would not damage John.'
\endgl
\xe
|endcodedisplay

The result is almost, but not quite, identical to (\lastx).  The
underfixes are positioned somewhat differently, not center aligned
under the word above them.

\framedisplay
\ex[glstyle=nlevel,glneveryline={\it,,\footnotesize},
   glnabovelineskip={,,-.4ex},extraglskip=0pt]
\begingl
Mary$_i$[Mary]
ist[is]
sicher,[sure]
dass[that]
es[it]
den[the/ACC]
Hans[Hans]
nicht[not]
st\"oren[annoy]
w\"urde[would]
seiner[his/DAT]
Freundin[girlfriend/DAT]
ihr$_i$[her/ACC]
Herz[heart/ACC]
auszusch\"utten.[out to throw]
\glft `Mary is sure that to reveal her heart to his girlfriend
would not damage John.'
\endgl
\xe
\endframedisplay
See (\getref{wapm2}) below for a gloss display in the nlevel style
which is closer to (\blastx).

\subsection Center alignment in glwords

\begininventory
\parameters*
\idx{|glwordalign|}& |left| or |center|& |left|\cr
\endinventory
When glwords are assembled, in either the wrap or nlevel styles, an
alignment inside a vbox is constructed.  The alignments use an |\hfil|
on the right, so that the text on the various lines of a glword is
left aligned.  The parameter |glwordalign| allows the choice of center
alignment.  This is occasionally useful.  Compare, for example,
(\nextx) with (\getref{wapm2}).

\framedisplay
\ex[glspace=1.5em,everygla=\hfil,glwordalign=center,
   everyglc=\hfil,aboveglbskip=-.2ex]<wapm2>
\begingl
\gla k- wapm -a -s'i -m -wapunin -uk //
\glb CL V AGR NEG AGR TNS AGR //
\glc 2 see {\sc 3acc} {} {\sc 2pl} preterit {\sc 3pl} //
\glft `you (pl) didn't see them'//
\endgl
\xe
\endframedisplay
\codedisplay~
\ex[glspace=1.5em,everygla=\hfil,glwordalign=center,
   everyglc=\hfil,aboveglbskip=-.2ex]
\begingl
\gla k- wapm -a -s'i -m -wapunin -uk //
\glb CL V AGR NEG AGR TNS AGR //
\glc 2 see {\sc 3acc} {} {\sc 2pl} preterit {\sc 3pl} //
\glft `you (pl) didn't see them'//
\endgl
\xe
|endcodedisplay

Center alignment in glwords is useful if you need to number the
glwords; to facilitate discussion, for example.

\framedisplay
\ex[glstyle=nlevel,glhangstyle=cascade,
   glneveryline={\insertno,\it,},
   glwordalign=center,
   glnabovelineskip={,-1pt},glspace=!.4em]
\putfnno
\count255=1
\def\insertno{\footnotesize(\the\count255)\global\advance\count255 by 1}%
\begingl[]
[Hom\^{a}o/\textsc{exist}]
[sa/one]
[\v{c}\^{o}/\textsc{clf}]
[p\^{o}/person]
[tha/old]
[\~{n}u/\textsc{3s}]
[nao/go]
[ng\u{a}/do]
[hmua./field]
[\~{N}u/\textsc{3s}]
[dj\u{a}/hold]
[g\u{a},/machete]
[\~{n}u/\textsc{3s}]
[dj\u{a}/hold]
[\v{c}\u{o}ng/hoe]
[\~{n}u,/\textsc{3s}]
[laih/and]
[gui/carry.on.back]
[r\^{e}o/back.basket]
[\~{n}u./\textsc{3s}]
[Todang/while]
[bboi/at]
[r\^{o}k/along]
[jolan/trail]
[\~{n}u/\textsc{3s}]
[nao/go]
[hma,/field]
[\~{n}u/\textsc{3s}]
[bb\^{o}h/see]
[sa/one]
[droi/\textsc{clf}]
[mr\u{a}/peacock]
[d\u{o}/stay]
[bboi/at]
[gah,/\textsc{drct}]
[a,/--]
[hruh/nest]
[\~{n}u./\textsc{3s}]
\glft
`There was an old person who went to work in the field. He took
along his machete, he took along his hoe, and he carried his
basket on his back. While he was on his way to the farm, he saw a
peacock beside its nest.'
\endgl
\xe
\endframedisplay
\def\goop{\quad $\vdots$ \par}%
\codedisplay~
\ex[glstyle=nlevel,glhangstyle=cascade,
   glneveryline={\insertno,\it,},glwordalign=center,
   glnabovelineskip={,-1pt},glspace=!.4em]
\begingl
(1)[Hom\^{a}o/\textsc{exist}]
(2)[sa/one]
(3)[\v{c}\^{o}/\textsc{clf}]
(4)[p\^{o}/person]
|goop
(35)[gah,/\textsc{drct}]
(36)[a,/--]
(37)[hruh/nest]
(38)[\~{n}u./\textsc{3s}]
\glft  |dots
\endgl
\xe
|endcodedisplay

Numbering can be automated.  One way to do this is below.

\codedisplay
\ex[glstyle=nlevel,glhangstyle=cascade,glneveryline={\hfil\insertno,\it,},
   glnabovelineskip={,-1pt},glspace=!.4em]
\count255=1|putfnno
\def\insertno{\eightrm(\the\count255)\global\advance\count255 by 1}%
\begingl[]
[Hom\^{a}o/\textsc{exist}]
[sa/one]
[\v{c}\^{o}/\textsc{clf}]
[p\^{o}/person]
[tha/old]
[\~{n}u/\textsc{3s}]
|goop
|endcodedisplay
\vfootnote{\the\fnno}{\Tex\ reserves |\count255| for scratchwork.
Since the assignment needs to be global, it is important to use a
count register that does not serve another purpose.  If a named count
register is available, it would be better to use that register.}%
Note carefully that |\begingl| has been supplied with an empty
argument.  Otherwise the parser would have been confused and |[Hom|\dots\
would have been interpreted as an optional argument.  An alternative
would have been |\begingl{}[Hom|\dots\ or \thinspace|\begingl\relax[Hom|\dots\thinspace.






