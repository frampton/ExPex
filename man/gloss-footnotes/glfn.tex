
\fnno=15

This footnote demonstrates the use of \verb:\ex: in footnotes.\footnote{%
Some text.
\ex an example\xe
More text.
\pex
\a first part
\a second part
\xe
More text.
\pex
\a first part
\a second part
\xe
Finishing text.}

\endinput
Some text with a footnote.\footnote{This is a footnote.}

\ex
\allocfns{A}
\begingl
\gla sh\=o'k {\=ar\`ab{\v\i}y\`a}\glfnA //
\glb foot \textsc{ass.cl2}:car //
\glft `tire, wheel (lit: foot of car)' //
\endgl
\glfn{{\sl \=ar\`ab{\v\i}y\`a\/} is a fast-speech contraction of
{\sl \=a r\`ab{\v\i}y\`a\/}.}
\xe

Some text with a footnote.\footnote{This is a footnote.}

\ex
\allocfns{A,B}
\begingl
\gla {k\'\i} dh\`an {g\`aj\`a g\`aj\`a g\`aj\`a} y\`a-\`a c\'e'd-k{\'\i} bw\`a
m\`o \'a c\'esh m\'un 'k\=a'b\=us\glfnB //
\glb \textsc{comp} great {\textsc{idph}\glfnA} go.\textsc{sing}-\textsc{lnk}
break-\textsc{ad1} stomach.\textsc{part} \textsc{mo} \textsc{loc.cl2} ground
\textsc{dem.loc.anph} shattered //
\glft `and he flailed through the air, breaking into pieces there on the
ground.' //
\endgl
\glfn{{\sl g\`aj\`a g\`aj\`a g\`aj\`a\/} is an ideophone for a long (thin)
thing going through the air, such as a spear with a wobbling shaft, or a tall
person waving their arms about flailingly.}
\glfn{{\sl 'k\=a'b\=us\/} denotes something broken into pieces, or something
which has become fragile after a heavy impact of some kind.}
\xe

The order of the footnotes is more straightforward in the nlevel style.

\ex[glstyle=nlevel]
\allocfns{A,B}
\begingl
{k\'\i}[\sc comp]
dh\`an[great]
{g\`aj\`a g\`aj\`a g\`aj\`a}[\sc idph\glfnA]
y\`a-\`a[go.\sc sing-lnk]
c\'e'd-k{\'\i}[break-\sc ad1]
bw\`a[stomach.\sc part]
m\`o[\textsc{mo}]
\'a[\textsc{loc.cl2}]
c\'esh[ground]
m\'un[\textsc{dem.loc.anph}]
'k\=a'b\=us\glfnB[shattered]
\glft `and he flailed through the air, breaking into pieces there on the
ground.'
\endgl
\glfn{{\sl g\`aj\`a g\`aj\`a g\`aj\`a\/} is an ideophone for a long (thin)
thing going through the air, such as a spear with a wobbling shaft, or a tall
person waving their arms about flailingly.}
\glfn{{\sl 'k\=a'b\=us\/} denotes something broken into pieces, or something
which has become fragile after a heavy impact of some kind.}
\xe


Some text with a footnote.\footnote{This is a footnote.}
\bye
\ex
\begingl
\gla sh\=o'k \=ar\`ab{\v\i}y\`a$^{\dag}$ //
\glb foot \textsc{ass.cl2}:car //
\glft `tire,? wheel (lit: foot of car)' //
\endgl

\medskip
$\dag\,${\sl \=ar\`ab{\v\i}y\`a\/} is a fast-speech contraction of
{\sl \=a r\`ab{\v\i}y\`a}.
\xe

Some text with a footnote.\footnote{This is a footnote.}

\ex
\begingl
\gla {k\'\i} dh\`an {g\`aj\`a g\`aj\`a g\`aj\`a} y\`a-\`a c\'e'd-k{\'\i} bw\`a
m\`o \'a c\'esh m\'un 'k\=a'b\=us //
\glb \textsc{comp} great \textsc{idph}$^{\dag}$ go.\textsc{sing}-\textsc{lnk}
break-\textsc{ad1} stomach.\textsc{part} \textsc{mo} \textsc{loc.cl2} ground
\textsc{dem.loc.anph} shattered$^{\ddag}$ //
\glft `and he flailed through the air, breaking into pieces there on the
ground.' //
\endgl
\medskip
$\dag\,${\sl g\`aj\`a g\`aj\`a g\`aj\`a\/} is an ideophone for a long (thin)
thing going through the air, such as a spear with a wobbling shaft, or a tall
person waving their arms about flailingly.

$\ddag\,${\sl 'k\=a'b\=us\/} denotes something broken into pieces, or something
which has become fragile after a heavy impact of some kind.
\xe

Some text with a footnote.\footnote{This is a footnote.}

\bye

