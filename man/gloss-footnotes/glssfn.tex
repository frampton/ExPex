
\let\\=\textsc
\lingset{everypanel=\footnotesize,ssratio=.5,glhangstyle=none}

\ex[glstyle=wrap]
\beginglpanel
\gla Hom\^{a}o$^1$ sa \v{c}\^{o} p\^{o} tha  \~{n}u nao ng\u{a}
hmua. \~{N}u dj\u{a} g\u{a}, \~{n}u dj\u{a} \v{c}\u{o}ng \~{n}u,
laih gui r\^{e}o \~{n}u. Todang bboi r\^{o}k jolan \~{n}u nao
hma, \~{n}u bb\^{o}h sa droi mr\u{a} d\u{o}$\,^4$ bboi gah, a, hruh
\~{n}u.//
\glb \\{exist} one \\{clf} person old \\{3s} go$^2$ do field
\\{3s} hold machete \\{3s} hold hoe \\{3s} and$^3$ carry.on.back
back.basket \\{3s} while at along trail \\{3s} go field \\{3s}
see one \\{clf} peacock stay at \\{drct} -- nest \\{3s}
//
\endgl
1.\enspace {\it hom\^{a}o} also means `have', reflecting the
strong tendency across languages to use the same word for
possession and the existential. {\it hom\^{a}o} is clause-initial
in existential clauses, but it comes after the subject in
possession clauses.

2.\enspace All verbs are glossed with a bare form, as Jarai has
no inflectional morphology. Although Jarai has lexical items that
encode tense, they are relatively infrequent in text.

3.\enspace The word {\it laih} is literally `after; finish', but
that is clearly not the meaning here. Probably {\it laih} here is
an abbreviation for {\it laih an\u{u}n}, `after that; and', hence
the gloss `and'.

4.\enspace {\it d\u{o}} `sit, stay' is used like a copula in
locative clauses, which is what I assume here (`a~peacock
[which was] beside its nest'); however, this could just as well
mean `a peacock sitting beside its nest', retaining the posture
semantics.
\endpanel
\bigskip
`There was an old person who went to work in the field. He took
along his machete, he took along his hoe, and he carried his
basket on his back. While he was on his way to the farm, he saw a
peacock beside its nest.'
\xe

\ex[glstyle=nlevel]
\beginglpanel
Hom\^{a}o$^1$[\\{exist}]  sa[one]  \v{c}\^{o}[\\{clf}]
p\^{o}[person]  tha[old]  \~{n}u[\\{3s}]  nao[go$^2$]
ng\u{a}[do]  hmua.[field]  \~{N}u[\\{3s}]  dj\u{a}[hold]
g\u{a},[machete]  \~{n}u[\\{3s}]  dj\u{a}[hold]
\v{c}\u{o}ng[hoe]  \~{n}u,[\\{3s}]  laih[and$^3$]
gui[carry.on.back]  r\^{e}o[back.basket]  \~{n}u.[\\{3s}]
Todang[while]  bboi[at]  r\^{o}k[along]  jolan[trail]
\~{n}u[\\{3s}]  nao[go]  hma,[field]  \~{n}u[\\{3s}]
bb\^{o}h[see]  sa[one]  droi[\\{clf}]  mr\u{a}[peacock]
d\u{o}$\,^4$[stay]  bboi[at]  gah,[\\{drct}]  a,[--]
hruh[nest]  \~{n}u.[\\{3s}]
\endgl
1.\enspace {\it hom\^{a}o} also means `have', reflecting the
strong tendency across languages to use the same word for
possession and the existential. {\it hom\^{a}o} is clause-initial
in existential clauses, but it comes after the subject in
possession clauses.

2.\enspace All verbs are glossed with a bare form, as Jarai has
no inflectional morphology. Although Jarai has lexical items that
encode tense, they are relatively infrequent in text.

3.\enspace The word {\it laih} is literally `after; finish', but
that is clearly not the meaning here. Probably {\it laih} here is
an abbreviation for {\it laih an\u{u}n}, `after that; and', hence
the gloss `and'.

4.\enspace {\it d\u{o}} `sit, stay' is used like a copula in
locative clauses, which is what I assume here (`a~peacock
[which was] beside its nest'); however, this could just as well
mean `a peacock sitting beside its nest', retaining the posture
semantics.
\endpanel

\bigskip
`There was an old person who went to work in the field. He took
along his machete, he took along his hoe, and he carried his
basket on his back. While he was on his way to the farm, he saw a
peacock beside its nest.'
\xe
\endinput


\endinput


\ex[everypanel=\footnotesize,glstyle=nlevel,glhangstyle=none,
   ssratio=.5]<panelex>
\beginglpanel
%\ex
%\begingl
Hom\^{a}o$^1$[\\{exist}]  sa[one]  \v{c}\^{o}[\\{clf}]
p\^{o}[person]  tha[old]  \~{n}u[\\{3s}]  nao[go$^2$]
ng\u{a}[do]  hmua.[field]  \~{N}u[\\{3s}]  dj\u{a}[hold]
g\u{a},[machete]  \~{n}u[\\{3s}]  dj\u{a}[hold]
\v{c}\u{o}ng[hoe]  \~{n}u,[\\{3s}]  laih[and$^3$]
gui[carry.on.back]  r\^{e}o[back.basket]  \~{n}u.[\\{3s}]
Todang[while]  bboi[at]  r\^{o}k[along]  jolan[trail]
\~{n}u[\\{3s}]  nao[go]  hma,[field]  \~{n}u[\\{3s}]
bb\^{o}h[see]  sa[one]  droi[\\{clf}]  mr\u{a}[peacock]
d\u{o}$\,^4$[stay]  bboi[at]  gah,[\\{drct}]  a,[--]
hruh[nest]  \~{n}u.[\\{3s}]
\endgl
%\xe
%\endinput
1.\enspace {\it hom\^{a}o} also means `have', reflecting the
strong tendency across languages to use the same word for
possession and the existential. {\it hom\^{a}o} is clause-initial
in existential clauses, but it comes after the subject in
possession clauses.

2.\enspace All verbs are glossed with a bare form, as Jarai has
no inflectional morphology. Although Jarai has lexical items that
encode tense, they are relatively infrequent in text.

3.\enspace The word {\it laih} is literally `after; finish', but
that is clearly not the meaning here. Probably {\it laih} here is
an abbreviation for {\it laih an\u{u}n}, `after that; and', hence
the gloss `and'.

4.\enspace {\it d\u{o}} `sit, stay' is used like a copula in
locative clauses, which is what I assume here (`a~peacock
[which was] beside its nest'); however, this could just as well
mean `a peacock sitting beside its nest', retaining the posture
semantics.
\endpanel
\bigskip
`There was an old person who went to work in the field. He took
along his machete, he took along his hoe, and he carried his
basket on his back. While he was on his way to the farm, he saw a
peacock beside its nest.'
\xe

