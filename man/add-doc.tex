
\ex[glstyle=nlevel,glhangindent=0pt,extraglskip=1ex]
\begingl
Mary$_i$[Mary] ist[is] sicher,[sure]
dass[that] es[it] den[the-\sc acc] Hans[Hans] nicht[not]
   st\"oren[annoy] w\"urde[would]
seiner[his-\sc dat] Freundin[girlfriend-\sc dat] ihr$_i$[her-\sc acc]
   Herz[heart-\sc acc] auszusch\"utten.[out to throw]
\endgl
\xe


%
%
%\ex[glstyle=wrap]
%\begingl
%\gla Mary$_i$ ist sicher, dass es den Hans nicht st\"oren w\"urde
%seiner Freundin ihr$_i$ Herz auszusch\"utten.//
%\glb Mary is sure that it the-{\sc acc} Hans not annoy would
%his-{\sc dat} girlfriend-{\sc dat} her-{\sc acc} heart-{\sc acc} {out to
%throw}//
%\glft  `Mary is sure that it would not annoy John to reveal her
%heart to his girlfriend.'//
%\endgl
%\xe

\endinput




%
%
%\ex[everygla=,glhangstyle=normal]<@period>
%\begingl
%\gla Fa'nu'i yu' ni [ [ {\it O} t{\it in\/}aitai-mu {\it t\/} ] na
%lepblu ] @ .//
%\glb show me Obl Op {\it WH\/}[obj].read-agr {} L book//
%\glft ``Show me the book that you read.''//
%\endgl
%\xe
%
%\ex \begingl[glstyle=nlevel]
%Fa'nu'i[show]
%yu'[me]
%ni[Obl]
%{[[\thinspace}[]@a
%{\it O\/}[Op]
%t{\it in\/}aitai-mu[{\it WH\/}{[obj]}.read-agr]
%{\it t\thinspace}[]@
%{]}[]
%na[L]
%lepblu{]}.[book]
%\endilg
%\glft ``Show me the book that you read.''//
%\endgl
%\xe
%
%
%
%
%
%
%

%\glft ``Show me the book that you read.''//
%\endgl
%
%
%\endinput

\section Additonal options for glosses

\subsection Glosses which span page breaks

\lingset{glbreaking=true}

\begininventory
\parameters*
\idx{|glbreaking|}& |true| or |false|& |false|\rlap{\quad (with default
value |true|)}\cr
\endinventory
In the wrap style (|glstyle=wrap|), there is now the option of
building glosses in a vbox or not.  Building a gloss in a vbox has the
advantage that its width can be more easily controlled, if this is
desired, and of preventing the gloss from being split between two
pages, if this is desired.  Sometimes, however, particularly with the
long glosses that result from glossing a narrative, it is desirable
for a glosses to continue over a page break.  The parameter
|glbreaking| controls whether or not the gloss is built in a vbox.  If
not (i.e. |glbreaking| is set to |true|), the gloss is generated as a
sequence of word glosses, which are fed into Tex's usual page building
machinery.

This is illustrated below.  The page is temporarily narrowed
(|\hsize=2.5in|) for illustrative purposes.  The coding is given
first.


\codedisplay
\ex
\hsize=2.5in
\let\\=\textsc
\begingl[glhangindent=0pt,glbreaking=true]
\gla Hom\^{a}o sa \v{c}\^{o} p\^{o} tha  \~{n}u nao ng\u{a}
hmua. \~{N}u dj\u{a} g\u{a}, \~{n}u dj\u{a} \v{c}\u{o}ng \~{n}u,
laih gui r\^{e}o \~{n}u. Todang bboi r\^{o}k jolan \~{n}u nao
hma, \~{n}u bb\^{o}h sa droi mr\u{a} d\u{o} bboi gah, a, hruh
\~{n}u.//
\glb \\{exist} one \\{clf} person old \\{3s} go do field
\\{3s} hold machete \\{3s} hold hoe \\{3s} and carry.on.back
back.basket \\{3s} while at along trail \\{3s} go field \\{3s}
see one \\{clf} peacock stay at \\{drct} -- nest \\{3s}
//
\endgl
\xe
|endcodedisplay

\ex
\hsize=2.5in
\let\\=\textsc
\begingl[glhangindent=0pt,glbreaking=true]
\gla Hom\^{a}o sa \v{c}\^{o} p\^{o} tha  \~{n}u nao ng\u{a}
hmua. \~{N}u dj\u{a} g\u{a}, \~{n}u dj\u{a} \v{c}\u{o}ng \~{n}u,
laih gui r\^{e}o \~{n}u. Todang bboi r\^{o}k jolan \~{n}u nao
hma, \~{n}u bb\^{o}h sa droi mr\u{a} d\u{o} bboi gah, a, hruh
\~{n}u.//
\glb \\{exist} one \\{clf} person old \\{3s} go do field
\\{3s} hold machete \\{3s} hold hoe \\{3s} and carry.on.back
back.basket \\{3s} while at along trail \\{3s} go field \\{3s}
see one \\{clf} peacock stay at \\{drct} -- nest \\{3s}
//
\endgl
\xe


If gloss breaking is in effect, the setting of |glwidth| has no
effect.



\subsection glstyle = nlevel

\begininventory
\macros*
\idx{|\glnft|}\endmc
\parameters
|glneveryline|& list of token lists& |{}|\cr
|glnabovelineextraskip|& list of dimensions& |{}|\cr
\endinventory

We begin by recoding (\getref{wapm}) from Section
\getref{basicglosssec}.

\framedisplay
\ex[glstyle=nlevel]
\begingl
k-[CL/2] wapm[V/see] -a[AGR/\sc 3acc] -s'i[NEG]
-m[AGR/\sc 2pl] -wapunin[TNS/preterit] -uk[AGR/\sc 3pl]
\endgl
\glnft
`you (pl) didn't see them'
\xe
\endframedisplay
\codedisplay~
\ex[glstyle=nlevel]
\begingl
k-[CL/2] wapm[V/see] -a[AGR/\sc 3acc] -s'i[NEG]
-m[AGR/\sc 2pl] -wapunin[TNS/preterit] -uk[AGR/\sc 3pl]
\endgl
\glnft `you (pl) didn't see them'
\xe
|endcodedisplay
%\endinput
%
%
%\framedisplay
%\ex[glstyle=nlevel]
%\begingl
%k-[CL/2] wapm[V/see] -a[AGR/\sc 3acc] -s'i[NEG]
%-m[AGR/\sc 2pl] -wapunin[TNS/preterit] -uk[AGR/\sc 3pl]
%\endilg
%\glft `you (pl) didn't see them'//
%\endgl
%\xe
%\endframedisplay
%\codedisplay
%\ex[glstyle=nlevel]
%\begingl
%k-[CL/2] wapm[V/see] -a[AGR/\sc 3acc] -s'i[NEG]
%-m[AGR/\sc 2pl] -wapunin[TNS/preterit] -uk[AGR/\sc 3pl]
%\endilg
%\glft `you (pl) didn't see them'//
%\endgl
%\xe
%|endcodedisplay
The big advantage to this style of coding complex glosses is that the
gloss of a word and the word are adjacent in the code. This makes the
code much more readable.  In effect, an aspect of WYSIWG is built into
the coding.

The list between |\begingl| and |\endgl| is processed as a space
separated list.  Spaces inside |[|\dots|]| are effectively hidden from
this parsing.  So, for example, the space in |-a[AGR/\sc 3acc]| does
not mislead the parser.  The material inside |[|\dots|]| is processed
as a |/| separated list.  Of course, |/| in this material must be
hidden from the parser so |/| cannot appear at the top level.  The
same is true of |]|, for obvious reasons.

\def\<#1>{$\langle$#1$\rangle$}

Formally, the syntax of nlevel glosses can be given as a system of
rewriting rules.  Quantities in angle brackets are either specified in
other rewriting rules or explained in words. \<spaces> consists of one
or more spaces.  $\vert$ is logical or.

\bigskip
\begingroup
\leftskip=\parindent
\parindent=0pt
\<gloss display> $\to$ |\begingl| \<interlinear gloss> \<spaces> |\endgl|

\<interlinear gloss> $\to$ \<word gloss> $\vert$
   \<word gloss> \<spaces> \<interlinear gloss>

\<word gloss> $\to$
   \<word> |[| \<gloss> |]| $\vert$
   \<word> |[| \<gloss> |]| \<diacritic>

\<gloss> $\to$ \<gloss item> $\vert$ \<gloss item> |/| \<gloss>

\<diacritic> $\to$ |@| $\vert$ |+|
\bigskip
\endgroup
\noindent
\<word>, and \<gloss item> must be material
which can appear in the context |\hbox{|\dots|}|. \<word> cannot
contain top-level spaces. \<gloss item> cannot contain either
top-level |/| or top-level |]| (but can contain top-level spaces).





The parameter |glneveryline| is illustrated below.

\framedisplay
\ex[glstyle=nlevel,glneveryline={\it,\sc,\sc}]
\begingl
k-[cl/2]
wapm[v/\rm see]
-a[agr/3acc]
-s'i[neg]
-m[agr/\sc 2pl]
-wapunin[tns/preterit]
-uk[agr/3pl]
\endgl
\xe
\endframedisplay
\codedisplay
\ex[glstyle=nlevel,glneveryline={\it,\sc,\sc}]
\begingl
k-[cl/2]
wapm[v/\rm see]
-a[agr/3acc]
-s'i[neg]
-m[agr/\sc 2pl]
-wapunin[tns/preterit]
-uk[agr/3pl]
\endgl
\xe
|endcodedisplay

The use of |glnabovelineskip| is illustrated below.  The vertical
space underneath the words is closed up by putting a vertical skip of
-2pt over the second line of the gloss.

\framedisplay
\ex[glstyle=nlevel,glneveryline={\it,\sc,\sc},
   glnabovelineextraskip={,-2pt}]
\begingl
k-[cl/2]
wapm[v/\rm see]
-a[agr/3acc]
-s'i[neg]
-m[agr/\sc 2pl]
-wapunin[tns/preterit]
-uk[agr/3pl]
\endgl
\xe
\endframedisplay
\codedisplay~
\ex[glstyle=nlevel,glneveryline={\it,\sc,\sc},
   glnabovelineextraskip={,-2pt}]
\begingl
k-[cl/2]
wapm[v/\rm see]
-a[agr/3acc]
-s'i[neg]
-m[agr/\sc 2pl]
-wapunin[tns/preterit]
-uk[agr/3pl]
\endgl
\xe
|endcodedisplay
Note that a list of dimensions |{,-2pt}| with dimensions missing is
acceptable; the missing dimensions are assumed to be |0pt|.  Missing
entries in the value of |glneveryline|, either because the list is
shorter than the number of lines in the interlinear gloss or because
of null entries on the list, are similarly assumed to be null and
cause no problem.  The material inside the brackets, or material
delimited by |/| can be missing as well, but the brackets are
mandatory.  Spaces after |[| or |/| are ignored, so for example,
|wapm[v/\rm see]| and |wapm[ v/ \rm see]| produce the same output.

One spacing problem that is handled automatically in the wrap style
must be handled by the user in the nlevel style.  It is not common,
but is worth mentioning because it might cause a perplexing problem if
it does arise.  Characters on lower levels which are extra tall can
cause misalignment of the baselines.  Suppose, for example, that you
want to use a gloss to give some information about the
morphophonological derivation of a surface form, as in (\nextx).

\def\AccentedBarredW{$\acute{\hbox{$\overline w$}}$}

\framedisplay
\ex[glstyle=nlevel,glneveryline={\it}]
\begingl m-[(mo-)] wope[(a\AccentedBarredW ope)] \endgl \xe
\endframedisplay
\codedisplay
\ex[glstyle=nlevel,glneveryline={\it}]
\begingl m-[(mo-)] wope[(a\AccentedBarredW ope)] \endgl \xe
|endcodedisplay
Not only is more space between the lines needed, but there is
misalignment of the baselines on the second level.

|glneveryline={,5pt}| would give more space over the second level but
it will not fix the misaligned baselines. But |glbaselineskip| can be
easily adjusted to give vertical room for the second line.

\framedisplay
\ex[glstyle=nlevel,glbaselineskip=!6pt]
\begingl m-[(mo-)] wope[(a\AccentedBarredW ope)] \endgl \xe
\endframedisplay
\codedisplay
\ex[glstyle=nlevel,glbaselineskip=!6pt]
\begingl m-[(mo-)] wope[(a\AccentedBarredW ope)] \endgl
\xe
|endcodedisplay
Note that the baselineskip within the word glosses in a particular
example can be set using the |!|dimension syntax as an offset from the
setting of |glbaselineskip| in the environment in which the example
occurs.

The nlevel style handles line breaking and hanging indentation as
expected.

\framedisplay
\ex
\let\\=\textsc
\lineskip=1ex
\begingl[glstyle=nlevel,glneveryline={\it},glhangindent=1em]
Hom\^{a}o[\\{exist}]  sa[one]  \v{c}\^{o}[\\{clf}]  p\^{o}[person]
tha[old]  \~{n}u[\\{3s}]  nao[go]  ng\u{a}[do]  hmua.[field]
\~{N}u[\\{3s}]  dj\u{a}[hold]  g\u{a},[machete]  \~{n}u[\\{3s}]
dj\u{a}[hold]  \v{c}\u{o}ng[hoe]  \~{n}u,[\\{3s}]  laih[and]
gui[carry.on.back]  r\^{e}o[back.basket]  {\~{n}u. }[\\{3s}]
Todang[while] bboi[at]  r\^{o}k[along]  jolan[trail]  \~{n}u[\\{3s}]
nao[go] hma,[field]  \~{n}u[\\{3s}]  bb\^{o}h[see]  sa[one]
droi[\\{clf}] mr\u{a}[peacock]  d\u{o}[stay]  bboi[at]  gah,[\\{drct}]
a,[] hruh[nest]  \~{n}u.[\\{3s}]
\endgl \xe
\endframedisplay
\codedisplay~
\ex
\let\\=\textsc
\lineskip=1ex
\begingl[glstyle=nlevel,glneveryline={\it},glhangindent=1em]
Hom\^{a}o[\\{exist}]  sa[one]  \v{c}\^{o}[\\{clf}]  p\^{o}[person]
tha[old]  \~{n}u[\\{3s}]  nao[go]  ng\u{a}[do]  hmua.[field]
\~{N}u[\\{3s}]  dj\u{a}[hold]  g\u{a},[machete]  \~{n}u[\\{3s}]
dj\u{a}[hold]  \v{c}\u{o}ng[hoe]  \~{n}u,[\\{3s}]  laih[and]
gui[carry.on.back]  r\^{e}o[back.basket]  {\~{n}u. }[\\{3s}]
Todang[while] bboi[at]  r\^{o}k[along]  jolan[trail]  \~{n}u[\\{3s}]
nao[go] hma,[field]  \~{n}u[\\{3s}]  bb\^{o}h[see]  sa[one]
droi[\\{clf}] mr\u{a}[peacock]  d\u{o}[stay]  bboi[at]  gah,[\\{drct}]
a,[] hruh[nest]  \~{n}u.[\\{3s}]
\endgl \xe
|endcodedisplay

\subsection The diacritics |@| and |+|


\def\ge#1{\langle\hbox{\sl #1}\,\rangle}

\noindent A word gloss $\ge{word}|[|\ge{gloss$_1$}|/|\ldots|]|$ can
optionally be immediately followed by one of |@| and |+|.  A |@|
cancels the usual space between the typeset word glosses.  A |+|
introduces a line break.  This should be compared with the slightly
different syntax of these diacritics in the wrap style; see sections ?
and ?.  The folowing code produces the same output as the code which
is given for (\getref{wiye}).

\codedisplay
\ex[glstyle=nlevel]
\begingl wiye[two] kepi[whitemen] e-[\sc 1p:3d-]@ ca[found] \endgl
\xe
|endcodedisplay
\vfil \break

\ex[glstyle=wrap]
\begingl
\gla Mary$_i$ ist sicher, + dass es den Hans nicht st\"oren w\"urde
+ seiner Freundin ihr$_i$ Herz auszusch\"utten.//
\glb Mary is sure that it the-{\sc acc} Hans not annoy would
his-{\sc dat} girlfriend-{\sc dat} her-{\sc acc} heart-{\sc acc} {out to
throw}//
\glft  `Mary is sure that it would not annoy John to reveal her
heart to his girlfriend.'//
\endgl
\xe

\ex[glstyle=nlevel,glhangindent=0pt]
\begingl
Mary$_i$[Mary] ist[is] sicher,[sure]+
dass[that] es[it] den[the-\sc acc] Hans[Hans] nicht[not]
   st\"oren[annoy] w\"urde[would]+
seiner[his-\sc dat] Freundin[girlfriend-\sc dat] ihr$_i$[her-\sc acc]
   Herz[heart-\sc acc] auszusch\"utten.[out to throw]
\endgl
\glnft  `Mary is sure that it would not annoy John to reveal her
%\glft  `Mary is sure that it would not annoy John to reveal her
heart to his girlfriend.'
%\endgl
\xe


\endinput

