%
%\beginss
%\pex[labeltype=alpha]
%\a First part.
%\a Second part.
%\xe |midss
%\pex[labeltype=alpha]
%\a First part.
%\a Second part.
%\xe
%\endss
%
%\beginss
%\pex[exno=1,labeltype=footnote]
%\a first
%\a second
%\a third
%\a fourth
%\xe|midss
%\pex[labeltype=footnote]
%\pex[exno=1,labeltype=footnote]
%\a first
%\a second
%\a third
%\a fourth
%\xe
%\endss
%
%\endinput


\makeatletter
\def\showpex{%
   \edef\resetexcnt{\noexpand\global\noexpand\excnt=\the\excnt}%
   \quad
   \psscalebox{1.5}{%
   \vrule height.5em depth4.6em width0pt
   \parindent=0pt
   \leavevmode
   \Lingset{numoffset=4.5em,preambleoffset=4em,labeloffset=3em,
      textoffset=4em,labelwidth=.8em,arrows=<->}
   \pnode(0,0){A}
   \lower10ex\vbox{\hsize=3.8in
   \excnt=23
   \pex
   \pnode(-\lingpreambleoffset,0){B3}\pnode{B4}
   This is the preamble.
   \a \pnode{E4}%
   This is an example.%
   \SpecialCoor
   \rput(A|B3){\pnode{B1}}
   \rput(B1){\pnode(\lingnumoffset,0){B2}}
   \rput(E4){\pnode(-\lingtextoffset,0){E3}}
   \rput(E3){\pnode(-\linglabelwidth,0){E2}}
   \rput(E2){\pnode(-\linglabeloffset,0){E1}}
   \psset{nodesep=0,labelsep=0}
   \pcline[offset=2.5ex](B1)(B2)
   \Aput{\strut\eighttt numoffset}
   \pcline[offset=2.5ex](B3)(B4)
   \Aput{\strut\eighttt preambleoffset}
   \pcline[offset=-1.5ex](E1)(E2)
   \Bput{\strut\eighttt labeloffset}
   \pcline[offset=-1.5ex](E3)(E4)
   \Bput{\strut\eighttt textoffset}
   \psset{offset=0,angle=90,linestyle=dotted,arrows=-}
   \XKV@for@n{1,2,3,4}\which{%
      \pcline(B\which)([nodesep=3ex]B\which)}%
   \XKV@for@n{1,2,3,4}\which{%
      \pcline([nodesep=1.5ex]E\which)([nodesep=-2ex]E\which)}%
   \xe}}%
   \resetexcnt
}
\resetatcatcode
\def\\#1em{#1\hskip.2em em}
%\lingset{lingstyle=fdabovecd}
%%%%%%%%%%%%%%%%%%%%%%%%%%%%%%%%%%%%%%%%%%%%%%%%%%%%%%%%%%%%%%
\section Examples with (and without) labeled parts:
Advanced\deftagsec{advanced}

\subsection Relative versus fixed dimensions

{\it Tex\/} has two kinds of dimensional units.  Dimensions
specified in term of ``pts'', ``inches'', ``cm'', etc. are fixed.
Dimensions specified in terms of the units ``em'' or ``ex'' are
relative to the particular text font that is current.
Historically, an em is the width of a capital M and an ex is the
height of a lowercase x.  This is still more or less true, but
each font is free to specify the equivalents in any way that it
sees fit.  The difference has important implications for
parameter setting.  A value specified in terms of em or ex units
can be used without change in both the main text and footnotes,
for example.  If we set |textoffset=1em| at the beginning of a
document, the proportions will stay the same whether we used 10pt
or 12pt type, or 8 or 9 pt type in footnotes.
If the document is set in 12pt type, and 1em is specified to be
12pt in that main text font, it makes no difference for
typesetting in the main text font whether we set |textoffset=1em|
or |textoffset=12pt|.  But it does make a difference in sections
of the document where a font with a different em dimension is
used.  If we switch to a font with a 10pt em unit, then the first
specification will give a physical offset of 10pt, but the second
will give a physical offset of 12pt.

Setting a dimension or skip parameter by incrementing the old
value specifies the new value as a fixed dimension, which will
not scale with font changes.  The same is true of specifying
setting |labelwidth| via |samplelabel|.  Adjusting a parameter by
incrementing the old value, or using |samplelabel| to set the
width of the label slot, should only be used to make local
adjustments, not at a level which has font size changes in its
scope.  For this reason, {\sl expex.tex\/} specifies the label width
in the label type |alpha| by |labelwidth=.72em|, not by
|samplelabel=a.|.\footnote{In the font that I happen to be using to
write this section, the alphabetical labels (including the period)
a--d have widths varying from \\.694em to \\.75em; the capital labels
A--D have widths varying from \\.917em to \\.972em; and the integer
labels for 1 to 9 all have width \\.75em. The label width settings for
the label types |alpha| (\\.72em) and |caps| (\\.94em) are a good
compromise.} Setting |labelwidth| via the second method would only be
satisfactory if the fonts in force at the points that |\pex| is
used is the same as the font in force when expex.tex is input and
the |alpha| style defined.

\subsection Anchoring (|labelanchor|,|preambleanchor|, and
|textanchor|

\deftagsec{anchors}%
Normally, the left edge of the label slot is determined by its
offset from the right edge of the number, the left edge of the
preamble is also determined by its offset from the left edge of
the number, and the left edge of the text is determined by its
offset from the right edge of the label slot.  The ``anchor'' for
the offset is parametrized.  Below, for example, is a more
refined version of example on page \getref{partialrepeat}; the
labels are anchored at the left edge of the example number rather
than the right edge.

\begingroup
\setss .55 .38
\beginss
\pex[exno={[47, partially repeated from
   p. 32]},labelanchor=numleft,
   exnoformat=X,labeloffset=1.5em]
\par
\a[label=b] first
\a[label=d] second
\a[label=g] third
\xe|midss
%\pex[exno={47, partially repeated from
%   p. 32},labelanchor=numleft,exnoformat={[X]},
%   labeloffset=1.5em]
\pex[exno={[47, partially repeated from
   p. 32]},labelanchor=numleft,exnoformat=X,
   labeloffset=1.5em]
\par
\a[label=b] first
\a[label=d] second
\a[label=g] third
\xe
\endss
\endgroup
\noindent Braces are needed in the specification of the
value of |exno|, otherwise |\pex| would think that the key |exno|
was set to the value |[47| and try to set the key |partially
repeated from p. 32| to its default value. |exnoformat| is set to |X|
so that parentheses are not put around the special exno.

The following would work equally well:

\def\stuff{$\vdots$}
\codedisplay
\pex[exno={47, partially repeated from
   p. 32},labelanchor=numleft,exnoformat={[X]},
   labeloffset=1.5em]
          |stuff
|endcodedisplay
Braces are needed around |[X]| so that the mechanism that reads the
optional argument of |\pex| does not interpret the right bracket as
the right delimiter of the optional argument.

The possible setting of the anchor keys are given below. They are all
choice keys which must be set to one of the possibilities listed.

\exdisplay\noexno
Anchoring (choice) parameters:\medskip
%\leftskip=0pt
\qquad\vtop{\halign{#\hfil&& \qquad #\hfil\cr
\hfil\it key& \hfil\it allowed values& \hfil\it initial value\cr
\noalign{\smallskip}
\idx{|labelanchor|}& |numright|, |numleft|, or |margin|&
   |numright|\cr
\idx{|preambleanchor|}& |numright|, |labelleft|, or |text|&
   |numright|\cr
\idx{|textanchor|}& |numleft| or |normal|& |normal|\cr
}}\xe
The |normal| setting of the |textanchor| parameter is the right
edge of the number inside |\ex| and right edge of the label slot
inside
|\pex|.

Here is another example:
\beginss
\lingset{textanchor=numleft,
   labelanchor=numleft,
   labeloffset=.35in,
   textoffset=.7in}
\pex[exno=9]
\a first
\a second
\xe
\bigskip
\pex[exno=10]
\a first
\a[label=aa] second
\xe|midss
\lingset{textanchor=numleft,
   labelanchor=numleft,
   labeloffset=.35in,
   textoffset=.7in}
\pex[exno=9]
\a first
\a second
\xe
\bigskip
\pex[exno=10]
\a first
\a[label=aa] second
\xe
\endss
\noindent Some publications
demand this style.  It is left over from typewriter days with
mechanical tabs (and from {\sl Word}, which is still rooted in that
typographic period).

\subsection Formatting the labels (|everylabel| and
|labelformat|)

%\parinventory
%& \idx{|everylabel|}& token list\qquad& |{}|\cr
%& \idx{|labelformat|}& formatting string (see below)\hidewidth\cr
%\endparinventory

\noindent The value of |everylabel| is a token list.  It is
inserted just before labels are typeset.  It is grouped so that
it affects only the label.  One use is to set the font used for
the labels if it differs from the font in the running text.  For
example:

\beginss
\pex[everylabel=\it]
\a one
\a two
\xe|midss
\pex[everylabel=\it]
\a one
\a two
\xe
\endss

There are other uses aside from setting the label font.

\beginss
\pex[everylabel=A,labeltype=numeric,
   samplelabel=A1.]
\a An example
\a An example
\a An example
\xe|midss
\pex[everylabel=A,labeltype=numeric,
   samplelabel=A1.]
\a An example
\a An example
\a An example
\xe
\endss

The effect of the value of |labelformat| is illustrated in
(\nextx).

\setss .6 .35

\beginss
\pex[labelformat=$\langle$A$\rangle$,
   samplelabel=$\langle$a$\rangle$]
\a first
\a second
\xe|midss
\pex[labelformat=$\langle A\rangle$,
   samplelabel=$\langle A\rangle$]
\a first
\a second
\xe
\endss

The example above is fanciful, but one sometimes see formats like the
following.

\beginss
\pex[exno=1,exnoformat=X.,labeltype=roman,
   labelformat=(A),samplelabel=(iii)]
\a first
\a second
\a third
\a fourth
\xe|midss
\pex[exno=1,exnoformat=X.,labeltype=roman,
   labelformat=(A),samplelabel=(iii)]
\a first
\a second
\a third
\a fourth
\xe
\endss

The label formatting mechanism is primitive.
|labelformat| must
be of the form\medskip
\centerline{$\rm \langle balanced\
text\rangle\thinspace|A|\thinspace \langle balanced\ text\rangle$}
\medskip
\noindent The pre-A text is inserted before the label (including
the material specified by |everypar|) and the post-A text is
inserted after the label. The balanced text cannot contain the
character |A|.  {\it Balanced text\/} is a string of tokens with
properly nested (explicit) braces. No error checking is done to
ensure that the format specification has the required form, so be
careful.  An error might lead to very obscure error messages.

Another style sometimes found in footnotes is like the one above,
except that the labels are right aligned in the label slot.

\subsection Aligning the labels (|labelalign|)

\parinventory*
& \idx{|labelalign|}& choice (\idx{|left|}, \idx{|right|},
\idx{|margin|})&
   \idx{|right|}\cr
\endparinventory
\bigskip

\noindent There is a choice of left, right, or center alignment of the
labels in the label slot.  This is chosen by the parameter
\idx{|labelalign|}, which can be set to |left|, |center|, or |right|.
For example,

\beginss
\pex[exno=43,labeltype=roman,
   labelformat=(A),labelalign=right,
   samplelabel=(iii)]
\a first
\a second
\a third
\a fourth
\xe|midss
\pex[exno=43,labeltype=roman,
   labelformat=(A),labelalign=right,
   samplelabel=(iii)]
\a first
\a second
\a third
\a fourth
\xe
\endss
This style looks odd to me, but this is the style for multipart examples
{\it in the main text\/} in Chomsky's {\it Lectures on Government
and Binding}.  In footnotes, the style is:

\beginss
\pex[exno=i,labeltype=alpha,
   samplelabel=(a),labelformat=(A)]
\a first
\a second
\a third
\a fourth
\xe|midss
\pex[exno=i,labeltype=alpha,
   samplelabel=(a),labelformat=(A)]
\a first
\a second
\a third
\a fourth
\xe
\endss

For more ordinary |\pex| constructions which use the letters or
numbers which have roughly the same width, label alignment is not
a significant concern.  But if labels include, for example, the
narrow letter ``i'' and the wide letter ``m'', as below, label
alignment has a noticible effect on the appearance.  Individual
tastes (and publisher's demands) may differ, but I prefer center
alignment in these cases.

\line{%
\lingset{belowpreambleskip=.4ex,interpartskip=.3ex,
   everyex=\advance\pexcnt by 8,samplelabel=m.}%
\hsize=.33\hsize
\vbox{%
\pex[labelalign=left]<triplea1>
{\it left aligned labels}
\a A typical example.
\a A typical example.
\a A typical example.
\a A typical example.
\a A typical example.
\a A typical example.
\xe}%
\vbox{%
\pex[labelalign=center]
{\it center aligned labels}
\a A typical example.
\a A typical example.
\a A typical example.
\a A typical example.
\a A typical example.
\a A typical example.
\xe}
\vbox{%
\pex[labelalign=right]<triplea3>
{\it right aligned labels}
\a A typical example.
\a A typical example.
\a A typical example.
\a A typical example.
\a A typical example.
\a A typical example.
\xe}}

If the labels are numeric, label alignment can have an even bigger
effect.  Again, individual tastes and publishers' demands may
differ.  My preference is right alignment in this case.

\exbreak[8ex]
\line{%
\lingset{interpartskip=.3ex,labeltype=numeric,
   everyex=\advance\pexcnt by 6,samplelabel=10.}%
\hsize=.33\hsize
\vbox{%
\pex[labelalign=left]<tripleb1>
{\it left aligned labels}
\a A typical example.
\a A typical example.
\a A typical example.
\a A typical example.
\a A typical example.
\a A typical example.
\xe}\hfil
\vbox{%
\pex[labelalign=center]
{\it center aligned labels}
\a A typical example.
\a A typical example.
\a A typical example.
\a A typical example.
\a A typical example.
\a A typical example.
\xe}\hfil
\vbox{%
\pex[labelalign=right]<tripleb3>
{\it right aligned labels}
\a A typical example.
\a A typical example.
\a A typical example.
\a A typical example.
\a A typical example.
\a A typical example.
\xe}}

The initial setting is left alignment for letters (either
uppercase or lowercase) and right alignment for numbers. Unless
numbers or letters of significantly different widths appear as
labels, most users will not notice the difference and can safely
ignore the issue.

%Particularly wide labels create another problem which needs
%attention.  Recall that the slot in which labels are typeset does
%not automatically adjust its width to fit the widest label.  It
%is fixed in advance by the setting of |labelwidth|. Wide labels
%will spill out of the label slot; to the right, left, or on both
%sides, depending on the setting of |labelalign|. Particularly in
%the case of double digit numeric labels, the width of the label
%slot needs to be changed from the default value. One can do this
%by saying something like |labelwidth=!1ex|, which will increase
%the label width by $1\,ex$.  This gives more or less
%satisfactory results. The pseudo parameter \idx{|samplelabel|} is
%provided to help in adjusting the label width.  If you say, for
%example,
%
%\codedisplay
%\lingset{samplelabel=10.}
%|endcodedisplay
%
%\noindent then |labelwidth| is set to the width of ``10.'' in the
%current font. Examples (\getref{triplea1}--\getref{triplea3}) used
%|\lingset{samplelabel=m.}| and
%(\getref{tripleb1}--\getref{tripleb3}) used
%|\lingset{samplelabel=10.}|

It is a side issue, but the reader may have wondered how
(\getref{triplea1}--\getref{triplea3}) and
(\getref{tripleb1}--\getref{tripleb3}) were typeset. The idea is
simple.  You say:

\codedisplay
\line{\divide\hsize by 3
   \vbox{\pex|verbdots \xe}\hss
   \vbox{\pex|verbdots\xe}\hss
   \vbox{\pex|verbdots \xe}}
|endcodedisplay
|\hss| is used to give a little stretch or shrink so
that dimensional rounding does not lead to an under or overfull
|\line{| \dots|}|.

Variations are useful.  One can easily imagine a situation in
which something like the following is appropriate for two side by
side examples.

\codedisplay
\line{%
   \vbox{\hsize=.55\hsize \pex|verbdots \xe}\hss
   \vbox{\hsize=.45\hsize \pex|verbdots \xe}}
|endcodedisplay


\subsection User designed labeling

\begingroup \exbreak \parindent=0pt
Macros:\quad \idx{|\definelabeltype|}, \idx{|\a[]|}\par\nobreak
Counter:\quad \idx{|\pexcnt|}\par\nobreak
\parinventory
& \idx{|labelgen|}& \idx{|char|}, \idx{|number|}, or \idx{|list|}&
   |char|\cr
& \idx{|pexcnt|}& integer& |97|\cr
& \idx{|labellist|}& comma separated list& |{}|\cr
\endparinventory
\endgroup

\noindent
If |labeltype| is set to |alpha|, the counter |\pexcnt| is set to
97, the character code of lowercase a in standard roman font
sets, and |labelgen| is set to |char|.  The successive labels are
generated by taking the character corresponding to |\pexcnt| and
stepping the counter by 1.  This relies on the fact that the
alphabetical sequence of characters corresponds to the numerical
order of the character codes of the characters.  Setting
|labeltype| to |caps| is almost the same, except that |\pexcnt|
is initialized to 65, the character code of uppercase A.  If
|labeltype| is set to |numeric|, |labelgen| is set to |number|
and |\pexcnt| is initialized to 1.  The labels are generated by
taking the number corresponding to value of |\pexcnt|.

\let\mit=\twelvei
Something like the following might be useful.  In the format I am
using, |\mit| selects a math italics font which has the lowercase
greek letters starting with the position 11.

\beginss
\pex[labelgen=char,pexcnt=11,
   everylabel=\mit]
\a
\a
\a
\a
\xe|midss
\pex[labelgen=char,pexcnt=11,
   everylabel=\mit]
\a
\a
\a
\a
\xe
\endss
\noindent This scheme has a quirk if labels above $\xi$ are
needed because the correspondence between numerical order and
alphabetical order breaks down at this point.  {\it o} will be
missing, with $\pi$ following $\xi$.

This labeling scheme can be defined by:
\medskip
\noindent
|\definelabeltype{greekmath}{labelgen=char,pexcnt=11,everylabel=\mit,|\par
\noindent
|   labelformat=A.}|
\medskip
\noindent Then |\pex[labeltype=greekmath]| is sufficient to
invoke this labeling style.

If a sequence of character labels is needed which does not appear
in sequence in a font, it is necessary to generate the labels
from a list.  Documents in Greek, for example, face the following
problem.  Roman letters were largely borrowed from the Greek
alphabet, but Greek alphabetical order was not.  Common greek
fonts place letters in the position of the borrowing, not in
their natural order in the Greek alphabet.  Documents written in
Greek, therefore, will have to generate the labels in multipart
examples from a list. The solution is to set |labelgen| to
|list|, and |labellist| to the desired list of labels.

For example, suppose the label type |greek| is defined by
\medskip
\noindent |\definelabeltype{greek}{labelgen=list,|\par
\noindent |   labellist={a,b,g,d,e,z,h,j,i,k,l,m,n,x,o,p,r,sv,t,u,f,q,y,w}}|
\medskip
\noindent and that |\gr| selects one of the grmn series of fonts
in the cb family of Greek fonts.

\begingroup
\let\gr=\twelverm  %% TEMPORARY %% TEMPORARY %% TEMPORARY %%
%\font\gr=grmn1200
\gr

\definelabeltype{greek}{labelgen=list,
   labellist={a,b,g,d,e,z,h,j,i,k,l,m,n,x,o,p,r,sv,t,u,f,q,y,w}}
\lingset{labeltype=greek}

\beginss
\gr
\pex[labeltype=greek]
\a {\rm a,b,g,d,e,z}
\a a,b,g,d,e,z
\a {\rm a,b,c,d,e,f}
\a a,b,c,d,e,f
\a {\rm A,B,G,D,E,Z}
\a A,B,G,D,E,Z
\xe|midss
\pex[labeltype=greek]
\a {\rm a,b,g,d,e,z}
\a a,b,g,d,e,z
\a {\rm a,b,c,d,e,f}
\a a,b,c,d,e,f
\a {\rm A,B,G,D,E,Z}
\a A,B,G,D,E,Z
\xe
\endss
\endgroup

\noindent ``sv'' appears in the list rather than ``v'' so that a
nonfinal sigma is produced rather than a final sigma.  They
differ.  ``v'' produces what amounts to a vertical strut of zero
width, making the sigma nonfinal, but contributing no visible
material. (Thanks to Christos Vlachos for this idea.)

Of course, if the entire document is written in Greek, then
\medskip
\noindent |\lingset{labeltype=greek}|
\medskip
\noindent should have global scope, so that parameter
setting is not necessary in each |\pex| construction.

The computation is faster if the label list is shorter, so
the label list should not be much longer than the longest list
you anticipate using.  If more labels are called for than the
list provides, labeling starts over again at the beginning and a
warning is issued.


\bigskip
\begininventory
\omit Macro: |\definelabeltype|\hidewidth\cr
\omit Count register: |\pexcnt|\hidewidth\cr
\omit Parameters:\hidewidth\cr
\hfil\it key& \hfil\it value& \hfil\it initial value\cr
\idx{|labelgen|}& |char|, |number|, or |list|& |char|\cr
\idx{|pexcnt|}& integer& |`a|\cr
\idx{|labellist|}& comma separated list& |{}|\cr
\endinventory

%\endinput
%\bigskip
%\noindent\hfil
%\psframebox[framesep=1ex,boxsep=true,doubleline=true]{\hbox{%
%\vtop{\halign{#\hfil&& \qquad #\hfil\cr
%\hfil\it key& \hfil\it value& \hfil\it initial value\cr
%\noalign{\smallskip}
%\idx{|labelgen|}& |char|, |number|, or |list|&
%   |char|\cr
%\idx{|pexcnt|}& integer& |`a|\cr
%\idx{|labellist|}& comma separated list& {}\cr
%}}}}
%\medskip

\subsection IJAL style format of multiline examples

\noindent The formatting demands of the International Journal of
American Linguistics (IJAL) require some additional
parametrization.  Multipart examples look like this:

\definelingstyle{IJAL}{%
   avoidnumlabelclash=true,
   labelanchor=numleft,labeloffset=0pt,
   textanchor=normal,textoffset=1.5em,
   preambleanchor=text,preambleoffset=0pt,
   labelformat=(A),
   everylabel=\actualexno,
   appendtopexarg={samplelabel=(\actualexno a)}
   }%
\begingroup
\lingset{lingstyle=IJAL}
\excnt=4
\pex<nonum>
\a first
\a second
\xe
\excnt=64
\pex~
Preamble
\a first
\a second
\xe
\excnt=1025
\pex~
Preamble
\a first
\a second
\xe
\endgroup

A first approximation is

\codedisplay
\lingset{labelwidth=2em,labelanchor=numleft,labeloffset=0pt,
   textanchor=normal,textoffset=1.5em,preambleanchor=text,
   preambleoffset=0pt,labelformat=(A),everylabel=\actualexno}
|endcodedisplay

\begingroup

\lingset{labelwidth=2em,labelanchor=numleft,labeloffset=0pt,
   labelformat=(A),everylabel=\actualexno,textanchor=normal,
   textoffset=1.5em,preambleanchor=text,preambleoffset=0pt}

%   appendtopexarg={samplelabel=(\actualexno a)}

\pex[exno=5]<nonum>
\a first
\a second
\xe

\pex~[exno=65]
Preamble
\a first
\a second
\xe

\pex~[exno=102]
Preamble
\a first
\a second
\xe

There are two things that need to be done.  First, in order to
avoid the overwriting in (\getref{nonum}), printing the example
number must be suppressed if there is no preamble. Second, in
order to avoid the shrinking gap between the label and the text
as more digits appear in the example number and it gets wider,
the label width must be made dependent on the particular example
number.

\expex\/ provides the boolean parameter |avoidnumlabelclash|
which, if set to |true|, suppresses printing the example number
in |\pex| constructions {\it if there is no preamble}.  It has
the XKV default value |true|, so that it can be set by giving the
key with no label.  Although it has the default value |true|, it
has the initial value |false|.  So, with the parameters set as
above, we get:

\beginss
\pex[exno=5,avoidnumlabelclash]
\a first
\a second
\xe|midss
\pex[exno=5,avoidnumlabelclash]
\a first
\a second
\xe
\endss

In order to solve the problem of the shrinking gap between the
label and text, we could try something like
|samplelabel=(\actualexno a)|. But it is considerably less than
elegant to have to write this in every |\pex| example.  It will
not produce the desired result to put
|\lingset{samplelabel=(\actualexno a)}| at the beginning of the
document.  |labelwidth| will be set to whatever the current width
of |(\actualexno a)| is, which is not likely to be what you want
for the entire document even if |\actualexno| happens to be
defined (which is unlikely) at the point the |\lingset| is
executed.  To circumvent the global/local problem, \expex\/
provides the parameter |appendtopexarg|.  Its value (unexpanded)
is appended to whatever other arguments are given to
|\pex| and evaluated locally.
\codedisplay
\lingset{appendtopexarg={samplelabel=(\actualexno a)}|endcodedisplay
accomplishes what we want.

We can summarize the discussion by defining the IJAL style.

\codedisplay
\definelingstyle{IJAL}{labelwidth=2em,labelanchor=numleft,
   labeloffset=0pt,labelformat=(A),everylabel=\actualexno,
   textanchor=normal,textoffset=1em,preambleanchor=text,
   preambleoffset=0pt,avoidnumlabelclash,
   appendtopexarg={samplelabel=(\actualexno a)}}|endcodedisplay
Then

\definelingstyle{IJAL}{labelwidth=2em,labelanchor=numleft,
   labeloffset=0pt,labelformat=(A),everylabel=\actualexno,
   textanchor=normal,textoffset=1em,preambleanchor=text,
   preambleoffset=0pt,avoidnumlabelclash,
   appendtopexarg={samplelabel=(\actualexno a)}}%
\edef\\{\noexpand\vskip\lingbelowexskip}%
\beginss
\lingset{lingstyle=IJAL}
\pex
\a first
\a second
\xe

\pex~[exno=65]
Preamble
\a first
\a second
\xe

\pex~[exno=1026]
Preamble
\a first
\a second
\xe|midss
\lingset{lingstyle=IJAL}
\pex~[exno=5]
\a first
\a second
\xe
\\
\pex[exno=65]
Preamble
\a first
\a second
\xe
\\
\pex~[exno=1026]
Preamble
\a first
\a second
\xe
\endss

\endgroup

\subsection Access to the plumbing

As you might guess, there is a lot going on under the surface
when |\pex| does its computations.  Sometimes a user would like
access to some of the dimensions involved.  The following are all
macros (not dimension registers) which expand to the
corresponding dimension

\ex
\vtop{\halign{#\hfil& \tspace #\hfil\cr
|\lingaboveexskip|&       |\lingpreambleoffset|\cr
|\lingbelowexskip|&       |\lingnumrightoffset|\cr
|\lingnumoffset|&         |\linglabelwidth|\cr
|\lingtextoffset|&        |\linglabeloffset|\cr
}}\xe

\endinput




{crskip}
{glspace}
{aboveglcskip}
{glhangindent}
{glwidth}
{aboveglcskip}
{aboveglftskip}
{abovegl\#1skip}
{abovemoreglskip}
{aboveglcskip}
{moregloffset}
{glufcloseup}
{glbracketsep}

\endinput
%%%%%%%%%%%%%%%%%%%%%%%%%%%%%%%%%%%%%%%%%%%%%%%%%%%%%%%%%%%%%%
%%%%%%%%%%%%%%%%%%%%%%%%%%%%%%%%%%%%%%%%%%%%%%%%%%%%%%%%%%%%%%
\subsection Formatting the preamble

\parinventory
& \idx{|preambleoffset|}& incrementable dimension& |1em|\cr
& \idx{|belowpreambleskip|}& skip& |1ex|\cr
& \idx{|nopreamble|}&& (default only)\cr
\endparinventory

\noindent Visible material which occurs before the first labeled
entry, as in (\getref{Pre}) for example, is called the preamble.
Although the initial settings produce the format in (\blastx),
the offset of the preamble can be set independently of the offset
of the labels and the extra vertical skip between the preamble
and the first part can also be set independently of the extra
vertical skip between the various parts.

\expex\ sets the label offset and the preamble offset to be equal,
so that the left edge of the preamble aligns with the left edge
of the labels.  But this is under the control of the user.

\framedisplay
\pex[labeltype=caps,labeloffset=!.8em]
{\it Principles of the Theory of Binding}
\a An anaphor is bound in its governing category.
\a A pronomial is free in its governing category.
\a An R-expression is free
\xe
\endframedisplay

\codedisplay
\pex[labeltype=caps,labeloffset=!.8em]
{\it Principles of the Theory of Binding}
\a An anaphor is bound in its governing category.
\a A pronomial is free in its governing category.
\a An R-expression is free
\xe |endcodedisplay


The different effects of |interpartskip| and |belowpreambleskip|
are illustrated below.

\framedisplay
\pex[labeltype=caps,belowpreambleskip=.75ex,interpartskip=.25ex]
{\it Principles of the Theory of Binding}
\a An anaphor is bound in its governing category.
\a A pronomial is free in its governing category.
\a An R-expression is free.
\xe
\endframedisplay

\goodbreak
\codedisplay
\pex[labeltype=caps,belowpreambleskip=.75ex,interpartskip=.25ex]
{\it Principles of the Theory of Binding}
\a An anaphor is bound in its governing category.
\a A pronomial is free in its governing category.
\a An R-expression is free.
\xe
|endcodedisplay

If |\a| or |\label| (see Section \getref{labelsec}) directly follows
|\pex| (and a possible tilde diacritic
and parameter settings), |\pex| assumes that
there is no preamble, otherwise it assumes that there is.
This poses a problem if you want to have nonprinting material
other than a label specification before the first part.  For example,
suppose you want to increase the baselineskip by 2pt.  You might
try (\nextx), but it fails to achieve what you want.

\beginss
\pex \openup2pt
\a
\a
\a
\xe|midss
\pex \openup2pt
\a
\a
\a
\xe
\endss
\noindent \expex\/ provides the parameter |nopreamble| to
solve the problem.  Setting it to ``true'' tells |\pex| that it
should assume that there is no preamble, in spite of superficial
appearances to the contrary.  So:

\beginss
\pex[nopreamble] \openup2pt
\a
\a
\a
\xe|midss
\pex[nopreamble] \openup2pt
\a
\a
\a
\xe
\endss
\noindent Invoking |nopreamble| with no specified value sets it
to the stipulated default value ``true''.  Setting
|nopreamble=false| has no effect since |\pex| always starts out
assuming that there is a preamble (i.e. that |nopreamble| has
been set to |false|) and this is always overridden
if a following |\a| or |\label| is detected, regardless of the
setting of |nopreamble|.

\font\titlett=txtt at 13.3pt
\subsection  Anchoring the labels, text, and preamble%
\deftag{\the\secno.\the\subsecno}{anchors}

%Choice parameter:\quad |labelanchor|\qquad set to |numright| or
%|margin|
\parinventory*
& \idx{|labelanchor|}& \idx{|numright|}, \idx{|numleft|},
   \idx{|margin|}& \idx{|numright|}\cr
& \idx{|textanchor|}& \idx{|normal|}, \idx{|numleft|}&
   \idx{|normal|}\cr
& \idx{|preambleanchor|}& \idx{|numright|}, \idx{|labelleft|}&
   \idx{|text|}\cr
\endparinventory

\noindent Ordinarily, the label offset is measured from the right
edge of the example number.  There is little choice if a label
occurs on the same line as the example number (i.e. there is no
preamble).  In example displays with a preamble, however, it is
sometimes desirable to measure the labeloffset from the left
margin, as shown below.

\ex[labelanchor=margin] \showpex \xe

One application of anchoring the label offset at the
left margin is in examples like the following.  If it is
impossible or undesirable to break a line (or graphic display of
some kind), (\anextx) is an alternative to (\nextx).  There is a
different kind of example on page \getref{partialrepeat}.

\framedisplay
\pex
\a A long example that will not fit in one line and
should not be broken if that is at all possible.
\a A short line might be more interesting linguistically, but less
interesting typographically.
\xe

\pex~[labelanchor=margin,labeloffset=1em] Make up a title.
Typographically, almost anything looks better than nothing.
\a A long example that will not fit in one line and
should not be broken if that is at all possible.
\a A short line might be more interesting linguistically, but less
interesting typographically.
\xe
\endframedisplay

It is produced by:

\vbox{\codedisplay
\pex[labelanchor=margin,labeloffset=1em]\par
\a A long example that will not fit in one line and
should not be broken if that is at all possible.
\a A short line might be more interesting linguistically, but less
interesting typographically.
\xe
|endcodedisplay}


\noindent Note that the preamble in the
second |\pex| is |\par|.  |\relax| would accomplish the same
thing.  Something must come before the first |\a|, otherwise
|\pex| will think there is no preamble.

Another way to handle situations like this is to use |\ex| and
insert explicit labels.  Something like the following works well.


\codedisplay
\ex \smallskip \leftskip=1em
a.\enspace This is a very long example that will not fit in one
line and cannot be broken for some reason.
\smallskip
b.\enspace A short line might be more interesting linguistically,
but less interesting typographically.
\xe |endcodedisplay

\framedisplay~
\ex \smallskip \leftskip=1em
a.\enspace
A long example that will not fit in one line and
should not be broken if that is at all possible.
\smallskip
b.\enspace A short line might be more interesting linguistically,
but less interesting typographically.
\xe
\endframedisplay

\noindent If you can think up a title, so that the top line
contains more than the example number, the appearance is
improved.

Anchoring the label offset at the left edge of the example number
(setting |labelanchor| to |numleft|) is unlikely to be useful.
The possibility is included mostly for the sake of completeness.

\exbreak[25ex]
\subsection Tabular format in multiline examples

\parinventory
& \idx{|textanchor|}&  choice\enspace
   (|numleft|, |numright|, or |labelleft|)& |numright|\cr
& \idx{|preambleanchor|}& choice\enspace
   (|numright|, |labelleft|, or |text|)& |numright|\cr
\endparinventory

Some publications specify that the text and label offset are measured
from the left margin.  \expex\ 3.3 measures the label offset from the
right edge of the example number and the textoffset from the right
edge of the ``label gutter'' (the slot which labels are typeset in).
\expex\ already had a parameter |labelanchor| (with values |margin|,
|numleft|, and |numright|) which determined how the value of
|labeloffset| was used to set the position of the labels. A parameter
has |textanchor| been introduced, with obvious meaning.  In multipart
examples in \expex\ 3.3, the position of a preamble in multipart
examples was measured with respect to the right edge of the typeset
example number.  The preamble is the stuff (if any) which appears
before the first labeled part of a multipart example.  \expex\
introduces the parameter |preambleanchor| (with possible values
|numright|, |numleft|, and |text|), with obvious meaning.

For example,
\begingroup
\codedisplay
\lingset{%
   textanchor=numleft,textoffset=50pt,
   labelanchor=numleft,labeloffset=25pt,
   preambleanchor=text,preambleoffset=0pt,
}
\pex[exno=137]
\a first
\a second
\xe

\pex~[labelwidth=2in]
Preamble
\a first
\a[label=bb] second
\xe
|endcodedisplay

\noindent produces

\lingset{%
   textanchor=numleft,textoffset=5em,
   labelanchor=numleft,labeloffset=2.5em,
   preambleanchor=text,preambleoffset=0pt,
}
\pex[exno=137]
\a first
\a second
\xe

\pex~[labelwidth=2in]
Preamble
\a first
\a[label=bb] second
\xe
The position of the labels and text is unaffected by the
width of the typeset example numer, the specified labelwidth, or
the actual label width.

\subsection Access to the plumbing

%{aboveexskip}
%{belowexskip}
%{numoffset}
%{textoffset}
{preambleoffset}
{labelwidth}
{labeloffset}

{glspace}
{glhangindent}
{glwidth}
{aboveglftskip}
{abovegl#1skip}
{abovemoreglskip}
{moregloffset}

{crskip}
{glufcloseup}
{glbracketsep}

