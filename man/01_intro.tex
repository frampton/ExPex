
\vfil
\eject
\global\pageno=1
\section Introduction

Many of the needs of linguists who wish to produce
typographically attractive papers using \Tex\ or \LaTex\
are not specific to linguistic papers.  There are therefore many
macro packages which deal with tables of contents, references,
section headings, font selection, indexing, etc. But linguistics
does have some special typographic needs. I addressed two of
these with the macro packages {\sl PST-JTree\/} and {\sl
PST-ASR\/}, which typeset syntactic trees and autosegmental
representations. \ExPex\ addresses the main remaining special
\Tex\ need in linguistics: formatting examples, examples with
multiple parts, glosses, and the like, and referring to examples
and parts of examples. The name comes from the two central
macros, |\ex| and |\pex|, used to typeset examples and examples
with labeled parts.

{\sl PST-JTree\/} and {\sl PST-ASR\/} rely heavily on Hendri
Adriaens' \XKeyVal\ package, which has become the standard for
\PSTricks\ based macro packages.  Although \ExPex\ is not based
on \PSTricks\relax, it does handle parameterization in the same
way that {\sl PST-JTree\/} and {\sl PST-ASR\/} do. When
{\sl expex.tex} is loaded, it immediately checks to see
whether {\sl xkeyval.tex} has already been loaded.  If not,
it does so.

The goal in writing a macro package for general use is to make it
simple to use if only simple things need to be done, but powerful
enough so that users who have complex needs can get those needs
satisfied if they are willing to deal with the complexities that
complex needs inevitably involve.  If you think there are simple
things that are not simple to do, or complex things that cannot
be done, please write to me at {\sl j.frampton@neu.edu}.  The
\ExPex\ macros have evolved over the last 15 or so years
and like anything which evolves, various features of the current
state may have more to do with history than with optimal design.
Please let me know about departures from optimal design.  Perhaps
the next version can be improved.

This User's Guide begins with four examples which demonstrate
\ExPex\ in action.  It serves as a ``A Quick Guide to {\sl
Expex\/}''.  Each page gives some code at the top, with the
product of this code below. Its main purpose is to give a sense
of how \ExPex\ works, so that curious readers have some
basis for determining whether they want to proceed with the
details.  It is possible to begin to use \ExPex\ solely on
the basis of the four demo pages and learn the more subtle
capabilities as needed.  A quick survey of the index and the table
of contents should give you some idea of what is available, if
you need it.

\subsection  Changes

There are a number of changes from earlier versions of
\ExPex\relax. This version does not maintain backward
compatibility with earlier versions, particularly with respect to
the glossing macros.  The older versions will still be available
on my website. {\sl expex2008.zip\/} contains
{\sl expex.tex}, {\sl expex.sty}, and
{\sl expex-doc.pdf}, Version 1.0 (beta).
{\sl expex2011.zip} contains {\sl expex.tex},
{\sl expex.sty}, and {\sl expex-doc.pdf}, Version 3.3.
My intention is that the current version will be stable for a
long time, except for bug fixes and extensions, so that backwards
compatibility will be maintained.

\medskip
\noindent 1.  The macros for creating glosses have been
substantially revised.  \ExPex\
currently supports only one gloss style, which is an extension of
the ``wrap style'' in earlier versions.\footnote{%
\ExPex\ supported several different gloss styles, selected by
the parameter |glstyle|. The parameter has been maintained, but
can only be set to |wrap|.  The parameter was retained in spite of the fact
that it currently has no use because my
intention is that other gloss styles will be added in the
future.}

\medskip
\noindent 2. It is now possible to specify the labels in
multipart examples by giving an explicit sequence of labels. This
allows labels to be drawn from a font set in which sequential
alphabetical order does not correspond to sequential character
code order in the font set.

\medskip
\noindent 3. The macros to control page breaking inside examples
have been revised and the macro |\goodpar| has been renamed
|\exbreak| to be consistent with \Tex's naming habits.

\medskip
\noindent 4. Several initial settings of the parameters have been
changed.

\subsection LaTex/Tex cooperation

\ExPex\ is designed to be equally usable by \Tex\ and \LaTex\
users.  \LaTex\ users say |\usepackage{expex}| and \Tex\ users
say |\input expex|.  All of the code for the examples in this
documentation should run equally well in either system, subject
to the notes below.

\subsubsection Note to \LaTex\ users

|\it| (now a deprecated \LaTex\ command) is used in a few places.
Because its use is so limited, no problems should arise.

\subsubsection Note to \Tex\ users

Three macros are used in the examples in this documentation which
are defined in \LaTex\ but not in {\it Plain Tex}: |\footnotesize|,
|\sc|, and |\textsc|.  Assuming, for example, that text is set in
10pt type, the following suffices for all the examples in this
documentation.

\codedisplay
\font\eightrm=cmr8
\font\eightit=cmti8
\def\footnotesize{\eightrm \let\it=\eightit \baselineskip=9pt}
\font\tensc=cmcsc10
\let\sc=\tensc
\def\textsc#1{{\sc #1}}
|endcodedisplay

Most \Tex\ users will have much more general size changing macros
at their disposal.  The code is barely sufficient to handle this
documentation, but it will do the job. For what it is worth, this
documentation was typeset using \Tex. |\twelvepoint| and
|\tenpoint| were defined modeled on pages 414--415 in the {\sl
TeXbook\/} and |\let\footnotesize=\tenpoint| was executed to define
|\footnotesize|.  The running text is \textdim{12 pt}. |\sc| was
defined by |\font\twelvesc=cmcsc10
scaled\magstep1|%
\footnote{cmcsc10 scaled was used rather than
cmcsc12 because Postscript fonts for cmcsc10 are more readily
available than those for cmcsc12.}
and
|\let\sc=\twelvesc|%
\footnote{Small caps are used only in text size}%
.

\subsection The file expex-demo.tex

This file contains the code for all of the examples in this
documentation.  It is constructed so that one can say either
\exfrag
|tex expex-demo|
\xe
or
\exfrag
|latex expex-demo|
\xe
to typeset the examples.

The examples were extracted to give users a broad sample of
easily accessible working code which can serve as starting points
for experimenting with \ExPex, or as useful models.


\subsection Acknowledgements

Many participants in the {\sl Ling-Tex\/} discussion group have
contributed to the development of \ExPex\relax, either by posing
good questions, solving problems, or providing informed
discussion of desirable features.  In particular, I thank Alexis
Dimitriadis, Claude Dionne, Kevin Donnely, Antonio Fortin, Daniel
Harbor, Joshua Jensen, Joost Kremers, John Lyon, Alan Munn,
Christos Vlachos, and Natalie Weber.





