
\vfil
\eject
\global\pageno=1
\section Introduction

Many of the needs of linguists who wish to produce
typographically attractive papers using \Tex\/ or {\sl Latex\/}
are not specific to linguistic papers.  There are therefore many
macro packages which deal with tables of contents, references,
section headings, font selection, indexing, etc. But linguistics
does have some special typographic needs. I addressed two of
these with the macro packages {\sl PST-JTree\/} and {\sl
PST-ASR\/}, which typeset syntactic trees and autosegmental
representations. \Expex\/ addresses the main remaining special
\Tex\/ need in linguistics: formatting examples, examples with
multiple parts, glosses, and the like, and referring to examples
and parts of examples. The name comes from the two central
macros, |\ex| and |\pex|, used to typeset examples and examples
with labeled parts.

{\sl PST-JTree\/} and {\sl PST-ASR\/} rely heavily on Hendri
Adriaens' \Xkeyval\/ package, which has become the standard for
\Pstricks\/ based macro packages.  Although \Expex\/ is not based
on \Pstricks, it does handle parameterization in the same way that
{\sl PST-JTree\/} and {\sl PST-ASR\/} do. When {\sl expex.tex\/}
is loaded, it immediately checks to see whether {\sl
xkeyval.tex\/} has already been loaded.  If not, it does so.

The goal in writing a macro package for general use is to make it
simple to use if only simple things need to be done, but powerful
enough so that users who have complex needs can get those needs
satisfied if they are willing to deal with the complexities that
complex needs inevitably involve.  If you think there are simple
things that are not simple to do, or complex things that cannot
be done, please write to me at {\sl j.frampton@neu.edu}.  The
{\sl ExPex\/} macros have evolved over the last 15 or so years
and like anything which evolves, various features of the current
state may have more to do with history than with optimal design.
Please let me know about departures from optimal design.  Perhaps
the next version can be improved.

This User's Guide begins with four examples which demonstrate
{\sl ExPex\/} in action.  It serves as a ``A Quick Guide to {\sl
Expex\/}''.  Each page gives some code at the top, with the
product of this code below. Its main purpose is to give a sense
of how {\sl ExPex\/} works, so that curious readers have some
basis for determining whether they want to proceed with the
details.  It is possible to begin to use {\sl ExPex\/} solely on
the basis of the four demo pages and learn the more subtle
capabilities as needed.  A quick survey of the index and the table
of contents should give you some idea of what is available, if
you need it.

\subsection  Changes

There are a number of changes in {\sl ExPex\/} from Version 1.0
(Beta) that was circulated two years ago.

\medskip
\noindent 1.  The macros for creating glosses have been
substantially revised.  There are four different gloss styles
available, selected by the parameter |glstyle|. Line breaking in
glosses is automatic if |glstyle| is set to |wrap|.

\medskip
\noindent 2. It is now possible to specify the labels in
multipart examples by giving an explicit sequence of labels. This
allows labels to be drawn from a font set in which sequential
alphabetical order does not correspond to sequential character
code order in the font set.

\medskip
\noindent 3. The macros to control page breaking inside examples
have been revised.

\medskip
On the whole, this version of \Expex\ is backwards compatibility
with the earlier version.  There are some minor differences. Some
default settings have changed and the macro |\goodpar| has been
renamed |\exbreak| to be consistent with \Tex's naming habits.
The old version will be available as {\sl expex2007.tex\/} and
{\sl expex2007.sty\/} in case you have old documents that you
want to be able to run with only trivial modification.

An early version of the new gloss macros has had limited
circulation as {\sl ExPexGL}.  Those macros have been modified in
some details in the present \Expex\ on the basis of user's
experience with them.






