
\def\small{\tenrm}
\lingset{glstyle=wrap,abovemoreglskip=1ex}
\def\suff#1{{-\small #1}}%
\everymath={}

\makeatletter
\def\glmw@printcolframed{%
   \psframebox[framesep=0]{\vtop{%
      \ling@everyglword
      \gl@loopmoretrue
      \loop\ifgl@loopmore
         \gl@lop\tempa\to\@tempa
         \gl@lop\aboveskiplist\to\@aboveskip
         \gl@lop\strutlist\to\@strut
         \@aboveskip\hbox{\@strut\@tempa}%
         \ifx\tempa\empty \gl@loopmorefalse \fi
      \repeat
      }}%
}
\def\framewords{\let\glmw@printcol=\glmw@printcolframed}
\resetatcatcode

\section
Glosses

\lingset{dima=1.3em}
\noindent\begininventory
\omit Macros:\enspace
\idx{|\begingl|}|[]|, \idx{|\endgl|}, \idx{|\gla|}|[]|,
   \idx{|\glb|}|[]|, \idx{|\glft|}|[]|\cr
\endinventory


\noindent Before I discuss the many parameters that modify gloss
displays, it is best to look at a few examples.

\framedisplay
\ex<wapm>
\begingl
\gla k- wapm -a -s'i -m -wapunin -uk //
\glb Cl V Agr Neg Agr Tns Agr //
\glb 2 see {3\sc ACC} {} 2{\sc PL} preterit 3{\sc PL} //
\glft `you (pl) didn't see them'//
\endgl
\xe
\endframedisplay
\codedisplay~
\ex
\begingl
\gla k- wapm -a -s'i -m -wapunin -uk //
\glb Cl V Agr Neg Agr Tns Agr //
\glb 2 see {3\sc ACC} {} 2{\sc PL} preterit 3{\sc PL} //
\glft `you (pl) didn't see them'//
\endgl
\xe
|endcodedisplay

\framedisplay
\ex
\begingl
\gla Mary$_i$ ist sicher, dass es den Hans nicht st\"oren w\"urde
seiner Freundin ihr$_i$ Herz auszusch\"utten.//
\glb Mary is sure that it the-{\sc ACC} Hans not annoy would
his-{\sc DAT} girlfriend-{\sc DAT} her-{\sc ACC} heart-{\sc ACC} {out to
throw}//
\glft  `Mary is sure that it would not annoy John to reveal her
heart to his girlfriend.'//
\endgl
\xe
\endframedisplay
\codedisplay~
\ex
\begingl
\gla Mary$_i$ ist sicher, dass es den Hans nicht st\"oren w\"urde
seiner Freundin ihr$_i$ Herz auszusch\"utten.//
\glb Mary is sure that it the-{\sc ACC} Hans not annoy would his-{\sc
DAT} girlfriend-{\sc DAT} her-{\sc ACC} heart-{\sc ACC} {out to
throw}//
\glft  `Mary is sure that it would not annoy John to reveal her
heart to his girlfriend.'//
\endgl
\xe
|endcodedisplay

The |\gla| and |\glb| lines are parsed as a sequence of space
separated items terminating in~|//|.  The parser only looks for spaces
at the top-level.  Consequently, in (\lastx), for example, it is not
sensitive to the space in items like |the-{\sc ACC}| since the space
is inside a group, therefore not at the top level.  Spaces that
directly preceeds terminating |//| are disregarded.  The |\glft| (free
translation) line must also be terminated by |//|.

Only one |\gla| line is permitted and it must come first. Only one
|\glft| line is permitted and it must come last.  There can be
multiple |\glb| lines, as there are in (\blastx).

The next example gives a small taste of how parameter settings can be
used to adapt gloss displays to your needs.  It also shows that there
are no restrictions on the number of lines of interlinear text.  There
are four in this case.

%\framedisplay
%\ex
%\begingl[glwidth=4in]
%\gla Mary$_i$ ist sicher, dass es den Hans nicht st\"oren w\"urde
%seiner Freundin ihr$_i$ Herz auszusch\"utten.//
%\glb Mary is sure that it the\suff{ACC} Hans not annoy would
%his\suff{DAT} girlfriend\suff{DAT} her\suff{ACC} heart\suff{ACC} {out to
%throw}//
%\glft  `Mary is sure that it would not annoy John to reveal her
%heart to his girlfriend.'//
%\endgl
%\xe
%\endframedisplay

%
%
%
%
%The |\gla| line comes first, and can only appear first.  The |\glft|
%(free translation) line comes last, and can only appear last.  |\glb|
%lines can repeat.  As we shall see, the user can define other lines;
%|\glcat|, for example, might be a line on which category designations
%appear.  For each of the |\gl|$x$ lines ($x$ being |a|, |b|, |ft|, or
%user defined), the user can specify the font to be used and extra
%vertical spacing to be inserted over that line in the gloss (except
%for the |\gla| line which cannot insert extra vertical spacing above
%it).  The initial settings are such that the |\gla| line is set in
%italic and there is no extra vertical spacing above the |\glb| line,
%and an extra $1\,\rm ex$ spacing above the free translation line.

\font\ips=xipasl10 at 12pt
\font\ipss=xipasl10 at 7pt
\def\mroot{$\surd$}
\def\L{\char'354}
\def\v#1{{\accent"07 #1}}
\def\C{{\accent"07 c}}
\def\W{$^{\hbox{\ipss w}}\mskip-2mu$}
%   {glstyle=multilevel,glspace=1em,everygla=\ips,moregloffset=1.5em,
%   abovemoreglskip=1ex,aboveglftskip=1.5ex}


\framedisplay
\ex
\begingl
\gla[everygla=\ips] hoi Ekn {\L}E {x\W}ElEP t{g\W}El' st{\'\i}m {hE\L}
   {kuPEcx\W ist} {\L a} Pa{\v c}sEtqEt//
\glb[everyglb=\ips] hoi {\mroot}PEkn {\L}E {x\W}ElEP t{g\W}El' {s +
   \mroot t\'\i m} {hE\L} ku-PEc-\mroot{x\W}ist {\L}a Pa{\v c}sEtqEt//
\glb then {\mroot}say det Meadowlark why {nomlz + \mroot what} conn
   2nom-cust-{\mroot}one.travels det {day time}//
\glb then {she said} det Meadowlark why {what is it} conn {you travel
   about} det {day time}//
\glft Then Meadowlark said, ``Why do you travel about in the day
   time?''//
\endgl
\xe
\endframedisplay
\def\goop{\thinspace\putfnno}%
\codedisplay~
\ex|goop
\begingl
\gla[everygla=\ips] hoi Ekn {\L}E {x\W}ElEP t{g\W}El' st{\'\i}m {hE\L}
   {kuPEcx\W ist} {\L a} Pa{\v c}sEtqEt//
\glb[everyglb=\ips] hoi {\mroot}PEkn {\L}E {x\W}ElEP t{g\W}El' {s +
   \mroot t\'\i m} {hE\L} ku-PEc-\mroot{x\W}ist {\L}a Pa{\v c}sEtqEt//
\glb then {\mroot}say det Meadowlark why {nomlz + \mroot what} conn
   2nom-cust-{\mroot}one.travels det {day time}//
\glb then {she said} det Meadowlark why {what is it} conn {you travel
   about} det {day time}//
\glft Then Meadowlark said, ``Why do you travel about in the day
   time?''//
\endgl
\xe
|endcodedisplay

\noindent
In my Tex setup, |\ips| invokes one of the slant tipa IPA fonts, |\L|
produces {\ips \L}, |\W| produces {\ips \W}, and |\mroot|
(morphological root) produces {\mroot}.




The interlinear gloss is built by gathering items into boxes as shown
below, which repeats (\getref{wapm}) above and subjecting these boxes
to {\sl Tex}'s usual paragraph building machinary as they are
constructed. A horizontal space (determined by the parameter
|glspace|) is inserted between these boxes.  I will refer later to
these boxes as ``gl words'' since, as far as {\sl Tex}'s paragraph building
machinary is concerned, they act like words.  The baseline of each gl
word is the baseline of the top item in the box.

\ex
\framewords
\begingl
\gla k- wapm -a -s'i -m -wapunin -uk //
\glb Cl V Agr Neg Agr Tns Agr //
\glb 2 see {3\sc ACC} {} 2{\sc PL} preterit 3{\sc PL} //
\glft `you (pl) didn't see them'//
\endgl
\xe




\hfill\noindent\begininventory
\hfil\it key& \hfil\it value& \hfil\it initial value\cr
\idx{|glspace|}& skip& |1em|\cr
\idx{|everygl|}& token list& \it empty\cr
\idx{|everygla|}& token list& |\it|\cr
\idx{|everyglb|}& token list& \it empty\cr
\idx{|everyglft|}& token list& \it empty\cr
\idx{|everyglword|}& token list& \it empty\cr
\idx{|aboveglbskip|}& skip& |0pt|\cr
\idx{|aboveglftskip|}& skip& |1ex|\cr
\idx{|abovemoreglskip|}& skip& |1ex|\cr
\idx{|glhangstyle|}& choice (|normal|, |none|, |cascade|)& |normal|\cr
\idx{|glhangindent|}& skip& |1em|\cr
\idx{|glwidth|}& skip& |0pt|\cr
\endinventory

Suppose you would like certain glosses to look like (\nextx).
Compared with (\getref{wapm}): the first line is set in roman type,
not italics; the second line has been set in a smaller
font; and the gap between the first and second lines has been
decreased.

\ex
\begingl
\gla[everygla=] k- wapm -a -s'i -m -wapunin -uk //
\glb[everyglb=\tenrm,aboveglbskip=-.5ex] Cl V Agr Neg Agr Tns Agr //
\glb 2 see {3\sc ACC} {} 2{\sc PL} preterit 3{\sc PL} //
\glft `you (pl) didn't see them'//
\endgl
\xe

Here is one way to accomplish this.

\codedisplay
\ex
\begingl
\gla[everygla=] k- wapm -a -s'i -m -wapunin -uk //
\glb[everyglb=\tenrm,aboveglbskip=-.5ex] Cl V Agr Neg Agr Tns Agr //
\glb 2 see {3\sc ACC} {} 2{\sc PL} preterit 3{\sc PL} //
\glft `you (pl) didn't see them'//
\endgl
\xe
|endcodedisplay

Here is a better way to do it, assuming that you have a number of
glosses that you want to do in this way.  First, you use
|\definemwlevels{cat}| to define a new level.  This makes |\glcat|
available, just like |\glb|.  It also make parameters |everyglcat| and
|aboveglcatskip| available for adjustment.  They are initialized to
the empty token string and to $0\,\rm pt$.  On the face of it, this
does not seem to accomplish anything.  It seems that you still need to
say:
\definemwlevels{cat}

\codedisplay
\ex
\begingl
\gla[everygla=] k- wapm -a -s'i -m -wapunin -uk //
\glcat[everyglcat=\tenrm,aboveglcatskip=-.5ex] Cl V Agr Neg Agr Tns Agr //
\glb 2 see {3\sc ACC} {} 2{\sc PL} preterit 3{\sc PL} //
\glft `you (pl) didn't see them'//
\endgl
\xe
|endcodedisplay

But defining the `cat' level allows you to say:

\codedisplay
\definelingstyle{Potawatomi}
   {everygla=,everyglcat=\tenrm,aboveglcatskip=-.5ex}
|endcodedisplay
\definelingstyle{Potawatomi}{everygla=,everyglcat=\tenrm,aboveglcatskip=-.5ex}

Then, you get (\lastx) by saying:

\codedisplay
\ex[lingstyle=Potawatomi]
\begingl
\gla k- wapm -a -s'i -m -wapunin -uk //
\glcat Cl V Agr Neg Agr Tns Agr //
\glb 2 see {3\sc ACC} {} 2{\sc PL} preterit 3{\sc PL} //
\glft `you (pl) didn't see them'//
\endgl
\xe
|endcodedisplay

\ex[lingstyle=Potawatomi]
\begingl
\gla k- wapm -a -s'i -m -wapunin -uk //
\glcat Cl V Agr Neg Agr Tns Agr //
\glb 2 see {3\sc ACC} {} 2{\sc PL} preterit 3{\sc PL} //
\glft `you (pl) didn't see them'//
\endgl
\xe








%\endinput

\definemwlevels{aa}
\lingset{everygla=\ips,everyglaa=\ips}

\framedisplay
\ex
\begingl
\gla hoi Ekn {\L}E {x\W}ElEP t{g\W}El' st{\'\i}m {hE\L}
   {kuPEcx\W ist} {\L a} Pa{\v c}sEtqEt//
\glaa hoi {\mroot}PEkn {\L}E {x\W}ElEP t{g\W}El' {s +
   \mroot t\'\i m} {hE\L} ku-PEc-\mroot{x\W}ist {\L}a Pa{\v c}sEtqEt//
\glb then {\mroot}say det Meadowlark why {nomlz + \mroot what} conn
   2nom-cust-{\mroot}one.travels det {day time}//
\glb then {she said} det Meadowlark why {what is it} conn {you travel
   about} det {day time}//
\glft Then Meadowlark said, ``Why do you travel about in the day
   time?''//
\endgl
\xe
\endframedisplay




\let\suff=\gobble
\section
Glosses

Macros:\quad \idx{|\begingl|}|[]|, \idx{|\endgl|}, \idx{|\gla|}|[]|,
\idx{|\glb|}|[]|, \idx{|\glft|}|[]|

\parinventory
& \idx{|glstyle|}& choice\quad (|wrap|, |3level|, |multilevel|, or |oldstyle|)&
|wrap|\cr
\endparinventory

\vskip-2ex
\noindent
There are four different styles, choosen by the parameter
|glstyle|, which can be set to |wrap|, |3level|, |multilevel|, or
|oldstyle|.  The last two are included for backwards
compatibility with the early versions of ExPex.  They are not
intended for current or future use, nor will this manual explain
their use.\footnote{Their use is explained in earlier versions of
this manual.  I assume that if these gloss styles are in use,
no further documentation is needed.}

In earlier sections, the complete array of relevant parameters
was given at the beginning of the section.  We depart from that
convention in this section and discuss the description syntax
before all the various parameters and their meanings are
identified.

\subsection
Glosses which wrap, ({\tt glstyle=wrap})

Macros:\quad \idx{|\gla|}, \idx{|\glb|}, \idx{|\glft|}

\parinventory
& \idx{|glwidth|}& dimension& |0pt|\cr
\endparinventory
\vskip-2ex
\lingset{glstyle=wrap}
\noindent First, an example, with the code below it.  (All of the
examples in this section assume that |glstyle| has been set to
|wrap|, typically by |\lingset{glstyle=wrap}|.)

\framedisplay
\ex
\begingl
\gla Mary$_i$ ist sicher, dass es den Hans nicht st\"oren w\"urde
seiner Freundin ihr$_i$ Herz auszusch\"utten.//
\glb Mary is sure that it the\suff{ACC} Hans not annoy would
his\suff{DAT} girlfriend\suff{DAT} her\suff{ACC} heart\suff{ACC} {out to
throw}//
\glft  `Mary is sure that it would not annoy John to reveal her
heart to his girlfriend.'//
\endgl
\xe
\endframedisplay

\def\glexample{\the\excnt}
\codedisplay
\ex
\begingl
\gla Mary$_i$ ist sicher, dass es den Hans nicht st\"oren w\"urde
seiner Freundin ihr$_i$ Herz auszusch\"utten.//
\glb Mary is sure that it the\suff{ACC} Hans not annoy would
his\suff{DAT} girlfriend\suff{DAT} her\suff{ACC} heart\suff{ACC} {out to
throw}//
\glft  `Mary is sure that it would not annoy John to reveal her
heart to his girlfriend.'//
\endgl
\xe
|endcodedisplay
(The example and the later related one are from an article by Idan
Landau.)

Sometimes it is desirable to override natural wrapping and
break up the gloss so that the syntax is emphasized, as in the
following.

\framedisplay
\ex
\begingl
\gla Mary$_i$ ist sicher, + dass es den Hans nicht st\"oren w\"urde
+ seiner Freundin ihr$_i$ Herz auszusch\"utten.//
\glb Mary is sure that it the\suff{ACC} Hans not annoy would
his\suff{DAT} girlfriend\suff{DAT} her\suff{ACC} heart\suff{ACC} {out to
throw}//
\glft  `Mary is sure that it would not annoy John to reveal her
heart to his girlfriend.'//
\endgl
\xe
\endframedisplay

\bigskip
This is accomplished by inserting `|+|'
appropriately, as shown in the code below.

\codedisplay
\ex
\begingl
\gla Mary$_i$ ist sicher, + dass es den Hans nicht st\"oren w\"urde
+ seiner Freundin ihr$_i$ Herz auszusch\"utten.//
\glb Mary is sure that it the\suff{ACC} Hans not annoy would
his\suff{DAT} girlfriend\suff{DAT} her\suff{ACC} heart\suff{ACC} {out to
throw}//
\glft  `Mary is sure that it would not annoy John to reveal her
heart to his girlfriend.'//
\endgl
\xe
|endcodedisplay

Sometimes it is desirable to omit the space between two entries.

\framedisplay
\ex
\begingl
\gla wiye kepi e- @ ca//
\glb two whitemen \sc1P:3D- found//
\endgl
\xe
\endframedisplay

This is accomplished by inserting `|@|' appropriately, as shown
in the code below.

\codedisplay
\ex
\begingl
\gla wiye kepi e- @ ca//
\glb two whitemen \sc1P:3D- found//
\endgl
\xe
|endcodedisplay
In the unlikely event that you need an entry which
would normally be entered as |@|, enter it as |{{@}}|
so that it is not interpreted as a directive to omit a space.

In the wrap style, the gloss is built in a vbox whose nominal
width is $h-l-r$, where $h$, $l$, and $r$, are the current values
of |\hsize|, |\leftskip|, and |\rightskip|,
respectively.  This implicit determination of the width of the
gloss is appropriate for use with the ExPex macros which typeset
examples because they adjust the leftskip appropriately inside
examples.

There are two instances in which the width of the vbox is not the
nominal width.  First, material which appears between
|\begingl| and |\gla| is typeset as a prefix to the
vbox and the width of the vbox is decreased from the nominal
width by the width of the prefix.  Second, the width of the vbox
can be set explicitly, ignoring the nominal width.

Suppose you want to precede a gloss with an explicit example
number, 14 for example.  Something like the following won't work.

\bigskip
\noindent 14. |\begingl|\kern1pt\dots|\endgl|
\bigskip

\noindent This produces an overfull line (or wraps the entire gloss box to
the next line) because the gloss box takes up the whole line. The
width of the gloss box must be decreased to make room for the
number and following space. This is done automatically if the
material to be typeset before the gloss is placed between
|\begingl| and |\gla|.  For example,

\framedisplay
\exdisplay
\begingl
14.\quad
\gla Mary$_i$ ist sicher, dass es den Hans nicht st\"oren w\"urde
seiner Freundin ihr$_i$ Herz auszusch\"utten.//
\glb Mary is sure that it the\suff{ACC} Hans not annoy would
his\suff{DAT} girlfriend\suff{DAT} her\suff{ACC} heart\suff{ACC}
{out to throw}//
\glft  `Mary is sure that it would not annoy John to reveal her
heart to his girlfriend.'//
\endgl
\xe
\endframedisplay

This is produced by:

\edef\tempa{{%
   \twelverm\quad$\vdots$\quad
      \raise2pt\hbox{(as in the code for Example \glexample\ above)}\par}}
\codedisplay
\begingl
14.\quad
\gla
|tempa
\endgl
|endcodedisplay

In addition to automatic adjustment of the width of the gloss
box, its width can be specified explicitly by setting the
parameter |glwidth|.  This parameter is ignored unless it is
positive.  The examples above were typeset with the default
setting |glwidth=0pt|, so that the width of the gloss box is
set automatically.

The following example illustrates the usefullness of the explicit
width option.

\ex
a.\quad
\begingl[glwidth=2.6in]
\gla Mary$_i$ ist sicher, dass es den Hans nicht st\"oren w\"urde
seiner Freundin ihr$_i$ Herz auszusch\"utten.//
\glb Mary is sure that it the\suff{ACC} Hans not annoy would
his\suff{DAT} girfriend\suff{DAT} her\suff{ACC} heart\suff{ACC} {out to
throw}//
\glft  `Mary is sure that it would not annoy John to reveal her
heart to his girlfriend.'//
\endgl
\hfil
b.\quad
\begingl[glwidth=2.6in]
\gla Mary$_i$ ist sicher, dass seiner Freunden ihr$_i$ Herz
auszuch\"utten dem Hans nicht schaden w\"urde.//
\glb Mary is sure that his\suff{DAT} girlfriend\suff{DAT} her\suff{ACC}
heart\suff{ACC} {out to throw} the\suff{DAT} Hans not damage would//
\glft `Mary is sure that to reveal her heart to his girlfriend
would not damage John.'//
\endgl
\xe

The code is
\codedisplay
\ex
a.\quad
\begingl[glwidth=2.6in]
\gla Mary$_i$ ist sicher, dass es den Hans nicht st\"oren w\"urde
seiner Freundin ihr$_i$ Herz auszusch\"utten.//
\glb Mary is sure that it the\suff{ACC} Hans not annoy would
his\suff{DAT} girfriend\suff{DAT} her\suff{ACC} heart\suff{ACC} {out to
throw}//
\glft  `Mary is sure that it would not annoy John to reveal her
heart to his girlfriend.'//
\endgl
\hfil
b.\quad
\begingl[glwidth=2.6in]
\gla Mary$_i$ ist sicher, dass seiner Freunden ihr$_i$ Herz
auszuch\"utten dem Hans nicht schaden w\"urde.//
\glb Mary is sure that his\suff{DAT} girlfriend\suff{DAT} her\suff{ACC}
heart\suff{ACC} {out to throw} the\suff{DAT} Hans not damage would//
\glft `Mary is sure that to reveal her heart to his girlfriend
would not damage John.'//
\endgl
\xe
|endcodedisplay

\subsection Changing the gloss parameters

\parinventory
%& \idx{|glwidth|}& token list& |{}|\cr
& \idx{|glspace|}& skip& |.6em|\cr
& \idx{|everygl|}& dimension& |0pt|\cr
& \idx{|everygla|}& token list& |{}|\cr
& \idx{|everyglb|}& token list& |{}|\cr
& \idx{|everyglword|}& token list& |{}|\cr
& \idx{|everyglc|}& token list& |{}|\cr
& \idx{|aboveglcskip|}& skip& |.5ex|\cr
& \idx{|glhangindent|}& dimension& |1em|\cr
\endparinventory

All of the examples above use the default settings of the
gloss parameters.  In order to illustrate the effects of the
parameters, several different ling styles are defined below,
then the same gloss done in the different styles is shown.

\definelingstyle{defaultgl}{glstyle=wrap,glspace=.6em,
   everygl=\openup.5ex,everyglword=,everygla=,everyglb=,everyglc=\it,
   aboveglcskip=.5ex,glhangindent=1em}
\definelingstyle{gergl}{glstyle=wrap,glspace=1.2em,everygl=\openup.5ex,
   everyglword=\openup-.7ex,everygla=\bf,everyglc=\it,
   aboveglcskip=0pt,glhangindent=1.5em}
\definelingstyle{mingl}{glstyle=wrap,everygl=,everyglword=,
   everygla=,everyglb=,everyglc=,aboveglcskip=0pt,glhangindent=1em}

\codedisplay
\definelingstyle{defaultgl}{glstyle=wrap,glspace=.6em,
   everygl=\openup.5ex,everyglword=,everygla=,everyglb=,everyglc=\it,
   aboveglcskip=.5ex,glhangindent=1em}
\definelingstyle{gergl}{glstyle=wrap,glspace=1.2em,everygl=\openup.5ex,
   everyglword=\openup-.7ex,everygla=\bf,everyglc=\it,
   aboveglcskip=0pt,glhangindent=1.5em}
\definelingstyle{mingl}{glstyle=wrap,everygl=,everyglword=,
   everygla=,everyglb=,everyglc=,aboveglcskip=0pt,glhangindent=1em}|endcodedisplay

Then

\codedisplay
\pex[interpartskip=3ex]
\a \begingl[lingstyle=mingl] |dots \endgl
\a \begingl[lingstyle=defaultgl] |dots \endgl
\a \begingl[lingstyle=gergl] |dots \endgl
|endcodedisplay \nobreak
\noindent produces

\lingset{lingstyle=mingl}
\framedisplay~
\pex[interpartskip=3ex]
\a \begingl[lingstyle=mingl]
\gla Mary$_i$ ist sicher, dass es den Hans nicht st\"oren w\"urde
seiner Freundin ihr$_i$ Herz auszusch\"utten.//
\glb Mary is sure that it the\suff{ACC} Hans not annoy would
his\suff{DAT} girlfriend\suff{DAT} her\suff{ACC} heart\suff{ACC} {out to
throw}//
\glft  `Mary is sure that it would not annoy John to reveal her
heart to his girlfriend.'//
\endgl

\a \begingl[lingstyle=defaultgl]
\gla Mary$_i$ ist sicher, dass es den Hans nicht st\"oren w\"urde
seiner Freundin ihr$_i$ Herz auszusch\"utten.//
\glb Mary is sure that it the\suff{ACC} Hans not annoy would
his\suff{DAT} girlfriend\suff{DAT} her\suff{ACC} heart\suff{ACC} {out to
throw}//
\glft  `Mary is sure that it would not annoy John to reveal her
heart to his girlfriend.'//
\endgl

\a \begingl[lingstyle=gergl]
\gla Mary$_i$ ist sicher, dass es den Hans nicht st\"oren w\"urde
seiner Freundin ihr$_i$ Herz auszusch\"utten.//
\glb Mary is sure that it the\suff{ACC} Hans not annoy would
his\suff{DAT} girlfriend\suff{DAT} her\suff{ACC} heart\suff{ACC} {out to
throw}//
\glft  `Mary is sure that it would not annoy John to reveal her
heart to his girlfriend.'//
\endgl
\xe
\endframedisplay

\subsection Multiple gloss lines ({\tt glstyle=multilevel})

%\noindent For simple glosses, it makes almost no difference
%whether |glstyle| is set to |multilevel| or |wrap|, as illustrated
%below.
%
%\framedisplay
%\pex[everygl=,aboveglcskip=0pt,aboveglftskip=0pt]
%\a \begingl[glstyle=multilevel]
%\gla Il semble au g\'en\'eral \^etre arriv\'e deux soldats
%en ville.//
%\glb there seems {to the} general {to be} arrived two soldiers
%in town//
%\glft `There seems to the general to have arrived two soldiers in
%town.'//
%\endgl
%\a \begingl[glstyle=wrap]
%\gla Il semble au g\'en\'eral \^etre arriv\'e deux soldats
%en ville.//
%\glb there seems {to the} general {to be} arrived two soldiers
%in town//
%\glft `There seems to the general to have arrived two soldiers in
%town.'//
%\endgl
%\xe
%\endframedisplay
%
%\codedisplay~
%\pex[everygl=,aboveglcskip=0pt,aboveglftskip=0pt]
%\a \begingl[glstyle=multilevel]
%\gla Il semble au g\'en\'eral \^etre arriv\'e deux soldats
%en ville.//
%\glb there seems {to the} general {to be} arrived two soldiers
%in town//
%\glft `There seems to the general to have arrived two soldiers in
%town.'//
%\endgl
%\a \begingl[glstyle=wrap]
%\gla Il semble au g\'en\'eral \^etre arriv\'e deux soldats
%en ville.//
%\glb there seems {to the} general {to be} arrived two soldiers
%in town//
%\glft `There seems to the general to have arrived two soldiers in
%town.'//
%\endgl
%|endcodedisplay

\noindent Glosses built with |glstyle=wrap| are limited to one
|\gla| line and one |\glb| line.  Glosses built with
|glstyle=multilevel| do not wrap automatically, but both |\gla|
and |\glb| can be used multiple times, in either order, a third
line |\glc| is defined, and there is provision to defining other
lines.

The display (\nextx) is typeset using the code which follows.
\framedisplay
\ex[glstyle=multilevel,everygla=,everyglb=,everyglc=,
   aboveglaskip=0pt,aboveglbskip=0pt,aboveglftskip=1ex]
\begingl
\gla k- wapm -a -s'i -m -wapunin -uk //
\glb Cl V Agr Neg Agr Tns Agr //
\glb 2 see {3\sc ACC} {} 2{\sc PL} preterit 3{\sc PL} //
\glft `you (pl) didn't see them'//
\endgl
\xe
\endframedisplay
\codedisplay~
\ex[glstyle=multilevel,everygla=,everyglb=,everyglc=,
   aboveglaskip=0pt,aboveglbskip=0pt,aboveglftskip=1ex]
\begingl
\gla k- wapm -a -s'i -m -wapunin -uk //
\glb Cl V Agr Neg Agr Tns Agr //
\glb 2 see {3\sc ACC} {} 2{\sc PL} preterit 3{\sc PL} //
\glft `you (pl) didn't see them'//
\endgl
\xe
|endcodedisplay
The example is from Hockett, as glossed by Halle and
Marantz.

It is worth noting that, in practice, it is most likely that
a style would be defined as follows.
Then \def\vd{\hbox{$\qquad \vdots$}}
\codedisplay
\definelingstyle{glabb}{glstyle=multilevel,everygla=,everyglb=,
   everyglc=,aboveglaskip=0pt,aboveglbskip=0pt,aboveglftskip=1ex}

\ex[lingstyle=glabb]
\begingl
|vd
\endgl
\xe
|endcodedisplay

\noindent would be sufficient.

Repeating |\glb| does not allow different fonts or spacing for the two
lines.  In order to permit this, another level |\glc| is
predefined.  For example, suppose a style is defined by

\codedisplay
\definelingstyle{glabc}{glstyle=multilevel,everygl=,everygla=,
   everyglb=\elevenpoint,aboveglbskip=.3ex,
   everyglc=\tenpoint,aboveglcskip=-.3ex,
   everyglft=\it,aboveglftskip=1ex}
|endcodedisplay

\noindent |\elevenpoint| and |\tenpoint| are the ways 11 point
and 10 point type are invoked in my font setup.  This will vary
from user to user, so no attempt is made here to detail the
macros I rely on (which are variations on the ones in the
TexBook).
\codedisplay

\definelingstyle{glabc}{glstyle=multilevel,everygl=,everygla=,
   everyglb=\elevenpoint,aboveglbskip=.3ex,
   everyglc=\tenpoint,aboveglcskip=-.3ex,
   everyglft=\it,aboveglftskip=1ex}
\def\tenpoint{\tenrm \let\sc\eightrm}%
\def\elevenpoint{\elevenrm}
\definegllevels{word}
|endcodedisplay

Then

\codedisplay
\ex[lingstyle=glabc]
\begingl
\gla k- wapm -a -s'i -m -wapunin -uk //
\glb Cl V Agr Neg Agr Tns Agr //
\glc 2 see {3\sc ACC} {} 2pl preterit 3pl //
\glft {`you (pl) didn't see them'}//
\endgl
\xe
|endcodedisplay
produces


\definelingstyle{glabc}{glstyle=multilevel,everygl=,everygla=,
   everyglb=\elevenpoint,aboveglbskip=.3ex,
   everyglc=\tenpoint,aboveglcskip=-.3ex,
   everyglft=\it,aboveglftskip=1ex}
\framedisplay
\ex[lingstyle=glabc]
\begingl
\gla k- wapm -a -s'i -m -wapunin -uk //
\glb Cl V Agr Neg Agr Tns Agr //
\glc 2 see {3\sc ACC} {} 2pl preterit 3pl //
\glft {`you (pl) didn't see them'}//
\endgl
\xe
\endframedisplay

\noindent (The settings for the style |glabc| are chosen for
illustrative purposes, not because this is a particularly
attractive display.)

The user can define additional levels by
\medskip
|\definegllevels{mylevela,mylevelb}|
\medskip
Then the macros |\glmylevela| and |\glmylevelb| will be defined
and the keys |\everyglmylevela|, |everyglmylevelb|,
|aboveglmylevelaskip|, and |aboveglmylevelbskip| will be
activated.



\medskip
\vfil\break
\noindent{\it Inline citations}\medskip

\noindent Macro:\quad \idx{|\rightcite|}
\medskip


\noindent Multilevel glosses make it easy to give inline citations in
several different styles, as illustrated below.

\framedisplay
\pex[glstyle=multilevel,everygla=,everyglb=,everyglc=,
   aboveglaskip=0pt,aboveglbskip=0pt,aboveglftskip=1ex]
\a \begingl
\gla k- wapm -a -s'i -m -wapunin -uk //
\glb Cl V Agr$_1$ Neg Agr$_2$ Tns Agr$_3$//
\glb 2 see {3\sc ACC} {} 2{\sc PL} preterit 3{\sc PL} //
\glft `you (pl) didn't see them'//
\endgl
\hfill (Hockett 1948, p. 143)
\a \begingl
\gla k- wapm -a -s'i -m -wapunin -uk //
\glb Cl V Agr$_1$ Neg Agr$_2$ Tns Agr$_3$//
\glb 2 see {3\sc ACC} {} 2{\sc PL} preterit 3{\sc PL} //
\glft `you (pl) didn't see them'//
\endgl
\hfil (Hockett 1948, p. 143)
\a \begingl
\gla k- wapm -a -s'i -m -wapunin -uk //
\glb Cl V Agr$_1$ Neg Agr$_2$ Tns Agr$_3$//
\glb 2 see {3\sc ACC} {} 2{\sc PL} preterit 3{\sc PL} //
\glft \rightcite{(Hockett 1948, p. 143)}
   `you (pl) didn't see them'//
\endgl
\xe
\endframedisplay
\codedisplay
\pex[glstyle=multilevel,everygla=,everyglb=,everyglc=,
   aboveglaskip=0pt,aboveglbskip=0pt,aboveglftskip=1ex]
\a \begingl
\gla k- wapm -a -s'i -m -wapunin -uk //
\glb Cl V Agr$_1$ Neg Agr$_2$ Tns Agr$_3$//
\glb 2 see {3\sc ACC} {} 2{\sc PL} preterit 3{\sc PL} //
\glft `you (pl) didn't see them'//
\endgl \hfill (Hockett 1948, p. 143)
\a \begingl
\gla k- wapm -a -s'i -m -wapunin -uk //
\glb Cl V Agr$_1$ Neg Agr$_2$ Tns Agr$_3$//
\glb 2 see {3\sc ACC} {} 2{\sc PL} preterit 3{\sc PL} //
\glft `you (pl) didn't see them'//
\endgl \hfil (Hockett 1948, p. 143)
\a \begingl
\gla k- wapm -a -s'i -m -wapunin -uk //
\glb Cl V Agr$_1$ Neg Agr$_2$ Tns Agr$_3$//
\glb 2 see {3\sc ACC} {} 2{\sc PL} preterit 3{\sc PL} //
\glft \rightcite{(Hockett 1948, p. 143)}
   `you (pl) didn't see them'//
\endgl
\xe
|endcodedisplay


%%%%%%%%%%%%%%%%%%%%%%%%%%%%%%
\font\ips=xipasl10 at 12pt
\font\ipss=xipasl10 at 7pt
\def\mroot{$\surd$}
\def\L{\char'354}
\def\v#1{{\accent"07 #1}}
\def\C{{\accent"07 c}}
\def\W{$^{\hbox{\ipss w}}\mskip-2mu$}

\medskip
\vfil\break

\noindent{\it Line breaking (non-automatic) in the multilevel
style}\medskip

\noindent
Macros:\quad \idx{|\moregl|}

\parinventory
& \idx{|moregloffset|}& dimension& 1em\cr
& \idx{|abovemoreglskip|}& skip& .5ex plus .2ex\cr
\endparinventory



Assuming that |\ips| selects the tipa slant font and |\mroot|, |\L|,
and |\W| are appropriately defined, the code below produces the
display below it.

\codedisplay
\definelingstyle{ips-gloss}%
   {glstyle=multilevel,glspace=1em,everygla=\ips,moregloffset=1.5em,
   abovemoreglskip=1ex,aboveglftskip=1.5ex}

\ex[lingstyle=ips-gloss]
\begingl
\gla hoi Ekn {\L}E {x\W}ElEP t{g\W}El' st{\'\i}m//
\gla hoi {\mroot}PEkn {\L}E {x\W}ElEP t{g\W}El' {s + \mroot t\'\i m}//
\glb then {\mroot}say det Meadowlark why
   {nomlz + \mroot what}//
\glb then {she said} det Meadowlark why {what is it}//
\moregl
\gla {hE\L} {kuPEcx\W ist} {\L a} Pa{\v c}sEtqEt//
\gla {hE\L} ku-PEc-\mroot{x\W}ist {\L}a Pa{\v c}sEtqEt//
\glb conn 2nom-cust-{\mroot}one.travels det {day time}//
\glb conn {you travel about} det {day time}//
\glft Then Meadowlark said, ``Why do you travel about in the day
time?''//
\endgl
\xe
|endcodedisplay

\definelingstyle{ips-gloss}%
   {glstyle=multilevel,glspace=1em,everygla=\ips,moregloffset=1.5em,
   abovemoreglskip=1ex,aboveglftskip=1.5ex}

\framedisplay~
\ex[lingstyle=ips-gloss]
\begingl
\gla hoi Ekn {\L}E {x\W}ElEP t{g\W}El' st{\'\i}m//
\gla hoi {\mroot}PEkn {\L}E {x\W}ElEP t{g\W}El' {s + \mroot t\'\i m}//
\glb then {\mroot}say det Meadowlark why
   {nomlz + \mroot what}//
\glb then {she said} det Meadowlark why {what is it}//
\moregl
\gla {hE\L} {kuPEcx\W ist} {\L a} Pa{\v c}sEtqEt//
\gla {hE\L} ku-PEc-\mroot{x\W}ist {\L}a Pa{\v c}sEtqEt//
\glb conn 2nom-cust-{\mroot}one.travels det {day time}//
\glb conn {you travel about} det {day time}//
\glft Then Meadowlark said, ``Why do you travel about in the day
time?''//
\endgl
\xe
\endframedisplay

\noindent This example was contributed by Shannon Bischoff.

%
%
%
%James Crippen provided a gloss with five gloss lines (and the
%free translation): standard orthography, phonetic form (as
%spoken), phonemic form (standardized), underlying morphology, and
%morpheme gloss.
%
%\font\ipa=xipa10 at 12pt
%\font\ipascript=xipa10 at 7pt
%\def\umlautunder#1{\oalign{#1\crcr\hidewidth
%   \vbox to .2ex{\hbox{\char'004}\vss}\hidewidth}}
%\def\barunder#1{\oalign{#1\crcr\hidewidth
%   \vbox to .2ex{\hbox{\char'011}\vss}\hidewidth}}
%\def\barover#1{$\bar{\hbox{\ipa #1}}$}
%\def\nullmorf{{\twelvesy\char'073}}
%
%\definelingstyle{tlingit}{glstyle=multilevel,glspace=1em,everygla=\ipa,
%   everyglb=\tenrm,moregloffset=2em,abovemoreglskip=1ex}
%\def\tlingitstyle{%
%   \lingset{lingstyle=tlingit}%
%   \let\"=\umlautunder
%   \def\G{\char'345}%
%   \def\L{\char'354}%
%   \def\W{$^{\hbox{\ipascript w}}$}%
%   \def\H{$^{\hbox{\ipascript h}}$}%
%}
%
%\ex \tlingitstyle
%\def\\#1{{\rm #1}}%
%\begingl
%\gla	Ch'u.\'a\`an, {gun\'ei kawdudln{\'\i}k} y\'aa haa//
%\gla  {tS'uP\umlautunder{\'a}a\`n} {{\G}Un\'e: k{\H}awtUt{\L}n{\'\i}k}
%   j\'a: ha //
%\gla  tS'uP\'aa\`n {qun\'e; k{\H}awtUt{\L}n{\'\i}k}
%   j\'a; ha://
%\gla  ch'u.\'a\`an dgun\'ei=\nullmorf-ka-wu-du-dli-n\'ik
%   y\'aa haa//
%\glb  RESUM INCEP=\\3O-HSFC-PFV-\\3OBV.S-CL[+D,\\1,+I]-\\{tell} DEM.PROX
%   \\1PL.PSS//
%\moregl
%\gla shag\'oon y\'aa saanyaa {\barunder {\rm K}}w\'a\`an.//
%\gla S2k{\W}\'u\`un j\'a; sa:nja: q{\H\W}{\barover A}:n//
%\gla Sak\'u\`un j\'a; sa:nja: q{\H\W}\'a\`an//
%\gla shag\'o\`on y\'aa saa-niyaa {\barunder k}w\'a\`an//
%\glb \\{history} DEM.PROX \\{southern direction} ???? //
%\glft `Anyway, they would begin telling our history of this
%Saanyaa Kw\'aan.'//
%\endgl
%\xe
%
%%%%%%%%%%%%%%%%%%%%%%%%%%%%%%%%%%%%%%%%%%%%%%%%%
\subsection An alternate gloss description syntax
({\tt glstyle=3level})

\noindent  The usual gloss description syntax uses a format which
imitates the intended typeset output.  The gloss type |3level|
provides an alternative, which some may find attractive.

\codedisplay
\ex[glstyle=3level]
\begingl
{k-/Cl/2} {wapm/V/see} {-a/Agr/3\sc ACC} {-s'i/Neg} {-m/Agr/\sc PL}
{-wapunin/Tns/preterit} {-uk/Agr/\sc PL}.
`you (pl) didn't see them'.
\endgl
\xe
|endcodedisplay
\framedisplay~
\ex[glstyle=3level,aboveglaskip=0pt,aboveglbskip=0pt]
\begingl
{k-/Cl/2} {wapm/V/see} {-a/Agr/3\sc ACC} {-s'i/Neg}
{-m/Agr/\sc PL} {-wapunin/Tns/preterit} {-uk/Agr/\sc PL}.
`you (pl) didn't see them'.
\endgl
\xe
\endframedisplay

\smallskip
\noindent The two lines which describe the gloss are terminated
by a period.  The brackets are required in the first line, which
is not processed as a space separated list.  The spaces between
the bracketed items are optional. The three gloss lines in the
output are called the gla, glb, and glc lines and parameters like
|everyglb| or |aboveglcskip| have the effect expected.  The last
line is the glft line and the parameters |everyglft| and
|aboveglft| also have the expected effect.

Gloss displays with two lines are possible.

\framedisplay
\ex
\begingl[glstyle=3level,everygla=\it]
{n-ku/1st-OK} {wapm-a/see}.
`OK I'll see him'
\endgl
\xe
\endframedisplay
\codedisplay~
\ex
\begingl[glstyle=3level,everygla=\it]
{n-ku/1st-OK} {wapm-a/see}.
`OK I'll see him'
\endgl
\xe
|endcodedisplay

Some might prefer to construct perfectly ordinary glosses using
this gloss type.

\framedisplay
\ex
\begingl[glstyle=3level]
{Il/there} {semble/seems} {au/to the} {g\'en\'eral/general}
{\^etre/to be} {arriv\'e/arrived} {deux/two} {soldats/soldiers}
{en/in} {ville/town.}.
`There seems to the general to have arrived two soldiers in
town{.}'.
\endgl
\xe
\endframedisplay
\codedisplay~
\ex
\begingl[glstyle=3level]
{Il/there} {semble/seems} {au/to the} {g\'en\'eral/general}
{\^etre/to be} {arriv\'e/arrived} {deux/two} {soldats/soldiers}
{en/in} {ville/town.}.
`There seems to the general to have arrived two soldiers in town{.}'.
\endgl
\xe
|endcodedisplay
\noindent You will be restricted to a single line, with no
wrapping.

\framedisplay
\ex[glstyle=3level]
\begingl
{pwa-/Neg} {min/V/give} {-kwa/Agr/2pl{\sc NOM}.3pl\sc ACC}
{-pun/Tns/preterit} .
`you (pl) didn't give them (something)'
\endgl
\xe
\endframedisplay

\codedisplay~
\ex[glstyle=3level]
\begingl
{pwa-/Neg} {min/V/give} {-kwa/Agr/2pl{\sc NOM}.3pl\sc ACC}
{-pun/Tns/preterit} .
`you (pl) didn't give them (something)'
\endgl
\xe
|endcodedisplay

Three level glosses are useful, particularly in
morphology.

\framedisplay
\ex[glstyle=3level]
a.\quad
\begingl
{pwa-/Neg} {min/V/give} {-kwa/Agr/2pl{\sc NOM}.3pl\sc ACC} {-pun/Tns/preterit} .
`you (pl) didn't give them (something)'
\endgl
\hfil
b.\quad
\begingl[everygla=\it,everyglft=,aboveglftskip=-.1ex]
{n-ku/1st-OK} {wapm-a/see}.
`OK I'll see him'
\endgl
\xe
\endframedisplay
\codedisplay~
\ex[glstyle=3level]
a.\quad
\begingl
{pwa-/Neg} {min/V/give} {-kwa/Agr/2pl{\sc NOM}.3pl\sc ACC}
{-pun/Tns/preterit} .
`you (pl) didn't give them (something)'
\endgl
\hfil
b.\quad
\begingl[everygla=\it,everyglft=,aboveglftskip=-.1ex]
{n-ku/1st-OK} {wapm-a/see}.
`OK I'll see him'
\endgl
\xe|endcodedisplay

%\subsection Parameters which affect 3 level glosses
%
%\parinventory
%& \idx{|everygla|}& token list& |{}|\cr
%& \idx{|everyglb|}& token list& |{}|\cr
%& \idx{|everyglc|}& token list& |{}|\cr
%& \idx{|everyglft|}& token list& |{\it}|\cr
%& \idx{|aboveglgextraskip|}& skip& |0pt|\cr
%& \idx{|abovelinethreeextraskip|}& skip& |0pt|\cr
%& \idx{|aboveglftskip|}& skip& |.5ex|\cr
%\endparinventory
%\vskip-1.5ex
%\noindent  Additionally, the parameters |glspace| and |everygl|
%apply and have the same meaning as in the wrap style.


