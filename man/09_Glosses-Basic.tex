
%\makeatletter
%\ex
%\lingset{glstyle=wrap}
%\writeln{\ling@glstyle, \noexpand\glw@glft, \meaning\glw@glft}
%\writeln{\ling@glstyle, \noexpand\glw@glft, \meaning\glpreamble}
%\begingl
%\glpreamble This is example ?? for Halle and Marantz.//
%\gla k- wapm -a -s'i -m -wapunin -uk //
%\glb CL V AGR NEG AGR TNS AGR //
%\glb 2 see {\sc 3acc} {} {\sc 2pl} preterit {\sc 3pl} //
%\glft `you (pl) didn't see them'//
%\endgl
%\xe
%\endinput

%\ex[everygla=,glhangstyle=normal,glstyle=nlevel]
%\def\brackspace{\hskip.1em}%
%\begingl
%Fa'nu'i[show] yu'[me] ni[Obl] {[[}[]@ {\it O}[Op]
%t{\it in\/}aitai-mu[{\it WH\/}{[obj]}.read-agr] {\it t\/}[]@ {]}[] na[L]
%lepblu[book]@ {].}[]
%\endgl
%\xe
%
%\endinput

\section Glosses
\deftag{\the\secno}{basicglosssec}

\begininventory
\macros
\idx{|\begingl[]|},
\idx{|\glpreamble[]|},
\idx{|\gla[]|},
\idx{|\glb[]|},
\idx{|\glc[]|},
\idx{|\glft[]|},
\idx{|\endgl|}
\endmc
\endinventory
%
Before I introduce the parameters that control the
characteristics of gloss displays, here are a few examples done
with the initial settings of the parameters.

\medskip
\framedisplay
\ex[belowpreambleskip=0pt,aboveglftskip=0pt,aboveglbskip=0pt,
   aboveglcskip=0pt,everygla=,everyglb=,everyglc=,
   everyglft=]<wapm>
\begingl
\glpreamble This is example ?? for Halle and Marantz.//
\gla k- wapm -a -s'i -m -wapunin -uk //
\glb CL V AGR NEG AGR TNS AGR //
\glb 2 see {\sc 3acc} {} {\sc 2pl} preterit {\sc 3pl} //
\glft `you (pl) didn't see them'//
\endgl
\xe
\endframedisplay
\codedisplay~
\ex
\begingl
\gla k- wapm -a -s'i -m -wapunin -uk //
\glb CL V AGR NEG AGR TNS AGR //
\glb 2 see {\sc 3acc} {} {\sc 2pl} preterit {\sc 3pl} //
\glft `you (pl) didn't see them'//
\endgl
\xe
|endcodedisplay

\framedisplay
\ex<sicher>
\begingl
\gla Mary$_i$ ist sicher, dass es den Hans nicht st\"oren w\"urde
seiner Freundin ihr$_i$ Herz auszusch\"utten.//
\glb Mary is sure that it the-{\sc acc} Hans not annoy would
his-{\sc dat} girlfriend-{\sc dat} her-{\sc acc} heart-{\sc acc} {out to
throw}//
\glft  `Mary is sure that it would not annoy John to reveal her
heart to his girlfriend.'//
\endgl
\xe
\endframedisplay
\codedisplay~
\ex
\begingl
\gla Mary$_i$ ist sicher, dass es den Hans nicht st\"oren w\"urde
seiner Freundin ihr$_i$ Herz auszusch\"utten.//
\glb Mary is sure that it the-{\sc acc} Hans not annoy would
his-{\sc dat} girlfriend-{\sc dat} her-{\sc acc}
heart-{\sc acc} {out to throw}//
\glft  `Mary is sure that it would not annoy John to reveal her
heart to his girlfriend.'//
\endgl
\xe
|endcodedisplay


\framedisplay
\ex<um>
\begingl
\glpreamble Um-\"asudda' h\"am yan i taotao ni si Juan
ilek-\~na nu guahu mal\"agu' gui
asudd\"a'-\~na.//
\gla Um-\"asudda' h\"am yan [ i taotao [ {\it O\/} ni si Juan
ilek-\~na nu guahu [ mal\"agu' gui [ asudd\"a'-\~na {\it
t\/} ] ] ] ] @ .//
\glb agr-meet we with the person Op Comp the Juan say-agr Obl me
agr.want he {\it WH\/}[obl].meet-agr//
\glft ``I met the person who Juan told me he wanted to
meet.''//
\endgl
\xe
\endframedisplay
\codedisplay~
\ex
\begingl
\glpreamble Um-\"asudda' h\"am yan i taotao ni si Juan
   ilek-\~na nu guahu mal\"agu' gui asudd\"a'-\~na.//
\gla Um-\"asudda' h\"am yan [ i taotao [ {\it O\/} ni si Juan
   ilek-\~na nu guahu [ mal\"agu' gui [ asudd\"a'-\~na {\it
   t\/} ] ] ] ] @ .//
\glb agr-meet we with the person Op Comp the Juan say-agr Obl me
   agr.want he {\it WH\/}[obl].meet-agr//
\glft ``I met the person who Juan told me he wanted to
   meet.''//
\endgl
\xe
|endcodedisplay

Glosses (|\begingl| \dots\ |\endgl|) have up to three
parts.  First, an optional preamble (|\glpreamble| \dots\ |//|).
Second, an interlinear gloss, which is also optional if there is
a preamble.\footnote{%
Contrary to first impressions, there are good reasons to allow the
interlinear gloss to be omitted in certain situations.}  Third, a free
translation (|\glft| \dots\ |//|), which can be omitted if there is an
interlinear gloss. They are illustrated below, which repeats (\lastx)
with a narrower hsize.

\begingroup
\input pstricks-add
\ex%[glbreaking=false]
\def\TOP{\pnode(3.4in,1.5ex)}%
\def\BOT{\pnode(0,-.5ex)}%
\begingl[glwidth=3.2in]
\glpreamble \TOP{A1}Um-\"asudda' h\"am yan i taotao ni si Juan
ilek-\~na nu guahu mal\"agu' gui
asudd\"a'-\~na.\BOT{B1}//
\gla \TOP{A2}Um-\"asudda' h\"am yan [ i taotao [ {\it O\/} ni si Juan
ilek-\~na nu guahu [ mal\"agu' gui [ asudd\"a'-\~na {\it
t\/} ] ] ] ] @ .//
\glb agr-meet we with the person Op Comp the Juan say-agr Obl me
agr.want he {\it WH\/}[obl].meet-agr\BOT{B2}//
\glft \TOP{A3}``I met the person who Juan told me he wanted to
meet.''\BOT{B3}//
\endgl
\SpecialCoor
\psbrace[ref=lC](A1|B1)(A1){ preamble}
\psbrace[ref=lC](A1|B2)(A2){ interlinear gloss}
\psbrace[ref=lC](A1|B3)(A3){ free translation}
\xe
\endgroup

The interlinear gloss consists of a sequence of lines of the form
$$\hbox{|\gl|{\it levelname\/} \dots\ |//|}$$
where {\it levelname\/} is |a|, |b|, or |c|.\footnote{%
This will be extended later to allow the user to define new level
names.}  There must be one and only one |\gla| line, which must come
first in the interlinear gloss. |\glb| and |\glc| lines can come in
any order and can be repeated arbitrarily.

The material delineated by |\gl|{\it levelname\/} and |//| is parsed
as a sequence of space separated items.  The parser only looks for
spaces at the top-level.  Consequently, in (\blastx), for example, it
is not sensitive to the space in items like |the-{\sc acc}| since the
space is inside a group, therefore not at the top level.  Spaces that
directly precede terminating |//| are disregarded.  A line in the
interlinear gloss can have fewer items on it than the line above it. It
acts as if it ended with empty |{}| items.  But {\it no line in the
interlinear gloss can have more items than the line above it}.  If
there are ``excess'' items on a line, they will act as if they are on
the higher line (recursively).

\subsection Parameters

\def\\#1{\noalign{\smallskip}\quad #1\cr}%
\hrule
\medskip
\parinventory
\\{Horizontal spacing\hidewidth}
\quad & \idx{|glspace|\user}& \inc dimension& \textdim{1 em}\cr
\\{Font changes, struts, etc.\hidewidth}
& \idx{|everygl|}& token list& |{}|\cr
& \idx{|everyglpreamble|}& token list& |{}|\cr
& \idx{|everygla|}& token list& |\it|\cr
& \idx{|everyglb|}& token list& |{}|\cr
& \idx{|everyglc|}& token list& |{}|\cr
& \idx{|everyglft|}& token list& |{}|\cr
\\{Vertical spacing\hidewidth}
& \idx{|belowglpreambleskip|\user}& (!)dimension& \textdim{1 ex}\cr
& \idx{|aboveglbskip|\user}& (!)dimension& \textdim{0 pt}\cr
& \idx{|aboveglcskip|\user}& (!)dimension& \textdim{0 pt}\cr
& \idx{|aboveglftskip|\user}& (!)dimension& \textdim{1 ex}\cr
\endparinventory
\vskip-1ex
\hrule
\medskip

Suppose for example that rather than the gloss in (\getref{wapm}) the
following style is desired.

\framedisplay
\ex
\begingl[glspace=2em]
\gla[everygla=] k- wapm -a -s'i -m -wapunin -uk //
\glb[everyglb=\sc,aboveglbskip=-.4ex]
   cl v agr neg agr tns agr //
\glb 2 see {\sc 3acc} {} {\sc 2pl} preterit {\sc 3pl} //
\glft[everyglft=\it,aboveglftskip=0pt] `you (pl) didn't see them'//
\endgl
\xe
\endframedisplay
There are several differences with (\getref{wapm}).  The words are
more widely separated; the font in the |\gla| line is not italic;  the
font in the free translation is italic; the first |\glb| line is set
in a smaller font and is moved up closer to the |\gla| line; and there
is no extra vertical skip between the free translation and the last
line of the interlinear gloss, as there is in (\getref{wapm}).

Here is one way to write the code:

\codedisplay
\ex
\begingl[glspace=2em]
\gla[everygla=] k- wapm -a -s'i -m -wapunin -uk //
\glb[everyglb=\sc,aboveglbskip=-.4ex]
   cl v agr neg agr tns agr //
\glb 2 see {\sc 3acc} {} {\sc 2pl} preterit {\sc 3pl} //
\glft[everyglft=\it,aboveglftskip=0pt] `you (pl) didn't see them'//
\endgl
\xe
|endcodedisplay

Here is an alternative which produces the same output.  It has the
advantage that all of the parameter settings are gathered in one
place, making it easier to modify.

\codedisplay
\ex
\begingl[glspace=2em,everygla=,everyglb=\sc,
   aboveglbskip=-.4ex,everyglft=\it,aboveglftskip=0pt]
\gla k- wapm -a -s'i -m -wapunin -uk //
\glb cl v agr neg agr tns agr //
\glc 2 see {\sc 3acc} {} {\sc 2pl} preterit {\sc 3pl} //
\glft `you (pl) didn't see them'//
\endgl
\xe
|endcodedisplay

If you use multiple instances of the same gloss format, a style should
be defined

\codedisplay
\definelingstyle{Potawatami}{glspace=2em,everygla=,everyglb=\sc,
   aboveglbskip=-.4ex,everyglft=\it,aboveglftskip=0pt}
|endcodedisplay

Then, typesetting a gloss in that style is done simply.

\codedisplay
\ex
\begingl[lingstyle=Potawatami]
\gla k- wapm -a -s'i -m -wapunin -uk //
\glb Ccl V Agr Neg Agr Tns Agr //
\glc 2 see {\sc 3acc} {} {\sc 2pl} preterit {\sc 3pl} //
\glft `you (pl) didn't see them'//
\endgl
\xe
|endcodedisplay

If you want a sequence of glosses to all be done in this style, you can say:

\def\goop{\quad $\vdots$}
\codedisplay
\begingroup
\lingset{lingstyle=Potawatami}
|goop
\endgroup
|endcodedisplay

\subsection The parameter {\tt glwidth}

If the parameter |glwidth| is set to a nonzero value, the gloss is
built in a vbox whose width is the setting of |glwidth|. The following
example illustrates the usefulness of the explicit width option.

\ex[glwidth=2.6in]<sicher2>
a.\quad
\begingl
\gla Mary$_i$ ist sicher, dass es den Hans nicht st\"oren w\"urde
seiner Freundin ihr$_i$ Herz auszusch\"utten.//
\glb Mary is sure that it the-{\sc acc} Hans not annoy would
his-{\sc dat} girlfriend-{\sc dat} her-{\sc acc} heart-{\sc acc} {out to
throw}//
\glft  `Mary is sure that it would not annoy John to reveal her
heart to his girlfriend.'//
\endgl
\hfil
b.\quad
\begingl
\gla Mary$_i$ ist sicher, dass seiner Freunden ihr$_i$ Herz
auszuch\"utten dem Hans nicht schaden w\"urde.//
\glb Mary is sure that his-{\sc dat} girlfriend-{\sc dat} her-{\sc acc}
heart-{\sc acc} {out to throw} the-{\sc dat} Hans not damage would//
\glft `Mary is sure that to reveal her heart to his girlfriend
would not damage John.'//
\endgl
\xe

\codedisplay~
\ex[glwidth=2.6in]
a.\quad
\begingl
\gla Mary$_i$ ist sicher, dass es den Hans nicht st\"oren w\"urde
seiner Freundin ihr$_i$ Herz auszusch\"utten.//
\glb Mary is sure that it the-{\sc acc} Hans not annoy would
his-{\sc dat} girlfriend-{\sc dat} her-{\sc acc} heart-{\sc acc} {out to
throw}//
\glft  `Mary is sure that it would not annoy John to reveal her
heart to his girlfriend.'//
\endgl
\hfil
b.\quad
\begingl
\gla Mary$_i$ ist sicher, dass seiner Freunden ihr$_i$ Herz
auszuch\"utten dem Hans nicht schaden w\"urde.//
\glb Mary is sure that his-{\sc dat} girlfriend-{\sc dat} her-{\sc acc}
heart-{\sc acc} {out to throw} the-{\sc dat} Hans not damage would//
\glft `Mary is sure that to reveal her heart to his girlfriend
would not damage John.'//
\endgl
\xe
|endcodedisplay



%
%
%
%\subsection The width of the gloss
%
%In the wrap style (the only gloss style currently implemented),
%the gloss is built in a vbox whose width is determined implicitly
%if the parameter |glwidth| is set to \textdim{0 pt}. The width is
%$h-l-r$, where $h$, $l$, and $r$, are the current values of
%|\hsize|, |\leftskip|, and |\rightskip|, respectively.  This
%implicit determination of the width of the gloss is appropriate
%for use with the \ExPex\ macros which typeset examples because
%they adjust the leftskip appropriately inside examples.
%
%If you want to supply an example number or explicit label, it will not
%work to say something like the following if |glwidth| is set to
%\textdim{0 pt}.
%
%\def\goop{\dots\ }
%\codedisplay
%[A]\quad \begingl |goop \endgl
%|endcodedisplay

%The vbox built by the gloss macro will not fit on the same line with
%the [A].
%
%You must say:
%
%\def\temp{\dots }
%\codedisplay
%\ex[exno=A,exnoformat={[X]}] \begingl |temp \endgl
%|endcodedisplay
%The braces around |[A]| are needed to so that the optional argument
%is correctly delineated.  For example:
%
%\framedisplay
%\ex[exno={(6), p. 14},exnoformat={[X]}]
%\begingl
%\gla Mary$_i$ ist sicher, dass es den Hans nicht st\"oren w\"urde
%seiner Freundin ihr$_i$ Herz auszusch\"utten.//
%\glb Mary is sure that it the-{\sc acc} Hans not annoy would
%his-{\sc dat} girlfriend-{\sc dat} her-{\sc acc} heart-{\sc acc} {out to
%throw}//
%\endgl
%\xe
%\endframedisplay
%\codedisplay~
%\ex[exno={(6), p. 14},exnoformat={[X]}]
%\begingl
%\gla Mary$_i$ ist sicher, dass es den Hans nicht st\"oren w\"urde
%seiner Freundin ihr$_i$ Herz auszusch\"utten.//
%\glb Mary is sure that it the-{\sc acc} Hans not annoy would
%his-{\sc dat} girlfriend-{\sc dat} her-{\sc acc} heart-{\sc acc} {out to
%throw}//
%\endgl
%\xe
%|endcodedisplay
%
%If the parameter |glwidth| is set to a nonzero dimension, the width of
%the vbox that the gloss is constructed in is the specified dimension.
%
%The following example illustrates the usefulness of the explicit
%width option.
%
%\ex[glwidth=2.6in]<sicher2>
%a.\quad
%\begingl
%\gla Mary$_i$ ist sicher, dass es den Hans nicht st\"oren w\"urde
%seiner Freundin ihr$_i$ Herz auszusch\"utten.//
%\glb Mary is sure that it the-{\sc acc} Hans not annoy would
%his-{\sc dat} girlfriend-{\sc dat} her-{\sc acc} heart-{\sc acc} {out to
%throw}//
%\glft  `Mary is sure that it would not annoy John to reveal her
%heart to his girlfriend.'//
%\endgl
%\hfil
%b.\quad
%\begingl
%\gla Mary$_i$ ist sicher, dass seiner Freunden ihr$_i$ Herz
%auszuch\"utten dem Hans nicht schaden w\"urde.//
%\glb Mary is sure that his-{\sc dat} girlfriend-{\sc dat} her-{\sc acc}
%heart-{\sc acc} {out to throw} the-{\sc dat} Hans not damage would//
%\glft `Mary is sure that to reveal her heart to his girlfriend
%would not damage John.'//
%\endgl
%\xe
%
%\codedisplay~
%\ex[glwidth=2.6in]
%a.\quad
%\begingl
%\gla Mary$_i$ ist sicher, dass es den Hans nicht st\"oren w\"urde
%seiner Freundin ihr$_i$ Herz auszusch\"utten.//
%\glb Mary is sure that it the-{\sc acc} Hans not annoy would
%his-{\sc dat} girlfriend-{\sc dat} her-{\sc acc} heart-{\sc acc} {out to
%throw}//
%|exbreak\glft  `Mary is sure that it would not annoy John to reveal her
%heart to his girlfriend.'//
%\endgl
%\hfil
%b.\quad
%\begingl
%\gla Mary$_i$ ist sicher, dass seiner Freunden ihr$_i$ Herz
%auszuch\"utten dem Hans nicht schaden w\"urde.//
%\glb Mary is sure that his-{\sc dat} girlfriend-{\sc dat} her-{\sc acc}
%heart-{\sc acc} {out to throw} the-{\sc dat} Hans not damage would//
%\glft `Mary is sure that to reveal her heart to his girlfriend
%would not damage John.'//
%\endgl
%\xe
%|endcodedisplay


\subsection Comments and citations

\begininventory
\macros
\idx{|\trailingcitation|}, \idx{|\rightcomment|}\endmc
\parameters*
|mincitesep|& dimension& \hfil \textdim{1.5 em}\cr
\endinventory

The following illustrates two different ways to append a comment or
citation to a gloss.

\framedisplay
\ex
\begingl
\gla \rightcomment{[Potawatami]}k- wapm -a -s'i -m -wapunin -uk //
\glb \rightcomment{category}Cl V Agr$_1$ Neg Agr$_2$ Tns Agr$_3$//
\glb 2 see {\sc 3acc} {} {\sc 2pl} preterit {\sc 3pl} //
\glft `you (pl) didn't see them'\trailingcitation{(Hockett 1948,
   p. 143)}//
\endgl
\xe
\endframedisplay
\codedisplay~
\ex
\begingl
\gla \rightcomment{[Potawatami]}k- wapm -a -s'i -m -wapunin -uk //
\glb \rightcomment{category}Cl V Agr$_1$ Neg Agr$_2$ Tns Agr$_3$//
\glb 2 see {\sc 3acc} {} {\sc 2pl} preterit {\sc 3pl} //
\glft `you (pl) didn't see them'\trailingcitation{(Hockett 1948,
   p. 143)}//
\endgl
\xe
|endcodedisplay
|\trailingcitation| will put the citation on the same line as the last
line of the free translation if there is enough room for it, otherwise at the
end of the  following line.  The parameter |mincitesep| determines the
minimum whitespace between the end of the free translation and the
citation that is tolerated; the default is \textdim{1.5 em}.

|\rightcomment{|\dots|}| must come first in the first item on the
line.  The macro is very primitive.  It does not
consider the width of the citation or the amount of whitespace at
the right of the gloss.  The citation will overlap the gloss if
there is not room for it at the right.  For example, if the gloss
(\lastx) is attempted with an hsize of \textdim{3.5 in} and
|glspace=.5em|, the result is (\nextx).

\framedisplay
\hsize=3.5in
\ex
\rightcomment{[Potawatami]}
\begingl
\gla k- wapm -a -s'i -m -wapunin -uk //
\glb \rightcomment{category}Cl V Agr$_1$ Neg Agr$_2$ Tns Agr$_3$//
\glb 2 see {\sc 3acc} {} {\sc 2pl} preterit {\sc 3pl} //
\glft `you (pl) didn't see them'\trailingcitation{(Hockett 1948,
   p. 143)}//
\endgl
\xe
\endframedisplay

In spite of its limitations, |\rightcomment| is occasionally
quite useful.

\subsection  Exceptional {\tt \char"5C gla} items
\deftag{\the\secno.\the\subsecno}{atdiacritic}

Items on the |\gla| line are generally associated with items on
the other lines of the interlinear gloss.  There are however a
few items, called here {\it exceptional items}, which are
interpreted in an exceptional fashion. There are three kinds of exceptional
items: the single character |+| or |@|, and any item which begins with
the control sequence |\nogloss|.\footnote{Version 4 of {\it Expex} had
two additional exceptional items, the single characters |[| and |]|.
In version 5, these characters still operate as in Version 4, but
their use is discouraged and their use as exceptional |\gla|
characters will disappear from {\it ExPex} at some point.  |\nogloss|
is more general, as explained below.}

\subsubsection {\tt +}
\deftag{\the\secno.\the\subsecno.\the\subsubsecno}{plusdiacritic}

\sidx{|+|, exceptional gloss item}%
Sometimes it is desirable to override natural wrapping and
break up the gloss so that the syntax is emphasized, as in the
following.

\framedisplay
\ex<Marysicher>
\begingl
\gla Mary$_i$ ist sicher, + dass es den Hans nicht st\"oren w\"urde
+ seiner Freundin ihr$_i$ Herz auszusch\"utten.//
\glb Mary is sure that it the-{\sc acc} Hans not annoy would
his-{\sc dat} girlfriend-{\sc dat} her-{\sc acc} heart-{\sc acc} {out to
throw}//
\glft  `Mary is sure that it would not annoy John to reveal her
heart to his girlfriend.'//
\endgl
\xe
\endframedisplay

\bigskip
This is accomplished by inserting `|+|' appropriately, as shown in the
code below.  When |+| is encountered, the line is broken and a new
line started, ignoring any hanging indentation.

\codedisplay
\ex\begingl
\gla Mary$_i$ ist sicher, + dass es den Hans nicht st\"oren w\"urde
+ seiner Freundin ihr$_i$ Herz auszusch\"utten.//
\glb Mary is sure that it the-{\sc acc} Hans not annoy would
his-{\sc dat} girlfriend-{\sc dat} her-{\sc acc} heart-{\sc acc} {out to
throw}//
|exbreak
\glft  `Mary is sure that it would not annoy John to reveal her
heart to his girlfriend.'//
\endgl
\xe
|endcodedisplay

\subsubsection {\tt @}
\deftag{\the\secno.\the\subsecno.\the\subsubsecno}{atdiacritic}

\sidx{|@|, exceptional gloss item}%
Sometimes it is desirable to omit the space between two entries.

\framedisplay
\ex<wiye>
\begingl
\gla wiye kepi e- @ ca//
\glb two whitemen {\sc 1p:3d}- found//
\endgl
\xe
\endframedisplay

This is accomplished by inserting `|@|' appropriately, as shown
in the code below.

\codedisplay
\ex
\begingl
\gla wiye kepi e- @ ca//
\glb two whitemen {\sc 1p:3d}- found//
\endgl
\xe
|endcodedisplay
In the unlikely event that you need a gla entry which
would normally be entered as |@|, enter it as |{\relax @}|
so that it is not interpreted as a directive to omit a
space.\footnote{This is a change.  In version 4, |{{@}}| was
suggested.  This no longer works because various parsing steps strip
away the grouping.  |{{{{@}}}}| does work, but |{\relax @}| is more straightforward.}
(\getref{@period}) below shows another use for the |@| diacritic.

\exbreak
\subsubsection {\tt \char"5C nogloss}

Suppose you want to produce gloss displays like the ones below.

\framedisplay
\pex[everygla=,glhangstyle=normal]<@period>
\a
\begingl
\gla Fa'nu'i yu' ni \nogloss{[[} @ {\it O} t{\it in\/}aitai-mu
\nogloss{{\it t\/}]} na {lepblu].}//
\glb show me Obl Op {\it WH\/}[obj].read-agr L book//
\glft ``Show me the book that you read.''//
\endgl
\a \begingl
\gla Um-\"asudda' h\"am yan \nogloss{[} @ i taotao \nogloss{[} @ {\it O\/} ni si Juan
ilek-\~na nu guahu \nogloss{[} @ mal\"agu' gui \nogloss{[} @ asudd\"a'-\~na
\nogloss{\it t\/} \nogloss{]]]].}//
\glb agr-meet we with the person Op Comp the Juan say-agr Obl me
agr.want he {\it WH\/}[obl].meet-agr//
\glft ``I met the person who Juan told me he wanted to meet.''//
\endgl
\xe
\endframedisplay
\noindent The brackets are not glossed.

(\lastx) is produced by:

\codedisplay
\pex[everygla=,glhangstyle=normal]<@period>
\a
\begingl
\gla Fa'nu'i yu' ni \nogloss{[[} @ {\it O} t{\it in\/}aitai-mu
\nogloss{{\it t\/}]} na {lepblu].}//
\glb show me Obl Op {\it WH\/}[obj].read-agr L book//
\glft ``Show me the book that you read.''//
\endgl
\a \begingl
\gla Um-\"asudda' h\"am yan \nogloss{[} @ i taotao \nogloss{[} @ {\it O\/}
ni si Juan ilek-\~na nu guahu \nogloss{[} @ mal\"agu' gui \nogloss{[} @
asudd\"a'-\~na \nogloss{\it t\/} \nogloss{]]]].}//
\glb agr-meet we with the person Op Comp the Juan say-agr Obl me agr.want
he {\it WH\/}[obl].meet-agr//
\glft ``I met the person who Juan told me he wanted to meet.''//
\endgl
\xe
|endcodedisplay


\everymath={}
\lingset{glwidth=2.4in,everygl=,glhangstyle=none}
\psset{framesep=1ex}

%\ex[textoffset=1in,dima=.5em]<Chung1>
%\begingl[extraglskip=1.2ex]
%\gla Ti ma'a'\~nao hao kumuentusi\pnode(\lingdima,0){B2} \pnode(-\lingdima,0){A1}ni h\'ayiyi ha'.//
%\glb \pnode(-.5em,0){A2}not agr.afraid you
%Infin.speak.to\pnode(\lingdima,0){B1} not anyone Emp//
%\glft ``You're not afraid to talk to anyone.''//
%\endgl
%\pcline{|*-|*}(A2)(A1)
%\pcline{<->}(A2)(A1)
%\nbput[ref=r]{glskip\ }
%\pcline{|*-|*}(B1|B2)(B1)
%\pcline{<->}(B1|B2)(B1)
%\naput[ref=l]{\ baselineskip}
%\xe


\def\minigloss#1#2{\outlinebox{%
   \vtop{\halign{\ministrut ##\hfil\cr #1\cr #2\cr}}}\hskip1pt}%
\let\ministrut=\relax
\def\miniglossbare#1#2{\vtop{\halign{##\hfil\cr
   #1\cr #2\cr}}}
\psset{linewidth=.4pt}
\def\outlinebox{\psframebox[framesep=0,boxsep=false,linewidth=.4pt]}

\subsection  Line spacing in wrapped glosses

\subsubsection Changes from version 4.0

\begininventory
\parameters*
\idx{|abovemoreglskip|}\user& (!)skip& \hfil \textdim{1 ex}& now obsolete\cr
\endinventory

\noindent
Executing the macro |\gloldstyle| will return \ExPex\ to a state in
which glosses written using version 4.0 should display as they
did using version 4.0.  The parameter |abovemoreglskip| which
was used to control wrap spacing in glosses is still recognized,
but it is now considered obsolete and should be avoided in future
work.

\subsubsection Simple control

\begininventory
\parameters*
\idx{|autoglskip|}& boolean& \hfil true\cr
\idx{|extraglskip|}\user& \inc skip& \hfil \textdim{0 pt}\cr
\endinventory

\noindent In deciding on a gloss format, the main decision is
the value of what I call the {\it glskip}, as illustrated
below.

%\ex[textoffset=1in,dima=.5em,glbreaking=false]<Chung1>
\ex[textoffset=1in,dima=.5em]<Chung1>
\begingl[extraglskip=1.2ex]
\gla Ti ma'a'\~nao hao kumuentusi\pnode(\lingdima,0){B2} \pnode(-\lingdima,0){A1}ni h\'ayiyi ha'.//
\glb \pnode(-.5em,0){A2}not agr.afraid you
Infin.speak.to\pnode(\lingdima,0){B1} not anyone Emp//
\glft ``You're not afraid to talk to anyone.''//
\endgl
\pcline{|*-|*}(A2)(A1)
\pcline{<->}(A2)(A1)
\nbput[ref=r]{glskip\ }
\pcline{|*-|*}(B1|B2)(B1)
\pcline{<->}(B1|B2)(B1)
\naput[ref=l]{\ baselineskip}
\xe
Generally, the desired glskip is equal to the baselineskip or
equal to the baselineskip plus some specified increment.  When I
have occasion to typeset glosses, I generally make the glskip
equal to the baselineskip plus \textdim{.6 ex}.  The extra space,
although small, makes it much easier to parse the gloss visually.
The extra glskip in (\lastx) is \textdim{1.2 ex}.

ExPex initializes the glossing algorithm so that it produces
glosses like~(\nextx). The glskip is equal to the baselineskip.
This is done in {\it expex.tex\/} by setting |autoglskip| to true
and |extraglskip| to \textdim{0 pt}. In adjusting the glskip,
struts are used.  It is assumed that the strut height plus the
strut depth is equal to the baselineskip. Most users will never
reset |autoglskip|.  The next section discusses vertical spacing
when |autoglskip| is set to false. (|glwidth| is set to
\textdim{2.4 in} in all of the examples in this section in the
interests of ``economy of space''.)

\setss .52 .48

\beginss
\ex
\begingl
\gla Ti ma'a'\~nao hao kumuentusi
   ni h\'ayiyi ha'.//
\glb not agr.afraid you
   Infin.speak.to not anyone Emp//
\endgl
\xe|midss
\ex<Ti>
\begingl
\gla Ti ma'a'\~nao hao kumuentusi ni h\'ayiyi ha'.//
\glb not agr.afraid you Infin.speak.to not anyone Emp//
\endgl
\xe
\endss

It is easy to produce glosses like (\nextx) by adjusting the
parameter |extraglskip|.  In (\nextx), the glskip is \textdim{1.2
ex} larger than the baselineskip.

\beginss
\ex
\begingl[extraglskip=1.2ex]
\gla Ti ma'a'\~nao hao kumuentusi
   ni h\'ayiyi ha'.//
\glb not agr.afraid you
   Infin.speak.to not anyone Emp//
\endgl
\xe|midss
\ex
\begingl[extraglskip=1.2ex]
\gla Ti ma'a'\~nao hao kumuentusi ni h\'ayiyi ha'.//
\glb not agr.afraid you Infin.speak.to not anyone Emp//
\endgl
\xe
\endss

If |extraglskip| is expressed in font dependent units, the format
for glosses will not need adjustment for glosses in footnotes
which are set in smaller type with tighter baselines. This
assumes that the footnote environment adjusts the strut size
appropriately, as it should, with the height plus depth of struts
made equal to the baselineskip. The text in (\nextx) is set in a
\textdim{12 pt} font with \textdim{14 pt} baselines.  Struts are
\textdim{10 pt} high and 4pt deep. |\fnenvironment| switches to a
\textdim{10 pt} font with \textdim{12 pt} baselines, with struts
that are 8.5pt high and 3.5pt deep.

\lingset{autoglskip=true}

\beginss
\ex
\begingl[extraglskip=1ex]
\gla Ti ma'a'\~nao hao kumuentusi
   ni h\'ayiyi ha'.//
\glb not agr.afraid you
   Infin.speak.to not anyone Emp//
\endgl
\xe|midss
\ex
\begingl[extraglskip=1ex]
\gla Ti ma'a'\~nao hao kumuentusi ni h\'ayiyi ha'.//
\glb not agr.afraid you Infin.speak.to not anyone Emp//
\endgl
\xe
\endss
\vskip-2ex

\beginss
\fnenvironment
\ex
\begingl[extraglskip=1ex]
\gla Ti ma'a'\~nao hao kumuentusi
   ni h\'ayiyi ha'.//
\glb not agr.afraid you
   Infin.speak.to not anyone Emp//
\endgl
\xe|midss
\lingset{autoglskip=true,extraglskip=1ex}
\baselineskip=12pt
\setbox\strutbox=\hbox{\vrule height 8.5pt depth 3.5pt width 0pt}
\footnotesize
\ex
\begingl[extraglskip=1ex]
\gla Ti ma'a'\~nao hao kumuentusi ni h\'ayiyi ha'.//
\glb not agr.afraid you Infin.speak.to not anyone Emp//
\endgl
\xe
\endss

Most users will never need to know more about glossing than that
provided in this section. But some, at some point, might face
some unusual situation requiring more control over the line
spacing than the control which |extraglskip| allows.  The
next section is for such users only.

\subsubsection The techonology behind vertical spacing in glosses

\begininventory
\parameters
\idx{|gllineskip|}\user & \inc skip& \hfil\textdim{1 ex}\cr
\idx{|glstruts|}& boolean& \hfil true\cr
\endinventory



\noindent  Wrapping is carried out by first forming a sequence of
boxes, as in (\nextx) and then feeding this sequence to Tex's
apparatus for breaking paragraphs into lines and stacking
sequences of lines into pages.  The baseline of these boxes is
the baseline of the top line.  As far as Tex is concerned, the
sequence of boxes is a sequence of words; a paragraph.  We use
(\getref{Ti}) as an illustration.  If struts are inserted before
every word of every box, we get (\nextx b), otherwise (\nextx a).
Generally, struts are inserted because this gives much better
control of the line spacing.

\pex[nopreamble,interpartskip=3ex]<seq>
\let\\=\minigloss
\spaceskip=\lingglspace
\a \\{Ti}{not}, \\{ma'a'\~nao}{agr.afraid}, \\{hao}{you},
\\{kumuentusi}{Infin.speak.to}, \\{ni}{not}, \\{h\'ayiyi}{anyone},
\\{ha'}{Emp}
\let\ministrut=\strut
\a \\{Ti}{not}, \\{ma'a'\~nao}{agr.afraid}, \\{hao}{you},
\\{kumuentusi}{Infin.speak.to}, \\{ni}{not}, \\{h\'ayiyi}{anyone},
\\{ha'}{Emp}
\xe
The boxes will be called {\it glwords}.

Suppose we feed (\lastx b), for example, to Tex's page formation
machinery.  The outcome, will depend on many Tex parameters.  Of
particular interest for the line spacing are |\baselineskip| and
|\lineskip|.  |\lineskiplimit| is set to \textdim{0 pt} in
glosses. There are nested environments; one governing stacking
words into glwords, the other governing stacking lines of
glwords. Call these the interior and exterior environments.  If
we consider (\nextx), for example, note that the dimension of
most interest, the vertical spacing between the {\it b\/} and
{\it c\/} lines is not a distance between baselines in either the
interior or exterior environments.  When glwords are built, the
baseline spacing is between the {\it a\/} and {\it b\/} lines
(the {c-d\/} spacing is the same).  When the glwords are formed
into a paragraph, the baseline spacing is between the {\it a\/}
and {\it c\/} lines. Line {\it a\/} is the baseline of the top
row of glwords and line {\it c\/} is the baseline of bottom row.

\ex[linestyle=dotted,linewidth=1pt]
\begingl[extraglskip=1ex]
\gla \pnode{A1}Ti ma'a'\~nao hao kumuentusi\pnode{B1} ni h\'ayiyi
ha'.\pnode{B3}//
\glb not agr.afraid you Infin.speak.to\pnode(1em,0){B2} not anyone
Emp\pnode{B4}//
\endgl
\pcline(A1|B1)(B2|B1)
\pcline(A1|B2)(B2|B2)
\pcline(A1|B3)(B2|B3)
\pcline(A1|B4)(B2|B4)
\rput(B2|B1){\pnode(1em,0){C1}}
\rput(C1){\tenit a}
\rput(C1|B2){\tenit b}
\rput(C1|B3){\tenit c}
\rput(C1|B4){\tenit d}
\xe
A desired value for the {\it b-c\/} line spacing must be achieved
indirectly.

ExPex leaves both the baselineskip in the exterior environment
the same as it is inside the glwords. This means that the
baselineskip will have no effect when the paragraph (\lastx a) or
(\lastx b) is typeset because the glwords are deeper than the
baselineskip.  \ExPex\ adjusts the |\lineskip| in the exterior
environment to achieve the desired {\it b-c\/} line spacing; that
is, to achieve the desired glskip.

(\nextx) illustrates the relation between |\lineskip| and the
glskip, assuming that strut insertion has taken place.

\ex[textoffset=.8in,aboveexskip=!1ex]
\def\outlinebox{\psframebox[framesep=0,boxsep=false,linewidth=.15pt]}
\let\\=\minigloss
\let\ministrut=\strut
\psscalebox{2}{\vtop{\leftskip=0pt \raggedright \hsize=2.4in
\lineskip=2ex
\leavevmode
\\{\pnode{Y1}Ti}{\pnode{Y3}not}\
\\{ma'a'\~nao}{agr.\pnode(0,-\dp\strutbox){A1}afraid}\
\\{hao}{you}\
\\{kumuentusi}{Infin.speak.to}\
\\{\pnode(-.2em,0){Y2}ni}{not}\
\\{h\'ayiyi\pnode(0,\ht\strutbox){A2}}{anyone}\
\\{ha'}{Emp}%
}}
\pcline[linewidth=1pt]{<->}(A1)(A1|A2)
\naput[ref=l]{lineskip}
\pcline[linewidth=1pt]{<->}(Y2|Y3)(Y2|Y2)
\pcline[linewidth=1pt]{|*-|*}(Y2|Y3)(Y2|Y2)
\nbput[ref=r]{glskip}
\xe

If autoglskip is turned on, |\lineskip| is adjusted so that the
glskip is equal to the baselineskip plus the value of
|extraglskip|.  If $|\baselineskip|=b$, $|\dp\strutbox|=d$,
$|\ht\strutbox|=h$, and $|extraglskip|=e$, then |\lineskip|
is set to $b+e-d-h$.

If |autoglskip| is turned off, struts are inserted only if
|glstruts| is turned on; |\lineskip| is made equal to
|gllineskip|.  With |glstruts| turned off, the glskip will depend
on |gllineskip|, the depth of the deepest character on the bottom
row of the top line of glwords, and the height of the tallest
character on the top row of the bottom line of glwords.

%\def\showchar#1{\setbox0=\hbox{#1}\writeln{#1: \the\ht0, \the\dp0}}
%\showchar{g}
%\showchar{p}
%\showchar{y}
%\showchar{\it h}
%\showchar{\it \'a}
%\showchar{\it '}
%\showchar{\it i}

The above will be clearer if you do the following exercise:
Calculate the glskip in the 3~examples below. The font is
\textdim{12 pt} with struts that are \textdim{10 pt} high and
\textdim{4 pt} deep.  The baselineskip is \textdim{18 pt}.  Roman
g, y, and p are all \textdim{2.60 pt} deep; italic h, italic ',
and italic i are all \textdim{7.96 pt} tall; and italic \'a is
\textdim{7.93 pt} tall.

\beginss
\ex
\begingl[autoglskip=true,
   extraglskip=4pt]
\gla Ti ma'a'\~nao hao kumuentusi
   ni h\'ayiyi ha'.//
\glb not agr.afraid you
   Infin.speak.to not anyone Emp//
\endgl
\xe|midss
\ex
\baselineskip=18pt
\begingl[autoglskip=true,extraglskip=4pt]
\gla Ti ma'a'\~nao hao kumuentusi ni h\'ayiyi ha'.//
\glb not agr.afraid you Infin.speak.to not anyone Emp//
\endgl
\xe
\endss

\beginss
\ex
\begingl[autoglskip=false,
   glstruts=true,gllineskip=8pt]
\gla Ti ma'a'\~nao hao kumuentusi
   ni h\'ayiyi ha'.//
\glb not agr.afraid you
   Infin.speak.to not anyone Emp//
\endgl
\xe|midss
\ex
\baselineskip=18pt
\begingl[autoglskip=false,glstruts=true,gllineskip=8pt]
\gla Ti ma'a'\~nao hao kumuentusi ni h\'ayiyi ha'.//
\glb not agr.afraid you Infin.speak.to not anyone Emp//
\endgl
\xe
\endss

\def\goop{$\,^{\the\fnno}$}
\beginss
\ex
\begingl[autoglskip=false,
   glstruts=false,gllineskip=8pt]
\gla Ti ma'a'\~nao hao kumuentusi
   ni h\'ayiyi ha'.//
\glb not agr.afraid you
   Infin.speak.to not anyone Emp//
\endgl
\xe|putfnno|midss
\ex
\baselineskip=18pt
\begingl[autoglskip=false,glstruts=false,gllineskip=8pt]
\gla Ti ma'a'\~nao hao kumuentusi ni h\'ayiyi ha'.//
\glb not agr.afraid you Infin.speak.to not anyone Emp//
\endgl
\xe
\endss
\footnotebody{\hbox{\rput[tr]{180}{Answers: \textdim{22 pt}, \textdim{22 pt},
and \textdim{18.56 pt}}}
}

\subsubsection How to avoid overfull boxes in glosses

\begininventory
\parameters
\idx{|glspace|}& skip& \textdim{.5 em} plus \textdim{.4 em} minus
   \textdim{.15 em}\hfil\cr
\idx{|glrightskip|}& skip& 0\dimskip pt plus .1|\hsize|\hfil \cr
\endinventory
%
Hopefully, you will never encounter the problem which the
parameterization in this section is designed to solve, the
``overfull line'' error message.  The default setting puts
\textdim{.5 em} of glue between the glwords on a line, plus
\textdim{.4 em} of stretchability and \textdim{.15 em} of
shrinkage.  The interglword spacing on a line can therefore vary
from \textdim{.35 em} up to \textdim{.9 em} to accommodate the
needs of the Tex linebreaking algorithm.\footnote{%
This is a much larger range than is typical in running text, but
it is appropriate in glosses, which typically have many large
whitespace gaps on one line or the other.}
Further, the default settings allow up to 10\% of the hsize of
whitespace to appear at the end of the line of glwords.  The
possible extra space at the right edge and
stretchability/shrinkability of the space between glwords means
that line breaking in glosses will rarely encounter problems with
overfull lines.

In unusual cases, there may be a problem.  Your publisher (or
you) may be particularly fussy and demand a particular spacing
and/or better right alignment in glosses.  Or you might want to make
an unusually narrow gloss, which increases the chances that there
might not be enough flexibility in the spacing.

If you do run into the problem of overfull lines in a gloss, two
parameters allow for a great deal of flexibility; |glspace| and
|glrightskip|. |glspace| is an incremental parameter, so you could say,
for example, |\lingset{glspace=!0pt plus .2em}|, increasing the
stretchability of the interglword space by \textdim{.2 em}. Or,
you might want to increase the stretchability of the rightskip
in a particular troublesome gloss to allow more whitespace at the
right edge.  An acceptable solution will depend on your
particular typesetting aesthetics. Solving line breaking problems
is often troublesome and requires experimentation.

\medskip
The Chamorro examples in (\getref{@period}) and (\getref{Chung1})
are adapted from Sandy Chung's {\it The Design of Agreement}. See
(59a) on page 237, (82a) on page 247, and (59b) on page 97.  The
Potawatami example (\getref{wapm}) is from Halle and Marantz's
``Distributed Morphology'' article. The German examples in
(\getref{sicher2}) are from an article by Idan Landau. The Kiowa
example (\getref{wiye}) was contributed by Daniel Harbour.

\endinput

\ex[everygla=,glhangstyle=normal]<@period>
\begingl
\gla Fa'nu'i yu' ni [ [ {\it O} t{\it in\/}aitai-mu {\it t\/} ] na
lepblu ] @ .//
\glb show me Obl Op {\it WH\/}[obj].read-agr {} L book//
\glft ``Show me the book that you read.''//
\endgl
\xe

\ex[everygla=,glhangstyle=normal]
\begingl
\gla Fa'nu'i yu' ni \nogloss{[[} @ {\it O} t{\it in\/}aitai-mu {\it
t\/} @ \nogloss{]} na lepblu @ \nogloss{].} //
\glb show me Obl Op {\it WH\/}[obj].read-agr {} L book//
\glft ``Show me the book that you read.''//
\endgl
\xe

\ex[everygla=,glhangstyle=normal]
\def\brackspace{\hskip.1em}%
\begingl
\gla Fa'nu'i yu' ni \nogloss{[[\brackspace} @ {\it O} t{\it in\/}aitai-mu {\it
t\/} @ \nogloss{\brackspace]} na
lepblu @ \nogloss{\brackspace]} @ .//
\glb show me Obl Op {\it WH\/}[obj].read-agr {} L book//
\glft ``Show me the book that you read.''//
\endgl
\xe

\ex[everygla=,glhangstyle=normal,glstyle=nlevel]
\def\brackspace{\hskip.1em}%
\begingl Fa'nu'i[show] yu'[me] ni[Obl] {[[}[]@ {\it O}[Op]
t{\it in\/}aitai-mu[{\it WH\/}{[obj]}.read-agr] {\it t\/}[]@ {]}[] na[L]
lepblu[book]@ {].}[]
\endgl
\xe

\ex[everygla=,glhangstyle=normal,glstyle=nlevel]
\begingl
Fa'nu'i[show] yu'[me] ni[Obl] \nogloss{[[}@ {\it O}[Op]
t{\it in\/}aitai-mu[{\it WH\/}{[obj]}.read-agr] \nogloss{{\it t\/}]} na[L]
{lepblu].}[book]
\endgl
\xe

\endinput


