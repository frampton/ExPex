
\ifnum\secno<1 \secno=7\fi

\section More on Glosses

\subsection User defined levels

\begininventory
\macros* |\defineglwlevels|\endmc
\endinventory
%
|\glb| and |\glc| are given definitions in {\it expex.tex\/}
by the command |\defineglwlevels{b,c}|.  The command also creates the
parameters |everyglb|, |everyglc|, |aboveglbskip|, and |aboveglcskip|.
|everyglb| and |everyglc| are initialized to empty token lists and
|aboveglbskip| and |aboveglcskip| to \textdim{0 pt}. The user may want
to use |\defineglwlevels| to create and name new gloss levels.

For example, suppose more suggestive level names are desired.

\codedisplay
\defineglwlevels{cat,gloss}
\lingset{everyglcat=\footnotesize,aboveglcatskip=-.5ex}

\ex
\begingl
\gla k- wapm -a -s'i -m -wapunin -uk //
\glcat Cl V Agr Neg Agr Tns Agr //
\glgloss 2 see 3{\sc acc} {} {2\sc pl} preterit {3\sc pl} //
\glft `you (pl) didn't see them'//
\endgl
\xe
|endcodedisplay
produces

\framedisplay
\defineglwlevels{cat,gloss}
\lingset{everyglcat=\footnotesize,aboveglcatskip=-.5ex}

\ex
\begingl
\gla k- wapm -a -s'i -m -wapunin -uk //
\glcat Cl V Agr Neg Agr Tns Agr //
\glgloss 2 see {3\sc acc} {} {2\sc pl} preterit {3\sc pl} //
\glft `you (pl) didn't see them'//
\endgl
\xe
\endframedisplay

Another case in which the user might want to define a new gloss level
or levels is if more than 3 lines of interlinear gloss are needed and
the desired flexibility cannot be obtained by repeated use of |\glb|
or |\glc|.


%%%%%%%%%%%%%%%%%%%%%%%%%%%%%%%%%%%%%%%%%%%%%%%%%
\subsection Positioning the free translation to the right of the
interlinear gloss

\begininventory
\parameters
\idx{|ftpos|}& choice (|below| or |right|)& |below|\cr
\idx{|sssep|}& dimension& |3em|\cr
\idx{|ssratio|}& decimal& |.6 |\cr
\idx{|ssrightskip|}& skip& |0pt plus 2em|\cr
\idx{|glhangstyle|}& |normal|, |none|, or |cascade|& |normal|\cr
\endinventory

\lingset{everygl=\openup.5ex,
   everyglword=\normalbaselines,everyglft=\normalbaselines,
   glhangindent=2em}

\framedisplay
\ex[glftpos=right,glhangstyle=none]
\let\\=\textsc
\begingl
\gla
Hom\^{a}o sa \v{c}\^{o} p\^{o} tha  \~{n}u nao ng\u{a} hmua. \~{N}u
dj\u{a} g\u{a}, \~{n}u dj\u{a} \v{c}\u{o}ng \~{n}u, laih gui r\^{e}o
\~{n}u. Todang bboi r\^{o}k jolan \~{n}u nao hma, \~{n}u bb\^{o}h sa
droi mr\u{a} d\u{o} bboi gah, a, hruh \~{n}u.//
\glb
\\{exist} one \\{clf} person old \\{3s} go do field \\{3s} hold
machete \\{3s} hold hoe \\{3s} and carry.on.back back.basket \\{3s}
while at along trail \\{3s} go field \\{3s} see one \\{clf} peacock
stay at \\{drct} -- nest \\{3s}//
\glft
`There was an old person who went to work in the field. He took
along his machete, he took along his hoe, and he carried his
basket on his back. While he was on his way to the farm, he saw a
peacock beside its nest.'//
\endgl
\xe
\endframedisplay

\noindent is achieved by
\codedisplay~
\ex[glftpos=right,glhangstyle=none]
\let\\=\textsc
\begingl
\gla
Hom\^{a}o sa \v{c}\^{o} p\^{o} tha  \~{n}u nao ng\u{a} hmua. \~{N}u
dj\u{a} g\u{a}, \~{n}u dj\u{a} \v{c}\u{o}ng \~{n}u, laih gui r\^{e}o
\~{n}u. Todang bboi r\^{o}k jolan \~{n}u nao hma, \~{n}u bb\^{o}h sa
droi mr\u{a} d\u{o} bboi gah, a, hruh \~{n}u.//
\glb
\\{exist} one \\{clf} person old \\{3s} go do field \\{3s} hold
machete \\{3s} hold hoe \\{3s} and carry.on.back back.basket \\{3s}
while at along trail \\{3s} go field \\{3s} see one \\{clf} peacock
stay at \\{drct} -- nest \\{3s}//
\glft
`There was an old person who went to work in the field. He took
along his machete, he took along his hoe, and he carried his
basket on his back. While he was on his way to the farm, he saw a
peacock beside its nest.'//
\endgl
\xe
|endcodedisplay
(This example, as well as (\getref{panelex}), was contributed by
Joshua Jensen.  It is from Jarai, an Austronesian language.  The
story teller was Hyech Ksor.  The orthography here is somewhat
simplified in order to keep the font requirements for the
examples in this documentation elementary.)

|ss| stands for ``side-by-side''. |sssep| gives the separation of the
gloss and the free translation. |ssratio| gives the proportion of the
available space (which is the leftskip minus the value of |sssep|)
which the gloss occupies. The point of hanging indentation is to
visually separate the free translation and the gloss, so
|glhangstyle=none| is completely satisfactory if the tree translation
is on the right. But \ExPex\ will happily use hanging (either
cascading or not) indentation with the free translation on the right.

Line breaking in the free translation is delicate because it will
generally set in a narrow column. The default setting of |ssrightskip|
allows up to \textdim{2 em} departure from right alignment. This usually avoids
overfull lines and awkward hyphenation.  |ssrightskip| can be
increased (all the way to \textdim{0 pt} plus \textdim{1 fil}) if there is a
problem, at the cost of a more ragged appearance.  This can be done
globally, or simply in troublesome examples.

\subsection Glosses with a side panel

\begininventory
\macros
\idx{|\beginglpanel|}, \idx{|\endpanel|}\endmc
\parameters*
\idx{|everypanel|}\user& token list& |{}|\cr
\endinventory
The mechanism for positioning the free translation to the right of the
interlinear gloss can be adapted to create a side panel for
glosses which can be used for other purposes, as illustrated
below.

\framedisplay
\ex[everypanel=\footnotesize]<panelex>
\let\\=\textsc
\beginglpanel[ssratio=.5,glhangstyle=none]
\gla Hom\^{a}o$^1$ sa \v{c}\^{o} p\^{o} tha  \~{n}u nao ng\u{a}
hmua. \~{N}u dj\u{a} g\u{a}, \~{n}u dj\u{a} \v{c}\u{o}ng \~{n}u,
laih gui r\^{e}o \~{n}u. Todang bboi r\^{o}k jolan \~{n}u nao
hma, \~{n}u bb\^{o}h sa droi mr\u{a} d\u{o}$\,^4$ bboi gah, a, hruh
\~{n}u.//
\glb \\{exist} one \\{clf} person old \\{3s} go$^2$ do field
\\{3s} hold machete \\{3s} hold hoe \\{3s} and$^3$ carry.on.back
back.basket \\{3s} while at along trail \\{3s} go field \\{3s}
see one \\{clf} peacock stay at \\{drct} -- nest \\{3s}
//
\endgl
1.\enspace {\it hom\^{a}o} also means `have', reflecting the
strong tendency across languages to use the same word for
possession and the existential. {\it hom\^{a}o} is clause-initial
in existential clauses, but it comes after the subject in
possession clauses.

2.\enspace All verbs are glossed with a bare form, as Jarai has
no inflectional morphology. Although Jarai has lexical items that
encode tense, they are relatively infrequent in text.

3.\enspace The word {\it laih} is literally `after; finish', but
that is clearly not the meaning here. Probably {\it laih} here is
an abbreviation for {\it laih an\u{u}n}, `after that; and', hence
the gloss `and'.

4.\enspace {\it d\u{o}} `sit, stay' is used like a copula in
locative clauses, which is what I assume here (`a~peacock
[which was] beside its nest'); however, this could just as well
mean `a peacock sitting beside its nest', retaining the posture
semantics.
\endpanel
\bigskip
`There was an old person who went to work in the field. He took
along his machete, he took along his hoe, and he carried his
basket on his back. While he was on his way to the farm, he saw a
peacock beside its nest.'
\xe
\endframedisplay

The syntax is:
\codedisplay
\beginglpanel |dots \endgl |dots \endpanel
|endcodedisplay
The first part is the gloss, with the usual syntax.  The second
part is put in a vbox and set alongside the gloss. The tokens
|lingeverypanel| are inserted when the vbox begins. All of the
parameters which are special to positioning the free translation
to the right of the gloss apply here as well, with the obvious
meanings.

The complete code for the example above is:

\codedisplay
\ex[everypanel=\footnotesize]<panelex>
\def\\#1{{\footnotesize\uppercase{#1}}}%
\let\\=\textsc
\beginglpanel[ssratio=.5,glhangstyle=none]
\gla Hom\^{a}o$^1$ sa \v{c}\^{o} p\^{o} tha  \~{n}u nao ng\u{a}
hmua. \~{N}u dj\u{a} g\u{a}, \~{n}u dj\u{a} \v{c}\u{o}ng \~{n}u,
laih gui r\^{e}o \~{n}u. Todang bboi r\^{o}k jolan \~{n}u nao
hma, \~{n}u bb\^{o}h sa droi mr\u{a} d\u{o}$\,^4$ bboi gah, a, hruh
\~{n}u.//
\glb \\{exist} one \\{clf} person old \\{3s} go$^2$ do field
\\{3s} hold machete \\{3s} hold hoe \\{3s} and$^3$ carry.on.back
back.basket \\{3s} while at along trail \\{3s} go field \\{3s}
see one \\{clf} peacock stay at \\{drct} -- nest \\{3s}//
\endgl
1.\enspace {\it hom\^{a}o} also means `have', reflecting the
strong tendency across languages to use the same word for
possession and the existential. {\it hom\^{a}o} is clause-initial
in existential clauses, but it comes after the subject in
possession clauses.

2.\enspace All verbs are glossed with a bare form, as Jarai has
no inflectional morphology. Although Jarai has lexical items that
encode tense, they are relatively infrequent in text.

3.\enspace The word {\it laih} is literally `after; finish', but
that is clearly not the meaning here. Probably {\it laih} here is
an abbreviation for {\it laih an\u{u}n}, `after that; and', hence
the gloss `and'.

4.\enspace {\it d\u{o}} `sit, stay' is used like a copula in
locative clauses, which is what I assume here (`a~peacock [which
was] beside its nest'); however, this could just as well mean `a
peacock sitting beside its nest', retaining the posture
semantics.
\endpanel
\bigskip
`There was an old person who went to work in the field. He took
along his machete, he took along his hoe, and he carried his
basket on his back. While he was on his way to the farm, he saw a
peacock beside its nest.'
\xe
|endcodedisplay
Note that the free translation here comes after |\endpanel| and is
typeset the way any material inside an |\ex| construction is
typeset.  This allows it to have full width, spanning both the
gloss and notes.  It could have been part of the gloss, with a
different result.

No support is given to side note numbering.  It must be done ``by
hand''.  If the construction turns out to be sufficiently useful
and hand numbering is sufficiently tedious, a more automatic
scheme might be possible.  It would not be trivial, because the
order in which the notes appear inside the gloss before it is
typeset is not necessarily the same as the order in which they
appear after it is typeset.

\subsection Cascading hanging indentation in glosses

\raggedbottom
\lingset{glhangindent=.25in,everygl=\openup.5ex,
   everyglword=\normalbaselines,everyglft=\normalbaselines,
   abovemoreglskip=1ex}

\lingset{glhangstyle=none}
\framedisplay
\ex[glhangstyle=cascade]
\let\\=\textsc
\begingl
\gla
Hom\^{a}o sa \v{c}\^{o} p\^{o} tha  \~{n}u nao ng\u{a} hmua. \~{N}u
dj\u{a} g\u{a}, \~{n}u dj\u{a} \v{c}\u{o}ng \~{n}u, laih gui r\^{e}o
\~{n}u. Todang bboi r\^{o}k jolan \~{n}u nao hma, \~{n}u bb\^{o}h sa
droi mr\u{a} d\u{o} bboi gah, a, hruh \~{n}u.//
\glb
\\{exist} one \\{clf} person old \\{3s} go do field \\{3s} hold
machete \\{3s} hold hoe \\{3s} and carry.on.back back.basket \\{3s}
while at along trail \\{3s} go field \\{3s} see one \\{clf} peacock
stay at \\{drct} -- nest \\{3s}//
\glft
`There was an old person who went to work in the field. He took
along his machete, he took along his hoe, and he carried his
basket on his back. While he was on his way to the farm, he saw a
peacock beside its nest.'//
\endgl
\xe
\endframedisplay

is achieved by
\codedisplay~
\ex[glhangstyle=cascade]
\let\\=\textsc
\begingl
\gla
Hom\^{a}o sa \v{c}\^{o} p\^{o} tha  \~{n}u nao ng\u{a} hmua. \~{N}u
dj\u{a} g\u{a}, \~{n}u dj\u{a} \v{c}\u{o}ng \~{n}u, laih gui r\^{e}o
\~{n}u. Todang bboi r\^{o}k jolan \~{n}u nao hma, \~{n}u bb\^{o}h sa
droi mr\u{a} d\u{o} bboi gah, a, hruh \~{n}u.//
\glb
\\{exist} one \\{clf} person old \\{3s} go do field \\{3s} hold
machete \\{3s} hold hoe \\{3s} and carry.on.back back.basket \\{3s}
while at along trail \\{3s} go field \\{3s} see one \\{clf} peacock
stay at \\{drct} -- nest \\{3s}//
\glft
`There was an old person who went to work in the field. He took
along his machete, he took along his hoe, and he carried his
basket on his back. While he was on his way to the farm, he saw a
peacock beside its nest.'//
\endgl
\xe
|endcodedisplay

\subsection  Gloss underfixes

\begininventory
\macros* |\gluf|\endmc
\parameters
\idx{|glufcloseup|}& dimension& |.4ex|\cr
\idx{|everygluf|}& token list& |{}|\cr
\endinventory

\noindent Sometimes, gloss displays like the following are
desired, with grammatical markings written below the gloss.

\framedisplay
\ex[glhangstyle=normal,glufcloseup=.4ex,everygluf=\footnotesize]
\begingl
\gla Mary$_i$ ist sicher, dass es den Hans nicht st\"oren
   w\"urde seiner Freundin ihr$_i$ Herz auszusch\"utten.//
\glb Mary is sure that it \gluf/the/ACC/ Hans not annoy would
   \gluf/his/DAT/ \gluf/girlfriend/DAT/ \gluf/her/ACC/
   \gluf/heart/ACC/ {out to throw}//
\glft `Mary is sure that to reveal her heart to his girlfriend
would not damage John.'//
\endgl
\xe
\endframedisplay

\ExPex\ provides the macro |\gluf| which can be used to
construct such a display.

\codedisplay
\ex[glhangstyle=normal,glufcloseup=.4ex,everygluf=\footnotesize]
\begingl
\gla Mary$_i$ ist sicher, dass es den Hans nicht st\"oren
   w\"urde seiner Freundin ihr$_i$ Herz auszusch\"utten.//
\glb Mary is sure that it \gluf/the/ACC/ Hans not annoy would
   \gluf/his/DAT/ \gluf/girlfriend/DAT/ \gluf/her/ACC/
   \gluf/heart/ACC/ {out to throw}//
\glft `Mary is sure that to reveal her heart to his girlfriend
would not damage John.'//
\endgl
\xe
|endcodedisplay

The grammatical markings are essentially ``underfixes'' (rather than
prefixes or suffixes), hence the name ``gluf'' (gl underfix). When the
underfix is typeset, the value of |everygluf| is first inserted.  It
is provided so that the user has control of the font used to typeset
the underfixes. The value of |glufcloseup| determines how much the
baselineskip between the underfix and the underfixed word is closed
up.  Without some closeup, the underfixes are not positioned close
enough to the glosses they modify (in my opinion). The macro |\gluf|
centers the underfix below the word it annotates.  Its syntax should
be clear from the example above.

\endinput
================================================
================================================

%\subsection An alternate gloss description syntax
%({\tt glstyle=3level})
%
%There are four different styles, choosen by the parameter
%|glstyle|, which can be set to |wrap|, |3level|, |multilevel|, or
%|oldstyle|.  The last two are included for backwards
%compatibility with the early versions of ExPex.  They are not
%intended for current or future use, nor will this manual explain
%their use.\footnote{Their use is explained in earlier versions of
%this manual.  I assume that if these gloss styles are in use,
%no further documentation is needed.}
%
%
%\noindent  The usual gloss description syntax uses a format which
%imitates the intended typeset output.  The gloss type |3level|
%provides an alternative, which some may find attractive.
%
%\codedisplay
%\ex[glstyle=3level]
%\begingl
%{k-/Cl/2} {wapm/V/see} {-a/Agr/3\sc ACC} {-s'i/Neg} {-m/Agr/\sc pl}
%{-wapunin/Tns/preterit} {-uk/Agr/\sc pl}.
%`you (pl) didn't see them'.
%\endgl
%\xe
%|endcodedisplay
%\framedisplay~
%\ex[glstyle=3level,aboveglaskip=0pt,aboveglbskip=0pt]
%\begingl
%{k-/Cl/2} {wapm/V/see} {-a/Agr/3\sc ACC} {-s'i/Neg}
%{-m/Agr/\sc pl} {-wapunin/Tns/preterit} {-uk/Agr/\sc PL}.
%`you (pl) didn't see them'.
%\endgl
%\xe
%\endframedisplay
%
%\smallskip
%\noindent The two lines which describe the gloss are terminated
%by a period.  The brackets are required in the first line, which
%is not processed as a space separated list.  The spaces between
%the bracketed items are optional. The three gloss lines in the
%output are called the gla, glb, and glc lines and parameters like
%|everyglb| or |aboveglcskip| have the effect expected.  The last
%line is the glft line and the parameters |everyglft| and
%|aboveglft| also have the expected effect.
%
%Gloss displays with two lines are possible.
%
%\framedisplay
%\ex
%\begingl[glstyle=3level,everygla=\it]
%{n-ku/1st-OK} {wapm-a/see}.
%`OK I'll see him'
%\endgl
%\xe
%\endframedisplay
%\codedisplay~
%\ex
%\begingl[glstyle=3level,everygla=\it]
%{n-ku/1st-OK} {wapm-a/see}.
%`OK I'll see him'
%\endgl
%\xe
%|endcodedisplay
%
%Some might prefer to construct perfectly ordinary glosses using
%this gloss type.
%
%\framedisplay
%\ex
%\begingl[glstyle=3level]
%{Il/there} {semble/seems} {au/to the} {g\'en\'eral/general}
%{\^etre/to be} {arriv\'e/arrived} {deux/two} {soldats/soldiers}
%{en/in} {ville/town.}.
%`There seems to the general to have arrived two soldiers in
%town{.}'.
%\endgl
%\xe
%\endframedisplay
%\codedisplay~
%\ex
%\begingl[glstyle=3level]
%{Il/there} {semble/seems} {au/to the} {g\'en\'eral/general}
%{\^etre/to be} {arriv\'e/arrived} {deux/two} {soldats/soldiers}
%{en/in} {ville/town.}.
%`There seems to the general to have arrived two soldiers in town{.}'.
%\endgl
%\xe
%|endcodedisplay
%\noindent You will be restricted to a single line, with no
%wrapping.
%
%\framedisplay
%\ex[glstyle=3level]
%\begingl
%{pwa-/Neg} {min/V/give} {-kwa/Agr/2pl{\sc NOM}.3pl\sc ACC}
%{-pun/Tns/preterit} .
%`you (pl) didn't give them (something)'
%\endgl
%\xe
%\endframedisplay
%
%\codedisplay~
%\ex[glstyle=3level]
%\begingl
%{pwa-/Neg} {min/V/give} {-kwa/Agr/2pl{\sc NOM}.3pl\sc ACC}
%{-pun/Tns/preterit} .
%`you (pl) didn't give them (something)'
%\endgl
%\xe
%|endcodedisplay
%
%Three level glosses are useful, particularly in
%morphology.
%
%\framedisplay
%\ex[glstyle=3level]
%a.\quad
%\begingl
%{pwa-/Neg} {min/V/give} {-kwa/Agr/2pl{\sc NOM}.3pl\sc ACC} {-pun/Tns/preterit} .
%`you (pl) didn't give them (something)'
%\endgl
%\hfil
%b.\quad
%\begingl[everygla=\it,everyglft=,aboveglftskip=-.1ex]
%{n-ku/1st-OK} {wapm-a/see}.
%`OK I'll see him'
%\endgl
%\xe
%\endframedisplay
%\codedisplay~
%\ex[glstyle=3level]
%a.\quad
%\begingl
%{pwa-/Neg} {min/V/give} {-kwa/Agr/2pl{\sc NOM}.3pl\sc ACC}
%{-pun/Tns/preterit} .
%`you (pl) didn't give them (something)'
%\endgl
%\hfil
%b.\quad
%\begingl[everygla=\it,everyglft=,aboveglftskip=-.1ex]
%{n-ku/1st-OK} {wapm-a/see}.
%`OK I'll see him'
%\endgl
%\xe|endcodedisplay
%
%\subsection Parameters which affect 3 level glosses
%
%\parinventory
%& \idx{|everygla|}& token list& |{}|\cr
%& \idx{|everyglb|}& token list& |{}|\cr
%& \idx{|everyglc|}& token list& |{}|\cr
%& \idx{|everyglft|}& token list& |{\it}|\cr
%& \idx{|aboveglgextraskip|}& skip& |0pt|\cr
%& \idx{|abovelinethreeextraskip|}& skip& |0pt|\cr
%& \idx{|aboveglftskip|}& skip& |.5ex|\cr
%\endparinventory
%\vskip-1.5ex
%\noindent  Additionally, the parameters |glspace| and |everygl|
%apply and have the same meaning as in the wrap style.


\makeatletter
\def\glw@printcolframed{%
   \psframebox[framesep=-.3pt]{\vtop{%
      \ling@everyglword
      \gl@loopmoretrue
      \loop\ifgl@loopmore
         \glw@lop\tempa\to\@tempa
         \glw@lop\aboveskiplist\to\@aboveskip
         \glw@lop\strutlist\to\@strut
         \expandafter\ifdim\@aboveskip=0pt \else
            \vskip\@aboveskip \fi
            \hbox{\psframebox[framesep=-.3pt]{\@strut\@tempa}}%
         \ifx\tempa\empty \gl@loopmorefalse \fi
      \repeat
      }}%
}
\def\framewords{\let\glw@printcol=\glw@printcolframed}
\resetatcatcode

\subsection Interline skip in the interlinear gloss

Macro: |everyglword|

\medskip

\noindent This section will not be of any use to most \ExPex\
users and should therefore be omitted unless curiosity or some special
need prompts you to read it.

The interlinear gloss is built by gathering items into boxes as shown
below, which repeats (\getref{wapm}) above.  {\sl Tex}'s usual
paragraph building machinary is used inside these boxes as they are
constructed.  {\sl Tex\/}'s paragraph building machinery is called
upon again to organize the sequence of boxes into the interlinear
gloss.  I will refer to these vboxes as ``gl words'' since these
vboxes act like words when these boxes are strung together to
construct the interlinear gloss.  The baseline of each gl word is the
baseline of the top item in the box. A horizontal space (determined by
the parameter |glspace|) is inserted between these boxes.

\ex[linewidth=.6pt]
\framewords
\begingl
\gla k- wapm -$\left\{{\hbox{a}\atop\hbox{e}}\right\}8$ -s'i -m -wapunin -uk //
\glb Cl V Agr Neg Agr Tns Agr //
\glb 2 see {3\sc ACC} {} {2\sc PL} preterit {3\sc PL} //
\glft `you (pl) didn't see them'//
\endgl
\xe






\ex
\begingl[everyglword=\offinterlineskip]
\gla k- wapm -a -s'i -m -wapunin -uk //
\glb Cl V Agr Neg Agr Tns Agr //
\glb 2 see {3\sc ACC} {} {2\sc PL} preterit {3\sc PL} //
\glft `you (pl) didn't see them'//
\endgl
\xe

\ex[abovemoreglskip=0pt]
\begingl
\gla Mary$_i$ ist sicher, dass es den Hans nicht st\"oren w\"urde
seiner Freundin ihr$_i$ Herz auszusch\"utten.//
\glb Mary is sure that it the-{\sc ACC} Hans not annoy would
his-{\sc DAT} girfriend-{\sc DAT} her-{\sc ACC} heart-{\sc ACC} {out to
throw}//
\glft  `Mary is sure that it would not annoy John to reveal her
heart to his girlfriend.'//
\endgl
\xe


\ex
\begingl
\gla k- wapm -a -s'i -m -wapunin -uk //
\glb[everyglb=\ninerm] Cl V Agr Neg Agr Tns Agr //
\glb 2 see {3\sc ACC} {} {2\sc PL} preterit {3\sc PL} //
\glft `you (pl) didn't see them'//
\endgl
\xe




\defineglwlevels{d,e,f}

\ex
\begingl[everygla=\ninerm,everyglb=\tenrm,
   everyglc=\elevenrm,everygld=\twelverm,everygle=\twelvebf,everyglf=\ninerm]
\gla Resist illegitimate authority.//
\glb Resist illegitimate authority.//
\glc Resist illegitimate authority.//
\gld Resist illegitimate authority.//
\gle Resist illegitimate authority.//
\gld Resist illegitimate authority.//
\glc Resist illegitimate authority.//
\glb Resist illegitimate authority.//
\glf Resist illegitimate authority.//
\endgl
\xe


\ex  |everyglword=\offinterlineskip|
\medskip

\leavevmode
\begingl[everygla=\ninerm,everyglb=\tenrm,
   everyglc=\elevenrm,everygld=\twelverm,everygle=\twelvebf,everyglf=\ninerm,
   everyglword=\offinterlineskip]
\gla Resist illegitimate authority.//
\glb Resist illegitimate authority.//
\glc Resist illegitimate authority.//
\gld Resist illegitimate authority.//
\gle Resist illegitimate authority.//
\gld Resist illegitimate authority.//
\glc Resist illegitimate authority.//
\glb Resist illegitimate authority.//
\glf Resist illegitimate authority.//
\endgl
\xe

\ex |everyglword={\baselineskip=0pt
\lineskip=.5ex}|
\smallskip

\leavevmode
\begingl[everygla=\ninerm,everyglb=\tenrm,
   everyglc=\elevenrm,everygld=\twelverm,everygle=\twelvebf,everyglf=\ninerm,
   everyglword={\baselineskip=0pt \lineskip=.5ex}]
\gla Resist illegitimate authority.//
\glb Resist illegitimate authority.//
\glc Resist illegitimate authority.//
\gld Resist illegitimate authority.//
\gle Resist illegitimate authority.//
\gld Resist illegitimate authority.//
\glc Resist illegitimate authority.//
\glb Resist illegitimate authority.//
\glf Resist illegitimate authority.//
\endgl
\xe

\ex manipulating |abovegl|x|skip|
\smallskip

\leavevmode
\begingl[everygla=\ninerm,everyglb=\tenrm,
   everyglc=\elevenrm,everygld=\twelverm,everygle=\twelvebf,everyglf=\ninerm,
   everyglword={\baselineskip=0pt \lineskip=.3ex}]
\gla Resist illegitimate authority.//
\glb Resist illegitimate authority.//
\glc Resist illegitimate authority.//
\gld Resist illegitimate authority.//
\gle[abovegleskip=2ex] Resist illegitimate authority.//
\gld[abovegldskip=2ex] Resist illegitimate authority.//
\glc Resist illegitimate authority.//
\glb Resist illegitimate authority.//
\glf Resist illegitimate authority.//
\endgl
\xe

\ex spacing using struts
\smallskip

\leavevmode
\begingl[everygla=\ninerm,everyglb=\tenrm,
   everyglc=\elevenrm,everygld=\twelverm,everygle=\twelvebf,everyglf=\ninerm,
   everyglword={\baselineskip=0pt \lineskip=.3ex}]
\gla Resist illegitimate authority.//
\glb Resist illegitimate authority.//
\glc Resist illegitimate authority.//
\gld Resist illegitimate authority.//
\gle {\vrule height2ex depth1ex width0pt Resist} illegitimate authority.//
\gld Resist illegitimate authority.//
\glc Resist illegitimate authority.//
\glb Resist illegitimate authority.//
\glf Resist illegitimate authority.//
\endgl
\xe


%%%%%%%%%%%%%%%%%%%%%%%%%%%%%%%%%%%%%%%%%%%%%%%%%
%% OMIT WHAT FOLLOWS (I THINK)
%% contains more examples of cascade style AND documentation
%% of multilevel style
%%%%%%%%%%%%%%%%%%%%%%%%%%%%%%%%%%%%%%%%%%%%%%%%%
\endinput

\lingset{glhangstyle=none}

\exdisplay\noexno
\begingl[glftpos=below]
[A]\quad \gla
Hmao sa co po tha  nu nao nga hmua. Nu dja ga, nu dja cog nu laih
gui reo ne. Todang bboi rok jolan nu nao hma, nu bboh sa droi mra
do bboi gah, a, hruh nu.//
\glb
exist one {clf} person old 3s go do field 3s hold machete 3s hold
hoe 3s pst carry.on.back back.basket 3s while at {along} trail 3s
go field 3s see one clf peacock stay at drct {?} nest 3s//
\glft
`There was an old person who went to work in the field. He took
along his machete, he took along his hoe, and he carried his
basket on his back. While he was on his way to the farm, he saw a
peacock beside its nest.'//
\endgl
\xe

\exdisplay\noexno
\begingl[glftpos=right]
[B]\quad \gla
Hmao sa co po tha  nu nao nga hmua. Nu dja ga, nu dja cog nu laih
gui reo ne. Todang bboi rok jolan nu nao hma, nu bboh sa droi mra
do bboi gah, a, hruh nu.//
\glb
exist one {clf} person old 3s go do field 3s hold machete 3s hold
hoe 3s pst carry.on.back back.basket 3s while at {along} trail 3s
go field 3s see one clf peacock stay at drct {?} nest 3s//
\glft
`There was an old person who went to work in the field. He took
along his machete, he took along his hoe, and he carried his
basket on his back. While he was on his way to the farm, he saw a
peacock beside its nest.'//
\endgl
\xe

\ex[glftpos=below]
\begingl
\gla
Hmao sa co po tha  nu nao nga hmua. Nu dja ga, nu dja cog nu laih
gui reo ne. Todang bboi rok jolan nu nao hma, nu bboh sa droi mra
do bboi gah, a, hruh nu.//
\glb
exist one {clf} person old 3s go do field 3s hold machete 3s hold
hoe 3s pst carry.on.back back.basket 3s while at {along} trail 3s
go field 3s see one clf peacock stay at drct {?} nest 3s//
\glft
`There was an old person who went to work in the field. He took
along his machete, he took along his hoe, and he carried his
basket on his back. While he was on his way to the farm, he saw a
peacock beside its nest.'//
\endgl
\xe


\ex[glftpos=right]
\begingl
\gla
Hmao sa co po tha  nu nao nga hmua. Nu dja ga, nu dja cog nu laih
gui reo ne. Todang bboi rok jolan nu nao hma, nu bboh sa droi mra
do bboi gah, a, hruh nu.//
\glb
exist one {clf} person old 3s go do field 3s hold machete 3s hold
hoe 3s pst carry.on.back back.basket 3s while at {along} trail 3s
go field 3s see one clf peacock stay at drct {?} nest 3s//
\glft
`There was an old person who went to work in the field. He took
along his machete, he took along his hoe, and he carried his
basket on his back. While he was on his way to the farm, he saw a
peacock beside its nest.'//
\endgl
\xe

\lingset{glhangstyle=normal}

\exdisplay\noexno
\begingl[glftpos=below]
[A]\quad \gla
Hmao sa co po tha  nu nao nga hmua. Nu dja ga, nu dja cog nu laih
gui reo ne. Todang bboi rok jolan nu nao hma, nu bboh sa droi mra
do bboi gah, a, hruh nu.//
\glb
exist one {clf} person old 3s go do field 3s hold machete 3s hold
hoe 3s pst carry.on.back back.basket 3s while at {along} trail 3s
go field 3s see one clf peacock stay at drct {?} nest 3s//
\glft
`There was an old person who went to work in the field. He took
along his machete, he took along his hoe, and he carried his
basket on his back. While he was on his way to the farm, he saw a
peacock beside its nest.'//
\endgl
\xe

\exdisplay\noexno
\begingl[glftpos=right]
[B]\quad \gla
Hmao sa co po tha  nu nao nga hmua. Nu dja ga, nu dja cog nu laih
gui reo ne. Todang bboi rok jolan nu nao hma, nu bboh sa droi mra
do bboi gah, a, hruh nu.//
\glb
exist one {clf} person old 3s go do field 3s hold machete 3s hold
hoe 3s pst carry.on.back back.basket 3s while at {along} trail 3s
go field 3s see one clf peacock stay at drct {?} nest 3s//
\glft
`There was an old person who went to work in the field. He took
along his machete, he took along his hoe, and he carried his
basket on his back. While he was on his way to the farm, he saw a
peacock beside its nest.'//
\endgl
\xe

\ex[glftpos=below]
\begingl
\gla
Hmao sa co po tha  nu nao nga hmua. Nu dja ga, nu dja cog nu laih
gui reo ne. Todang bboi rok jolan nu nao hma, nu bboh sa droi mra
do bboi gah, a, hruh nu.//
\glb
exist one {clf} person old 3s go do field 3s hold machete 3s hold
hoe 3s pst carry.on.back back.basket 3s while at {along} trail 3s
go field 3s see one clf peacock stay at drct {?} nest 3s//
\glft
`There was an old person who went to work in the field. He took
along his machete, he took along his hoe, and he carried his
basket on his back. While he was on his way to the farm, he saw a
peacock beside its nest.'//
\endgl
\xe


\ex[glftpos=right]
\begingl
\gla
Hmao sa co po tha  nu nao nga hmua. Nu dja ga, nu dja cog nu laih
gui reo ne. Todang bboi rok jolan nu nao hma, nu bboh sa droi mra
do bboi gah, a, hruh nu.//
\glb
exist one {clf} person old 3s go do field 3s hold machete 3s hold
hoe 3s pst carry.on.back back.basket 3s while at {along} trail 3s
go field 3s see one clf peacock stay at drct {?} nest 3s//
\glft
`There was an old person who went to work in the field. He took
along his machete, he took along his hoe, and he carried his
basket on his back. While he was on his way to the farm, he saw a
peacock beside its nest.'//
\endgl
\xe


\lingset{glhangstyle=cascade}

\exdisplay\noexno
\begingl[glftpos=below]
[A]\quad \gla
Hmao sa co po tha  nu nao nga hmua. Nu dja ga, nu dja cog nu laih
gui reo ne. Todang bboi rok jolan nu nao hma, nu bboh sa droi mra
do bboi gah, a, hruh nu.//
\glb
exist one {clf} person old 3s go do field 3s hold machete 3s hold
hoe 3s pst carry.on.back back.basket 3s while at {along} trail 3s
go field 3s see one clf peacock stay at drct {?} nest 3s//
\glft
`There was an old person who went to work in the field. He took
along his machete, he took along his hoe, and he carried his
basket on his back. While he was on his way to the farm, he saw a
peacock beside its nest.'//
\endgl
\xe

\exdisplay\noexno
\begingl[glftpos=right]
[B]\quad \gla
Hmao sa co po tha  nu nao nga hmua. Nu dja ga, nu dja cog nu laih
gui reo ne. Todang bboi rok jolan nu nao hma, nu bboh sa droi mra
do bboi gah, a, hruh nu.//
\glb
exist one {clf} person old 3s go do field 3s hold machete 3s hold
hoe 3s pst carry.on.back back.basket 3s while at {along} trail 3s
go field 3s see one clf peacock stay at drct {?} nest 3s//
\glft
`There was an old person who went to work in the field. He took
along his machete, he took along his hoe, and he carried his
basket on his back. While he was on his way to the farm, he saw a
peacock beside its nest.'//
\endgl
\xe

\ex[glftpos=below]
\begingl
\gla
Hmao sa co po tha  nu nao nga hmua. Nu dja ga, nu dja cog nu laih
gui reo ne. Todang bboi rok jolan nu nao hma, nu bboh sa droi mra
do bboi gah, a, hruh nu.//
\glb
exist one {clf} person old 3s go do field 3s hold machete 3s hold
hoe 3s pst carry.on.back back.basket 3s while at {along} trail 3s
go field 3s see one clf peacock stay at drct {?} nest 3s//
\glft
`There was an old person who went to work in the field. He took
along his machete, he took along his hoe, and he carried his
basket on his back. While he was on his way to the farm, he saw a
peacock beside its nest.'//
\endgl
\xe


\ex[glftpos=right]
\begingl
\gla
Hmao sa co po tha  nu nao nga hmua. Nu dja ga, nu dja cog nu laih
gui reo ne. Todang bboi rok jolan nu nao hma, nu bboh sa droi mra
do bboi gah, a, hruh nu.//
\glb
exist one {clf} person old 3s go do field 3s hold machete 3s hold
hoe 3s pst carry.on.back back.basket 3s while at {along} trail 3s
go field 3s see one clf peacock stay at drct {?} nest 3s//
\glft
`There was an old person who went to work in the field. He took
along his machete, he took along his hoe, and he carried his
basket on his back. While he was on his way to the farm, he saw a
peacock beside its nest.'//
\endgl
\xe








