


\def\fnexnoprint{(\romannumeral\the\excnt)}
\everyfootnote={%
   \tenpoint
   \noindent
   \excnt=1
   \keepexcntlocal
   \let\exnoprint=\fnexnoprint
}
\def\specialexnoprint{[\lingspecialexno]}
\def\raiseexcnt{\xdef\resetexcnt{\noindent\excnt=\the\excnt}%
   \aftergroup\resetexcnt}

\ex one\xe

\line{\hsize=.5\hsize
\vtop{\ex first\xe}\hss\vtop{\ex second\xe}}

\ex one\xe


\ex[exno=75] one\xe

\ex two\footnote{%
A footnote with examples:

\pex fn
\a part one
\a part two
\xe
And a second example.
\ex[belowexskip=0pt] fn\xe
}\xe

\ex three\footnote{%
Another footnote with examples:
\pex[exno={13, p. 49},labelanchor=numleft] fn
\a part one
\a part two
\xe
And a second example.
\ex fn\xe
}\xe

\endinput




\def\emwidth#1#2{\dimen0=#1 \dimen1=1em
   \dimen0=256\dimen0
   \divide\dimen1 by 256
   \divide\dimen0 by \dimen1
   \edef#2{\the\dimen0}\ignorespaces}
\emwidth{6pt}\foop
\def\foop{0.05pt}
\def\ggg#1pt{#1em}
{\expandafter\ggg\foop}
%\writeln{====\sloop====}

