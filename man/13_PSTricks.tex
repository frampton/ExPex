
\everymath={}%

\section ExPex and PSTricks

Several features of \ExPex\relax, listed in (\nextx), come into
play only if {\sl pstricks.tex\/} has been loaded at the point
that {\sl expex.tex\/} is loaded.  They are intended to make it
easier to use {\sl PSTricks\/} in examples.

\pex[labeltype=numeric]
\a \idx{|\Lingset|} is activated.  It works like |\lingset|, but if
there are parameters which are not in the family {\sl ling}, they are
passed to |\psset|.
\a The family {\sl ling\/} is added to the set of parameter families
which |\psset| can set.
\a The optional argument of |\ex|, |\pex|, and |\exdisplay| is
passed to |\Lingset|.
\xe

\noindent |\Lingset| first scans its argument from left to right
and sets all the parameters from the family {\sl ling}.  The
remaining keys are then passed to |\psset|.  If {\sl PSTricks\/}
has not been loaded, |\Lingset| is defined, but its meaning is
the meaning of |\lingset|.

This is illustrated by (\nextx).

\medskip
\framedisplay%[exclude]
\ex[angle=-90,nodesep=0pt,arrows=->,labelsep=.25ex,linearc=.7ex,
   dima=.2em]
\def\\#1(#2){\rnode{#2}{\strut #1}}%
%
\vrule height0pt depth5.3ex width0pt
\\Whom(A) did John persuade \\t(B) [ PRO to visit \\whom(C) ]
\ncbar[armA=3.5ex,offsetB=\lingdima]{B}{A}
\bput{0}{$M_{sp}=2$}
\ncbar[armA=4.5ex,offsetB=-\lingdima]{C}{A}
\bput{0}(1.2){$M_{sp}=6$}
\xe
\endframedisplay

\noindent (The example is from Juan Uriagereka, in {\sl The Role
of Economy Principles in Linguistic Theory}.)

In the code below, note that the optional argument of |\ex| is
used to set the \PSTricks\ parameters |angle|, |arrows|,
|nodesep|, |labelsep|, and |linearc|, and the \ExPex\ parameter
|dima|.  Note also that the \ExPex\ scratch dimension
|\lingdima| is used seemlessly by the {\sl PST-Node\/} macros.

\CLnum
\ex[angle=-90,nodesep=0pt,arrows=->,dima=.2em, labelsep=.25ex,
   linearc=.7ex]
\def\\#1(#2){\rnode{#2}{\strut #1}}%
%
\vrule height0pt depth5.3ex width0pt
\\Whom(A) did John persuade \\t(B) [ PRO to visit \\whom(C) ]
\ncbar[armA=3.5ex,offsetB=\lingdima]{B}{A}
\bput{0}{$M_{sp}=2$}
\ncbar[armA=4.5ex,offsetB=-\lingdima]{C}{A}
\bput{0}(1.2){$M_{sp}=6$}
\xe |endCLnum

\noindent Since the connections which \PSTricks\ draws are
dimensionless, a zero width |\vrule| is used to give correct
spacing.  (The width can first be made nonzero so it is visible,
and the depth adjusted.)

The following gives the same result.

\CLnum
\ex
\psset{angle=-90,nodesep=0pt,arrows=->,dima=.2em,labelsep=.25ex,
   linearc=.7ex}|endCLnum
\removelastskip
\hskip.7in $\vdots$
\vskip1em

|\Lingset| and |\psset| are not entirely equivalent, even when
\PSTricks\ is active.  If a key name in the family {\sl ling\/}
is the same as a key name in another family which has also been
added as part of the \PSTricks\ extended family, |\Lingset| will
treat it unambiguously as in the {\sl ling\/} family.   |\psset|,
on the other hand, does not give priority to keys in the {\sl
ling\/} family.  Exactly how |\psset| establishes priority is a
complex matter, depending on the history of the formation of the
\PSTricks\ extended family.


