
\ifnum\secno < 1 \secno=11 \excnt=34 \fi

\section Referring to examples and labeled parts of examples

\vskip-1ex
\subsection Unnamed reference\deftagsec{unnamedref}

\beginmacinventory
Macros:\quad \idx{|\lastx|}, \idx{|\nextx|}, \idx{|\blastx|},
\idx{|\anextx|}, \idx{|\bblastx|}
\endmacinventory

\noindent If $x$ is the number of the last example, these
produce $x$, $x+1$, $x-1$, $x+2$, and $x-2$, respectively.
The example number
counter is incremented early in the expansion of |\ex| and |\pex|,
so if one of these macros is inside an example, with ``last
example'' taking on the meaning of ``current example''.  If any
of these macros is used in setting parameters in the optional
argument of |\ex| or |\pex|, they act as if they were being
evaluated just before |\ex| or |\pex|.

It is potentially dangerous to use macros like |\bblastx| or
|\anextx| for reference to an example because later additions or
deletions in the document can throw off the reference.  This kind
of misreference is particularly easy to overlook in proofing a
document. It is better to assign names to the things you want to
refer to and to refer to them by name, particularly in a document
that will undergo a lot of rewriting.  If reference by name is
used and an intervening example is deleted or added, no problem
arises.  If the example which is referred to is deleted, then
\Tex\/ will report a missing reference.

\subsection Named reference

\deftagsec{named-reference}
\beginmacinventory
\idx{|\deftag|}, \idx{|\deftagex|}, \idx{|\deftaglabel|},
   \idx{|\deftagpage|}, \idx{|\getref|}, \idx{|\getfullref|}%
\endmacinventory

\noindent The core macros are \idx{|\deftag|} and
\idx{|\getref|}. |\deftag| takes two (obligatory) arguments.
After |\deftag{�reference�}{�tag�}| is executed, |\getref{�tag�}|
expands to {\sl reference}, where the value of the reference is
determined at the point the |\deftag| command was evaluated.
\idx{|\deftagex|}|{�tag�}| expands to
|\deftag{\the\exno}{�tag�}|. \idx{|\deftagpage|}|{�tag�}| expands
to |\deftag{\the\pageno}{�tag�}|, but the expansion is delayed
until after \Tex's page breaking mechanism has decided what page
the |\deftag| command appears on.  The tag/page pair which is
generated in this case is written to a file only, so |\getref|
cannot recover a page number until a tag-ref file is read.

The evaluation of \idx{|\deftaglabel|}|{�tag�}| is more complex.
|\pexcnt|, the setting of |labeltype|, and the setting of
|everylabel| are used to construct the reference (either a letter
or a number).  The tag which is submitted to |\deftag| is $\sl
tag'\!|.|tag$, where $\sl tag'$ is the example number tag.

The code below produces (\nextx).

\codedisplay
\pex[interpartskip=0pt]
\a First\deftag{the first part of example \lastx}{FP}
\a Second\deftagex{snoopy}\deftaglabel{dog}
\a Third\deftaglabel{a}
\xe|endcodedisplay

\framedisplay~
\pex[interpartskip=0pt]
\a First\deftag{the first part of example \lastx}{FP}
\a Second\deftagex{snoopy}\deftaglabel{dog}
\a Third\deftaglabel{a}
\xe
\endframedisplay

\noindent Afterwards, assuming that the tag |snoopy| has not been
redefined in the interim\par\nobreak\medskip

\hskip3em
\sidx{|\getfullref|}%
\vtop{\openup1pt
\halign{#\hfil& \quad$\to$\quad #\hfil\cr
|\getref{snoopy}|& \getref{snoopy}\cr
|\getref{snoopy.dog}|& \getref{snoopy.dog}\cr
|\getfullref{snoopy.dog}|& \getfullref{snoopy.dog}\cr
|\getref{snoopy.a}|& \getref{snoopy.a}\cr
|\getfullref{snoopy.a}|& \getfullref{snoopy.a}\cr
|\getref{dog}|& \it undefined tag warning, or spurious reference\cr
|\getref{FP}|& \getref{FP}\cr
}}

\medskip
\noindent |\getref{dog}| will produce a spurious reference if the
tag |dog| is defined somewhere else in your document.
|\deftaglabel| must know, at the point that it is expanded, what
tag has been attached to the example number. If the example
number has not been tagged at that point, a warning message is
issued.

There is a shortcut for tagging example numbers and labels,
which, unlike most shortcuts, makes the code easier to read as
well. After

\codedisplay
\pex[interpartskip=0pt,labeltype=alpha]<snoopy>
\a First
\a<dog> Second
\a<A> Third
\xe |endcodedisplay

\noindent |\getref{snoopy}|, |\getref{snoopy.A}|, and
|\getfullref{snoopy.A}| are all interpreted as~expected.

If |\getref{sometag}| is evaluated and the tag is undefined, a
warning message to this effect is generated
and~\hbox{$\bullet$|sometag|} is typeset instead of the intended
reference.

\subsection Proofing references

\beginmacinventory*
|\refproofing|
\endmacinventory
It is tedious to check that references to example numbers and the
like are correct (i.e. refer to what you think they refer to).
\Expex\ provides a little help. If you execute
\idx{|\refproofing|}, then all the references that are generated
by |\bblastx|, |\blastx|, |\lastx|, |\nextx|, |\anextx|,
|\getref|, and |\getfullref| are highlighted.  If \Pstricks\/ is
loaded at the point that |\refproofing| is evaluated, the
highlighting consists of a box around the reference.  If not, the
reference is under- and overlined. Highlighting is a significant
aid in copy editing.  The difference is illustrated below.

\pex
\a Consider (\getfullref{snoopy.dog}), for
example.\qquad default behavior
\refproofing
\a Consider (\getfullref{snoopy.dog}), for example.\qquad
|\refproofing|, \Pstricks\/ available
\a {\let\PSTricksLoaded\relax \refproofing
\makeatletter
\let\@highlightprint\mathhigh@lightref
\resetatcatcode
Consider (\getfullref{snoopy.dog}), for example.}\qquad
|\refproofing|, \Pstricks\/ not available
\xe



\subsection The tag/reference file

\beginmacinventory
|\gathertags|, |tagfilesuffix|
\endmacinventory

\noindent Forward references require writing tag/reference
associations to a file in one run, then reading this file in a
second run. Forward reference is only one reason for such a
system.  If you work on a document in pieces, say one section at
a time, and need to refer to things in prior sections, then the
tag/reference pairs from previous sections must be stored in a
file so that they can be used in the section that you are working
on. If you say \idx{|\gatherftags|}, the tag/reference pairs will
be written to a file as they are established.  If your main file
is named {\sl foop.tex}, for example, the default is to write the
tag/ref pairs to {\sl foop-tags.tex}.  But you can modify this by
using the command \idx{|\tagfilesuffix|}.  {\sl expex.tex\/}
contains |\tagfilesuffix{-tags}|, but you can overrule this with
|\tagfilesuffix{�your suffix�}|.

When the first |\getref|, |\getfullref|, or |\gathertags| command
is encountered, \Expex\/ checks to see if there is a tag file.  If
the file does exist, it is read and all the tag/reference pairs
it encodes are established.  No testing is done for conflicting
tag/reference pairs with the same tag, so it is the
responsibility of the user to see that this does not occur to a
bad effect.  If the user wants to be absolutely sure that bad
references have not accidentally occurred because of using the
same tag twice, he/she can look directly at the tag file in a
text editor.  If the lines are alphabetized, it is relatively
easy to find tags for which multiple references have been
established.  If you are a careful copy editor, this may be worth
doing in a complex project (a book length manuscript) during
final copy editing.  Of course, the printed final manuscript
should be checked in any event to make sure that the references
refer to what you want them to refer to.

For what it is worth, my own style of work is to break up
projects into multiple pieces and have a main file which calls
the various pieces.  Then I simply comment out the calls to
pieces that I am not working on.  From time to time I will run
the main file calling on all the various pieces, invoking
|\gathertags|. Then I comment out |\gathertags|.  The tag-ref
file which is created then remains intact until |\gathertags| is
invoked at some future time.  All of the tag/reference pairs it
encodes are available in subsequent work until a new tags file is
created.

\subsection References to references, references as values

\beginmacinventory*
\idx{|\lastlabel|}
\endmacinventory
%
If, for some reason, you want to refer to a specific
part of a multipart example without tagging the example number,
it can be done as follows, assuming that the part labels are
alphabetical.

\codedisplay
\pex[everylabel=\it]
\a First Example.
\a Second Example.\deftag{\lastx\lastlabel}{snoopy}
\a Third Example.
\xe
|endcodedisplay
\framedisplay
\pex[everylabel=\it]
\a First Example.
\a Second Example.\deftag{\lastx\lastlabel}{snoopy}
\a Third Example.
\xe
\endframedisplay

\noindent You can then use |\getref{snoopy}| to retrieve
\getref{snoopy}. |\lastlabel| expands to the most recent label,
in the form in which it is printed (i.e. containing the expansion
of |\everylabel|). If the part labels are numerical, then
|\deftag{\lastx.\lastlabel}| is required.

You can also say:

\sidx{|exno|}%
\codedisplay
\ex[exno=\getref{snoopy}] Second example.\xe|endcodedisplay

\framedisplay~
\ex[exno=\getref{snoopy}] Second example.\xe
\endframedisplay
\medskip

Or you can say:

\codedisplay
\ex[exno={\getref{snoopy}, repeated},exnoformat={[X]}]
Second example.\xe
|endcodedisplay

\framedisplay~
\ex[exno={\getref{snoopy}, repeated},exnoformat={[X]}]
Second example.\xe
\endframedisplay

\noindent Note that braces must hide the comma in the value which
sets the key.

If you say

\codedisplay
\pex[labeltype=numeric]<dog>
\a First Example.
\a<G> Second Example.
\a Third Example.
\xe|endcodedisplay

\framedisplay~
\pex[labeltype=numeric]<dog>
\a First Example.
\a<G> Second Example.
\a Third Example.
\xe\endframedisplay

\noindent then, if you want to repeat (\getfullref{dog.G}) at
some point, you can use:

\codedisplay
\ex[exno=\getfullref{dog.G}] Second example\xe
|endcodedisplay

\framedisplay~
\ex[exno={\getfullref{dog.G}}] Second example\xe
\endframedisplay

\subsection Extensions of the tag/reference mechanism

The tag/reference mechanism can easily be extended to reference
to chapters, sections, and subsections.  Suppose, for example,
that there are counters for the chapter number, section number,
subsection number, and subsubsection number.  One might define:

\codedisplay
\def\currsec{\the\chapterno
   \ifnum\secno>0 .\the\secno
   \ifnum\subsecno>0 .\the\subsecno
   \ifnum\subsubsecno>0 .\the\subsubsecno \fi\fi\fi}
\def\deftagsec#1{\deftag\currsec{#1}}|endcodedisplay

\noindent This assumes that there are counters |\chapterno|,
|\secno|, |\subsecno|, and |\subsecsecno| and that when a chapter
is initiated, |\secno| is set to 0, when a section is initiated,
|\secsecno| is set to 0, and that when a subsection is initiated,
|\subsubsecno| is set to 0.  (Lines 1--5, commented out,
are included at the end of {\sl expex.tex\/} for the user to
use, modify, or ignore.)


\subsection The parameter {\tt fullreformat}

\begininventory
\parameters
\idx{|labelformat|}& $\ldots\,|A|\,\ldots$ & |A.|\cr
\idx{|fullrefformat|}& $\ldots\,|X|\,\ldots\,|A|\,\ldots$ & |XA|\cr
\endinventory

\begingroup
\parindent=0pt \leftskip=1.2em

1. |labelformat| determines how the labels are formatted.
Generally they are followed by a period, but this parameter
allows other possibilities.
\smallskip

2. |fullrefformat| determines how references retrieved by
|\getfullref| are formatted.  Generally, example numbers and
labels are concatenated if the labels are characters and
separated by a period if the labels are numbers.  This parameter
allows other possibilities.  The format specification is
primitive.  The value of |fullrefformat| is analyzed as
|#1X#2A#3|, with |X| preceding |A|.  Then the full reference is
found by substituting the example number for |X| and the part
label for |A|.
\smallskip
\endgroup

Consider the artificial example below, with unusual formatting
for illustrative purposes.

\begingroup
\setss .65 .35
\beginss
\pex[labeltype=numeric,labelformat={[A]},
   fullrefformat=X-A,samplelabel={[1]}]<K>
\a first
\a<A> second
\xe|midss
\pex[labeltype=numeric,labelformat={[A]},
   fullrefformat=X-A,samplelabel={[1]}]<K>
\a first
\a<A> second
\xe
\endss
\endgroup
|\getref{K.A}| produces \getref{K.A} and |\getfullref{K.A}|
produces \getfullref{K.A}.

\subsection Reference to a part of a multipart example

Suppose there is a multipart example which you want to partially
repeat at a later point, giving only some of the parts. There are
many opportunities to make a mistake.  The reference to the
example number may be wrong, the references to the part labels my
be wrong, or a particular part may not be repeated the way it
originally appeared.  Here is one way to (partially) automate the
references to avoid most of the possible reference
errors.

\exbreak[15ex]
%\vskip 0pt plus 20ex \penalty0 \vskip 0pt plus -20ex

\begingroup
\setss .53 .43
\beginss
\deftag{Someone loves everyone.}{1st}
\deftag{Everyone loves someone.}{2nd}

\pex<A>
\a   \dots
\a<X> \getref{1st}
\a  \dots
\a<Z> \getref{2nd}
\xe

\ex An intervening example.\xe

The two crucial examples in
(\getref{A}) are repeated below:

\pex[exno=\getref{A}]
\a[label=\getref{A.X}]  \getref{1st}
\a[label=\getref{A.Z}]  \getref{2nd}
\xe

\ex Numbering resumes.\xe|midss
\deftag{Someone loves everyone.}{1st}
\deftag{Everyone loves someone.}{2nd}

\pex<A>
\a   \dots
\a<X> \getref{1st}
\a  \dots
\a<Z> \getref{2nd}
\xe

\bigskip
\ex An intervening example.\xe

\bigskip
The two crucial examples in
(\getref{A}) are repeated below:

\bigskip
\deftag{\the\secno.\the\subsecno}{explicit-sec}%
\pex[exno=\getref{A}]
\a[label=\getref{A.X}]  \getref{1st}
\a[label=\getref{A.Z}]  \getref{2nd}
\xe

\bigskip
\ex Numbering resumes.\xe
\endss
%
{\bf Important limitation}:\enspace Inside an example that sets
|exno| to a nonempty value, the use of |\deftaglabel| or
|\deftagex|, or their implicit versions using the |<|\dots|>|
notation, or |\label|, will result in an error or (more
dangerously) an unexpected outcome.  This is a minor limitation
because there should be no need to tag items which themselves are
referenced by tags.  It is important only because violating it
can generate Tex errors.\endgroup

\subsection Support for the LaTex \ttcs{label} and
\ttcs{ref} commands%

\deftag{\the\secno.\the\subsecno}{labelsec}

\sidx{|\label|, use of the LaTex macro}
\sidx{|\ref|, use of the LaTex macro}
\noindent LaTex users, if so inclined, may wish to use the LaTex
|\label|-|\ref| mechanism to tag example numbers and example part
labels.

Assuming that LaTex is in use,

\beginss
\pex \label{A}
\a Tom\label{B}
\a Dick\label{C}
\a Harry\label{D}, etc.
\xe

\ex~ Intervening example.\xe

\noindent Consider example (\ref{A}).
Tom is in example (\ref{B}), Dick is
in example (\ref{C}), and Harry is in
example (\ref{D}).|midss
%%%%
\pex[aboveexskip=1ex,nopreamble]\deftagex{A}
\a Tom\deftaglabel{B}
\a Dick\deftaglabel{C}
\a Harry\deftaglabel{D}, etc.
\xe

\bigskip
\ex~ Intervening example\xe

\bigskip
\noindent Consider example (\getref{A}). Tom is in example
(\getfullref{A.B}), Dick is in example (\getfullref{A.C}), and
Harry is in example (\getfullref{A.D}).
\medskip
%%%%
\endss

There are important differences between the LaTex mechanism and
the |\deftag|-|\getref| mechanism.

\smallskip
\noindent 1.\enspace
|\ref{|{\it tag}|}| cannot be used as a
special example number, nor can it be given as an optional
argument to |\a|.

\smallskip
\noindent 2.\enspace
The |\label| name space is flat.  No distinction is made between
names of examples and names of parts of examples. This makes it
much harder to remember and manage the names, at least for me.

\smallskip
\noindent 3.\enspace
LaTex provides no control over the generation of the file
containing the tag-reference pairs.  One cannot work on Chapter 5
(for example) and have all the tag-reference information for the
earlier chapters available without actually submitting all of the
Chapters 1 to 4 to LaTex while you are working on Chapter 5.


%%%%%%%%%%%%%%%%%%%%%%%%%%%%%%%%%%%%%%%%%%%%%%%%%%
\endinput



\endinput

\ex dog days\xe

\ex[everylabel=\it,
   exno={\getref{dog}\getref{dog.B}, repeated}]<dog2>
Second example.
\xe

\ex happy days\xe

\deftag{\getref{dog}.\getref{dog.A}}{coreexample}

\refproofing
\getfullref{dog.A}\qquad
\getref{snoopy}\qquad
\getref{coreexample}

\endinput
Although |\lastx| and |\getref{�tag�}| (with the reference of the
tag an example number) usually produce integers, their expansions
are actually complex.  The expansion of |\lastx| contains the
code for highlighting the result if |\refproofing| has been
executed. The expansion of |\getref{�tag�}| has this code and
also code to warn the user if the tag is undefined.  Because of
this, |\lastx| and |\getref| cannot be used in most contexts
that expect a number.



\Expex\/ makes special provisions to allow |\lastx| and |\getref|
to be used in certain contexts where it is convenient to be able
to use them, but would not be allowed without special provisions.
The following illustrates one such context. Suppose you want to
refer to (\nextx b) at some later point in your document.  You
can tag the needed reference using |\deftag{\lastx b}{mi-ti}|.
Then |\getref{mi-ti}| will recover the needed reference.
Recall that |\lastx| gives the current example number inside
|\ex| \dots|\xe|.  Inside the first argument of |\deftag|,
|\lastx| is redefined to strip off the highlighting code, which
would otherwise cause problems in this context.  \sidx{|\deftag|,
using |\lastx| in}

\framedisplay\ex[dima=1em]
a.\tspace
\judge*\begingl
\gla Te mostr\'o a m\'\i/nosotros /
\glb you showed-3sg to me/us /
\glc{`He showed you to me/us'}
\endgl
\hfil
b.\tspace
\deftag{\lastx b}{mi-ti}
\judge*\begingl
\gla Me mostr\'o a t\'\i/vosotros /
\glb me showed-3sg to you/you-pl /
\glc{`He showed me to you/you guys'}
\endgl
\xe\endframedisplay

\codedisplay~
\ex[dima=1em]
a.\tspace
\judge*\begingl
\gla Te mostr\'o a m\'\i/nosotros /
\glb you showed-3sg to me/us /
\glc{`He showed you to me/us'}
\endgl
\hfil
b.\tspace
\deftag{\lastx b}{mi-ti}           % \lastx inside \deftag
\judge*\begingl
\gla Me mostr\'o a t\'\i/vosotros /
\glb me showed-3sg to you/you-pl /
\glc{`He showed me to you/you guys'}
\endgl
\xe |endcodedisplay

Although you can use |\lastx| inside the first argument of
|\deftag|, don't try to use |\nextx|, |\blastx|, etc. or
|\getref| in this context. The result is not likely to be what
you want.

Another case in which it is useful to use |\getref{�tag�}| as if
it were simply a number is in repeating earlier examples.
Suppose that an example number has been given the tag |foo|,
either through using |\deftagex{foo}| or the |<|\dots|>| mechanism and
that later in your paper you want to repeat this example with its
original number.  It is sufficient to use

\sidx{|exno|, setting using |\getref|}%
\codefragment
|\ex[exno=\getref{foo}]| \dots
\endcodefragment
In this context, |\getref| is redefined to strip away testing for
tag definition and highlighting, so that it is usable as a value
associated with |exno|.  Since no testing is done to first
check if the tag is defined, an undefined tag will not produce a
user friendly warning message, simply a \Tex\/ crash of the
familiar kind.  If the tag |foo| is undefined, you will be
informed that |\lingtag@foo| is undefined.



\ex[exno=12] example\xe

\refproofing

\ex[exno=\getref{clitics}] example\xe

\getref{mi-ti}\quad \getref{clitics}
\endinput


\subsection A complex example

|\deftaglabel| works only for tables which use the
|\labels|-|\tl| system of assigning labels.  In many complex
tables which are ``handmade'', the labels are most conveniently
inserted directly.  In this case, |\deftag| must be used to
assign references to tags.  Consider the table below, from a paper by
Juan Uriagereka\footnote goop
"endfootnote
.  All the examples are ungrammatical, by the way.  Aside from
the use of |\deftag| and |\getref|, the example illustrates table
construction and glosses.

\framedisplay
\vbox{At various points in the paper, reference must be made to
(\nextx), (\nextx.1c), and (\nextx.2b).

\exdisplay[dimb=2em,dima=.6em,aboveglcskip=0pt,
   labeloffset=.8em,textoffset=.6em]
\deftagex{Juan-example}
\halign{#\hfil& \tspace[labeloffset]#\hfil& \tspace#\hfil&&
   \tspace[textoffset]#\hfil& \tspace[dimb]#\hfil\cr
\exnoprint& 1.& a.&
\begingl
\gla Me mostr\'o a m\'\i/nosotros /
\glb me showed-3sg to me/us /
\glc{`He showed me to me/us'}
\endgl
& b.&
\begingl
\gla Te mostr\'o a m\'\i/nosotros /
\glb you showed-3sg to me/us /
\glc{`He showed you to me/us'}
\endgl
\cr
\noalign{\medskip}
&& c.&
\begingl
\gla Nos mostr\'o a m\'\i/nosotros /
\glb us showed-3sg to me/us /
\glc{`He showed us to me/us'}
\endgl
\deftag{\lastx.1c}{mi/nos}
& d.&
\begingl
\gla Os mostr\'o a m\'\i/nosotros /
\glb you-pl showed-3sg to me/us /
\glc{`He showed you guys to me/us'}
\endgl
\cr
\noalign{\medskip}
&2.& a.&
\begingl
\gla Me mostr\'o a t\'\i/vosotros /
\glb me showed-3sg to you/you-pl /
\glc{`He showed me to you/you guys'}
\endgl
& b.&
\begingl
\gla Te mostr\'o a t\'\i/vosotros /
\glb you showed-3sg to you/you-pl /
\glc{`He showed you to you/you guys'}
\endgl
\deftag{\lastx.2b}{ti/vos}
\cr
\noalign{\medskip}
&& c.&
\begingl
\gla Nos mostr\'o a t\'\i/vosotros /
\glb us showed-3sg to you/you-pl /
\glc{`He showed us to you/you guys'}
\endgl
& d.&
\begingl
\gla Os mostr\'o a t\'\i/vosotros /
\glb you-pl showed-3sg to you/you-pl /
\glc{`He showed you guys to you/you guys'}
\endgl
\cr
}
\xe

\noindent You should be able to refer by name to the example
number (\getref{Juan-example}), and the labeled parts
(\getref{mi/nos}) and (\getref{ti/vos}).}\endframedisplay

\codedisplay~
At various points in the paper, reference must be made to
(\nextx), (\nextx.1c), and (\nextx.2b).

\exdisplay[dimb=2em,dima=.6em,aboveglcskip=0pt,
   labeloffset=.8em,textoffset=.6em]
% dima is the width of the gutter between the number labels and
%   the letter labels to the right
% dimb is the width of the gutter between the two columns
\deftagex{Juan-example}
\halign{#\hfil& \tspace[labeloffset]#\hfil& \tspace#\hfil&&
   \tspace[textoffset]#\hfil& \tspace[dimb]#\hfil\cr
\exnoprint& 1.& a.&
\begingl
\gla Me mostr\'o a m\'\i/nosotros /
\glb me showed-3sg to me/us /
\glc{`He showed me to me/us'}
\endgl
& b.&
\begingl
\gla Te mostr\'o a m\'\i/nosotros /
\glb you showed-3sg to me/us /
\glc{`He showed you to me/us'}
\endgl
\cr
\noalign{\medskip}|exbreak
&& c.&
\begingl
\gla Nos mostr\'o a m\'\i/nosotros /
\glb us showed-3sg to me/us /
\glc{`He showed us to me/us'}
\endgl
\deftag{\lastx.1c}{mi/nos}
& d.&
\begingl
\gla Os mostr\'o a m\'\i/nosotros /
\glb you-pl showed-3sg to me/us /
\glc{`He showed you guys to me/us'}
\endgl
\cr
\noalign{\medskip}|exbreak
&2.& a.&
\begingl
\gla Me mostr\'o a t\'\i/vosotros /
\glb me showed-3sg to you/you-pl /
\glc{`He showed me to you/you guys'}
\endgl
& b.&
\begingl
\gla Te mostr\'o a t\'\i/vosotros /
\glb you showed-3sg to you/you-pl /
\glc{`He showed you to you/you guys'}
\endgl
\deftag{\lastx.2b}{ti/vos}
\cr
\noalign{\medskip}|exbreak
&& c.&
\begingl
\gla Nos mostr\'o a t\'\i/vosotros /
\glb us showed-3sg to you/you-pl /
\glc{`He showed us to you/you guys'}
\endgl
& d.&
\begingl
\gla Os mostr\'o a t\'\i/vosotros /
\glb you-pl showed-3sg to you/you-pl /
\glc{`He showed you guys to you/you guys'}
\endgl
\cr
}
\xe

\noindent You should be able to refer by name to the example
number (\getref{Juan-example}), and the labeled parts
(\getref{mi/nos}) and (\getref{ti/vos}).|endcodedisplay










