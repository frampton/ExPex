
\section Judgment marks

\deftagsec{starsec}
\begininventory
\macros \idx{|\judge|}, \idx{|\ljudge|}\endmc
\parametersdef*
\hfil |*|\sidx{|*|, parameter}& sample judgment string& \hfil |*|& \hfil |*|\cr
\endinventory

\noindent
In examples without parts, not much needs to be said.

\codedisplay
\ex *Jack and Jill wented up the hill.\xe|endcodedisplay

\framedisplay~
\ex *Jack and Jill wented up the hill.\xe
\endframedisplay

In my view (\nextx), with a little whitespace inserted between
the asterisk and the example sentence, looks somewhat better than
(\lastx), but the difference is slight.

\codedisplay
\ex \judge* Jack and Jill is going up the hill.\xe|endcodedisplay

\framedisplay~
\ex \judge* Jack and Jill wented up the hill.\xe
\endframedisplay

\noindent \ExPex\ provides the macro |\judge| to
accomplish this. |\judge| takes one argument.  A multi-character
judgment diacritic therefore needs to be surrounded in braces.
|\judge| also ignores following spaces.  So
|\judge{??}Mary|\dots\ and \hbox{|\judge{??} Mary|\dots} produce
the same thing, as do |\judge*Mary|\dots\ and |\judge* Mary|\dots

Multipart examples are more complex, if alignment is to be
maintained. If you find (\nextx) satisfactory, what follows will
not be of much interest.  But it you would like the text (not
including the judgment marks) to be aligned, read on.

\framedisplay
\pex
\a There is a pair of pants on the floor.
\a \judge{?*}There are a pair of pants on the floor.
\a \judge*There is the pair of pants on the floor.
\xe
\endframedisplay

\ExPex\ provides the macro |\ljudge| which pushes the judgment
diacritics into the gap between the labels and the examples,
instead of pushing the examples to the right to make room for the
judgment diacritics.  So

\codedisplay
\pex
\a There is a pair of pants on the floor.
\a \ljudge{?*}There are a pair of pants on the floor.
\a \ljudge*There is the pair of pants on the floor.
\xe |endcodedisplay

\noindent produces

\framedisplay
\pex
\a There is a pair of pants on the floor.
\a \ljudge{?*}There are a pair of pants on the floor.
\a \ljudge*There is the pair of pants on the floor.
\xe
\endframedisplay

\noindent Unfortunately, depending on the setting of
|textoffset|, there is unlikely to be sufficient room for
judgment diacritics in between the labels and the examples.
|textoffset| needs to be increased to make room.

\ExPex\ provides the pseudo parameter to
facilitate adjusting the text offset.

\codedisplay
\pex[*=?*]
\a There is a pair of pants on the floor.
\a \ljudge{?*}There are a pair of pants on the floor.
\a \ljudge*There is the pair of pants on the floor.
\xe |endcodedisplay

\framedisplay
\pex[*=?*]
\a There is a pair of pants on the floor.
\a \ljudge{?*}There are a pair of pants on the floor.
\a \ljudge*There is the pair of pants on the floor.
\xe
\endframedisplay

\noindent |textoffset| is increased by the width of the judgment
diacritic which is furnished as the value of the parameter |*|.

If you say |\lingset{*}|, with no value assigned to |*|, it is
given a default value, which happens to be |*|.  So |\lingset{*}|
is equivalent to |\lingset{*=*}|.  So, for example:

\codedisplay
\pex[*]
\a There is a pair of pants on the floor.
\a \ljudge* There are a pair of pants on the floor.
\a \ljudge* There is the pair of pants on the floor.
\xe |endcodedisplay

\framedisplay~
\pex[*]
\a There is a pair of pants on the floor.
\a \ljudge* There are a pair of pants on the floor.
\a \ljudge* There is the pair of pants on the floor.
\xe
\endframedisplay

If you think that the text offset in (\blastx) is too large,
|textoffset| can be further adjusted directly, so you could write

\codedisplay
\pex[*=?*,textoffset=!-.3em]
\a There is a pair of pants on the floor.
\a \ljudge{?*} There are a pair of pants on the floor.
\a \ljudge* There is the pair of pants on the floor.
\xe |endcodedisplay

\framedisplay~
\pex[*=?*,textoffset=!-.3em]
\a There is a pair of pants on the floor.
\a \ljudge{?*} There are a pair of pants on the floor.
\a \ljudge* There is the pair of pants on the floor.
\xe
\endframedisplay


