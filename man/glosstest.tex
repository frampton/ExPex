
\def\small{\tenrm}
\lingset{glstyle=wrap,abovemoreglskip=1ex}
\def\suff#1{{-\small #1}}%
\everymath={}

\makeatletter
\def\glmw@printcolframed{%
   \psframebox[framesep=0]{\vtop{%
      \ling@everyglword
      \gl@loopmoretrue
      \loop\ifgl@loopmore
         \lop\tempa\to\@tempa
         \lop\aboveskiplist\to\@aboveskip
         \lop\strutlist\to\@strut
         \@aboveskip\hbox{\@strut\@tempa}%
         \ifx\tempa\empty \gl@loopmorefalse \fi
      \repeat
      }}%
}
\def\framewords{\let\glmw@printcol=\glmw@printcolframed}
\resetatcatcode

\ex<wapm>
\begingl
\gla k- wapm -a -s'i -m -wapunin -uk //
\glb Cl V Agr Neg Agr Tns Agr //
\glb 2 see {3\sc ACC} {} 2{\sc PL} preterit 3{\sc PL} //
\glft `you (pl) didn't see them'//
\endgl
\xe

\ex
\begingl
\gla Mary$_i$ ist sicher, dass es den Hans nicht st\"oren w\"urde
seiner Freundin ihr$_i$ Herz auszusch\"utten.//
\glb Mary is sure that it the-{\sc ACC} Hans not annoy would
his-{\sc DAT} girlfriend-{\sc DAT} her-{\sc ACC} heart-{\sc ACC} {out to
throw}//
\glft  `Mary is sure that it would not annoy John to reveal her
heart to his girlfriend.'//
\endgl
\xe

\font\ips=xipasl10 at 12pt
\font\ipss=xipasl10 at 7pt
\def\mroot{$\surd$}
\def\L{\char'354}
\def\v#1{{\accent"07 #1}}
\def\C{{\accent"07 c}}
\def\W{$^{\hbox{\ipss w}}\mskip-2mu$}


\ex
\begingl
\gla[everygla=\ips] hoi Ekn {\L}E {x\W}ElEP t{g\W}El' st{\'\i}m {hE\L}
   {kuPEcx\W ist} {\L a} Pa{\v c}sEtqEt//
\glb[everyglb=\ips] hoi {\mroot}PEkn {\L}E {x\W}ElEP t{g\W}El' {s +
   \mroot t\'\i m} {hE\L} ku-PEc-\mroot{x\W}ist {\L}a Pa{\v c}sEtqEt//
\glb then {\mroot}say det Meadowlark why {nomlz + \mroot what} conn
   2nom-cust-{\mroot}one.travels det {day time}//
\glb then {she said} det Meadowlark why {what is it} conn {you travel
   about} det {day time}//
\glft Then Meadowlark said, ``Why do you travel about in the day
   time?''//
\endgl
\xe

\ex
\framewords
\begingl
\gla k- wapm -a -s'i -m -wapunin -uk //
\glb Cl V Agr Neg Agr Tns Agr //
\glb 2 see {3\sc ACC} {} 2{\sc PL} preterit 3{\sc PL} //
\glft `you (pl) didn't see them'//
\endgl
\xe

\ex
\begingl
\gla[everygla=] k- wapm -a -s'i -m -wapunin -uk //
\glb[everyglb=\tenrm,aboveglbskip=-.5ex] Cl V Agr Neg Agr Tns Agr //
\glb 2 see {3\sc ACC} {} 2{\sc PL} preterit 3{\sc PL} //
\glft `you (pl) didn't see them'//
\endgl
\xe

\definemwlevels{cat}
\definelingstyle{Potawatomi}{everygla=,everyglcat=\tenrm,aboveglcatskip=-.5ex}

\ex[lingstyle=Potawatomi]
\begingl
\gla k- wapm -a -s'i -m -wapunin -uk //
\glcat Cl V Agr Neg Agr Tns Agr //
\glb 2 see {3\sc ACC} {} 2{\sc PL} preterit 3{\sc PL} //
\glft `you (pl) didn't see them'//
\endgl
\xe

\ex
\begingl
\gla Mary$_i$ ist sicher, dass es den Hans nicht st\"oren w\"urde
seiner Freundin ihr$_i$ Herz auszusch\"utten.//
\glb Mary is sure that it the\suff{ACC} Hans not annoy would
his\suff{DAT} girlfriend\suff{DAT} her\suff{ACC} heart\suff{ACC} {out to
throw}//
\glft  `Mary is sure that it would not annoy John to reveal her
heart to his girlfriend.'//
\endgl
\xe

\def\glexample{\the\excnt}
\ex
\begingl
\gla Mary$_i$ ist sicher, + dass es den Hans nicht st\"oren w\"urde
+ seiner Freundin ihr$_i$ Herz auszusch\"utten.//
\glb Mary is sure that it the\suff{ACC} Hans not annoy would
his\suff{DAT} girlfriend\suff{DAT} her\suff{ACC} heart\suff{ACC} {out to
throw}//
\glft  `Mary is sure that it would not annoy John to reveal her
heart to his girlfriend.'//
\endgl
\xe
\ex
\begingl
\gla wiye kepi e- @ ca//
\glb two whitemen \sc1P:3D- found//
\endgl
\xe

\exdisplay
\begingl
14.\quad
\gla Mary$_i$ ist sicher, dass es den Hans nicht st\"oren w\"urde
seiner Freundin ihr$_i$ Herz auszusch\"utten.//
\glb Mary is sure that it the\suff{ACC} Hans not annoy would
his\suff{DAT} girlfriend\suff{DAT} her\suff{ACC} heart\suff{ACC}
{out to throw}//
\glft  `Mary is sure that it would not annoy John to reveal her
heart to his girlfriend.'//
\endgl
\xe

\ex
a.\quad
\begingl[glwidth=2.6in]
\gla Mary$_i$ ist sicher, dass es den Hans nicht st\"oren w\"urde
seiner Freundin ihr$_i$ Herz auszusch\"utten.//
\glb Mary is sure that it the\suff{ACC} Hans not annoy would
his\suff{DAT} girfriend\suff{DAT} her\suff{ACC} heart\suff{ACC} {out to
throw}//
\glft  `Mary is sure that it would not annoy John to reveal her
heart to his girlfriend.'//
\endgl
\hfil
b.\quad
\begingl[glwidth=2.6in]
\gla Mary$_i$ ist sicher, dass seiner Freunden ihr$_i$ Herz
auszuch\"utten dem Hans nicht schaden w\"urde.//
\glb Mary is sure that his\suff{DAT} girlfriend\suff{DAT} her\suff{ACC}
heart\suff{ACC} {out to throw} the\suff{DAT} Hans not damage would//
\glft `Mary is sure that to reveal her heart to his girlfriend
would not damage John.'//
\endgl
\xe


\definelingstyle{defaultgl}{glstyle=wrap,glspace=.6em,
   everygl=\openup.5ex,everyglword=,everygla=,everyglb=,everyglc=\it,
   aboveglcskip=.5ex,glhangindent=1em}
\definelingstyle{gergl}{glstyle=wrap,glspace=1.2em,everygl=\openup.5ex,
   everyglword=\openup-.7ex,everygla=\bf,everyglc=\it,
   aboveglcskip=0pt,glhangindent=1.5em}
\definelingstyle{mingl}{glstyle=wrap,everygl=,everyglword=,
   everygla=,everyglb=,everyglc=,aboveglcskip=0pt,glhangindent=1em}

%\lingset{lingstyle=mingl}
\pex[interpartskip=3ex]
\a \begingl[lingstyle=mingl]
\gla Mary$_i$ ist sicher, dass es den Hans nicht st\"oren w\"urde
seiner Freundin ihr$_i$ Herz auszusch\"utten.//
\glb Mary is sure that it the\suff{ACC} Hans not annoy would
his\suff{DAT} girlfriend\suff{DAT} her\suff{ACC} heart\suff{ACC} {out to
throw}//
\glft  `Mary is sure that it would not annoy John to reveal her
heart to his girlfriend.'//
\endgl

\a \begingl[lingstyle=defaultgl]
\gla Mary$_i$ ist sicher, dass es den Hans nicht st\"oren w\"urde
seiner Freundin ihr$_i$ Herz auszusch\"utten.//
\glb Mary is sure that it the\suff{ACC} Hans not annoy would
his\suff{DAT} girlfriend\suff{DAT} her\suff{ACC} heart\suff{ACC} {out to
throw}//
\glft  `Mary is sure that it would not annoy John to reveal her
heart to his girlfriend.'//
\endgl

\a \begingl[lingstyle=gergl]
\gla Mary$_i$ ist sicher, dass es den Hans nicht st\"oren w\"urde
seiner Freundin ihr$_i$ Herz auszusch\"utten.//
\glb Mary is sure that it the\suff{ACC} Hans not annoy would
his\suff{DAT} girlfriend\suff{DAT} her\suff{ACC} heart\suff{ACC} {out to
throw}//
\glft  `Mary is sure that it would not annoy John to reveal her
heart to his girlfriend.'//
\endgl
\xe

\subsection Multiple gloss lines ({\tt glstyle=multilevel})

\ex[glstyle=multilevel,everygla=,everyglb=,everyglc=,
   aboveglaskip=0pt,aboveglbskip=0pt,aboveglftskip=1ex]
\begingl
\gla k- wapm -a -s'i -m -wapunin -uk //
\glb Cl V Agr Neg Agr Tns Agr //
\glb 2 see {3\sc ACC} {} 2{\sc PL} preterit 3{\sc PL} //
\glft `you (pl) didn't see them'//
\endgl
\xe

\codedisplay~
\ex[glstyle=multilevel,everygla=,everyglb=,everyglc=,
   aboveglaskip=0pt,aboveglbskip=0pt,aboveglftskip=1ex]
\begingl
\gla k- wapm -a -s'i -m -wapunin -uk //
\glb Cl V Agr Neg Agr Tns Agr //
\glb 2 see {3\sc ACC} {} 2{\sc PL} preterit 3{\sc PL} //
\glft `you (pl) didn't see them'//
\endgl
\xe
|endcodedisplay
The example is from Hockett, as glossed by Halle and
Marantz.

It is worth noting that, in practice, it is most likely that
a style would be defined as follows.
Then \def\vd{\hbox{$\qquad \vdots$}}
\codedisplay
\definelingstyle{glabb}{glstyle=multilevel,everygla=,everyglb=,
   everyglc=,aboveglaskip=0pt,aboveglbskip=0pt,aboveglftskip=1ex}

\ex[lingstyle=glabb]
\begingl
|vd
\endgl
\xe
|endcodedisplay

\noindent would be sufficient.

Repeating |\glb| does not allow different fonts or spacing for the two
lines.  In order to permit this, another level |\glc| is
predefined.  For example, suppose a style is defined by

\codedisplay
\definelingstyle{glabc}{glstyle=multilevel,everygl=,everygla=,
   everyglb=\elevenpoint,aboveglbskip=.3ex,
   everyglc=\tenpoint,aboveglcskip=-.3ex,
   everyglft=\it,aboveglftskip=1ex}
|endcodedisplay

\noindent |\elevenpoint| and |\tenpoint| are the ways 11 point
and 10 point type are invoked in my font setup.  This will vary
from user to user, so no attempt is made here to detail the
macros I rely on (which are variations on the ones in the
TexBook).
\codedisplay

\definelingstyle{glabc}{glstyle=multilevel,everygl=,everygla=,
   everyglb=\elevenpoint,aboveglbskip=.3ex,
   everyglc=\tenpoint,aboveglcskip=-.3ex,
   everyglft=\it,aboveglftskip=1ex}
\def\tenpoint{\tenrm \let\sc\eightrm}%
\def\elevenpoint{\elevenrm}
\definegllevels{word}
|endcodedisplay

Then

\codedisplay
\ex[lingstyle=glabc]
\begingl
\gla k- wapm -a -s'i -m -wapunin -uk //
\glb Cl V Agr Neg Agr Tns Agr //
\glc 2 see {3\sc ACC} {} 2pl preterit 3pl //
\glft {`you (pl) didn't see them'}//
\endgl
\xe
|endcodedisplay
produces


\definelingstyle{glabc}{glstyle=multilevel,everygl=,everygla=,
   everyglb=\elevenpoint,aboveglbskip=.3ex,
   everyglc=\tenpoint,aboveglcskip=-.3ex,
   everyglft=\it,aboveglftskip=1ex}
\framedisplay
\ex[lingstyle=glabc]
\begingl
\gla k- wapm -a -s'i -m -wapunin -uk //
\glb Cl V Agr Neg Agr Tns Agr //
\glc 2 see {3\sc ACC} {} 2pl preterit 3pl //
\glft {`you (pl) didn't see them'}//
\endgl
\xe
\endframedisplay

\noindent (The settings for the style |glabc| are chosen for
illustrative purposes, not because this is a particularly
attractive display.)

The user can define additional levels by
\medskip
|\definegllevels{mylevela,mylevelb}|
\medskip
Then the macros |\glmylevela| and |\glmylevelb| will be defined
and the keys |\everyglmylevela|, |everyglmylevelb|,
|aboveglmylevelaskip|, and |aboveglmylevelbskip| will be
activated.



\medskip
\vfil\break
\noindent{\it Inline citations}\medskip

\noindent Macro:\quad \idx{|\rightcite|}
\medskip


\noindent Multilevel glosses make it easy to give inline citations in
several different styles, as illustrated below.

\framedisplay
\pex[glstyle=multilevel,everygla=,everyglb=,everyglc=,
   aboveglaskip=0pt,aboveglbskip=0pt,aboveglftskip=1ex]
\a \begingl
\gla k- wapm -a -s'i -m -wapunin -uk //
\glb Cl V Agr$_1$ Neg Agr$_2$ Tns Agr$_3$//
\glb 2 see {3\sc ACC} {} 2{\sc PL} preterit 3{\sc PL} //
\glft `you (pl) didn't see them'//
\endgl
\hfill (Hockett 1948, p. 143)
\a \begingl
\gla k- wapm -a -s'i -m -wapunin -uk //
\glb Cl V Agr$_1$ Neg Agr$_2$ Tns Agr$_3$//
\glb 2 see {3\sc ACC} {} 2{\sc PL} preterit 3{\sc PL} //
\glft `you (pl) didn't see them'//
\endgl
\hfil (Hockett 1948, p. 143)
\a \begingl
\gla k- wapm -a -s'i -m -wapunin -uk //
\glb Cl V Agr$_1$ Neg Agr$_2$ Tns Agr$_3$//
\glb 2 see {3\sc ACC} {} 2{\sc PL} preterit 3{\sc PL} //
\glft \rightcite{(Hockett 1948, p. 143)}
   `you (pl) didn't see them'//
\endgl
\xe
\endframedisplay
\codedisplay
\pex[glstyle=multilevel,everygla=,everyglb=,everyglc=,
   aboveglaskip=0pt,aboveglbskip=0pt,aboveglftskip=1ex]
\a \begingl
\gla k- wapm -a -s'i -m -wapunin -uk //
\glb Cl V Agr$_1$ Neg Agr$_2$ Tns Agr$_3$//
\glb 2 see {3\sc ACC} {} 2{\sc PL} preterit 3{\sc PL} //
\glft `you (pl) didn't see them'//
\endgl \hfill (Hockett 1948, p. 143)
\a \begingl
\gla k- wapm -a -s'i -m -wapunin -uk //
\glb Cl V Agr$_1$ Neg Agr$_2$ Tns Agr$_3$//
\glb 2 see {3\sc ACC} {} 2{\sc PL} preterit 3{\sc PL} //
\glft `you (pl) didn't see them'//
\endgl \hfil (Hockett 1948, p. 143)
\a \begingl
\gla k- wapm -a -s'i -m -wapunin -uk //
\glb Cl V Agr$_1$ Neg Agr$_2$ Tns Agr$_3$//
\glb 2 see {3\sc ACC} {} 2{\sc PL} preterit 3{\sc PL} //
\glft \rightcite{(Hockett 1948, p. 143)}
   `you (pl) didn't see them'//
\endgl
\xe
|endcodedisplay


%%%%%%%%%%%%%%%%%%%%%%%%%%%%%%
\font\ips=xipasl10 at 12pt
\font\ipss=xipasl10 at 7pt
\def\mroot{$\surd$}
\def\L{\char'354}
\def\v#1{{\accent"07 #1}}
\def\C{{\accent"07 c}}
\def\W{$^{\hbox{\ipss w}}\mskip-2mu$}

\medskip
\vfil\break

\noindent{\it Line breaking (non-automatic) in the multilevel
style}\medskip

\noindent
Macros:\quad \idx{|\moregl|}

\parinventory
& \idx{|moregloffset|}& dimension& 1em\cr
& \idx{|abovemoreglskip|}& skip& .5ex plus .2ex\cr
\endparinventory



Assuming that |\ips| selects the tipa slant font and |\mroot|, |\L|,
and |\W| are appropriately defined, the code below produces the
display below it.

\codedisplay
\definelingstyle{ips-gloss}%
   {glstyle=multilevel,glspace=1em,everygla=\ips,moregloffset=1.5em,
   abovemoreglskip=1ex,aboveglftskip=1.5ex}

\ex[lingstyle=ips-gloss]
\begingl
\gla hoi Ekn {\L}E {x\W}ElEP t{g\W}El' st{\'\i}m//
\gla hoi {\mroot}PEkn {\L}E {x\W}ElEP t{g\W}El' {s + \mroot t\'\i m}//
\glb then {\mroot}say det Meadowlark why
   {nomlz + \mroot what}//
\glb then {she said} det Meadowlark why {what is it}//
\moregl
\gla {hE\L} {kuPEcx\W ist} {\L a} Pa{\v c}sEtqEt//
\gla {hE\L} ku-PEc-\mroot{x\W}ist {\L}a Pa{\v c}sEtqEt//
\glb conn 2nom-cust-{\mroot}one.travels det {day time}//
\glb conn {you travel about} det {day time}//
\glft Then Meadowlark said, ``Why do you travel about in the day
time?''//
\endgl
\xe
|endcodedisplay

\definelingstyle{ips-gloss}%
   {glstyle=multilevel,glspace=1em,everygla=\ips,moregloffset=1.5em,
   abovemoreglskip=1ex,aboveglftskip=1.5ex}

\framedisplay~
\ex[lingstyle=ips-gloss]
\begingl
\gla hoi Ekn {\L}E {x\W}ElEP t{g\W}El' st{\'\i}m//
\gla hoi {\mroot}PEkn {\L}E {x\W}ElEP t{g\W}El' {s + \mroot t\'\i m}//
\glb then {\mroot}say det Meadowlark why
   {nomlz + \mroot what}//
\glb then {she said} det Meadowlark why {what is it}//
\moregl
\gla {hE\L} {kuPEcx\W ist} {\L a} Pa{\v c}sEtqEt//
\gla {hE\L} ku-PEc-\mroot{x\W}ist {\L}a Pa{\v c}sEtqEt//
\glb conn 2nom-cust-{\mroot}one.travels det {day time}//
\glb conn {you travel about} det {day time}//
\glft Then Meadowlark said, ``Why do you travel about in the day
time?''//
\endgl
\xe
\endframedisplay

\noindent This example was contributed by Shannon Bischoff.

%
%
%
%James Crippen provided a gloss with five gloss lines (and the
%free translation): standard orthography, phonetic form (as
%spoken), phonemic form (standardized), underlying morphology, and
%morpheme gloss.
%
%\font\ipa=xipa10 at 12pt
%\font\ipascript=xipa10 at 7pt
%\def\umlautunder#1{\oalign{#1\crcr\hidewidth
%   \vbox to .2ex{\hbox{\char'004}\vss}\hidewidth}}
%\def\barunder#1{\oalign{#1\crcr\hidewidth
%   \vbox to .2ex{\hbox{\char'011}\vss}\hidewidth}}
%\def\barover#1{$\bar{\hbox{\ipa #1}}$}
%\def\nullmorf{{\twelvesy\char'073}}
%
%\definelingstyle{tlingit}{glstyle=multilevel,glspace=1em,everygla=\ipa,
%   everyglb=\tenrm,moregloffset=2em,abovemoreglskip=1ex}
%\def\tlingitstyle{%
%   \lingset{lingstyle=tlingit}%
%   \let\"=\umlautunder
%   \def\G{\char'345}%
%   \def\L{\char'354}%
%   \def\W{$^{\hbox{\ipascript w}}$}%
%   \def\H{$^{\hbox{\ipascript h}}$}%
%}
%
%\ex \tlingitstyle
%\def\\#1{{\rm #1}}%
%\begingl
%\gla	Ch'u.\'a\`an, {gun\'ei kawdudln{\'\i}k} y\'aa haa//
%\gla  {tS'uP\umlautunder{\'a}a\`n} {{\G}Un\'e: k{\H}awtUt{\L}n{\'\i}k}
%   j\'a: ha //
%\gla  tS'uP\'aa\`n {qun\'e; k{\H}awtUt{\L}n{\'\i}k}
%   j\'a; ha://
%\gla  ch'u.\'a\`an dgun\'ei=\nullmorf-ka-wu-du-dli-n\'ik
%   y\'aa haa//
%\glb  RESUM INCEP=\\3O-HSFC-PFV-\\3OBV.S-CL[+D,\\1,+I]-\\{tell} DEM.PROX
%   \\1PL.PSS//
%\moregl
%\gla shag\'oon y\'aa saanyaa {\barunder {\rm K}}w\'a\`an.//
%\gla S2k{\W}\'u\`un j\'a; sa:nja: q{\H\W}{\barover A}:n//
%\gla Sak\'u\`un j\'a; sa:nja: q{\H\W}\'a\`an//
%\gla shag\'o\`on y\'aa saa-niyaa {\barunder k}w\'a\`an//
%\glb \\{history} DEM.PROX \\{southern direction} ???? //
%\glft `Anyway, they would begin telling our history of this
%Saanyaa Kw\'aan.'//
%\endgl
%\xe
%
%%%%%%%%%%%%%%%%%%%%%%%%%%%%%%%%%%%%%%%%%%%%%%%%%
\subsection An alternate gloss description syntax
({\tt glstyle=3level})

\noindent  The usual gloss description syntax uses a format which
imitates the intended typeset output.  The gloss type |3level|
provides an alternative, which some may find attractive.

\codedisplay
\ex[glstyle=3level]
\begingl
{k-/Cl/2} {wapm/V/see} {-a/Agr/3\sc ACC} {-s'i/Neg} {-m/Agr/\sc PL}
{-wapunin/Tns/preterit} {-uk/Agr/\sc PL}.
`you (pl) didn't see them'.
\endgl
\xe
|endcodedisplay
\framedisplay~
\ex[glstyle=3level,aboveglaskip=0pt,aboveglbskip=0pt]
\begingl
{k-/Cl/2} {wapm/V/see} {-a/Agr/3\sc ACC} {-s'i/Neg}
{-m/Agr/\sc PL} {-wapunin/Tns/preterit} {-uk/Agr/\sc PL}.
`you (pl) didn't see them'.
\endgl
\xe
\endframedisplay

\smallskip
\noindent The two lines which describe the gloss are terminated
by a period.  The brackets are required in the first line, which
is not processed as a space separated list.  The spaces between
the bracketed items are optional. The three gloss lines in the
output are called the gla, glb, and glc lines and parameters like
|everyglb| or |aboveglcskip| have the effect expected.  The last
line is the glft line and the parameters |everyglft| and
|aboveglft| also have the expected effect.

Gloss displays with two lines are possible.

\framedisplay
\ex
\begingl[glstyle=3level,everygla=\it]
{n-ku/1st-OK} {wapm-a/see}.
`OK I'll see him'
\endgl
\xe
\endframedisplay
\codedisplay~
\ex
\begingl[glstyle=3level,everygla=\it]
{n-ku/1st-OK} {wapm-a/see}.
`OK I'll see him'
\endgl
\xe
|endcodedisplay

Some might prefer to construct perfectly ordinary glosses using
this gloss type.

\framedisplay
\ex
\begingl[glstyle=3level]
{Il/there} {semble/seems} {au/to the} {g\'en\'eral/general}
{\^etre/to be} {arriv\'e/arrived} {deux/two} {soldats/soldiers}
{en/in} {ville/town.}.
`There seems to the general to have arrived two soldiers in
town{.}'.
\endgl
\xe
\endframedisplay
\codedisplay~
\ex
\begingl[glstyle=3level]
{Il/there} {semble/seems} {au/to the} {g\'en\'eral/general}
{\^etre/to be} {arriv\'e/arrived} {deux/two} {soldats/soldiers}
{en/in} {ville/town.}.
`There seems to the general to have arrived two soldiers in town{.}'.
\endgl
\xe
|endcodedisplay
\noindent You will be restricted to a single line, with no
wrapping.

\framedisplay
\ex[glstyle=3level]
\begingl
{pwa-/Neg} {min/V/give} {-kwa/Agr/2pl{\sc NOM}.3pl\sc ACC}
{-pun/Tns/preterit} .
`you (pl) didn't give them (something)'
\endgl
\xe
\endframedisplay

\codedisplay~
\ex[glstyle=3level]
\begingl
{pwa-/Neg} {min/V/give} {-kwa/Agr/2pl{\sc NOM}.3pl\sc ACC}
{-pun/Tns/preterit} .
`you (pl) didn't give them (something)'
\endgl
\xe
|endcodedisplay

Three level glosses are useful, particularly in
morphology.

\framedisplay
\ex[glstyle=3level]
a.\quad
\begingl
{pwa-/Neg} {min/V/give} {-kwa/Agr/2pl{\sc NOM}.3pl\sc ACC} {-pun/Tns/preterit} .
`you (pl) didn't give them (something)'
\endgl
\hfil
b.\quad
\begingl[everygla=\it,everyglft=,aboveglftskip=-.1ex]
{n-ku/1st-OK} {wapm-a/see}.
`OK I'll see him'
\endgl
\xe
\endframedisplay
\codedisplay~
\ex[glstyle=3level]
a.\quad
\begingl
{pwa-/Neg} {min/V/give} {-kwa/Agr/2pl{\sc NOM}.3pl\sc ACC}
{-pun/Tns/preterit} .
`you (pl) didn't give them (something)'
\endgl
\hfil
b.\quad
\begingl[everygla=\it,everyglft=,aboveglftskip=-.1ex]
{n-ku/1st-OK} {wapm-a/see}.
`OK I'll see him'
\endgl
\xe|endcodedisplay

%\subsection Parameters which affect 3 level glosses
%
%\parinventory
%& \idx{|everygla|}& token list& |{}|\cr
%& \idx{|everyglb|}& token list& |{}|\cr
%& \idx{|everyglc|}& token list& |{}|\cr
%& \idx{|everyglft|}& token list& |{\it}|\cr
%& \idx{|aboveglgextraskip|}& skip& |0pt|\cr
%& \idx{|abovelinethreeextraskip|}& skip& |0pt|\cr
%& \idx{|aboveglftskip|}& skip& |.5ex|\cr
%\endparinventory
%\vskip-1.5ex
%\noindent  Additionally, the parameters |glspace| and |everygl|
%apply and have the same meaning as in the wrap style.


\ifnum\subsecno<2 \subsecno=4 \secno=8 \fi

\lingset{glstyle=wrap}

\subsection Bracketing in glosses

\framedisplay[doubleline=true]
\bigskip\noindent
\parinventory
& \idx{|glbrackbracksep|}& dimension& |.1em|\cr
& \idx{|glbrackwordsep|}& dimension& |.2em|\cr
\endparinventory
\endframedisplay
\medskip

\noindent Sometimes it is desirable to displays brackets which
delimit grammatical constituents.  The display below is easy to
create, but is not as attractive as it might be.

\smallskip
\framedisplay
\ex[glstyle=wrap]
\begingl
\gla Mary$_i$ ist sicher, $[$dass es den Hans nicht st\"oren
w\"urde $[$seiner Freundin ihr$_i$ Herz auszusch\"utten$]]$//
\glb Mary is sure that it the Hans not annoy would
his girlfriend her heart {out to throw}//
\glft `Mary is sure that to reveal her heart to his girlfriend
would not damage John.'//
\endgl
\xe
\endframedisplay
\smallskip

\codedisplay
\ex[glstyle=wrap]
\begingl
\gla Mary$_i$ ist sicher, $[$dass es den Hans nicht st\"oren
w\"urde $[$seiner Freundin ihr$_i$ Herz auszusch\"utten$]]$//
\glb Mary is sure that it the Hans not annoy would
his girlfriend her heart {out to throw}//
\glft `Mary is sure that to reveal her heart to his girlfriend
would not damage John.'//
\endgl
\xe
|endcodedisplay

\framedisplay \makeatletter
\ex[glstyle=wrap,glbrackwordsep=.1em,glbrackbracksep=.04em]
\begingl
\gla Mary$_i$ ist sicher, $[\,$ @ dass es den Hans nicht st\"oren
w\"urde $[\,$ @ seiner Freundin ihr$_i$ Herz auszusch\"utten
@ $\,]].$//
\glb Mary is sure {} that it the Hans not annoy would
{} his girlfriend her heart {out to throw}//
\glft `Mary is sure that to reveal her heart to his girlfriend
would not damage John.'//
\endgl
\xe
\endframedisplay

\smallskip
\framedisplay \makeatletter
\ex[glstyle=wrap,glbrackwordsep=.1em,glbrackbracksep=.0em]
\begingl
\gla Mary$_i$ ist sicher, [ dass es den Hans nicht st\"oren
w\"urde [ seiner Freundin ihr$_i$ Herz auszusch\"utten ] @ $].$//
\glb Mary is sure that it the Hans not annoy would
his girlfriend her heart {out to throw}//
\glft `Mary is sure that to reveal her heart to his girlfriend
would not damage John.'//
\endgl
\xe
\endframedisplay
\smallskip

\framedisplay
\ex[glstyle=wrap]
\begingl
\gla Mary$_i$ ist sicher, $[\,$ @ dass es den Hans nicht st\"oren
w\"urde $[\,$ @ seiner Freundin ihr$_i$ Herz auszusch\"utten$\,]]\,.$//
\glb Mary is sure {} that it the Hans not annoy would
{} his girlfriend her heart {out to throw}//
\glft `Mary is sure that to reveal her heart to his girlfriend
would not damage John.'//
\endgl
\xe
\endframedisplay
\smallskip

\noindent Note that the gloss of words following brackets is
aligned with the word, not the bracket.  Note also that some
extra space is inserted between the brackets and the words, and
between adjacent brackets.

One relatively easy way to produce a display like this is to use
the |@| mark to close up the space where needed and math mode to
insert the brackets.  Placeholders (|{}|) must be inserted in the
|\glb| line because the left delimiters are treated as items in
the |\gla| line.  The code used for the above was.

\codedisplay
\ex[glstyle=wrap]
\begingl
\gla Mary$_i$ ist sicher, $[\,$ @ dass es den Hans nicht st\"oren
w\"urde $[\,$ @ seiner Freundin ihr$_i$ Herz
{auszusch\"utten$\,]]\,.$} //
\glb Mary is sure {} that it the Hans not annoy would {} his
girlfriend her heart {out to throw} //
\glft `Mary is sure that to reveal her heart to his girlfriend
would not damage John.'//
\endgl
\xe
|endcodedisplay

\noindent Recall that |\,| inserts a math `thin space' in math
mode. It is not necessary to enclose the last item on the |\gla|
line in brackets, but doing so makes the logic of the code
somewhat clearer.

{\it ExPex\/} provides a way to automate this process without the
need to insert placeholders on the |\glb| line. The code below
produces the display which follows.

\codedisplay
\pex[glstyle=wrap,everygla=,nopreamble]
\a \begingl       % 59a, p. 237
\gla Fa'nu'i yu' ni [ [ {\it O} t{\it in\/}aitai-mu {\it t\/} ] na
lepblu ] @ .//
\glb show me Obl Op {\it WH\/}[obj].read-agr {} L book//
\glft ``Show me the book that you read.''//
\endgl
\a \begingl       % 82a, p. 247
\gla Um-\"asudda' h\"am yan [ i taotao [ {\it O\/} ni si Juan
ilek-\~na nu guahu + [ mal\"agu' gui [ asudd\"a'-\~na {\it
t\/} ] ] ] ] @ .//
\glb agr-meet we with the person Op Comp the Juan say-agr Obl me
agr.want he {\it WH\/}[obl].meet-agr//
\glft ``I met the person who Juan told me he wanted to meet.''//
\endgl
\xe
|endcodedisplay
\framedisplay
\pex[glstyle=wrap,everygla=,nopreamble]
\a \begingl       % 59a, p. 237
\gla Fa'nu'i yu' ni [ [ {\it O} t{\it in\/}aitai-mu {\it t\/} ] na
lepblu ] @ .//
\glb show me Obl Op {\it WH\/}[obj].read-agr {} L book//
\glft ``Show me the book that you read.''//
\endgl
\a \begingl       % 82a, p. 247
\gla Um-\"asudda' h\"am yan [ i taotao [ {\it O\/} ni si Juan
ilek-\~na nu guahu + [ mal\"agu' gui [ asudd\"a'-\~na {\it
t\/} ] ] ] ] @ .//
\glb agr-meet we with the person Op Comp the Juan say-agr Obl me
agr.want he {\it WH\/}[obl].meet-agr//
\glft ``I met the person who Juan told me he wanted to meet.''//
\endgl
\xe
\endframedisplay

\noindent The examples are adapted from {\it The Design of
Agreement\/} by Sandra Chung.  See (59a) on page 237 and (82a) on
page 247.


%%% PROBLEM
%%% PROBLEM
%\ex[glstyle=wrap,everygluf=\tenrm]
%\gla Mary$_i$ ist sicher, dass es den Hans nicht st\"oren
%w\"urde + seiner Freundin ihr$_i$ Herz auszusch\"utten.//
%\glb Mary is sure that it \gluf{the}{ACC} Hans not annoy would
% \gluf{his}{DAT} \gluf{girlfriend}{DAT} \gluf{her}{ACC}
% \gluf{heart}{ACC}
%{out to throw}//
%\glft `Mary is sure that to reveal her heart to his girlfriend
%would not damage John.'\endgraf//   %% PROBLEM
%\xe
%%% PROBLEM

\subsection Cascading hanging indentation in glosses

%\hsize=4in

\raggedbottom
\lingset{glstyle=wrap,glhangindent=.25in,everygl=\openup.5ex,
   everyglword=\normalbaselines,everyglft=\normalbaselines}

\lingset{glhangstyle=none}
\framedisplay
\ex[glhangstyle=cascade]
\begingl
\gla
Hmao sa co po tha  nu nao nga hmua. Nu dja ga, nu dja cog nu laih
gui reo ne. Todang bboi rok jolan nu nao hma, nu bboh sa droi mra
do bboi gah, a, hruh nu.//
\glb
exist one {clf} person old 3s go do field 3s hold machete 3s hold
hoe 3s pst carry.on.back back.basket 3s while at {along} trail 3s
go field 3s see one clf peacock stay at drct {?} nest 3s//
\glft
`There was an old person who went to work in the field. He took
along his machete, he took along his hoe, and he carried his
basket on his back. While he was on his way to the farm, he saw a
peacock beside its nest.'//
\endgl
\xe
\endframedisplay
is achieved by

\codedisplay
\ex[glhangstyle=cascade]
\begingl
\gla
Hmao sa co po tha  nu nao nga hmua. Nu dja ga, nu dja cog nu laih
gui reo ne. Todang bboi rok jolan nu nao hma, nu bboh sa droi mra
do bboi gah, a, hruh nu.//
\glb
exist one {clf} person old 3s go do field 3s hold machete 3s hold
hoe 3s pst carry.on.back back.basket 3s while at {along} trail 3s
go field 3s see one clf peacock stay at drct {?} nest 3s//
\glft
`There was an old person who went to work in the field. He took
along his machete, he took along his hoe, and he carried his
basket on his back. While he was on his way to the farm, he saw a
peacock beside its nest.'//
\endgl
\xe
|endcodedisplay

If |cascade| in the above is changed to |normal| you get (\nextx)
and if it is changed to |none| you get (\anextx).

\framedisplay
\ex[glhangstyle=normal]
\begingl
\gla
Hmao sa co po tha  nu nao nga hmua. Nu dja ga, nu dja cog nu laih
gui reo ne. Todang bboi rok jolan nu nao hma, nu bboh sa droi mra
do bboi gah, a, hruh nu.//
\glb
exist one {clf} person old 3s go do field 3s hold machete 3s hold
hoe 3s pst carry.on.back back.basket 3s while at {along} trail 3s
go field 3s see one clf peacock stay at drct {?} nest 3s//
\glft
`There was an old person who went to work in the field. He took
along his machete, he took along his hoe, and he carried his
basket on his back. While he was on his way to the farm, he saw a
peacock beside its nest.'//
\endgl
\xe
\endframedisplay

\framedisplay
\ex[glhangstyle=none]
\begingl
\gla
Hmao sa co po tha  nu nao nga hmua. Nu dja ga, nu dja cog nu laih
gui reo ne. Todang bboi rok jolan nu nao hma, nu bboh sa droi mra
do bboi gah, a, hruh nu.//
\glb
exist one {clf} person old 3s go do field 3s hold machete 3s hold
hoe 3s pst carry.on.back back.basket 3s while at {along} trail 3s
go field 3s see one clf peacock stay at drct {?} nest 3s//
\glft
`There was an old person who went to work in the field. He took
along his machete, he took along his hoe, and he carried his
basket on his back. While he was on his way to the farm, he saw a
peacock beside its nest.'//
\endgl
\xe
\endframedisplay


\section Cascading hanging indentation in glosses

%\hsize=4in

\raggedbottom
\lingset{glstyle=wrap,glhangindent=.25in,everygl=\openup.5ex,
   everyglword=\normalbaselines,everyglft=\normalbaselines}

\lingset{glhangstyle=none}
\framedisplay
\ex[glhangstyle=cascade]
\begingl
\gla
Hmao sa co po tha  nu nao nga hmua. Nu dja ga, nu dja cog nu laih
gui reo ne. Todang bboi rok jolan nu nao hma, nu bboh sa droi mra
do bboi gah, a, hruh nu.//
\glb
exist one {clf} person old 3s go do field 3s hold machete 3s hold
hoe 3s pst carry.on.back back.basket 3s while at {along} trail 3s
go field 3s see one clf peacock stay at drct {?} nest 3s//
\glft
`There was an old person who went to work in the field. He took
along his machete, he took along his hoe, and he carried his
basket on his back. While he was on his way to the farm, he saw a
peacock beside its nest.'//
\endgl
\xe
\endframedisplay
is achieved by

\codedisplay
\ex[glhangstyle=cascade]
\begingl
\gla
Hmao sa co po tha  nu nao nga hmua. Nu dja ga, nu dja cog nu laih
gui reo ne. Todang bboi rok jolan nu nao hma, nu bboh sa droi mra
do bboi gah, a, hruh nu.//
\glb
exist one {clf} person old 3s go do field 3s hold machete 3s hold
hoe 3s pst carry.on.back back.basket 3s while at {along} trail 3s
go field 3s see one clf peacock stay at drct {?} nest 3s//
\glft
`There was an old person who went to work in the field. He took
along his machete, he took along his hoe, and he carried his
basket on his back. While he was on his way to the farm, he saw a
peacock beside its nest.'//
\endgl
\xe
|endcodedisplay

If |cascade| in the above is changed to |normal| you get (\nextx)
and if it is changed to |none| you get (\anextx).

\framedisplay
\ex[glhangstyle=normal]
\begingl
\gla
Hmao sa co po tha  nu nao nga hmua. Nu dja ga, nu dja cog nu laih
gui reo ne. Todang bboi rok jolan nu nao hma, nu bboh sa droi mra
do bboi gah, a, hruh nu.//
\glb
exist one {clf} person old 3s go do field 3s hold machete 3s hold
hoe 3s pst carry.on.back back.basket 3s while at {along} trail 3s
go field 3s see one clf peacock stay at drct {?} nest 3s//
\glft
`There was an old person who went to work in the field. He took
along his machete, he took along his hoe, and he carried his
basket on his back. While he was on his way to the farm, he saw a
peacock beside its nest.'//
\endgl
\xe
\endframedisplay

\framedisplay
\ex[glhangstyle=none]
\begingl
\gla
Hmao sa co po tha  nu nao nga hmua. Nu dja ga, nu dja cog nu laih
gui reo ne. Todang bboi rok jolan nu nao hma, nu bboh sa droi mra
do bboi gah, a, hruh nu.//
\glb
exist one {clf} person old 3s go do field 3s hold machete 3s hold
hoe 3s pst carry.on.back back.basket 3s while at {along} trail 3s
go field 3s see one clf peacock stay at drct {?} nest 3s//
\glft
`There was an old person who went to work in the field. He took
along his machete, he took along his hoe, and he carried his
basket on his back. While he was on his way to the farm, he saw a
peacock beside its nest.'//
\endgl
\xe
\endframedisplay


\definemwlevels{c,d,e,f}
\lingset{glstyle=wrap,abovegldskip=0pt}

%\ex[everyglword=\offinterlineskip] \let\strut\relax
%\begingl
%\gla x kp . uv//
%\gld u v//
%\endgl
%\xe
%
%\endinput
%
%\ex
%\begingl[glstyle=wrap,everygla=\ninerm,everyglb=\tenrm,
%   everyglc=\elevenrm,everygld=\twelvermlf=\ninerm]
%\gla x y  illegitimate authority.//
%\gld Resist illegitimate authority.//
%\endgl
%\xe
%
%
%
%\endinput

\ex
\begingl[glstyle=wrap,everygla=\ninerm,everyglb=\tenrm,
   everyglc=\elevenrm,everygld=\twelverm,everygle=\twelvebf,everyglf=\ninerm]
\gla Resist illegitimate authority).//
\glb Resist illegitimate authority.//
\glc Resist illegitimate authority).//
\gld Resist illegitimate authority.//
\gle Resist illegitimate authority.//
\gld Resist illegitimate authority.//
\glc Resist illegitimate authority.//
\glb Resist illegitimate authority.//
\glf Resist illegitimate authority.//
\endgl
\xe

\ex  |everyglword=\offinterlineskip|
\medskip

\leavevmode
\begingl[glstyle=wrap,everygla=\ninerm,everyglb=\tenrm,
   everyglc=\elevenrm,everygld=\twelverm,everygle=\twelvebf,everyglf=\ninerm,
   everyglword=\offinterlineskip]
\gla Resist illegitimate authority.//
\glb Resist illegitimate authority.//
\glc Resist illegitimate authority.//
\gld Resist illegitimate authority.//
\gle Resist illegitimate authority.//
\gld Resist illegitimate authority.//
\glc Resist illegitimate authority.//
\glb Resist illegitimate authority.//
\glf Resist illegitimate authority.//
\endgl
\xe

\ex |everyglword={\baselineskip=0pt
\lineskip=.5ex}|
\smallskip

\leavevmode
\begingl[glstyle=wrap,everygla=\ninerm,everyglb=\tenrm,
   everyglc=\elevenrm,everygld=\twelverm,everygle=\twelvebf,everyglf=\ninerm,
   everyglword={\baselineskip=0pt \lineskip=.5ex}]
\gla Resist illegitimate authority.//
\glb Resist illegitimate authority.//
\glc Resist illegitimate authority.//
\gld Resist illegitimate authority.//
\gle Resist illegitimate authority.//
\gld Resist illegitimate authority.//
\glc Resist illegitimate authority.//
\glb Resist illegitimate authority.//
\glf Resist illegitimate authority.//
\endgl
\xe

\ex manipulating |abovegl|x|skip|
\smallskip

\leavevmode
\begingl[glstyle=wrap,everygla=\ninerm,everyglb=\tenrm,
   everyglc=\elevenrm,everygld=\twelverm,everygle=\twelvebf,everyglf=\ninerm,
   everyglword={\baselineskip=0pt \lineskip=.3ex}]
\gla Resist illegitimate authority.//
\glb Resist illegitimate authority.//
\glc Resist illegitimate authority.//
\gld Resist illegitimate authority.//
\gle[abovegleskip=2ex] Resist illegitimate authority.//
\gld[abovegldskip=2ex] Resist illegitimate authority.//
\glc Resist illegitimate authority.//
\glb Resist illegitimate authority.//
\glf Resist illegitimate authority.//
\endgl
\xe

\ex spacing using struts
\smallskip

\leavevmode
\begingl[glstyle=wrap,everygla=\ninerm,everyglb=\tenrm,
   everyglc=\elevenrm,everygld=\twelverm,everygle=\twelvebf,everyglf=\ninerm,
   everyglword={\baselineskip=0pt \lineskip=.3ex}]
\gla Resist illegitimate authority.//
\glb Resist illegitimate authority.//
\glc Resist illegitimate authority.//
\gld Resist illegitimate authority.//
\gle {\vrule height2ex depth1ex width0pt Resist} illegitimate authority.//
\gld Resist illegitimate authority.//
\glc Resist illegitimate authority.//
\glb Resist illegitimate authority.//
\glf Resist illegitimate authority.//
\endgl
\xe



\endinput
\subsection
{Positioning the free translation to the right of the word by word gloss}

\parinventory
& \idx{|ftpos|}& choice (|below| or |right|)& |below|\cr
& \idx{|sssep|}& dimension& |3em|\cr
& \idx{|ssratio|}& decimal& |.6 |\cr
& \idx{|ssrightskip|}& skip& |0pt plus 2em|\cr
\endparinventory




\lingset{glstyle=wrap,everygl=\openup.5ex,
   everyglword=\normalbaselines,everyglft=\normalbaselines,
   glhangindent=2em}

\framedisplay
\ex[glftpos=right,glhangstyle=none]
\begingl
\gla
Hmao sa co po tha  nu nao nga hmua. Nu dja ga, nu dja cog nu laih
gui reo ne. Todang bboi rok jolan nu nao hma, nu bboh sa droi mra
do bboi gah, a, hruh nu.//
\glb
exist one {clf} person old 3s go do field 3s hold machete 3s hold
hoe 3s pst carry.on.back back.basket 3s while at {along} trail 3s
go field 3s see one clf peacock stay at drct {?} nest 3s//
\glft
`There was an old person who went to work in the field. He took
along his machete, he took along his hoe, and he carried his
basket on his back. While he was on his way to the farm, he saw a
peacock beside its nest.'//
\endgl
\xe
\endframedisplay
\noindent is achieved by
\codedisplay
\ex[glftpos=right,glhangstyle=none]
\begingl
\gla
Hmao sa co po tha  nu nao nga hmua. Nu dja ga, nu dja cog nu laih
gui reo ne. Todang bboi rok jolan nu nao hma, nu bboh sa droi mra
do bboi gah, a, hruh nu.//
\glb
exist one {clf} person old 3s go do field 3s hold machete 3s hold
hoe 3s pst carry.on.back back.basket 3s while at {along} trail 3s
go field 3s see one clf peacock stay at drct {?} nest 3s//
\glft
`There was an old person who went to work in the field. He took
along his machete, he took along his hoe, and he carried his
basket on his back. While he was on his way to the farm, he saw a
peacock beside its nest.'//
\endgl
\xe
|endcodedisplay

|ss| stands for ``side-by-side''.
|sssep| gives the separation of the gloss and the free
translation. |ssratio| gives the proportion of the available
space (which is the leftskip minus the value of |sssep|) which
the gloss occupies.  Line breaking in the free translation is
delicate because it will generally be narrow. The default setting
of |ssrightskip| allows  up to 2em departure from right
alignment. This usually avoids overfull lines and awkward
hyphenation.  |ssrightskip| can be increased (all the way to
\hbox{|0pt plus 1fil|}) if there is a problem, at the cost of a
more ragged appearance.  This can be done globally, or simply in
troublesome examples.

The point of hanging indentation is to visually separate the free
translation and the gloss, so |glhangstyle=none| is completely
satisfactory if the tree translation is on the right. But {it
ExPex\/} will happily use hanging (either cascading or not)
indentation with the free translation on the right.


\framedisplay
\ex[glftpos=right,glhangstyle=none,ssratio=.7,sssep=1.5em,
ssrightskip=0pt plus .4em]
\begingl
\gla
Hmao sa co po tha  nu nao nga hmua. Nu dja ga, nu dja cog nu laih
gui reo ne. Todang bboi rok jolan nu nao hma, nu bboh sa droi mra
do bboi gah, a, hruh nu.//
\glb
exist one {clf} person old 3s go do field 3s hold machete 3s hold
hoe 3s pst carry.on.back back.basket 3s while at {along} trail 3s
go field 3s see one clf peacock stay at drct {?} nest 3s//
\glft
`There was an old person who went to work in the field. He took
along his machete, he took along his hoe, and he carried his
basket on his back. While he was on his way to the farm, he saw a
peacock beside its nest.'//
\endgl
\xe
\endframedisplay

\endinput

\lingset{glhangstyle=none}

\exdisplay\noexno
\begingl[glftpos=below]
[A]\quad \gla
Hmao sa co po tha  nu nao nga hmua. Nu dja ga, nu dja cog nu laih
gui reo ne. Todang bboi rok jolan nu nao hma, nu bboh sa droi mra
do bboi gah, a, hruh nu.//
\glb
exist one {clf} person old 3s go do field 3s hold machete 3s hold
hoe 3s pst carry.on.back back.basket 3s while at {along} trail 3s
go field 3s see one clf peacock stay at drct {?} nest 3s//
\glft
`There was an old person who went to work in the field. He took
along his machete, he took along his hoe, and he carried his
basket on his back. While he was on his way to the farm, he saw a
peacock beside its nest.'//
\endgl
\xe

\exdisplay\noexno
\begingl[glftpos=right]
[B]\quad \gla
Hmao sa co po tha  nu nao nga hmua. Nu dja ga, nu dja cog nu laih
gui reo ne. Todang bboi rok jolan nu nao hma, nu bboh sa droi mra
do bboi gah, a, hruh nu.//
\glb
exist one {clf} person old 3s go do field 3s hold machete 3s hold
hoe 3s pst carry.on.back back.basket 3s while at {along} trail 3s
go field 3s see one clf peacock stay at drct {?} nest 3s//
\glft
`There was an old person who went to work in the field. He took
along his machete, he took along his hoe, and he carried his
basket on his back. While he was on his way to the farm, he saw a
peacock beside its nest.'//
\endgl
\xe

\ex[glftpos=below]
\begingl
\gla
Hmao sa co po tha  nu nao nga hmua. Nu dja ga, nu dja cog nu laih
gui reo ne. Todang bboi rok jolan nu nao hma, nu bboh sa droi mra
do bboi gah, a, hruh nu.//
\glb
exist one {clf} person old 3s go do field 3s hold machete 3s hold
hoe 3s pst carry.on.back back.basket 3s while at {along} trail 3s
go field 3s see one clf peacock stay at drct {?} nest 3s//
\glft
`There was an old person who went to work in the field. He took
along his machete, he took along his hoe, and he carried his
basket on his back. While he was on his way to the farm, he saw a
peacock beside its nest.'//
\endgl
\xe


\ex[glftpos=right]
\begingl
\gla
Hmao sa co po tha  nu nao nga hmua. Nu dja ga, nu dja cog nu laih
gui reo ne. Todang bboi rok jolan nu nao hma, nu bboh sa droi mra
do bboi gah, a, hruh nu.//
\glb
exist one {clf} person old 3s go do field 3s hold machete 3s hold
hoe 3s pst carry.on.back back.basket 3s while at {along} trail 3s
go field 3s see one clf peacock stay at drct {?} nest 3s//
\glft
`There was an old person who went to work in the field. He took
along his machete, he took along his hoe, and he carried his
basket on his back. While he was on his way to the farm, he saw a
peacock beside its nest.'//
\endgl
\xe

\lingset{glhangstyle=normal}

\exdisplay\noexno
\begingl[glftpos=below]
[A]\quad \gla
Hmao sa co po tha  nu nao nga hmua. Nu dja ga, nu dja cog nu laih
gui reo ne. Todang bboi rok jolan nu nao hma, nu bboh sa droi mra
do bboi gah, a, hruh nu.//
\glb
exist one {clf} person old 3s go do field 3s hold machete 3s hold
hoe 3s pst carry.on.back back.basket 3s while at {along} trail 3s
go field 3s see one clf peacock stay at drct {?} nest 3s//
\glft
`There was an old person who went to work in the field. He took
along his machete, he took along his hoe, and he carried his
basket on his back. While he was on his way to the farm, he saw a
peacock beside its nest.'//
\endgl
\xe

\exdisplay\noexno
\begingl[glftpos=right]
[B]\quad \gla
Hmao sa co po tha  nu nao nga hmua. Nu dja ga, nu dja cog nu laih
gui reo ne. Todang bboi rok jolan nu nao hma, nu bboh sa droi mra
do bboi gah, a, hruh nu.//
\glb
exist one {clf} person old 3s go do field 3s hold machete 3s hold
hoe 3s pst carry.on.back back.basket 3s while at {along} trail 3s
go field 3s see one clf peacock stay at drct {?} nest 3s//
\glft
`There was an old person who went to work in the field. He took
along his machete, he took along his hoe, and he carried his
basket on his back. While he was on his way to the farm, he saw a
peacock beside its nest.'//
\endgl
\xe

\ex[glftpos=below]
\begingl
\gla
Hmao sa co po tha  nu nao nga hmua. Nu dja ga, nu dja cog nu laih
gui reo ne. Todang bboi rok jolan nu nao hma, nu bboh sa droi mra
do bboi gah, a, hruh nu.//
\glb
exist one {clf} person old 3s go do field 3s hold machete 3s hold
hoe 3s pst carry.on.back back.basket 3s while at {along} trail 3s
go field 3s see one clf peacock stay at drct {?} nest 3s//
\glft
`There was an old person who went to work in the field. He took
along his machete, he took along his hoe, and he carried his
basket on his back. While he was on his way to the farm, he saw a
peacock beside its nest.'//
\endgl
\xe


\ex[glftpos=right]
\begingl
\gla
Hmao sa co po tha  nu nao nga hmua. Nu dja ga, nu dja cog nu laih
gui reo ne. Todang bboi rok jolan nu nao hma, nu bboh sa droi mra
do bboi gah, a, hruh nu.//
\glb
exist one {clf} person old 3s go do field 3s hold machete 3s hold
hoe 3s pst carry.on.back back.basket 3s while at {along} trail 3s
go field 3s see one clf peacock stay at drct {?} nest 3s//
\glft
`There was an old person who went to work in the field. He took
along his machete, he took along his hoe, and he carried his
basket on his back. While he was on his way to the farm, he saw a
peacock beside its nest.'//
\endgl
\xe


\lingset{glhangstyle=cascade}

\exdisplay\noexno
\begingl[glftpos=below]
[A]\quad \gla
Hmao sa co po tha  nu nao nga hmua. Nu dja ga, nu dja cog nu laih
gui reo ne. Todang bboi rok jolan nu nao hma, nu bboh sa droi mra
do bboi gah, a, hruh nu.//
\glb
exist one {clf} person old 3s go do field 3s hold machete 3s hold
hoe 3s pst carry.on.back back.basket 3s while at {along} trail 3s
go field 3s see one clf peacock stay at drct {?} nest 3s//
\glft
`There was an old person who went to work in the field. He took
along his machete, he took along his hoe, and he carried his
basket on his back. While he was on his way to the farm, he saw a
peacock beside its nest.'//
\endgl
\xe

\exdisplay\noexno
\begingl[glftpos=right]
[B]\quad \gla
Hmao sa co po tha  nu nao nga hmua. Nu dja ga, nu dja cog nu laih
gui reo ne. Todang bboi rok jolan nu nao hma, nu bboh sa droi mra
do bboi gah, a, hruh nu.//
\glb
exist one {clf} person old 3s go do field 3s hold machete 3s hold
hoe 3s pst carry.on.back back.basket 3s while at {along} trail 3s
go field 3s see one clf peacock stay at drct {?} nest 3s//
\glft
`There was an old person who went to work in the field. He took
along his machete, he took along his hoe, and he carried his
basket on his back. While he was on his way to the farm, he saw a
peacock beside its nest.'//
\endgl
\xe

\ex[glftpos=below]
\begingl
\gla
Hmao sa co po tha  nu nao nga hmua. Nu dja ga, nu dja cog nu laih
gui reo ne. Todang bboi rok jolan nu nao hma, nu bboh sa droi mra
do bboi gah, a, hruh nu.//
\glb
exist one {clf} person old 3s go do field 3s hold machete 3s hold
hoe 3s pst carry.on.back back.basket 3s while at {along} trail 3s
go field 3s see one clf peacock stay at drct {?} nest 3s//
\glft
`There was an old person who went to work in the field. He took
along his machete, he took along his hoe, and he carried his
basket on his back. While he was on his way to the farm, he saw a
peacock beside its nest.'//
\endgl
\xe


\ex[glftpos=right]
\begingl
\gla
Hmao sa co po tha  nu nao nga hmua. Nu dja ga, nu dja cog nu laih
gui reo ne. Todang bboi rok jolan nu nao hma, nu bboh sa droi mra
do bboi gah, a, hruh nu.//
\glb
exist one {clf} person old 3s go do field 3s hold machete 3s hold
hoe 3s pst carry.on.back back.basket 3s while at {along} trail 3s
go field 3s see one clf peacock stay at drct {?} nest 3s//
\glft
`There was an old person who went to work in the field. He took
along his machete, he took along his hoe, and he carried his
basket on his back. While he was on his way to the farm, he saw a
peacock beside its nest.'//
\endgl
\xe


\endinput
\bye

\ex[glftpos=below,glhangstyle=normal] \mkgloss\hmao \xe

\ex[glftpos=right,glhangstyle=normal] \mkgloss\hmao \xe

\ex[glftpos=below,glhangstyle=cascade] \mkgloss\hmao \xe

\ex[glftpos=right,glhangstyle=cascade,glhangindent=1em] \mkgloss\hmao \xe

\ex \mkgloss[glftpos=right,glhangstyle=cascade,glhangindent=1em]\hmao \xe


\noindent
\begingl
%\relax[A]\quad
\gla
Hmao sa co po tha  nu nao nga hmua. Nu dja ga, nu dja cog nu laih
gui reo ne. Todang bboi rok jolan nu nao hma, nu bboh sa droi mra
do bboi gah, a, hruh nu.//
\glb
exist one {clf} person old 3s go do field 3s hold machete 3s hold
hoe 3s pst carry.on.back back.basket 3s while at {along} trail 3s
go field 3s see one clf peacock stay at drct {?} nest 3s//
\glft
`There was an old person who went to work in the field. He took
along his machete, he took along his hoe, and he carried his
basket on his back. While he was on his way to the farm, he saw a
peacock beside its nest.'//
\endgl


