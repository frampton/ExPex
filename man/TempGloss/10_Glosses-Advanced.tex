
\section More on Glosses

\vskip-2ex
\subsection User defined levels

\begininventory
\macros* \idx{|\defineglwlevels|}\endmc
\endinventory
%
|\glb| and |\glc| are given definitions in {\it expex.tex\/}
by the command |\defineglwlevels{b,c}|.  The command also creates the
parameters |everyglb|, |everyglc|, |aboveglbskip|, and |aboveglcskip|.
|everyglb| and |everyglc| are initialized to empty token lists and
|aboveglbskip| and |aboveglcskip| to \textdim{0 pt}. The user may want
to use |\defineglwlevels| to create and name new gloss
levels.

For example, suppose more suggestive level names are
desired.

\codedisplay
\defineglwlevels{cat,gloss}
\lingset{everyglcat=\footnotesize,aboveglcatskip=-.5ex}|exbreak

\ex
\begingl
\gla k- wapm -a -s'i -m -wapunin -uk //
\glcat Cl V Agr Neg Agr Tns Agr //|par|allowbreak
\glgloss 2 see 3{\sc acc} {} {2\sc pl} preterit {3\sc pl} //
\glft `you (pl) didn't see them'//
\endgl
\xe
|endcodedisplay
produces

\framedisplay
\defineglwlevels{cat,gloss}
\lingset{everyglcat=\footnotesize,aboveglcatskip=-.5ex}

\ex
\begingl
\gla k- wapm -a -s'i -m -wapunin -uk //
\glcat Cl V Agr Neg Agr Tns Agr //
\glgloss 2 see {3\sc acc} {} {2\sc pl} preterit {3\sc pl} //
\glft `you (pl) didn't see them'//
\endgl
\xe
\endframedisplay

Another instance in which the user might want to define a new gloss level
or levels is if more than 3 lines of interlinear gloss are needed and
the desired flexibility cannot be obtained by repeated use of |\glb|
or |\glc|.


%%%%%%%%%%%%%%%%%%%%%%%%%%%%%%%%%%%%%%%%%%%%%%%%%
\subsection Positioning the free translation to the right of the
interlinear gloss

\begininventory
\parameters
\idx{|glftpos|}& choice (|below| or |right|)& |below|\cr
\idx{|sssep|}& dimension& 3\dimskip em\cr
\idx{|ssratio|}& decimal& |.6 |\cr
\idx{|ssrightskip|}& skip& 0\dimskip pt plus 2\dimskip em\cr
\idx{|glhangstyle|}& |normal|, |none|, or |cascade|& |normal|\cr
\endinventory

\lingset{everygl=\openup.5ex,
   everyglword=\normalbaselines,everyglft=\normalbaselines,
   glhangindent=2em}

\framedisplay
\ex[glftpos=right,glhangstyle=none]
\let\\=\textsc
\begingl
\gla
Hom\^{a}o sa \v{c}\^{o} p\^{o} tha  \~{n}u nao ng\u{a} hmua. \~{N}u
dj\u{a} g\u{a}, \~{n}u dj\u{a} \v{c}\u{o}ng \~{n}u, laih gui r\^{e}o
\~{n}u. Todang bboi r\^{o}k jolan \~{n}u nao hma, \~{n}u bb\^{o}h sa
droi mr\u{a} d\u{o} bboi gah, a, hruh \~{n}u.//
\glb
\\{exist} one \\{clf} person old \\{3s} go do field \\{3s} hold
machete \\{3s} hold hoe \\{3s} and carry.on.back back.basket \\{3s}
while at along trail \\{3s} go field \\{3s} see one \\{clf} peacock
stay at \\{drct} -- nest \\{3s}//
\glft
`There was an old person who went to work in the field. He took
along his machete, he took along his hoe, and he carried his
basket on his back. While he was on his way to the farm, he saw a
peacock beside its nest.'//
\endgl
\xe
\endframedisplay

\noindent is achieved by
\codedisplay~
\ex[glftpos=right,glhangstyle=none]
\let\\=\textsc
\begingl
\gla
Hom\^{a}o sa \v{c}\^{o} p\^{o} tha  \~{n}u nao ng\u{a} hmua. \~{N}u
dj\u{a} g\u{a}, \~{n}u dj\u{a} \v{c}\u{o}ng \~{n}u, laih gui r\^{e}o
\~{n}u. Todang bboi r\^{o}k jolan \~{n}u nao hma, \~{n}u bb\^{o}h sa
droi mr\u{a} d\u{o} bboi gah, a, hruh \~{n}u.//
\glb
\\{exist} one \\{clf} person old \\{3s} go do field \\{3s} hold
machete \\{3s} hold hoe \\{3s} and carry.on.back back.basket \\{3s}
while at along trail \\{3s} go field \\{3s} see one \\{clf} peacock
stay at \\{drct} -- nest \\{3s}//
\glft
`There was an old person who went to work in the field. He took
along his machete, he took along his hoe, and he carried his
basket on his back. While he was on his way to the farm, he saw a
peacock beside its nest.'//
\endgl
\xe
|endcodedisplay
(This example, as well as (\getref{panelex}), was contributed by
Joshua Jensen.  It is from Jarai, an Austronesian language.  The
story teller was Hyech Ksor.  The orthography here is somewhat
simplified in order to keep the font requirements for the
examples in this documentation elementary.)

|ss| stands for ``side-by-side''. |sssep| gives the separation of the
gloss and the free translation. |ssratio| gives the proportion of the
available width that the gloss occupies. The point of hanging
indentation is to visually separate the free translation and the
gloss, so |glhangstyle=none| is completely satisfactory if the
tree translation is on the right. But \ExPex\ will happily use
hanging indentation with the free translation on the right.

Line breaking in the free translation is delicate because it will
generally set in a narrow column. The default setting of |ssrightskip|
allows up to \textdim{2 em} departure from right alignment. This usually avoids
overfull lines and awkward hyphenation.  |ssrightskip| can be
increased (all the way to \textdim{0 pt} plus \textdim{1 fil}) if there is a
problem, at the cost of a more ragged appearance.  This can be done
globally, or simply in troublesome examples.

\subsection Glosses with a side panel

\begininventory
\macros
\idx{|\beginglpanel[]|}, \idx{|\endpanel|}\endmc
\parameters*
\idx{|everypanel|\user}& token list& |{}|\cr
\endinventory
The mechanism for positioning the free translation to the right of the
interlinear gloss can be adapted to create a side panel for
glosses which can be used for other purposes, as illustrated
below.

\framedisplay
\ex[everypanel=\footnotesize]<panelex>
\let\\=\textsc
\beginglpanel[ssratio=.5,glhangstyle=none]
\gla Hom\^{a}o$^1$ sa \v{c}\^{o} p\^{o} tha  \~{n}u nao ng\u{a}
hmua. \~{N}u dj\u{a} g\u{a}, \~{n}u dj\u{a} \v{c}\u{o}ng \~{n}u,
laih gui r\^{e}o \~{n}u. Todang bboi r\^{o}k jolan \~{n}u nao
hma, \~{n}u bb\^{o}h sa droi mr\u{a} d\u{o}$\,^4$ bboi gah, a, hruh
\~{n}u.//
\glb \\{exist} one \\{clf} person old \\{3s} go$^2$ do field
\\{3s} hold machete \\{3s} hold hoe \\{3s} and$^3$ carry.on.back
back.basket \\{3s} while at along trail \\{3s} go field \\{3s}
see one \\{clf} peacock stay at \\{drct} -- nest \\{3s}
//
\endgl
1.\enspace {\it hom\^{a}o} also means `have', reflecting the
strong tendency across languages to use the same word for
possession and the existential. {\it hom\^{a}o} is clause-initial
in existential clauses, but it comes after the subject in
possession clauses.

2.\enspace All verbs are glossed with a bare form, as Jarai has
no inflectional morphology. Although Jarai has lexical items that
encode tense, they are relatively infrequent in text.

3.\enspace The word {\it laih} is literally `after; finish', but
that is clearly not the meaning here. Probably {\it laih} here is
an abbreviation for {\it laih an\u{u}n}, `after that; and', hence
the gloss `and'.

4.\enspace {\it d\u{o}} `sit, stay' is used like a copula in
locative clauses, which is what I assume here (`a~peacock
[which was] beside its nest'); however, this could just as well
mean `a peacock sitting beside its nest', retaining the posture
semantics.
\endpanel
\bigskip
`There was an old person who went to work in the field. He took
along his machete, he took along his hoe, and he carried his
basket on his back. While he was on his way to the farm, he saw a
peacock beside its nest.'
\xe
\endframedisplay

The syntax is:
\codedisplay
\beginglpanel |dots \endgl |dots \endpanel
|endcodedisplay
The first part is the gloss, with the usual syntax.  The second
part is put in a vbox and set alongside the gloss. The tokens
|lingeverypanel| are inserted when the vbox begins. All of the
parameters which are special to positioning the free translation
to the right of the gloss apply here as well, with the obvious
meanings.

The complete code for the example above is:

\codedisplay
\ex[everypanel=\footnotesize]<panelex>
\def\\#1{{\footnotesize\uppercase{#1}}}%
\let\\=\textsc
\beginglpanel[ssratio=.5,glhangstyle=none]
\gla Hom\^{a}o$^1$ sa \v{c}\^{o} p\^{o} tha  \~{n}u nao ng\u{a}
hmua. \~{N}u dj\u{a} g\u{a}, \~{n}u dj\u{a} \v{c}\u{o}ng \~{n}u,
laih gui r\^{e}o \~{n}u. Todang bboi r\^{o}k jolan \~{n}u nao
hma, \~{n}u bb\^{o}h sa droi mr\u{a} d\u{o}$\,^4$ bboi gah, a, hruh
\~{n}u.//
\glb \\{exist} one \\{clf} person old \\{3s} go$^2$ do field
\\{3s} hold machete \\{3s} hold hoe \\{3s} and$^3$ carry.on.back
back.basket \\{3s} while at along trail \\{3s} go field \\{3s}
see one \\{clf} peacock stay at \\{drct} -- nest \\{3s}//
\endgl
1.\enspace {\it hom\^{a}o} also means `have', reflecting the
strong tendency across languages to use the same word for
possession and the existential. {\it hom\^{a}o} is clause-initial
in existential clauses, but it comes after the subject in
possession clauses.

2.\enspace All verbs are glossed with a bare form, as Jarai has
no inflectional morphology. Although Jarai has lexical items that
encode tense, they are relatively infrequent in text.

3.\enspace The word {\it laih} is literally `after; finish', but
that is clearly not the meaning here. Probably {\it laih} here is
an abbreviation for {\it laih an\u{u}n}, `after that; and', hence
the gloss `and'.

4.\enspace {\it d\u{o}} `sit, stay' is used like a copula in
locative clauses, which is what I assume here (`a~peacock [which
was] beside its nest'); however, this could just as well mean `a
peacock sitting beside its nest', retaining the posture
semantics.
\endpanel
\bigskip
`There was an old person who went to work in the field. He took
along his machete, he took along his hoe, and he carried his
basket on his back. While he was on his way to the farm, he saw a
peacock beside its nest.'
\xe
|endcodedisplay
Note that the free translation here comes after |\endpanel| and is
typeset the way any material inside an |\ex| construction is
typeset.  This allows it to have full width, spanning both the
gloss and notes.  It could have been part of the gloss, with a
different result.

No support is given to side note numbering.  It must be done ``by
hand''.  If the construction turns out to be sufficiently useful
and hand numbering is sufficiently tedious, a more automatic
scheme might be possible.  It would not be trivial, because the
order in which the notes appear inside the gloss before it is
typeset is not necessarily the same as the order in which they
appear after it is typeset.

\subsection Cascading hanging indentation in glosses

\raggedbottom
\lingset{glhangindent=.25in,everygl=\openup.5ex,
   everyglword=\normalbaselines,everyglft=\normalbaselines,
   extraglskip=1ex}

\lingset{glhangstyle=none}
\framedisplay
\ex[glhangstyle=cascade]
\let\\=\textsc
\begingl
\gla
Hom\^{a}o sa \v{c}\^{o} p\^{o} tha  \~{n}u nao ng\u{a} hmua. \~{N}u
dj\u{a} g\u{a}, \~{n}u dj\u{a} \v{c}\u{o}ng \~{n}u, laih gui r\^{e}o
\~{n}u. Todang bboi r\^{o}k jolan \~{n}u nao hma, \~{n}u bb\^{o}h sa
droi mr\u{a} d\u{o} bboi gah, a, hruh \~{n}u.//
\glb
\\{exist} one \\{clf} person old \\{3s} go do field \\{3s} hold
machete \\{3s} hold hoe \\{3s} and carry.on.back back.basket \\{3s}
while at along trail \\{3s} go field \\{3s} see one \\{clf} peacock
stay at \\{drct} -- nest \\{3s}//
\glft
`There was an old person who went to work in the field. He took
along his machete, he took along his hoe, and he carried his
basket on his back. While he was on his way to the farm, he saw a
peacock beside its nest.'//
\endgl
\xe
\endframedisplay

is achieved by
\codedisplay~
\ex[glhangstyle=cascade]
\let\\=\textsc
\begingl
\gla
Hom\^{a}o sa \v{c}\^{o} p\^{o} tha  \~{n}u nao ng\u{a} hmua. \~{N}u
dj\u{a} g\u{a}, \~{n}u dj\u{a} \v{c}\u{o}ng \~{n}u, laih gui r\^{e}o
\~{n}u. Todang bboi r\^{o}k jolan \~{n}u nao hma, \~{n}u bb\^{o}h sa
droi mr\u{a} d\u{o} bboi gah, a, hruh \~{n}u.//
\glb
\\{exist} one \\{clf} person old \\{3s} go do field \\{3s} hold
machete \\{3s} hold hoe \\{3s} and carry.on.back back.basket \\{3s}
while at along trail \\{3s} go field \\{3s} see one \\{clf} peacock
stay at \\{drct} -- nest \\{3s}//
\glft
`There was an old person who went to work in the field. He took
along his machete, he took along his hoe, and he carried his
basket on his back. While he was on his way to the farm, he saw a
peacock beside its nest.'//
\endgl
\xe
|endcodedisplay

\subsection  Gloss underfixes

\begininventory
\macros* \idx{|\gluf|}\endmc
\parameters
\idx{|glufcloseup|}& dimension& \textdim{.4 ex}\cr
\idx{|everygluf|}& token list& |{}|\cr
\endinventory

\noindent Sometimes, gloss displays like the following are
desired, with grammatical markings written below the gloss.

\framedisplay
\ex[glhangstyle=normal,glufcloseup=.4ex,everygluf=\footnotesize]
\begingl
\gla Mary$_i$ ist sicher, dass es den Hans nicht st\"oren
   w\"urde seiner Freundin ihr$_i$ Herz auszusch\"utten.//
\glb Mary is sure that it \gluf/the/ACC/ Hans not annoy would
   \gluf/his/DAT/ \gluf/girlfriend/DAT/ \gluf/her/ACC/
   \gluf/heart/ACC/ {out to throw}//
\glft `Mary is sure that to reveal her heart to his girlfriend
would not damage John.'//
\endgl
\xe
\endframedisplay

\ExPex\ provides the macro |\gluf| which can be used to
construct such a display.

\codedisplay
\ex[glhangstyle=normal,glufcloseup=.4ex,everygluf=\footnotesize]
\begingl
\gla Mary$_i$ ist sicher, dass es den Hans nicht st\"oren
   w\"urde seiner Freundin ihr$_i$ Herz auszusch\"utten.//
\glb Mary is sure that it \gluf/the/ACC/ Hans not annoy would
   \gluf/his/DAT/ \gluf/girlfriend/DAT/ \gluf/her/ACC/
   \gluf/heart/ACC/ {out to throw}//
\glft `Mary is sure that to reveal her heart to his girlfriend
would not damage John.'//
\endgl
\xe
|endcodedisplay

The grammatical markings are essentially ``underfixes'' (rather than
prefixes or suffixes), hence the name ``gluf'' (gl underfix). When the
underfix is typeset, the value of |everygluf| is first inserted.  It
is provided so that the user has control of the font used to typeset
the underfixes. The value of |glufcloseup| determines how much the
baselineskip between the underfix and the underfixed word is closed
up.  Without some closeup, the underfixes are not positioned close
enough to the glosses they modify (in my opinion). The macro |\gluf|
centers the underfix below the word it annotates.  Its syntax should
be clear from the example above.


