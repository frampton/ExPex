
\section User defined styles

\begininventory
\macros* \idx{|\definelingstyle|}\cr
\parameters
\idx{|everyex|}& token list& |{}|\cr
\idx{|Everyex|}& token list& |{}|\cr
\endinventory
%
Aside from all the \Expex\/ parameters introduced to this point,
there are numerous \Tex\/ parameters that affect the appearance
of examples: line spacing, hsize, font selection, etc.  These are
all quantities that \Tex\/ itself provides mechanisms for setting
to suit the user.  You may want to make adjustments of these
parameters every time a particular kind of example is typeset, so
that special settings hold inside these examples independently of
the contextual settings.  For example, suppose you have many
examples that, for some reason, you want to format like~(\nextx),
with a narrow width, italic font, an oversized gap between the
example number and the text, and a somewhat greater than normal
separation between baselines.

\framedisplay
\leftline{\vbox{%
\ex[textoffset=3em]
\hsize=3in \it \advance\baselineskip by 2pt
Und hier k\"onnen wir sehen was f\"ur Unfug wird gemacht
wenn er einen ganz langen Satz binnen kriegt.
\xe
}\hfil}
\endframedisplay

One straightforward way to do this is:

\codedisplay~
\ex[textoffset=3em]
\hsize=3in \it \advance\baselineskip by 2pt
Und hier k\"onnen wir sehen was f\"ur Unfug wird gemacht
wenn er einen ganz langen Satz binnen kriegt.
\xe
|endcodedisplay

If you have many examples that you want to typeset in this way
that are scattered throughout a document, it is awkward to have
to remember all of these settings and enter them each time an
example of this kind needs to be formatted.  Furthermore, if you
change your mind about some detail of this special formatting,
you need to go through the document and change each instance of
this formatting.  \Expex\ allows you to package all the
formatting changes into one named unit, called a ``style''.  If
this style is called ``narrow italic'', then you can write the following
to achieve the format of (\lastx).

\codedisplay
\ex[lingstyle=narrow italic]
|endcodedisplay

The various format specifications are packaged into a style
by saying:

\codedisplay
\definelingstyle{narrow italic}{textoffset=3em,
   everyex={\hsize=3in \it \advance\baselineskip by 2pt}}
|endcodedisplay
Thereafter, setting the parameter |lingstyle| to |narrow italic|
has the effect of setting all the parameters as specified in the
style definition (the two parameters |textoffset| and |everyex|
in this example). The parameter |everyex| works in the following
way.  If |expex| is set to {\sl value\/}, a macro |\lingeveryex|
is defined whose expansion is {\sl value}.  When
an |\ex| or |\pex| construction is initiated, {\it after\/} any
parameter settings take effect, |\lingeveryex| is executed.

\Expex\ provides a second parameter, |Everyex|, which is similar to
|expex|.  If |Expex| is set to value, a macro |ling@Everypex| is
defined. When
an |\ex| or |\pex| construction is initiated, {\it before\/} after any
local parameter changes take effect, |\lingEveryex| is executed.
The reasons for providing both |everyex| and |Everyex| are
subtle. Different stages of the writing/rewriting/editing process
may call for different treatments of example formatting,
particularly if the final aim is a camera ready product.  Suppose
you normally make double spaced drafts with an hsize of
\textdim{6.5 in} and that your final aim is to produce camera ready copy with
an hsize of \textdim{4.375 in}.  Suppose also that you are at
editing stage where you want to see exactly how the examples
will be formatted in the finished product, but still want full width
double spaced text.  One way to accomplish this is to say

\codedisplay
\lingset{everyex={\hsize=4.375in \normalbaselines}}
|endcodedisplay
at the beginning of your document.  |\normalbaselines|
is a standard \Tex\/ macro which establishes the normal line
spacing for the current font.  This makes the hsize and line
spacing inside examples independent of the hsize and line spacing
you select for the text (outside examples).

This will accomplish what you want.  But it makes the further use
of |everyex| in your document awkward.  If you want a particular
example to be set in italics, for example, you might think to
use:

\codedisplay
\ex[everyex=\it]
|endcodedisplay

\noindent This will certainly produce an italicized example
because |\it| will be evaluated early on in typesetting the
example.  But it has the unfortunate consequence of overwriting
the initial setting of |everyex|, removing the special line
spacing within examples.

|Everyex| is provided to accommodate situations like this. The
intention is that special settings that should hold inside
examples throughout the document are assigned to |Everyex|, with
|expex| reserved for local variations.  Note that since
|\lingEveryex| is evaluated before local parameter changes take
effect, parameter settings specified by |Expex| will be overridden
by any local parameter settings.

\endinput





\codedisplay~
\ex[textoffset=3em,
   everyex={\hsize=3in \it \advance\baselineskip by 2pt}]
Und hier k\"onnen wir sehen was f\"ur Unfug wird gemacht
wenn er einen ganz langen Satz binnen kriegt.
\xe |endcodedisplay




The parameters |everyex| and |Everyex| are provided by \Expex\/
to facilitate making these kind of adjustments every time an
example is printed. The effect of

\codefragment
|\lingset{everyex=�value�}|
|\lingset{everyex=�value�}|
\endcodefragment

\noindent is to define a macro |\lingeveryex| which expands to
{\sl value}.  When |\ex| is
expanded, a group is first initiated and
the parameter settings furnished to |\ex| are
passed to |\lingset|, then |\lingeveryex| is executed.





In order to illustrate this, suppose you have many examples that
you want to typeset in the following way:

\endinput






consider the code below:

\codedisplay
\ex[textoffset=3em,
   everyex={\hsize=3in \it \advance\baselineskip by 2pt}]
Und hier k\"onnen wir sehen was f\"ur Unfug wird gemacht
wenn er einen ganz langen Satz binnen kriegt.
\xe |endcodedisplay

\noindent The XKV evaluator will remove surrounding braces from
values associated with keys.  The braces around the value
associated with |everyex| above are not required by the syntax,
but do make the code much more readable.  If a value contains a
comma, however, braces are mandatory.  The XKV evaluator uses
commas to parse the key/value pairs and a comma inside a value
which is not hidden by braces will be incorrectly interpreted as
separating key/value pairs.  Chaos will ensue.

The code above produces:

Since the following code produces the same thing, the utility of
the parameter |everyex| is not yet evident. It does have the
advantage that all the special settings are put in one place, the
optional argument of |\ex|.  More substantial advantages will
become clearer after the parameter |lingstyle| is discussed in
the next section.

\codedisplay
\ex[textoffset=3em]
\hsize=3in \it \advance\baselineskip by 2pt
Und hier k\"onnen wir sehen was f\"ur Unfug wird gemacht
wenn er einen ganz langen Satz binnen kriegt.
\xe |endcodedisplay

The reasons for providing both |everyex| and |Everyex| are
subtle. Different stages of the writing/rewriting/editing process
may call for different treatments of example formatting,
particularly if the final aim is a camera ready product.  Suppose
you normally make double spaced drafts with an hsize of $6.5\,\rm
in$ and that your final aim is to produce camera ready copy with
an hsize of $4.375\,\rm in$.  Suppose also that you are at
editing stage where you want to see exactly what the examples
will look like in the finished product, but still want full width
double spaced text.  One way to accomplish this is to say

\codefragment
|\lingset{everyex={\hsize=4.375in \normalbaselines}}|
\endcodefragment

\noindent at the beginning of your document.  |\normalbaselines|
is a standard \Tex\/ macro which establishes the normal line
spacing for the current font.  This makes the hsize and line
spacing inside examples independent of the hsize and line spacing
you select for the text (outside examples).

This will accomplish what you want.  But it makes the further use
of |everyex| in your document awkward.  If you want a particular
example to be set in italics, for example, you might think to
use:

\codefragment
|\ex[everyex=\it]|
\endcodefragment

\noindent This will certainly produce an italicized example
because |\it| will be evaluated early on in typesetting the
example.  But it has the unfortunate consequence of overwriting
the initial setting of |everyex|, removing the special line
spacing within examples.

The parameter |Expex| is provided to accommodate this situation.
The intention is that special settings that should hold
throughout the document are assigned to |Expex|, with |expex|
reserved for local variations.

\subsection User defined styles

\noindent Macro:\quad \idx{|\definelingstyle|}\par
\parinventory*
& \idx{|lingstyle|}& name of user defined style\cr
\endparinventory

\noindent After

\codefragment
|\definelingstyle{�name�}{�key$_1$�=�value$_1$�,� \dots\ �,�key$_n$�=�value$_n$�}|
\endcodefragment

\noindent then

\codefragment
|\lingset{lingstyle=�name�}|
\endcodefragment

\noindent is equivalent to

\codefragment
|\lingset{�key$_1$�=�value$_1$�,� \dots\ �,�key$_n$�=�value$_n$�}|
\endcodefragment

If you have multiple special examples in your document which all
require the same special parameter settings, the array of
parameter settings (perhaps only a single change) which is
required can be given a style name and the parameter changes can
be simply made.

Suppose, for example, that you have multiple examples that you
want to format in the same way as (\lastx). If you first define
the style  ``narrow italic'' by:

\codefragment
|\definelingstyle{narrowitalic}{textoffset=3em,|\par\nobreak
|   everyex={\hsize=3in \it \advance\baselineskip by 2pt}}|
\endcodefragment
\definelingstyle{narrowitalic}%
   {textoffset=3em,everyex={\hsize=3in \it \advance\baselineskip
   by 2pt}}%

\noindent Then examples like (\lastx) can be simply formatted by
using:

\codefragment
|\ex[lingstyle=narrowitalic]|
\endcodefragment

Aside from relatively trivial benefit of saving typing, the
ability to define styles has all the well-known advantages of
uniformity, ease of modification, and clarity of typographic
organization. |lingstyle| is a parameter like all others in that
it can be set along with other parameters with standard left to
right key setting.

You could say, for example:\exbreak

\codedisplay
\ex[lingstyle=narrowitalic,textoffset=1em]
Und hier k\"onnen wir sehen was f\"ur Unfug wird gemacht
wenn er einen ganz langen Satz binnen kriegt.
\xe |endcodedisplay

\framedisplay~ \leftline{\vbox{%
\ex[lingstyle=narrowitalic,textoffset=1em]
Und hier k\"onnen wir sehen was f\"ur Unfug wird gemacht
wenn er einen ganz langen Satz binnen kriegt.
\xe
}}\endframedisplay









