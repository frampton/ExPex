
%\lingset{aboveexskip=0pt,glspace=1em plus .5em minus.2em,
%   glrightskip=0pt plus 4em,aboveglftskip=0pt,glneveryline={\it},
%   glstyle=nlevel,everygl={\hangindent=1.2em \hangafter=1}}
\lingset{aboveexskip=0pt,glspace=1em plus .5em minus.2em,
   glrightskip=0pt plus 4em,aboveglftskip=0pt,
   glstyle=nlevel,everygl={\hangindent=1.2em \hangafter=1}}

\hsize=5.2in
%%%%%%%%%%%%%%%%%%%%%%%%%%%%%%%%%%%%%%%%%%%%%%%%%%%%%%%%%%%%%%%%%%%%%%%

\subsection Illustrative running glossed text

\bigskip

\noindent
The text is a story from Erromangan, found in Chapter 9 of Terry
Crowley's `An Erromangan (Sye) Grammar'.

\bigskip
\exdisplay
\begingl
Yoco-nompi[\textsc{1sg:fut-mr:}do]
stori\putfnno[story]%
   \footnotebody{Corrected editorially to:
   \exdisplay[aboveexskip=1ex,belowexskip=0pt]\begingl
   Yoco-nompi[\textsc{1sg:fut-mr:}do] uvuvu[story] \endilg
   \glft I will tell a story//\endgl\xe
   }
gi[\sc obl]
lakih.[rat]
\endilg
\glft I will tell a story about the rat.//
\endgl
 \xe

\exdisplay
\begingl
Lakih[rat]
-etti[\textsc{3sg:distpast-br:}give.birth]
ra[\sc loc]
{novkilyenau $(<$novkilye-n}[dried.leaf-\sc const]
nau$)$.[bamboo]
\endilg
\glft The rat gave birth in the dried leaves of the bamboo.//
\endgl
\xe

\exdisplay\begingl
M-ante[\textsc{sg:es-mr:hab}]
{mandgi $(<$ m-and{\ips @}g-i$\mskip2mu)$}[\textsc{sg:es-mr:}hear-\textsc{const}]
nam[talk]
orog-orog[big-\sc redup]
u-ntemne[\textsc{loc}-village]
\endilg
\glft And she would always hear loud talk in the village.//
\endgl
\xe

\exdisplay
\begingl
Ovoteme[\textsc{pl}:person]
rum-ante[\textsc{3pl:pasthab-mr:}stat]
u-ntemne[\textsc{loc}-village]
rum-nam[\textsc{3pl:pasthab-mr:}talk]
orog-orog.[big-\textsc{redup}]
\endilg
\glft The people in the village would speak loudly.//
\endgl
\xe

\exdisplay
\begingl
Yandyoc\putfnno[\textsc{3sg:pasthab-mr:}pick.up]%
   \footnotebody{%
   The sequence {\it yandyoc nitni\/} would appear in more general
   speech as:
   \exdisplay[aboveexskip=1ex,belowexskip=0pt]
   \begingl Yem-andyok-i[\textsc{3sg:pasthab-mr:}pick.up-\textsc{const}]
   nitni[child:\textsc{3sg}] \endilg
   \glft She would pick up her child. //\endgl\xe
   }
nitni[child:\textsc{3sg}]
m-ampai[\textsc{sg:es-mr:}take]
me-hac[\textsc{sg:es-}ascend]
m-aglu:[\textsc{sg:es-mr:}say]
\endilg
\glft She would pick up her child and take them up and say. //
\endgl
\xe

\exdisplay
\begingl
Is,[hey]
{ucohpe $(<$u-oc{\ips @}h-pe$\mskip1mu )$\putfnno}%
      [\textsc{2pl:imp-br:}look\textsc{-prec}]%
   \footnotebody{Here again, the construct suffix that is present
   with younger speakers is absent.  This results in the
   underlying schwa being realized as {\it o} rather than zero.
   The sequence {\it uchope nitug\/} would appear in the speech
   of most people as:
   \exdisplay\begingl
   uchipe{$(<$u-oc{\ips @}h-i-per}%
      [\textsc{2pl:imp-br:}look\textsc{-const-prec}]
   nitu-g[child\textsc{-1sg}]
   \endilg \endgl\xe}
nitu-g[child\textsc{-1sg}]
\endilg
\glft Hey, you all look at my child first! //
\endgl
\xe


\endinput


