\def\ExPexMessage
   {6/20/2017 v5.1c ExPex linguistics example formatter (JF)}
% -------------------------------------------------------
% jf 2011/12/09  (v4.0a)
%  1) fixed bug in how exceptional gla items are detected
%  2) added everytrailingcitation parameter
%  3) made \glft a long definition to allow \par
%  4) fixed bug in \printrbrack and \printlbrack, now not math mode
% jf 2011/12/09  (v4.0b)
%  1) allow {} gla item
% jf 2012/03/10  (v4.0c)
%  1) fixed problem with IJAL \deftaglabel caused by nonexpansion
%     of \@actualexno
% jf 2012/12/01  (v4.1)
%  1) added auto glskip mode to glosses
%     /abovemoreglskip/ now obsolete
%  2) fixed bug in \@setinckey so that value is expanded when set
% jf 2013/01/13  (v4.1a)
%  1) added \let\pageno=\c@page to LaTex specializations
% jf 2013/06/27  (v4.1b)
%  1) fixed bug in skip over \glft entries
% jf 2013/07/27
%  1) LaTex \include/\includeonly mechanism now works for \excnt
% jf 2013/10/08
%  1) revise the definition of \glw@printcol (now named
%   \glw@printglword) to use token lists rather than edefs, and to build
%   glwords as \vtop{\halign{#\hfil\cr...} structures
%   rather than stacked hboxes. This allows items like {\hfil
%   word}, which results in center alignment.
%   2) removed w from the prefix in \glw@append, \glw@lop, and
%   \glw@lopff since these will be used in other gloss styles
% jf 2013/10/10 fixed belowpreambleskip bug
% jf 2014/01/27  (v5.0)
%   MAJOR REVISIONS: vertical spacing in glosses, introduction of
%   nlevel style glosses (see User's Manual, also revised at this time)
%   for more detail about the changes
% jf 2014/03/08  (v5.0b)
%   1. restored allowing {} empty items in wrap glosses
%   2. introduced <glbreaking> parameter
% jf 2014/03/14  (v5.0c)
%   1. fixed reference for IJAL style
%   2. fixed initialization for roman number parts in \pex
% jf 2015/01/06  (v5.0d)
%   glftpos=right now works with the nlevel style
%   many internal names were changed to the general form
%     \gl@<style>@<ftpos>@<name> to make the code for switching style and
%     ftpos much more transparent
% jf 2015/09/30  (v5.0e)
%   right panel in nlevel style
% jf 4/11/2016  (v5.0f)
%   fixed \ep@setlist so that it does not make global asssigment
%   changed some names of the append commands to better match their effect
% jf 3/9/2017   (v5.1)
%   fixed bug so that \glbreaking works if extraglskip has stretch and/or
%   shrink components
% jf 5/1/2017   (v5.1a)
%   1. made \sc equivalent to \scshape if the LaTex engine is used
%   2. made \it equivalent to \itshape if the LaTex engine is used
%   3. fixed bug in \getref in the IJAL style
% Vanya Kapitinov and jf 5/13/2017  (v5.1b)
%   define \rm and \tt in LaTex for modern LaTex compatibility
% jf 6/20/2017 (v5.1c)
%   restored \tagfilesuffix
% --------------------------------------------------------
\edef\resetatcatcode{\catcode`\noexpand\@\the\catcode`\@\relax}
\catcode`\@=11\relax
\ifx\XKeyValLoaded\endinput \else
   \input xkeyval \fi
\newif\ifeplatex
\ifx\ProvidesFile\@undefined
      \eplatexfalse
      \message{\ExPexMessage}
   \else
      \ProvidesFile{expex.tex}[\ExPexMessage]
      \eplatextrue
      \@addtofilelist{expex.tex}
      \let\it=\itshape
      \let\sc=\scshape
      \def\rm{\normalfont\rmfamily}   % VK 5/13/2017
      \def\tt{\normalfont\ttfamily}   % VK 5/13/2017
      \let\pageno=\c@page
% 2013/07/27
      \g@addto@macro\cl@@ckpt
         {\immediate\write\@partaux{\string\excnt=\the\excnt}}
   \fi
%! define eplain primitives, if necessary
\ifx\eplain\@undefined  % eplain stuff
   \def\@futurenonspacelet#1{\def\@cs{#1}%
      \afterassignment\@stepone\let\@nexttoken= }%
   \def\@stepone{\expandafter\futurelet\@cs\@steptwo}%
   \def\@steptwo{\expandafter\ifx\@cs\@sptoken\let\@@next=\@stepthree
      \else\let\@@next=\@nexttoken\fi \@@next}%
   \def\@stepthree{\afterassignment\@stepone\let\@@next= }%
   \def\@getoptionalarg#1{%
      \let\@optionaltemp = #1%
      \let\@optionalnext = \relax
      \@futurenonspacelet\@optionalnext\@bracketcheck
   }
   \def\@bracketcheck{%
      \ifx [\@optionalnext
         \expandafter\@@getoptionalarg
      \else
         \let\@optionalarg = \empty
         \expandafter\@optionaltemp
      \fi }
   \def\@@getoptionalarg[#1]{%
      \def\@optionalarg{#1}%
      \@optionaltemp}
\fi
%----- end of eplain inclusions
\def\ep@identity#1{#1}
\def\ep@expandonce{\expandafter\noexpand}
\def\ep@expandtwice{\expandafter\expandafter\expandafter\noexpand}
\def\ep@expandafterafter#1{\expandafter#1\expandafter}
\def\ep@gobble#1{}
\def\@getoptionaltag#1{%
   \let\@@optionaltemp = #1%
   \let\@optionaltag\empty
   \@ifnextcharacter<\@@gettag\@@optionaltemp
}
\def\@@gettag<#1>{\def\@optionaltag{#1}\@@optionaltemp}

\newif\if@tilde
\def\@tildecheck#1{%
   \@ifnextcharacter~%
      {\@tildetrue\expandafter#1\ep@gobble}%
      {\@tildefalse#1}%
}
%------- XKV parametrization ------
\def\define@lingkey{\define@key{ling}}
\def\define@ling@cmdkeys{\define@cmdkeys{ling}[ling@]}
\def\define@lingcmdkeys{\define@cmdkeys{ling}[ling]}
%
\def\define@linginckey#1#2{%
   \define@key{ling}{#2}{%
      \ep@expandafterafter\@setinckey
         #1\csname ling#2\endcsname ##1\@nil\relax}%  test \relax %%%%%%%%%%%%%%%%%%%%%%%
}
\let\XKVforn=\XKV@for@n   % added 2016/04/23 for easy use
\def\define@lingincdimenkeys#1{\XKVforn{#1}\@key{%
   \expandafter\define@linginckey\expandafter\dimen\expandafter{\@key}}}
\def\define@lingincskipkeys#1{\XKVforn{#1}\@key{%
   \expandafter\define@linginckey\expandafter\skip\expandafter{\@key}}}
\def\@setinckey#1#2#3#4\@nil{%
   \ifx#3!%
         #1 0=#2%
         \advance#1 0 by #4%
      \else
         #1 0=#3#4%
      \fi
      \edef#2{\the #10}%
}
%
\def\lingset#1{\setkeys{ling}{#1}\ignorespaces}
% \Lingset first sets ling keys, if there are non-ling keys
%   remaining, these are then passed to \psset
\def\Lingset#1{\setkeys*{ling}{#1}%
   \ifx\XKV@rm\@empty \else
      \ep@expandafterafter\psset{\XKV@rm}\fi
}
\def\Ling@usearg{%
   \ifx\@optionalarg\empty
      \else \ep@expandafterafter\Lingset{\@optionalarg}\fi
}
\def\ling@usearg{\ep@expandafterafter\lingset{\@optionalarg}}
% styles
\def\e@let#1#2{%
   \ep@expandafterafter\let#1\csname #2\endcsname\ignorespaces
}
\define@lingkey{lingstyle}{%
   \e@let\temp{ling@#1style}
      \ep@expandafterafter\Lingset{\temp}}
\def\definelingstyle#1#2{%
   \expandafter\def\csname ling@#1style\endcsname{#2}}
% if PST available, allow \psset to set ling parameters,
% otherwise cancel \Lingset's ability to set PST parameters
\ifx\PSTricksLoaded\endinput
      \pst@addfams{ling}
   \else
      \let\Lingset=\lingset
   \fi
%------ scratch dimensions and counts -----
\newdimen\ep@dima
\newdimen\ep@dimb
\newdimen\ep@dimc
\newcount\ep@cnta
\newcount\epc@b
%------ \ex -----
\newcount\excnt
\excnt=1
\newbox\numbox
\newdimen\epd@numright
\newif\if@specialexno
\define@lingincskipkeys{aboveexskip,belowexskip}
\define@lingincdimenkeys{numoffset,textoffset}
\define@lingcmdkeys{Everyex,everyex,exbreakfil,exbreakpenalty,splitpartspenalty}
\define@lingkey{exskip}%
   {\edef\lingaboveexskip{#1}\edef\lingbelowexskip{#1}}
\def\ep@settosum#1#2#3{#1=#2\relax \advance#1 by#3}
\define@choicekey{ling}{textanchor}%
   [\scratch\ep@textanchor]{numleft,normal}{}
\def\ex{\bgroup \@tildecheck\ex@a}
\def\ex@a{\def\@optionaltag{}\def\@specialexno{}%
   \@getoptionalarg\ex@b}
\def\ex@b{\@getoptionaltag\ex@c}
\def\ex@c{%
   \ex@setup
   \leavevmode
   \setbox\numbox=\hbox{\hskip\lingnumoffset\ep@actualexnoprint}%
   \epd@numright=\wd\numbox
   \ifcase\ep@textanchor                          % numleft
         \ep@settosum\leftskip\lingnumoffset\lingtextoffset
      \or                                         % normal
         \ep@settosum\leftskip\epd@numright\lingtextoffset
      \fi
   \llap{\hbox to\leftskip{\unhbox\numbox \hss}}%
   \lingeveryex
   \latex@tagex
   \ignorespaces
}
\def\actualexno
   {\if@specialexno {\ep@specialexno}\else \the\excnt \fi}
\def\ep@actualexnoprint{{%
   \if@specialexno
      \ep@specialexnoprint
   \else \ifx\ling@sampleexno\empty
      \exnoprint
   \else
      \hbox to \epd@sampleexnowidth{\exnoprint\hss}%
   \fi\fi
}}
\define@choicekey{ling}{exnotype}[\ling@exnotype\@N]%
   {arabic,chapter.arabic,roman}{%
      \ifcase\@N
            \def\ep@rawexnoprint{\number\excnt}%
         \or
            \def\ep@rawexnoprint{\thechapter.\number\excnt}%
         \or
            \def\ep@rawexnoprint{\romannumeral\the\excnt}%
         \fi
}
\lingset{exnotype=arabic}
\def\exnoprint{\ep@exnoformat{\ep@rawexnoprint}}
\def\ep@specialexnoprint{\ep@exnoformat{\ep@specialexno}}  % unformatted
\def\ep@globalstepexcnt{\global\advance\excnt by 1 }
\def\ep@localstepexcnt{\advance\excnt by 1 }
\let\stepexcnt=\ep@globalstepexcnt
\def\keepexcntlocal{\let\stepexcnt=\ep@localstepexcnt}
\def\ex@setup{%             also used by \pex
%% TENTATIVE, insert \par
%   \par
   \global\@specialexnofalse
   \latex@tagex
   \lingEveryex
   \let\reset@refproofing\@printref
   \let\@printref\ep@identity    % turn off refproofing
   \Ling@usearg
   \let\@printref\reset@refproofing
   \if@specialexno
         \def\@actualexno{\ep@specialexno}%
      \else
         \edef\@actualexno{\ep@rawexnoprint}%
      \fi
   \ifx\@optionaltag\empty
         \let\@localextag=\empty
      \else
         \edef\@localextag{\@optionaltag}%
         \deftag{\@actualexno}{\@optionaltag}
      \fi
   \exbreak
   \if@tilde \else \vskip\lingaboveexskip\fi
   \parindent=0pt
}
\def\noexno{\global\advance\excnt by -1}
\def\exbreak{\endgraf\bgroup\@getoptionalarg\exbreak@a}
\def\exbreak@a{%
   \ifx\@optionalarg\empty
         \skip255=\lingexbreakfil
      \else
         \skip255= 0pt plus\@optionalarg
      \fi
   \vskip\skip255
   \penalty\lingexbreakpenalty
   \vskip-\skip255
   \egroup
}
\def\xe{%
   \expandafter\vskip\lingbelowexskip
   \egroup
   \if@specialexno \else \stepexcnt \fi
   \allowbreak
   \prevdepth\dp\strutbox
   \noindent
}
\def\exdisplay{\bgroup\@tildecheck\exdisplay@a}
\def\exdisplay@a{\@getoptionalarg\exdisplay@b}
\def\exdisplay@b{\let\@optionaltag=\empty \ex@setup}
%------ \pex -----
\newcount\pexcnt
\newdimen\epd@labelleft
\newdimen\epd@labelright
\newdimen\epd@textleft
\newdimen\epd@preambleleft
\newif\if@firstlabel
\define@lingincdimenkeys{preambleoffset,labelwidth,labeloffset}
\define@lingincskipkeys{belowpreambleskip,interpartskip}
\define@choicekey{ling}{preambleanchor}
   [\scratch\ep@preambleanchor]{numright,labelleft,text}{}
\define@boolkey{ling}[ling@]{avoidnumlabelclash}[true]{}
\define@ling@cmdkeys{appendtopexarg}
\newdimen\epd@sampleexnowidth
\define@lingkey{sampleexno}{%
   \setbox0=\hbox{#1}%
   \epd@sampleexnowidth=\wd0
   \def\ling@sampleexno{#1}%
}
\lingset{sampleexno=}
%\define@lingcmdkeys{splitexpenalty}
%\define@lingcmdkeys{belowpreambleskip,interpartskip,splitexpenalty}
\define@choicekey{ling}{labelalign}[\ling@labelalign\nr]%
   {left,center,right}{%
      \ifcase\nr
            \def\ep@labelprint{\ep@labelformat\ep@label\hss}%
         \or
            \def\ep@labelprint{\hss \ep@labelformat\ep@label\hss}%
         \or
            \def\ep@labelprint{\hss \ep@labelformat\ep@label}%
         \fi
}
\define@key{ling}{samplelabel}{%
   \setbox0=\hbox{#1}%
   \lingset{labelwidth=\wd0}%
}
\define@boolkey{ling}[ling@]{nopreamble}[true]{}
\def\pex{\bgroup\@tildecheck\pex@a}
%\def\pexns{\bgroup \@tildetrue\pex@a}
\def\pex@a{\def\@optionaltag{}\def\@specialexno{}%
   \@getoptionalarg\pex@b}
\def\pex@b{%
   \ifx\ling@appendtopexarg\empty \else
      \XKV@addtolist@o\@optionalarg{\ling@appendtopexarg}\fi
   \@getoptionaltag\pex@c}
\def\pex@c{\ling@nopreambletrue
   \@futurenonspacelet\temp\pex@d}
\def\pex@d{%
   \ifx\temp\a \let\nextpex@\pex@e
      \else \ifx\temp\label \let\nextpex@\pex@f
      \else \ling@nopreamblefalse \let\nextpex@\pex@e
      \fi\fi
   \ex@setup
   \nextpex@
}
\def\pex@f#1#2{\label{#2}\@futurenonspacelet\temp\pex@g}
\def\pex@g{\ifx\temp\a \let\next\pex@h
   \else \let\next\pex@e \ling@nopreamblefalse \fi \next}
\def\pex@h#1\a{\pex@e\a}
\def\pex@e{\pex@i \lingeveryex }
\def\pex@i{%
   \setbox\numbox=\hbox{\hskip\lingnumoffset\ep@actualexnoprint}%
   \ep@setdimensions
   \ep@pexcntinit
   \@firstlabeltrue
   \let\a\ep@putlabel
   \ifling@nopreamble
         \leftskip=\epd@textleft
      \else
         \ep@setdimpreambleleft
         \leftskip=\epd@preambleleft
      \fi
   \def\next{\llap{\hbox to\leftskip{\unhbox\numbox \hss}}}%
   \ifling@nopreamble
         \ifling@avoidnumlabelclash \let\next\relax \fi\fi
   \leavevmode
   \next
}
\def\ep@setdimensions{% \epd@numright=\wd\numbox \epd@labelleft=\linglabeloffset
   \epd@numright=\wd\numbox
   \epd@labelleft=\linglabeloffset
   \advance\epd@labelleft by
      \ifcase\ep@labelanchor
         \epd@numright
      \or
         \lingnumoffset
      \or
         0pt
      \fi
   \ep@settosum\epd@labelright\epd@labelleft\linglabelwidth
   \ifcase\ep@textanchor   % numleft
         \ep@settosum\epd@textleft\lingnumoffset\lingtextoffset
      \or                 % normal
         \ep@settosum\epd@textleft\epd@labelright\lingtextoffset
      \fi
}
\def\ep@setdimensions{%
   \epd@numright=\wd\numbox
   \epd@labelleft=\linglabeloffset
   \advance\epd@labelleft by
      \ifcase\ep@labelanchor
            \epd@numright
         \or
            \lingnumoffset
         \or
            0pt
         \fi
   \ep@settosum\epd@labelright\epd@labelleft\linglabelwidth
   \ifcase\ep@textanchor   % numleft
         \ep@settosum\epd@textleft\lingnumoffset\lingtextoffset
      \or                 % normal
         \ep@settosum\epd@textleft\epd@labelright\lingtextoffset
      \fi
}
%\def\lingnumrightoffset{\the\epd@numright}%
\def\ep@setdimpreambleleft{%
   \epd@preambleleft=\lingpreambleoffset
   \advance\epd@preambleleft by \ifcase\ep@preambleanchor
      \epd@numright\or \epd@labelleft\or \epd@textleft \fi
}
\def\ep@pexcntinit{\ifnum\ep@labelgen=2\else
   \pexcnt=\ling@pexcnt \fi}
\define@key[epx@]{labels}{tag}{\def\@optionaltag{#1}}
\define@key[epx@]{labels}{label}{\def\@specialexno{#1}}
\def\ep@setlabelkeys{\setkeys[epx@]{labels}}
\def\ep@useoptionallabelarg{%
   \expandafter\ep@setlabelkeys\expandafter{\@optionalarg}}
\define@lingkey{tag}{\def\@optionaltag{#1}}
\newtoks\ep@everylabel  % \ep@everylabel is a token list
\define@lingkey{everylabel}{\ep@everylabel{#1}}
%
\def\ep@putlabel{%
   \if@firstlabel
         \ifling@nopreamble \else
            \vskip\lingbelowpreambleskip
            \leftskip=\epd@textleft
            \fi
         \@firstlabelfalse
      \else
         \par\penalty\lingsplitpartspenalty
         \vskip\linginterpartskip
      \fi
   \def\@specialexno{}\def\@optionaltag{}%
   \@getoptionalarg\ep@putlabel@a
}
% 03/14/2014 changed defs to edefs below
\def\ep@putlabel@a{%
   \ep@useoptionallabelarg
   \ifx\@specialexno\empty
         \ifcase\ep@labelgen
            \edef\ep@label{\the\ep@everylabel \char\the\pexcnt}%
%            \edef\ep@label{\the\ep@everylabel \char\noexpand\the\pexcnt}%
            \advance\pexcnt by 1
         \or
            \edef\ep@label{\the\ep@everylabel \number\pexcnt}%
%            \edef\ep@label{\the\ep@everylabel \number\noexpand\pexcnt}%
            \advance\pexcnt by 1
         \or
            \ep@popLL
         \or
            \edef\ep@label{\the\ep@everylabel \romannumeral\noexpand\pexcnt}%
            \advance\pexcnt by 1
         \fi
      \else
         \def\ep@label{\the\ep@everylabel\@specialexno}%
      \fi
   \xdef\resumepexcnt{\noexpand\pexcnt\the\pexcnt}%
   \@getoptionaltag
   \ep@putlabel@b
}
\def\ep@putlabel@b{%
   \ifx\@optionaltag\empty \else
      \deftaglabel{\@optionaltag}%
      \fi
   \leavevmode
   \llap{\hbox to\leftskip{\hskip\epd@labelleft
      \hbox to\linglabelwidth{\ep@labelprint}%
   \hfil}}%
   \latex@tagexlabel
   \ignorespaces
}
%
\define@choicekey{ling}{labelanchor}[\scratch\ep@labelanchor]%
   {numright,numleft,margin}[]{}
\define@lingkey{pexcnt}{\edef\ling@pexcnt{#1}}
% IJAL style
\definelingstyle{IJAL}{labelwidth=2em,labelanchor=numleft,
   labeloffset=0pt,labelformat=(A),everylabel=\actualexno,
   textanchor=normal,textoffset=1em,preambleanchor=text,
   preambleoffset=0pt,avoidnumlabelclash,
   appendtopexarg={samplelabel=(\actualexno a)}}
%-----------------------------------------
%----- judgments -----
\def\judge#1{\rm #1\kern .1em \ignorespaces}
\def\ljudge#1{\llap{\judge{#1}}\ignorespaces}
\define@key{ling}{*}[*]%
   {\setbox0=\hbox{#1}%
   \lingset{textoffset=!\wd0}%
}
%------ table support -----
\define@lingcmdkeys{dima,dimb,dimc}
\lingset{dima=2.4em}
\def\tspace{\@getoptionalarg\ep@tabelspace}
\def\ep@tabelspace{\hskip
   \ifx\@optionalarg\empty
         \lingdima
      \else
         \csname ling\@optionalarg\endcsname
      \fi
}
\def\labels{\@getoptionalarg\ep@labels}
\def\ep@labels{%
   \ifcase\ep@labelgen
         \def\ep@label{\the\ep@everylabel \char\the\pexcnt}%
      \or
         \def\ep@label{\the\ep@everylabel \number\pexcnt}%
      \or
      \or
         \def\ep@label{\the\ep@everylabel \romannumeral\pexcnt}%
      \fi
   \ling@usearg
   \dimen0=\lingtextoffset
   \advance\dimen0 by \linglabelwidth
   \edef\ling@labelskip{\the\dimen0}%
   \ep@pexcntinit
   \let\tl\ep@inserttabellabel
   \let\nl\ep@omitlabel
   \ignorespaces
}
\def\ep@inserttabellabel{\@getoptionaltag\ep@inserttablelabel@a}
\def\ep@inserttablelabel@a{%
%   \global\advance\pexcnt by 1
   \ifx\@optionaltag\empty \else
      \deftaglabel{\@optionaltag}%
      \fi
   \edef\foop{\ep@label.}\foop
   \global\advance\pexcnt by 1
}
\def\ep@omitlabel{\omit\hskip\linglabeloffset\hfil}
%\def\endpextable{\egroup\egroup \par \prevdepth=\dp\strutbox}
\def\hwit#1{\hidewidth \it #1\hidewidth}
\define@lingcmdkeys{crskip}
\lingset{crskip=.6em}
\def\crs{\cr\noalign{\vskip\lingcrskip}}
\def\crnb{\cr\noalign{\par\nobreak}}
% LL is "label list"
\define@lingkey{labellist}{%
   \edef\ling@LL{#1,}%
   \edef\@currLL{#1,}%  current LL
}
\def\ep@popLL{%
   \ifx\@currLL\empty
      \@expexwarn{Not enough labels in labellist}%
      \let\@currLL=\ling@LL  % start over
      \ep@popLL
   \else
      \expandafter\ep@popLL@a\@currLL\@nil
   \fi
}
\def\ep@popLL@a#1,#2\@nil{%
   \def\ep@label{\the\ep@everylabel #1}\def\@currLL{#2}}
\define@choicekey{ling}{labelgen}[\ling@labelgen\ep@labelgen]%
   {char,number,list,romannumber}{}
\def\definelabeltype#1#2{%
   \expandafter\def\csname ling@#1labeltype\endcsname{#2}}
\define@lingkey{labeltype}{%
   \e@let\temp{ling@#1labeltype}%
   \ep@expandafterafter\Lingset{\temp}}
\define@lingkey{labelformat}{\ep@omitlabelformat #1\@nil}
\def\ep@omitlabelformat #1A#2\@nil{%
   \def\ep@labelformat##1{#1{##1}#2}}
\define@lingkey{exnoformat}{\ep@mkexnoformater #1\@nil}
\def\ep@mkexnoformater #1X#2\@nil{%
   \def\ep@exnoformat##1{#1{##1}#2}}
\lingset{exnoformat=(X)}
\define@lingkey{fullrefformat}{\@fullrefformat #1\@nil}
\def\@fullrefformat #1X#2A#3\@nil{%
   \def\ep@fullrefformat##1##2{#1##1#2##2#3}}
%------ support for LaTex \label macro -----
\let\latex@tagex\relax
\let\latex@tagexlabel\relax
\ifx\label\relax \else    % else = LaTex is loaded
   \def\latex@tagexlabel{\def\@currentlabel
      {\ep@fullrefformat{{\the\excnt}}{\ep@label}}}%
   \def\latex@tagex{\edef\@currentlabel{\the\excnt}}%
   \fi
%-----------------------------------------
\definelabeltype{alpha}{labelgen=char,pexcnt=`a,labelformat=A.,
   fullrefformat=XA,labelalign=left,labelwidth=.72em}
\definelabeltype{caps}{labelgen=char,pexcnt=`A,labelformat=A.,
   fullrefformat=XA,labelalign=left,labelwidth=.92em}
\definelabeltype{numeric}{labelgen=number,pexcnt=1,labelformat=A.,
   fullrefformat=X.A,labelalign=right,labelwidth=.75em}
% 03/14/14 bug fix (wrong start to roman series labels)
\definelabeltype{roman}{labelgen=romannumber,pexcnt=2,labelformat=(A),
%\definelabeltype{roman}{labelgen=romannumber,pexcnt=1,labelformat=(A),
   fullrefformat=XA,labelalign=left,labelwidth=1.5em}
%-- tags and reference -----
%----- local reference to example numbers -----
%\def\nextx{{\@printref{\number\excnt}}}
%\def\anextx{{\@printref{\advance\excnt by 1 \number\excnt}}}
%\def\lastx{{\@printref{\advance\excnt by -1 \number\excnt}}}
%\def\blastx{{\@printref{\advance\excnt by -2 \number\excnt}}}
%\def\bblastx{{\@printref{\advance\excnt by -3 \number\excnt}}}
\def\nextx{{\@printref{\ep@rawexnoprint}}}
\def\anextx{{\@printref{\advance\excnt by 1 \ep@rawexnoprint}}}
\def\lastx{{\@printref{\advance\excnt by -1 \ep@rawexnoprint}}}
\def\blastx{{\@printref{\advance\excnt by -2 \ep@rawexnoprint}}}
\def\bblastx{{\@printref{\advance\excnt by -3 \ep@rawexnoprint}}}
%------ defining tags -----
\def\deftag#1#2{%
   {\let\@printref=\ep@identity
   \expandafter\xdef\csname lingtag@#2\endcsname{#1}%
   \if@g@thertags
      \immediate\write@tags{\noexpand\@fd@f {#2} {{#1}} }%
      \fi}%
   \ignorespaces
}
\def\deftaglabel#1{%
   \expandafter\xdef\csname lingtag@\@localextag.#1\endcsname%
      {{{\ep@label}}%
       {{\ep@fullrefformat{\@actualexno}\ep@label}}%
      }%
   \if@g@thertags
      \immediate\write@tags{%
         \noexpand\@fd@f
         {\@localextag.#1}
         {{{\ep@label}}%
          {{\ep@fullrefformat{\@actualexno}\ep@label}}}%
         }%
      \fi
   \ignorespaces
}
\def\deftagex#1{\edef\@localextag{#1}%
   \expandafter\xdef\csname lingtag@#1\endcsname{{\ep@rawexnoprint}}%
   \if@g@thertags
      \immediate\write@tags{\noexpand\@fd@f {#1} {{\ep@rawexnoprint}}}%
      \fi
   \ignorespaces
}
\def\deftagpage#1{%
   \if@g@thertags
      \write@tags{\noexpand\@fd@f #1 {{\the\pageno}}}%
      \fi
   \ignorespaces
}
\def\lastlabel{{\ep@label}}
\def\@expexwarn#1{\immediate\write16{====> EXPEX WARNING: #1.}}
\def\@expexerror#1{\immediate\write16{====> Fatal EXPEX ERROR: #1.}}
\newif\ifep@highlightref
\ep@highlightreffalse
\def\refproofing{\ep@highlightreftrue}
\def\mathhigh@lightref#1{$\overline{\underline{\hbox{#1}}}$}
\def\psthigh@lightref{\psframebox[boxsep=false,framesep=2pt,linewidth=.2ex]}
\ifx\PSTricksLoaded\endinput
      \let\@highlightprint\psthigh@lightref
   \else
      \let\@highlightprint\mathhigh@lightref
   \fi
\def\@printref#1{%
   \ifep@highlightref \@highlightprint{#1}\else #1\fi}
%%%%
\newbox\exnobox
\define@key{ling}{exno}{%
   \global\@specialexnotrue
   \let\latex@tagexlabel\ep@gobble
   \let\latex@tagex\ep@gobble
   \setbox\exnobox=\hbox{#1}%
   \def\ep@specialexno{\unhcopy\exnobox}%
}
%------ opening the tag file -----
\newif\if@g@thertags
\@g@thertagsfalse
\def\tagfilesuffix#1{\edef\@tagfilename{\jobname#1}}
\tagfilesuffix{-tag}
\newwrite\ling@tagsout
\newread\ling@tagsin
\def\write@tags{\write\ling@tagsout}
\def\gathertags{%
   \@setupreadtags
   \@g@thertagstrue
   \immediate\openout\ling@tagsout=\@tagfilename\relax
   \immediate\write@tags{\noexpand\relax}%
}
%------ reading the tag file and defining the tags it encodes -----
\gdef\@fd@f#1 #2 {%
   \expandafter\ifx\csname lingtag@#1\endcsname\relax
      \expandafter\gdef\csname lingtag@#1\endcsname{#2}%
      \fi
}
\newif\if@readtags
\@readtagstrue
\def\@setupreadtags{\if@readtags
   \do@readtags \global\@readtagsfalse \fi}
\def\do@readtags{%
   \immediate\openin\ling@tagsin=\@tagfilename\relax
   \ifeof\ling@tagsin \else
      \closein\ling@tagsin
      {\catcode`@=11 \input \@tagfilename\relax}%
   \fi
}
%!
%!------ tagging sections, adapt to your needs -----
%! If \tagsec is used with section macros that do not define
%! counters \secno,\subsecno,\subsubsecno, and \subsubsubsecno,
%! then \currsec must be redefined to whatever is appropriate.
%!\def\chapscurrsec{\ifnum\chapno>0 \the\chapno
%!   \ifnum\secno>0 .\the\secno
%!   \ifnum\subsecno>0 .\the\subsecno
%!   \ifnum\subsubsecno>0 .\the\subsubsecno \fi\fi\fi\fi}
%!\def\nochapscurrsec{\ifnum\secno>0 .\the\secno
%!   \ifnum\subsecno>0 .\the\subsecno
%!   \ifnum\subsubsecno>0 .\the\subsubsecno \fi\fi\fi}
%! choose one of the following twos
%!\let\currsec\nochapscurrsec
%!\let\currsec\chapscurrsec
%!\def\deftagsec#1{\deftag\currsec{#1}}
%!/
%\def\deftaglabel#1{%
%   \expandafter\xdef\csname lingtag@\@localextag.#1\endcsname
%      {%
%      {\ep@expandonce\ep@label}%
%      {\ep@fullrefformat{\@actualexno}\ep@expandonce\ep@label}%
%      }%
%   \ignorespaces
%}
\def\getref@aa#1#2{#1}%
\def\getref@ab#1#2{#2}%
\def\getref#1{\getref@a{#1}\getref@aa}
%\def\getfullref#1{\getref@a{#1}\getref@ab}
%\def\getref@a#1#2{%
%   \if@readtags \@setupreadtags \fi
%   \expandafter \ifcsname lingtag@#1\endcsname
%         \edef\temp{\ep@expandtwice\csname lingtag@#1\endcsname}%
%         \ifx\temp\empty
%               \@expexwarn{+++tag #1 has no full reference}%
%               \@printref{Missing!}%
%            \else
%               {\@printref{\temp}}%
%            \fi
%      \else
%         \@expexwarn{tag #1 is called but not defined}%
%         {\@printref{\tt [#1]}}%
%      \fi
%}
\newif\ifpartlabel
\newif\iffullref
\def\ep@ispartlabelcheck#1{\ep@ispart@a#1.\@nil}
\def\ep@ispart@a#1.#2\@nil{\def\temp{#2}%
   \ifx\temp\empty \partlabelfalse \else \partlabeltrue\fi}
\def\getref{\fullreffalse \getref@a}
\def\getfullref{\fullreftrue \getref@a}
\def\getref@a#1{%
   \if@readtags \@setupreadtags \fi
   \ep@ispartlabelcheck{#1}%
   \ifpartlabel
         \iffullref
               \let\@chooseref\chooseref@a
            \else
               \let\@chooseref\chooseref@g
            \fi
      \else
         \let\@chooseref\relax
      \fi
   \expandafter\ifx\csname lingtag@#1\endcsname \relax
         \@expexwarn{tag #1 is called but not defined}%
         {\@printref{\tt [#1]}}%
      \else
         \expandafter\let\expandafter\temp
            \csname lingtag@#1\endcsname
         \@printref{\expandafter\@chooseref\temp}%
      \fi
}
\def\chooseref@a#1#2{#2}
\def\chooseref@g#1#2{#1}
% ----- glosses -----
\def\ling@glstyle{wrap}
\def\ling@glftpos{below}
\define@choicekey{ling}{glstyle}[\ling@glstyle\gl@style@num]{wrap,nlevel}{%
   \ifnum\gl@style@num=0 \glw@assignlevels \fi
   \gl@setprefix
}
\def\begingl{\bgroup\@getoptionalarg\gl@begingl}
% bug fix 2105/10/27 eliminate stray space
%\def\gl@begingl{%
%   \ling@usearg
%   \ling@everygl
%   \gl@beginglstyle@a
%}
\def\gl@begingl{%
   \let\everylist=\empty
   \ling@usearg
   \ling@everygl
   \gl@beginglstyle@a
}
\def\gl@beginglstyle@a{\@ifnextchar\@space\gl@beginglstyle@b\gl@beginglstyle}
\def\gl@beginglstyle@b #1{\gl@beginglstyle}
% end bug fix
% parameters which are used in both gloss styles
\define@ling@cmdkeys{everygl,everyglpreamble,everyglilg, everyglft,
   everyglword,glrightskip,glhangindent,glwidth}
\define@lingincskipkeys{glspace,aboveglftskip,belowglpreambleskip,extraglskip}
\newbox\glstrutbox
\def\glstrut{\unhcopy\glstrutbox}
\define@boolkey{ling}[ling@]{glstruts}{}%
\newdimen\ep@cascadeindent
\newdimen\ep@hangindentamount
\newcount\ep@cascadecount
\define@choicekey{ling}{glhangstyle}[\temp\ep@glhangstyle]%
   {none,normal,cascade}%
   {\ifnum\ep@glhangstyle=2
      \ep@cascadecount=10
      \ep@cascadeindent=0pt
      \ep@hangindentamount=\ling@glhangindent
    \fi }
\lingset{glhangstyle=normal,glrightskip=0pt plus .1\hsize}
% dimensions
% switches
\newif\if@glpreamble
\newif\if@glft
%
% list manipulation macros (ala Knuth) that are used in both gloss styles
\newtoks\gltoks@a
\newtoks\gltoks@b
\def\gl@push #1\to #2{%
   \gltoks@a={\\{#1}}%
   \gltoks@b=\expandafter{#2}%
   \edef#2{\the\gltoks@a\the\gltoks@b}%
}
\def\gl@xappend #1\to #2{%  renamed from \gl@append
   \gltoks@a={\\{#1}}%
   \gltoks@b=\expandafter{#2}%
   \xdef#2{\the\gltoks@b\the\gltoks@a}%
}
\def\gl@lop#1\to#2{\ifx#1\empty
   \let#2\empty \else\expandafter\gl@lopoff#1\gl@lopoff#1#2\fi}
\long\def\gl@lopoff\\#1#2\gl@lopoff#3#4{\def#4{#1}\def#3{#2}}
\def\gl@exappend{\expandafter\gl@xappend}   % renamed from \gl@eappend
\newif\ifgl@loopmore
\def\gl@lopTL#1\to#2{%
   \ifx#1\empty #2={}\else
      \expandafter\gl@lopoffTL#1\gl@lopoffTL#1#2\fi}
\long\def\gl@lopoffTL\\#1#2\gl@lopoffTL#3#4{#4={#1}\def#3{#2}}
%%%%%%%%%%%%%%%%%%%% wrap style glosses %%%%%%%%%%%%%%%%%%%%
% \gl@wrap@below@begin sets up the environment
% \gla, \glb, etc. fill in the various lists which are initialized in
% the environment
% The input (something like the following)
% \gla x1 x2 x3 //
% \glb y1 y2 y3 //
% \glc z1 z2 z3 //
% gets converted into a list of lists (using Knuth's list macros)
%    {{x1,y1,z1},{x2,y2,z2},{x3,y3,z3}}
% auxiliary lists are also made at the same time
%    list of struts, list of "everygl<name>"
% Then these lists are popped, one position at a time, and vboxes
% are built.  The vboxes are fed into Tex's regular paragraph
% building machinary.
%%%%%%%%%%%%%%%%%%%%%%%%%%%%
% introduced "glbreaking" boolean parameter
\define@boolkey{ling}[ling@]{glbreaking}[true]{}
\lingset{glbreaking=false}
\def\ep@setlist#1#2{\def#1{}\XKVforn{#2}\@this{\gl@eappend\@this\to#1}}
% 2016/04/18 support for LaTex color and xcolor packages
\define@ling@cmdkeys{glacolor}
\define@lingkey{glnlinecolors}{\ep@setlist\colorlist{#1}}
\lingset{glnlinecolors={},glacolor=}
% various modifications of gloss building macros which follow were made
%    to make use of these parameters
\def\gl@wrap@below@begin{\gl@wrap@below@begin@a}
\def\gl@wrap@below@begin@a{%
   \bgroup
   \parindent0pt
   \ep@setglstrut
   \ifdim\ling@glwidth=0pt
      \ifling@glbreaking
         \bgroup
       \else
         \vtop\bgroup
            \advance\hsize by -\leftskip
            \advance\hsize by -\rightskip
            \leftskip=0pt
       \fi
   \else
      \vtop\bgroup
         \hsize=\ling@glwidth
         \leftskip=0pt
   \fi
   \leavevmode
   \bgroup
   \parindent=0pt
   \@glpreamblefalse
   \@glftfalse
   \let\mainlist=\empty
   \let\itemtypelist=\empty
   \let\aboveskiplist=\empty
   \let\strutlist=\empty
   \let\everylist=\empty
   \let\colorlist=\empty
}
% As the list of lists is assembled, heights and depths of each
% item are computed and maximum row heights and maximum row depths
% are computed in each row.  These are used to make struts which
% are used to assemble the vboxes.
%
\newdimen\glw@maxht
\newdimen\glw@maxdp
\def\glw@initializemaxhtdp{%
   \ifling@glstruts
      \glw@maxht=\ht\glstrutbox \glw@maxdp=\dp\glstrutbox
   \else
      \glw@maxht=0pt \glw@maxdp=0pt
   \fi
}
\define@ling@cmdkeys{everygla,everyglb,everyglc}
\def\@alinelabel{a}
\def\glw@gla{%
   \bgroup
   \ling@everyglilg
   \def\lingaboveglaskip{0pt}%
   \def\gl@linelabel{a}%
   \glw@initializemaxhtdp
   \@getoptionalarg
   \glw@gla@a
}
\def\glw@gla@a #1//{\bgroup\ling@usearg
   \glw@gla@b #1 \@nil }
\def\glw@gla@b{\@ifnextchar\@nil\glw@gla@c\glw@gla@d}
\def\glw@gla@c#1{\glw@updatelists \egroup \ignorespaces}
\def\ep@samecharcode#1#2{\ep@expandonce#1\ep@expandonce#2\ignorespaces}
\newcount\@itemtype
\newif\ifglw@word
% \@setitemtype sets the count \@itemtype, the macro \@mainlistappend, and
% the switch \glw@word
% 1=+, 2=@, 3=[, 4=], 5=\nogloss, 0=other
\def\@setitemtype #1#2\@nil{%
   \glw@wordfalse
   \def\temp{#2}%
   \ifx\temp\empty
      \if\ep@samecharcode #1+\@itemtype=1
      \else\if\ep@samecharcode #1@\@itemtype=2
      \else\if\ep@samecharcode #1[\@itemtype=3
      \else\if\ep@samecharcode #1]\@itemtype=4
      \else \@itemtype=0
      \fi\fi\fi\fi
   \else
      \glw@wordtrue
      \ifx#1\nogloss\@itemtype=5 \def\@itembody{#2}%
      \else \@itemtype=0 \fi
   \fi
}
\def\glw@gla@d #1 {%
   \def\temp{#1}%
   \ifx\temp\empty
% action for empty words added 2014/03/08
      \glw@wordfalse
      \@itemtype=0
      \gl@xappend \\{\glstrut}\to\mainlist
      \gl@exappend 0\to\itemtypelist
   \else
      \@setitemtype #1\@nil
      \ifnum\@itemtype=0 \gl@xappend \\{\glstrut #1}\to\mainlist
      \else \ifnum\@itemtype=5 \gl@exappend\@itembody\to\mainlist
      \else \gl@xappend \\{}\to\mainlist
      \fi\fi
      \gl@exappend \the\@itemtype \to\itemtypelist
   \fi
   \ifglw@word
      \setbox0=\hbox{\ling@everygla #1}%
      \ifdim\glw@maxht<\ht0 \glw@maxht=\ht0 \fi
      \ifdim\glw@maxdp<\dp0 \glw@maxdp=\dp0 \fi
   \fi
   \glw@gla@b
}
\newdimen\gl@maxdplast
\def\glw@updatelists{%
   \edef\temp{\csname ling@gl\gl@linelabel color\endcsname}%
   \gl@exappend\temp \to\colorlist
   \edef\temp{\csname lingabovegl\gl@linelabel skip\endcsname}%
   \gl@exappend\temp \to\aboveskiplist
   \edef\temp{\vrule width0pt height\the\glw@maxht\space depth\the\glw@maxdp\space }%
   \global\gl@maxdplast=\glw@maxdp
   \gl@exappend\temp\to\strutlist
   \expandafter\expandafter\expandafter
      \gl@xappend\csname ling@everygl\gl@linelabel\endcsname
      \to\everylist
   \gltoks@a=\expandafter{\everylist}%
}
\def\glw@assignlevels{%
   \expandafter\XKVforn\expandafter{\glw@levels}\levelname
   {\glw@assign@level\levelname}%
}
\def\glw@assign@level#1{%
   \edef\Temp{#1}\def\XTemp{ft}%
   \ifx\Temp\XTemp \let\temp\relax \else
      \edef\temp{\noexpand\let
         \expandafter\noexpand\csname gl#1\endcsname
         \expandafter\noexpand\csname glw@gl#1\endcsname}\fi
   \temp
}
% \glw@gla and \gl@wrap@below@ft get their definitions directly
% \glw@gl<name> gets defined via \defineglwlevels, which adds <name> to the
% list \glw@levels of defined glw levels
\def\glw@levels{a,ft}
\def\define@glw@level#1{%
   \expandafter\ifx\csname glw@gl#1\endcsname\relax
      \define@ling@cmdkeys{gl#1color}%
      \define@ling@cmdkeys{everygl#1}%
      \define@linginckey\skip{abovegl#1skip}%
      \XKV@addtolist@o\glw@levels{#1}%
      \lingset{everygl#1=,abovegl#1skip=0pt,gl#1color=}
      \expandafter\def\csname glw@gl#1\endcsname{\glw@glx{#1}}%
      \ifx\gl@style@num\undefined \else
         \ifnum\gl@style@num=0 \glw@assign@level{#1}\fi\fi
   \else
      \@expexwarn{Level #1 is already defined}
   \fi
}
\def\defineglwlevels#1{\XKVforn{#1}\thislevel
   {\expandafter\define@glw@level\expandafter{\thislevel}}%
}
\def\glw@glx#1{%   x suggests any label (as argument to \glw@glx)
   \def\worklist{}%
   \glw@initializemaxhtdp
   \def\gl@linelabel{#1}%
   \@getoptionalarg\glw@glx@a
}

\def\glw@glx@a #1// {\bgroup
   \ling@usearg
   \expandafter\let\expandafter\@every
      \csname ling@everygl\gl@linelabel\endcsname
   \glw@glx@b #1 \@nil }
\def\glw@glx@b{\@ifnextchar\@nil\glw@glx@c\glw@glx@d}
\def\glw@glx@c#1{\glw@updatelists \glw@mergerow\worklist\to\mainlist \egroup}
\def\glw@mergerow#1\to #2{%
   \let\itlist=\itemtypelist
   \let\alist=#1%
   \let\blist=#2%
   \let\clist=\empty
   \gl@loopmoretrue
   \loop
      \ifx\itlist\empty \gl@loopmorefalse \fi
      \ifgl@loopmore
         \gl@lop\itlist\to\@@itlist
         \@itemtype=\@@itlist
         \gl@lop\blist\to\currb
         \ifnum\@itemtype=0
            \gl@lop\alist\to\curra
            \gl@exappend\curra\to\currb
            \gl@exappend\currb\to\clist
         \else \ifnum\@itemtype=5
            \gl@exappend\currb\to\clist
         \else
            \gl@xappend\\{}\to\clist
         \fi\fi
   \repeat
   \global\let#2=\clist
}
\def\glw@glx@d #1 {%
   \setbox0=\hbox{\@every #1}%
   \ifdim\glw@maxht<\ht0 \glw@maxht=\ht0 \fi
   \ifdim\glw@maxdp<\dp0 \glw@maxdp=\dp0 \fi
   \gl@xappend {\glstrut #1}\to\worklist
   \glw@glx@b
}
\defineglwlevels{b,c}
\newif\ifglw@spacebefore
\long\def\gl@wrap@below@preamble #1// {%
   \@glpreambletrue
   {\hangindent=0pt \hangafter=0
   \leavevmode\ling@everyglpreamble #1\strut
   \par
   \vskip\lingbelowglpreambleskip }%
}
\def\gl@wrap@below@ft{\@getoptionalarg\gl@wrap@below@ft@a}
\long\def\gl@wrap@below@ft@a #1//{%
   \glw@printilgsetup
   \glw@printilg@a
   \vskip\lingaboveglftskip
   \nointerlineskip
   \egroup
   \@glfttrue
   \ifx\glstrut\strut
      \ifdim\gl@maxdplast>\dp\strutbox \prevdepth=\gl@maxdplast \fi
      \fi
   \ling@usearg
   \@ilgborderadjustment
   \ling@everyglft
   \strut #1\par
}
\def\gl@wrap@below@end{%
   \if@glft \else \glw@printilg\egroup \fi
   \egroup\egroup\egroup
}
\def\glw@printilg{\glw@printilgsetup \glw@printilg@a}
\def\glw@printilgsetup{%
   \@glspacefalse
   \rightskip=\ling@glrightskip
   \lineskiplimit=0pt
   \lineskip=\lingextraglskip
% bug fix, 3/9/2017
%   \ifdim\lingextraglskip=0pt \global\let\@ilgborderadjustment=\relax
   \ifdim\lineskip=0pt \global\let\@ilgborderadjustment=\relax
      \else \xdef\@ilgborderadjustment{\vskip\the\lineskip}\fi
   \if@glpreamble \@ilgborderadjustment  \fi
   \ifcase\ep@glhangstyle
      \or
         \hangindent=\ling@glhangindent
         \hangafter=1
      \or
         \ep@glmkcascade
   \fi
}

\def\glw@closegroups{\egroup\egroup\egroup}
\def\glw@printilg@a{\glw@printilg@b \par }
\def\glw@printilg@b{%
   \glw@spacebeforefalse
   \@glpostbrackfalse
   \leavevmode
   \gl@loopmoretrue
   \@glaparsestate=1
   \loop\ifgl@loopmore
      \gl@lop\mainlist\to\@currentitem\relax       % \@currentitem is one column
      \gl@lop\itemtypelist\to\@currentitemtype\relax
      \ifcase\@currentitemtype
         \glw@printilgspace
         \glw@printglword
      \or
         \vskip\lingextraglskip
         \ifcase\ep@glhangstyle
         \or
            \hangafter=0
            \hangindent=\ling@glhangindent
         \or
            \glhangcarry
            \leavevmode
         \fi
         \@glaparsestate=1
         \leavevmode
      \or
         \@glaparsestate=1
      \or
         \ifnum\@glaparsestate=0 \hskip\lingglspace
         \else\ifnum\@glaparsestate=2 \hskip\lingglbrackbracksep
         \else\ifnum\@glaparsestate=3 \hskip\lingglspace
         \fi\fi\fi
         \printlbrack\nobreak
         \@glaparsestate=2
      \or
         \nobreak
         \ifnum\@glaparsestate=0 \hskip\lingglbrackwordsep
         \else\ifnum\@glaparsestate=2 \hskip\lingglspace
         \else\ifnum\@glaparsestate=3 \hskip\lingglbrackbracksep
         \fi\fi\fi
         \printrbrack
         \@glaparsestate=3
      \or
         \ifnum\@glaparsestate=0 \hskip\lingglspace \fi
         \@glaparsestate=0
         {\@currentitem}% grouping to prevent font change contagion
         \leavevmode
      \fi
      \ifx\itemtypelist\empty \gl@loopmorefalse \fi
      \repeat
   \par \egroup
}
\def\glw@printilgspace{%
   \ifcase\@glaparsestate
      \hskip\lingglspace
   \or
   \or \hskip\lingglbrackwordsep
   \or \hskip\lingglspace
   \fi
   \@glaparsestate=0
}
\def\glw@print@i{%
   \glw@printilgspace
   \glw@printglword
}
\newtoks\gltoks@sofar
\newtoks\gltoks@every
\newtoks\gltoks@next
\newbox\debug@glword
\def\ep@mklinecolorop{%
   \ifx\@linecolor\empty \let\@linecolorop=\empty
   \else \ep@mklinecolorop@a  \fi}
\def\ep@mklinecolorop@a{%
   \edef\@linecolorop{\noexpand\noexpand\noexpand\color{\@linecolor}}%
}
\def\pstglcolors{%
   \def\ep@mklinecolorop@a{%
      \edef\@@linecolor{\noexpand\csname\@linecolor\noexpand\endcsname}%
      \edef\@@@linecolor{\ep@expandtwice\@@linecolor}%
      \edef\@linecolorop{\expandafter\noexpand
         \expandafter\noexpand\expandafter\noexpand\@@@linecolor}%
      }%
}
\def\glw@printglword{%
   \begingroup
   \gltoks@sofar={}%
   \gl@loopmoretrue
   \loop\ifgl@loopmore
      \gl@lop\colorlist\to\@linecolor
      \ep@mklinecolorop
      \gl@lop\aboveskiplist\to\@aboveskip
      \expandafter\ifdim\@aboveskip=0pt \def\@above{}\else
         \edef\@above{\noalign{\vskip\@aboveskip}}\fi
      \gl@lop\strutlist\to\@strut
      \gl@lopTL\everylist\to\gltoks@every
      \gl@lopTL\@currentitem\to\gltoks@next
      \edef\temp{\the\gltoks@sofar \@above\@strut
         {\@linecolorop\the\gltoks@every \the\gltoks@next}\cr}%
      \gltoks@sofar=\expandafter{\temp}%
      \ifx\@currentitem\empty \gl@loopmorefalse \fi
      \repeat
   \setbox0\vtop{%
      \ling@everyglword
      \halign{\glwordalign{##}\cr \the\gltoks@sofar }}%
   \box0
   \endgroup
}

\def\gl@wordalignleft#1{#1\hfil}
\def\gl@wordaligncenter#1{\hfil#1\hfil}
\define@choicekey{ling}{glwordalign}[\ling@glwordalign\ep@glwordalign]
   {left,center}{%
   \ifcase\ep@glwordalign
      \let\glwordalign=\gl@wordalignleft
   \or
      \let\glwordalign=\gl@wordaligncenter
   \fi
}
%%%%%%%%%%%%%%%%%%%% nlevel style glosses %%%%%%%%%%%%%%%%%%%%
\def\ep@setglstrut{%
   \ifling@glstruts  \setbox\glstrutbox=\copy\strutbox
   \else \setbox\glstrutbox=\hbox{}%
   \fi
}
%%%%%%%%%%%%%%%%%%%%%%%%%%%%%%%%%%%%%%%%%%%%%%
\def\gl@nlevel@below@begin{\gl@nlevel@below@begin@a}
\def\gl@nlevel@below@begin@a{%
   \bgroup
   \parindent0pt
   \bgroup
   \@glspacefalse
   \ep@setglstrut
   \ifdim\ling@glwidth=0pt
      \bgroup
   \else
      \vtop\bgroup
         \hsize=\ling@glwidth
         \leftskip=0pt
   \fi
   \futurelet\tempt\gl@nlevel@below@begin@b
}
\def\gl@nlevel@below@begin@b{%
   \ifx\tempt\glpreamble \let\next=\gl@nlevel@below@preamble@c
   \else \let\next=\gln@ilg \fi \next}
\def\gl@nlevel@below@preamble@c #1{\gl@nlevel@below@preamble@d}
\def\gl@nlevel@below@preamble@d #1\endpreamble{%
   \bgroup
   \ling@everyglpreamble
   #1\strut
   \vskip\lingbelowglpreambleskip
   \vskip\lingextraglskip
   \egroup
   \gln@ilg
}
%\def\gl@nlevel@below@begin@b{%
%   \ifx\temp\glpreamble \let\next=\relax
%   \else \let\next=\gln@ilg \fi \next}
%\def\gl@nlevel@below@preamble #1\endpreamble{%
%   \ling@everyglpreamble
%   #1\strut
%   \vskip\lingbelowglpreambleskip
%   \vskip\lingextraglskip
%   \egroup
%   \bgroup
%   \gln@ilg
%}
\def\gln@ilg{%
   \ling@everyglilg
   \lineskip=\lingextraglskip
   \rightskip=\ling@glrightskip
   \ifcase\ep@glhangstyle
      \or
         \hangindent=\ling@glhangindent
         \hangafter=1
      \or
         \ep@glmkcascade
   \fi
   \leavevmode
   \gln@ilg@a
}
%%%%%%%%%%%%%%%%%%%%%%%%%%%%%%%%%%%%%%%%%%%%%%
\newif\if@glspace
\def\gln@ilg@a{\futurelet\temp\gln@ilg@b}
\def\gln@ilg@b{\ifx\temp\endgl \let\next=\relax
   \else \ifx\temp\nogloss \let\next=\gln@noglossword
   \else \ifx\temp\glft \let\next=\relax
   \else \let\next=\gln@word \fi\fi\fi \next }
\def\gln@noglossword{\expandafter\gln@word}
\def\gln@word #1[#2]#3 {%
   \def\@currentitem{\\{#1}}%
   \def\@diacritic{#3}%
   \gln@ilg@f #2/\@nil
   \gln@ilg@a
}
\def\nogloss#1{{#1}[]}%
\def\gln@ilg@f{\@ifnextchar\@space\gln@ilg@fi\gln@ilg@fii}
\def\gln@ilg@fi #1#2/{\gl@xappend #2\to\@currentitem \gln@ilg@g}
\def\gln@ilg@fii#1/{\gl@xappend #1\to\@currentitem \gln@ilg@g}
\def\gln@ilg@g{\@ifnextchar\@nil\gln@ilg@h\gln@ilg@f}
\def\gln@ilg@h #1{%
   \gln@printglword
   \ifx\@diacritic\empty
      \else \expandafter\gln@diacritic\@diacritic\@nil \fi
}
\def\gln@printglword{%
   \if@glspace \hskip\lingglspace \else \@glspacetrue \fi
   \begingroup
   \gltoks@sofar={}%
   \lineskip=0pt
   \ling@everyglword
   \gl@loopmoretrue
   \loop\ifgl@loopmore
      \gl@lop\colorlist\to\@linecolor
      \ep@mklinecolorop
      \gl@lop\gln@abovelineskip\to\@aboveskip
      \ifx\@aboveskip\empty \let\@above=\empty
         \else \edef\@above{\noalign{\vskip\@aboveskip}}\fi
      \gl@lopTL\gln@everyline\to\gltoks@every
      \gl@lopTL\@currentitem\to\gltoks@next
      \edef\temp{\the\gltoks@sofar \@above
         \glstrut {\@linecolorop \the\gltoks@every \the\gltoks@next}\cr}%
      \gltoks@sofar=\expandafter{\temp}%
      \ifx\@currentitem\empty \gl@loopmorefalse \fi
      \repeat
   \vtop{%
      \lineskip=0pt
      \ling@everyglword
      \halign{\glwordalign{##}\cr \the\gltoks@sofar}}%
   \endgroup
}
\def\gln@diacritic#1#2\@nil{%
   \def\temp{#2}\ifx\temp\empty \else \ep@diacriticerror{#1#2}\fi
   \if\noexpand #1\noexpand @\let\next\@glspacefalse
   \else \if\noexpand #1\noexpand +\let\next\gln@after@c
   \else \ep@diacriticerror{#1}\fi\fi
   \next
}
\def\gln@after@c{%
   \par
   \ifcase\ep@glhangstyle
   \or
      \hangindent=\ling@glhangindent \hangafter=0
   \or
      \glhangcarry
   \fi
   \leavevmode
   \@glspacefalse
}
\def\ep@diacriticerror#1{%
   \@expexerror{bad gloss diacritic: #1 (only @ and + permitted)}\end}
\def\gl@nlevel@below@ft{%
   \vskip\lingaboveglftskip
   \vskip\lingextraglskip
   \egroup\bgroup
   \nointerlineskip
   \strut
   \ling@everyglft
}
\def\gl@nlevel@below@end{\par\egroup\egroup\egroup\egroup}
\define@lingkey{glneveryline}{\ep@setlist\gln@everyline{#1}}
\define@lingkey{glnabovelineskip}{\ep@setlist\gln@abovelineskip{#1}%
   \gl@lop\gln@abovelineskip\to\temp   % no first line abovelineskip
   \gl@push{}\to\gln@abovelineskip
}
% bug fix 4/9/2016
\def\gl@append #1\to #2{%
   \gltoks@a={\\{#1}}%
   \gltoks@b=\expandafter{#2}%
   \edef#2{\the\gltoks@b\the\gltoks@a}%
}
\def\gl@eappend{\expandafter\gl@append}
% end bug fix
\lingset{glneveryline={\it},glnabovelineskip={}}
%%%%%%%%%%%%%%%% end nlevel %%%%%%%%%%%%%%%%%%%%

% ----- brackets -----
\newif\if@glpostbrack
\newcount\@glaparsestate          % 0 normal, 1 post @, 2 post [, 3 post ]
\define@lingincdimenkeys{glbrackbracksep,glbrackwordsep}
\lingset{glbrackbracksep=.05em,glbrackwordsep=.1em}
% 2011-12-09 introduce hook into \printlback and \printrbrack to allow font selection
\define@ling@cmdkeys{everybrack}
\lingset{everybrack=\rm}
\def\printlbrack{{\ling@everybrack [}}
\def\printrbrack{{\ling@everybrack ]}}
% ----- cascading hanging indentation -----
\newdimen\glwcascade@pshapeindent
\newdimen\glwcascade@pshapelinewd
\def\glwcascade@parshapetarget{}
\def\glwcascade@mkshapeaux{%
   \edef\glwcascade@parshapetarget
      {\glwcascade@parshapetarget\space
         \the\glwcascade@pshapeindent\space\the\glwcascade@pshapelinewd}%
   \advance\glwcascade@pshapeindent by \ling@glhangindent
   \advance\glwcascade@pshapelinewd by -\ling@glhangindent
}
\def\ep@glmkcascade{{%
   \ep@cnta=\ep@cascadecount
   \ep@dima=\hsize
   \advance\ep@dima by -\ep@cascadeindent
   \edef\cascadeshape{}
   \loop\ifnum\ep@cnta >0
      \edef\cascadeshape{\cascadeshape \the\ep@cascadeindent\space\the\ep@dima\space}%
      \advance\ep@cascadeindent by \ling@glhangindent
      \advance\ep@dima by -\ling@glhangindent
      \advance\ep@cnta by -1
      \repeat
   \xdef\cascadeshape{\noexpand\parshape\the\ep@cascadecount\space\cascadeshape}}%
   \cascadeshape
   \ignorespaces
}
\def\glhangcarry{%
   \edef\next{\the\prevgraf}%
   \advance\ep@cascadecount by -\next
   \ep@dimc=\ling@glhangindent
   \advance\ep@cascadeindent by \next\ep@dimc
   \ep@glmkcascade
}
% ----- side by side (ss) gloss style -----
\def\gl@setprefix{\edef\gl@prefix{gl@\ling@glstyle @\ling@glftpos @}}
\def\glft{\csname \gl@prefix ft\endcsname}
\def\gl@beginglstyle{\csname \gl@prefix begin\endcsname}
\def\endgl{\csname \gl@prefix end\endcsname}
\def\glpreamble{\csname \gl@prefix preamble\endcsname}

\define@choicekey{ling}{glftpos}[\ling@glftpos\gl@ftpos@num]%
   {below,right}{\gl@setprefix}

\define@lingcmdkeys{sssep,ssratio,ssrightskip}
\lingset{sssep=2em,ssratio=.6,ssrightskip=0pt plus 2em}
\newdimen\ssleftwd
\newdimen\ssrightwd
\def\ep@setssdims{%
   \dimen0 =\hsize
   \advance\dimen0 by -\leftskip
   \advance\dimen0 by -\lingsssep
   \ssleftwd=\lingssratio\dimen0
   \ssrightwd=\dimen0
   \advance\ssrightwd by -\ssleftwd
}
\def\gl@wrap@right@begin{%
   \ep@setssdims
   \leavevmode\bgroup\hbox\bgroup
      \hsize=\ssleftwd
      \lingset{glwidth=\ssleftwd}
      \gl@wrap@below@begin
}
\def\gl@wrap@right@end{\egroup\egroup}
\def\gl@wrap@right@ft #1//{%
   \gl@wrap@below@end
   \hskip\lingsssep
   \vtop{%
      \leftskip=0pt
      \rightskip=\lingssrightskip
      \parindent=0pt
      \hsize=\ssrightwd
      \ling@everyglft
      #1}%
      \ignorespaces
}
\expandafter\def\expandafter\gl@nlevel@right@medial\expandafter
   {\gl@nlevel@below@end \gl@nlevel@right@medial@a}
\def\gl@nlevel@right@medial@a{\egroup
   \def\gl@prefix{gl@nlevel@right@}%
   \hskip\lingsssep
   \vtop\bgroup
      \leftskip=0pt
      \rightskip=\lingssrightskip
      \parindent=0pt
      \hsize=\ssrightwd
      \ling@everyglft
}
\def\gl@nlevel@right@begin{%
   \ep@setssdims
   \leavevmode
   \vtop\bgroup
      \hsize=\ssleftwd
      \leftskip=0pt
      \def\gl@prefix{gl@nlevel@below@}
      \let\gl@nlevel@below@ft=\gl@nlevel@right@medial \begingl }
      \def\gl@nlevel@right@end{\egroup\egroup}
% ----- gloss with a side panel
%2015 added macros for side panel in nlevel style
\define@lingcmdkeys{everypanel}
\lingset{everypanel={}}
\def\beginglpanel{\@getoptionalarg\beginglpanel@a}
\def\beginglpanel@a{%
   \bgroup
   \ifcase\gl@style@num
      \let\endgl=\gl@wrap@panel@end
   \else
      \let\endgl=\gl@nlevel@panel@end
   \fi
   \begin@glpanel@right
}
\def\begin@glpanel@right{%
   \ling@usearg
   \ep@setssdims
   \leavevmode
   \lingset{glwidth=\ssleftwd}
   \begingl
}
\def\gl@wrap@panel@end{%
   \gl@wrap@below@end
   \hfill
   \vtop\bgroup
   \hsize=\ssrightwd
   \leftskip=0pt
   \rightskip=\lingssrightskip
   \lingeverypanel
}
\def\gl@nlevel@panel@end{%
   \gl@nlevel@below@end
   \hfill
   \vtop\bgroup
   \hsize=\ssrightwd
   \leftskip=0pt
   \rightskip=\lingssrightskip
   \lingeverypanel
}%
\def\endpanel{\egroup\egroup\par}
% ----- underfixes -----
% removed 11/4/2015
% put back temporarily 3/9/2017 (still needed in manual)
\def\gluf/#1/#2/{%
   \vtop{\offinterlineskip\halign{\hfil##\hfil\cr
      \strut #1\cr
      \noalign{\vskip-\ling@glufcloseup}
      \ling@everygluf \strut#2\cr
}}}
% ----------
\define@ling@cmdkeys{everygluf,glufcloseup}
\lingset{glufcloseup=.4ex,everygluf=\sc}
% ----- gloss comments and citations -----
\def\rightcomment#1{\leavevmode\rlap{%
   \hbox to\hsize{\hfil \rm #1\hskip\leftskip}}\ignorespaces}
\let\rightcite=\rightcomment % for backwards compatability
\define@ling@cmdkeys{mincitesep}
\lingset{mincitesep=1.5em}
% jf 2011-12-09 introduce everytrailingcitation hook
\define@ling@cmdkeys{everytrailingcitation}
\def\trailingcitation#1{%
   \hskip\ling@mincitesep plus 1fill
   \penalty100\null\nobreak \hskip 0pt plus 1fill
   \hbox{\ling@everytrailingcitation #1}%
}
\lingset{everytrailingcitation=}
%! ----- initial settings -----
\lingset{%
   aboveexskip=2.7ex plus .8ex minus .8ex,
   belowexskip=2.7ex plus .8ex minus .8ex,
   Everyex=,
   everyex=,
   numoffset=0pt,
   labelanchor=numright,
   labeloffset=1em,
   labelwidth=.78em,
   textanchor=normal,
   textoffset=1em,
   preambleanchor=numright,
   preambleoffset=1em,
   avoidnumlabelclash=false,
   appendtopexarg=,
   labeltype=alpha,
   everylabel=,
   labelalign=left,
   belowpreambleskip=1ex,
   interpartskip=1ex,
%   splitexpenalty=200,
   exbreakfil=0pt plus 4ex,
   exbreakpenalty=-50,
   splitpartspenalty=200,
% auxiliary parameters used for building tables
   dima=2.4em,
   crskip=.6em
}
\lingset{%
% parameters used in glosses
   glspace=.5em plus.4em minus.15em,
   glrightskip=0pt plus .1\hsize,
   aboveglcskip=0pt,
   aboveglftskip=1ex,
   belowglpreambleskip=1ex,
   everyglpreamble=,
   glhangindent=1em,
   everygla=\it,
   everyglb=,
   everyglc=,
   everyglft=,
   everygl=,
   everyglilg=,
   everyglword=,
   glwordalign=left,
   glwidth=0pt,
   glufcloseup=.4ex,
   everygluf=,
   glstyle=wrap,
   extraglskip=.5ex,
   mincitesep=1.5em,
   glstruts=true
}
%%%%%%%%%%%%%%%% additions to the CTAN file
\def\tspacea{\hskip\lingdima}
\def\tspaceb{\hskip\lingdimb}
\def\tspacec{\hskip\lingdimc}
\lingset{dimb=1.5em,dimc=1.5em}

\resetatcatcode
