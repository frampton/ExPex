
\edef\resetatcatcode{\catcode`\noexpand\@\the\catcode`\@\relax}%
\catcode`\@11\relax
\ifx\XKeyValLoaded\endinput \else
   \input xkeyval \fi
\ifx\ProvidesFile\@undefined
   \message{2008/05/08 v1.0(beta) ExPex linguistics
      example formatter (JF)}
\else
   \ProvidesFile{expex.tex}[2008/05/08 v1.0(beta)
      ExPex linguistics example formatter (JF)]
   \@addtofilelist{expex.tex}
\fi
% define eplain primitives, if necessary
\ifx\eplain\@undefined  % eplain stuff
\def\@futurenonspacelet#1{\def\@cs{#1}%
   \afterassignment\@stepone\let\@nexttoken= }%
\def\@stepone{\expandafter\futurelet\@cs\@steptwo}%
\def\@steptwo{\expandafter\ifx\@cs\@sptoken\let\@@next=\@stepthree
   \else\let\@@next=\@nexttoken\fi \@@next}%
\def\@stepthree{\afterassignment\@stepone\let\@@next= }%
\def\@getoptionalarg#1{%
   \let\@optionaltemp = #1%
   \let\@optionalnext = \relax
   \@futurenonspacelet\@optionalnext\@bracketcheck
}%
\def\@bracketcheck{%
   \ifx [\@optionalnext
      \expandafter\@@getoptionalarg
   \else
      \let\@optionalarg = \empty
      \expandafter\@optionaltemp
   \fi
}%
\def\@@getoptionalarg[#1]{%
   \def\@optionalarg{#1}%
   \@optionaltemp
}%
\def\identity#1{#1}%
\fi
% end of eplain inclusions
\def\@getoptionaltag#1{%
   \let\@optionaltemp = #1%
   \@ifnextcharacter<\@@gettag
      {\let\@optionaltag\empty \@optionaltemp}%
}
\def\@@gettag<#1>{\def\@optionaltag{#1}\@optionaltemp}

\newif\if@tilde
\def\@tildecheck#1{%
   \@ifnextcharacter~%
      {\@tildetrue\expandafter#1\@gobble}%
      {\@tildefalse#1}%
}
% ------ XKV parametrization ------
\def\define@lingkey{\define@key{ling}}
\def\define@ling@inckey#1{%
   \define@key{ling}{#1}%
      {\expandafter\@setinckey\csname ling@#1\endcsname##1\@nil}%
}
% user friendly inc keys
\def\define@linginckey#1{%
   \define@key{ling}{#1}%
      {\expandafter\@setinckey\csname ling#1\endcsname##1\@nil}%
}
\def\define@ling@cmdkeys{\define@cmdkeys{ling}[ling@]}
\def\define@lingcmdkeys{\define@cmdkeys{ling}[ling]}
\def\@setinckey#1#2#3\@nil{%
   \ifx#2!%
      \dimen0=#1
      \advance\dimen0 by #3
   \else
      \dimen0=#2#3
   \fi
   \edef#1{\the\dimen0}%
}
\def\lingset#1{\setkeys{ling}{#1}\ignorespaces}
% \Lingset first sets ling keys, if there are non-ling keys
% remaining, these are then passed to \psset
\def\Lingset#1{\setkeys*{ling}{#1}%
   \ifx\XKV@rm\@empty \else
   \expandafter\psset\expandafter{\XKV@rm}\fi}
\def\Ling@usearg{\expandafter\Lingset\expandafter{\@optionalarg}}
\def\ling@usearg{\expandafter\lingset\expandafter{\@optionalarg}}
% styles
\define@lingkey{lingstyle}{%
   \edef\temp{\csname ling@#1style\endcsname}%
      \expandafter\Lingset\expandafter{\temp}}
\def\definelingstyle#1#2{%
   \expandafter\def\csname ling@#1style\endcsname{#2}}
% if PST available, allow \psset to set ling parameters,
% otherwise cancel \Lingset's ability to set PST parameters
\ifx\PSTricksLoaded\endinput
   \pst@addfams{ling}
\else
   \let\Lingset=\lingset
\fi
% ----- \ex -----
\newcount\excnt
\excnt=0
\define@linginckey{numoffset}
\define@linginckey{textoffset}
\define@ling@cmdkeys{Everyex,everyex,
   aboveexskip,belowexskip,
   exnosplitfil,
   exnosplitpenalty,
   goodparpenalty,
   goodparfil
}
\define@lingkey{exskip}%
   {\edef\ling@aboveexskip{#1}%
   \edef\ling@belowexskip{#1}%
}
\lingset{%
   numoffset=0pt,
   textoffset=1em,
   Everyex=,
   everyex=,
   aboveexskip=2.7ex plus .4ex minus .4ex,
   belowexskip=2.7ex plus .4ex minus .4ex,
   exnosplitfil=0pt plus .3\vsize,
   exnosplitpenalty=300,
   goodparfil=0pt plus .15\vsize,
   goodparpenalty=-10
}
\newif\if@judge
\newif\if@specialexno
\newcount\ex@type
\def\ex{\bgroup \ex@type=0
   \@tildecheck\ex@a}
\def\ex@a{\@getoptionalarg\ex@aa}
\def\ex@aa{\@getoptionaltag\ex@d}
%\def\ex@aa{\@getoptionaltag\ex@b}
%\def\ex@b{\@ifnextchar({\@judgetrue\ex@c}{\@judgefalse\ex@d}}
%\def\ex@c(#1){\def\@judgment{#1}\ex@d}
\def\ex@d{%
   \ex@setup
   \setbox0=\hbox{%
      \hskip\lingnumoffset
      \if@specialexno \specialexnoprint\else\exnoprint\fi
      \hskip\lingtextoffset}%
   \leftskip=\wd0
   \par\nobreak
   \leavevmode\llap{\unhbox0}%
%   \if@judge\ljudge{\@judgment}\fi
   \ignorespaces
}
\def\exnoprint{(\the\excnt)}
\def\specialexnoprint{(\ling@specialexno)}
\def\ex@setup{%
   \@specialexnofalse
   \global\advance\excnt by 1
   \ling@Everyex
   \Ling@usearg
   \ifx\@optionaltag\empty
      \let\@localextag=\empty
   \else
      \edef\@localextag{\@optionaltag}%
      \deftag{\the\excnt}{\@optionaltag}%
   \fi
   \if@specialexno \global\advance\excnt by -1 \fi
   \if@tilde \else \vskip\ling@aboveexskip \fi
   \skip255=\ling@exnosplitfil
   \vskip\skip255
   \penalty\ling@exnosplitpenalty
   \vskip-\skip255
   \parindent=0pt
   \ling@everyex
}
\def\noexno{\global\advance\excnt by -1}
\def\goodpar{\endgraf
   \skip255=\ling@goodparfil
   \vskip\skip255
   \penalty\ling@goodparpenalty
   \multiply\skip255 by -1
   \vskip\skip255
}
\def\xe{\expandafter\vskip\ling@belowexskip
   \allowbreak \egroup \prevdepth\dp\strutbox \noindent}
\def\exdisplay{\bgroup\@tildecheck\exdisplay@a}
\def\exdisplay@a{\@getoptionalarg\exdisplay@b}
\def\exdisplay@b{\let\@optionaltag=\empty \ex@setup}
% ----- \pex -----
\newcount\pexcnt
\newdimen\@pexoffset
\newif\if@ispreamble
\define@ling@inckey{preambleoffset}
\define@linginckey{labelwidth}
\define@linginckey{labeloffset}
\define@ling@cmdkeys{belowpreambleskip,interpartskip,splitexpenalty}
\define@choicekey{ling}{labeltype}[\ling@labeltype\@@labeltype]%
   {alpha,numeric,caps}{%
      \ifnum\@@labeltype=1
            \def\@@label{\the\pexcnt}%
            \lingset{labelalign=right}%
         \else
            \def\@@label{\char\the\pexcnt}%
            \lingset{labelalign=left}%
         \fi
}
\define@choicekey{ling}{labelalign}[\ling@labelalign\nr]%
   {left,center,right}{%
      \ifcase\nr
            \def\@labelprint{\@@label.\hss}%
         \or
            \def\@labelprint{\hss\@@label.\hss}%
         \or
            \def\@labelprint{\hss\@@label.}%
         \fi
}
\define@key{ling}{samplelabel}{%
   \setbox0=\hbox{#1}%
   \lingset{labelwidth=\wd0}%
}
\lingset{%
   labeltype=alpha,
   labelalign=left,
   labeloffset=1em,
   labelwidth=.78em,
   preambleoffset=1em,
   belowpreambleskip=1ex,        % vskip after the preamble
   interpartskip=1ex,            % vskip between parts
   splitexpenalty=200,
}
\let\nopreamble=\ignorespaces
\def\pex{\bgroup\@tildecheck\pex@a}
\def\pex@a{\@getoptionalarg\pex@ab}
\def\pex@ab{\@getoptionaltag\pex@b}
\def\pex@b{\@ispreamblefalse\@futurenonspacelet\temp\pex@c}
\def\pex@c{%
   \ifx\temp\a
      \else \ifx\temp\nopreamble
         \else \@ispreambletrue \fi\fi
   \pex@d
}
\newif\if@firstlabel
\def\pex@d{%
   \ex@setup
   \@pexoffset=\linglabeloffset
   \advance\@pexoffset by \linglabelwidth
   \advance\@pexoffset by \lingtextoffset
   \pexcnt@init
   \@firstlabeltrue
   \let\a\put@label
   \ling@everyex
   \setbox0=\hbox{%
      \hskip\lingnumoffset
      \if@specialexno \specialexnoprint\else\exnoprint\fi
      \if@ispreamble
         \hskip\ling@preambleoffset
      \else
         \hskip\@pexoffset
      \fi }%
   \leftskip=\wd0
   \leavevmode\llap{\unhbox0}%
}
\def\pexcnt@init{%
   \ifcase\@@labeltype
      \pexcnt=96 \or \pexcnt=0 \or \pexcnt=64
      \fi
}
\def\put@label{%
   \if@firstlabel
      \if@ispreamble
         \par
         \vskip\ling@belowpreambleskip
         \advance\leftskip by -\ling@preambleoffset
         \advance\leftskip by \@pexoffset
         \fi
      \@firstlabelfalse
   \else
      \vskip\ling@interpartskip
   \fi
   \put@label@a
}
\def\put@label@a{\@getoptionaltag\put@label@d}
%\def\put@label@a{\@getoptionaltag\put@label@b}
\def\put@label@b{\@ifnextchar(%
   {\@judgetrue\put@label@c}{\@judgefalse\put@label@d}}
\def\put@label@c(#1){\def\@judgment{#1}\put@label@d}
\define@choicekey{ling}{labelanchor}[\ling@labelanchor\@@labelanchor]%
   {numright,numleft,margin}[numright]{}
\lingset{labelanchor}
\def\put@label@d{%
   \advance\pexcnt by 1
   \ifx\@optionaltag\empty
      \else
         \deftaglabel{\@optionaltag}%
      \fi
   \ifnum\@@labelanchor>0
      \par
      \leftskip=\@pexoffset
      \ifnum\@@labelanchor=1
         \advance\leftskip by \lingnumoffset \fi
   \fi
   \leavevmode
   \llap{\hbox to\linglabelwidth{\@labelprint}%
      \hskip\lingtextoffset}%
%   \if@judge \ljudge{\@judgment}\fi
   \ignorespaces
}
%----- judgments -----
\def\judge#1{\rm #1\kern .1em \ignorespaces}
\def\ljudge#1{\llap{\judge{#1}}\ignorespaces}
\define@key{ling}{*}[*]%
   {\setbox0=\hbox{#1}%
   \lingset{textoffset=!\wd0}%
}
%----- local reference to example numbers -----
\def\nextx{\@printref{\advance\excnt by 1 \number\excnt}}
\def\anextx{\@printref{\advance\excnt by 2 \number\excnt}}
\def\lastx{\@printref{\number\excnt}}
\def\currx{\the\excnt\relax}
\def\blastx{\@printref{\advance\excnt by -1 \number\excnt}}
\def\bblastx{\@printref{\advance\excnt by -2 \number\excnt}}
%----- gloss tools
\define@ling@inckey{glspace}
\define@ling@inckey{abovemoreglskip}
\define@ling@inckey{aboveglcskip}
\define@ling@inckey{moregloffset}
\define@ling@cmdkeys{everygla,everyglb,everyglc}
\lingset{glspace=.6em,everygla=\it,everyglb=,everyglc=,
   aboveglcskip=.5ex,moregloffset=0pt,abovemoreglskip=.25ex}
\def\begingl{\leavevmode\vtop\bgroup
   \@getoptionalarg\begingl@ss}
\def\begingl@ss{%
   \ling@usearg
   \vtop\bgroup
      \let\ling@@everygl\ling@everygla
      \halign\bgroup ##\hfil &&
         \kern\ling@glspace ##\hfil\cr
}
\def\moregl{%
   \egroup\egroup \vskip\ling@abovemoreglskip
   \vtop\bgroup
      \halign\bgroup \kern\ling@moregloffset ##\hfil &&
         \kern\ling@glspace ##\hfil\cr
}
\def\endgl{\egroup\egroup\egroup}
\def\gla{\global\let\@everygl\ling@everygla \gl@ss}
\def\glb{\global\let\@everygl\ling@everyglb \gl@ss}
\def\gl@ss#1 {\@everygl #1\@ifnextchar/{\strut\cr\@gobble}{& \gl@ss}}
\def\glc{\noalign\bgroup\@tildecheck\glc@}
\def\glc@#1{%
   \if@tilde \else \vskip\ling@aboveglcskip \fi \egroup
   \omit \ling@everyglc #1\strut\hidewidth\cr
}
%  table building tools
\define@lingcmdkeys{dima,dimb,dimc}
\define@ling@cmdkeys{everylabel}
\lingset{dima=2.4em,everylabel=}
\def\tspace{\@getoptionalarg\tsp@ce}
\def\tsp@ce{\hskip \ifx\@optionalarg\empty
   \lingdima \else \csname ling\@optionalarg\endcsname\fi}
% labels
\def\labels{\@getoptionalarg\expex@labels}
\def\expex@labels{%
   \ling@usearg
   \dimen0\lingtextoffset
   \advance\dimen0 \linglabelwidth
   \edef\ling@labelskip{\the\dimen0}%
   \pexcnt@init
   \let\tl\@inserttabellabel
   \let\nl\@omitlabel
   \ignorespaces
}
\def\pexcnt@init{%
   \ifcase\@@labeltype
      \pexcnt=96 \or \pexcnt=0 \or \pexcnt=64
      \fi
}
\def\@inserttabellabel{%
   \global\advance\pexcnt by 1
   {\ling@everylabel\@@label.}%
}
\def\@omitlabel{\omit\hskip\linglabeloffset\hfil}
\def\endpextable{\egroup\egroup \par \prevdepth=\dp\strutbox}
%------??
%\def\omitlabel{\omit\hskip\ling@labelskip}
\def\hwit#1{\hidewidth \it #1\hidewidth}
\define@ling@inckey{crskip}
\lingset{crskip=.6em}
\def\crs{\cr\noalign{\vskip\ling@crskip}}
\def\crnb{\cr\noalign{\par\nobreak}}

% ----- tags and references -----
% ----- defining tags -----
\def\deftagex#1{\edef\@localextag{#1}\deftag{\the\excnt}{#1}}
\def\deftag#1#2{%
   {\let\@printref=\identity
   \let\getref=\@baregetref
   \expandafter\xdef\csname lingtag@#2\endcsname{#1}%
   \if@g@thertags
      \immediate\write@tags{\noexpand\@fd@f #2 #1 }%
      \fi}%
   \ignorespaces
}
\def\deftagpage#1{%
   \if@g@thertags
      \write@tags{\noexpand\@fd@f #1 {\the\pageno}}%
   \fi
   \ignorespaces
}
\def\lastlabel{{\@@label}}
\def\deftaglabel#1{%
   \ifx\@localextag\empty
      \@expexwarn{Example number for [#1] is not tagged}%
   \else
      \expandafter\xdef\csname lingtag@\@localextag.#1\endcsname
         {\lastlabel}%
      \if@g@thertags
         \immediate\write@tags
            {\noexpand\@fd@f \@localextag.#1 {\@@label}}\fi
   \fi
   \ignorespaces
}
% ----- getting references from tags -----
\def\getref#1{%
   \@getref{#1}\temp
   \ifx\temp\empty
   \else
      \@printref{\temp}%
   \fi
}
\def\@getref#1#2{%
   \@setupreadtags
   \expandafter\ifx\csname lingtag@#1\endcsname\relax
      \let#2\empty
      \@expexwarn{The tag [#1] is not defined}%
      {\tt $\bullet$#1}%
   \else
      \expandafter\let\expandafter#2\csname lingtag@#1\endcsname
   \fi
}
\def\@expexwarn#1{\immediate\write16{====> ExPex WARNING: #1.}}
\def\getfullref#1{%
   \@getref{#1}\temp@ak
   \ifx\temp@ak\empty \else \@getfullref #1\@nil \fi
}
\def\@getfullref #1.#2\@nil{%
   \@getref{#1}\temp@bk
   \ifx\temp@bk\relax
   \else
      \@printref{\temp@bk\expandafter\@ling@putperiod\temp@ak}%
   \fi
}

% ----- ref proofing -----
\let\@printref\relax
\def\refproofing{%
   \ifx\PSTricksLoaded\endinput
      \let\@printref\@psthighlightref
   \else
      \let\@printref\@highlightref
   \fi\relax
}
\def\@psthighlightref{%
   \leavevmode
   \psframebox[boxsep=false,framesep=4pt,linewidth=.24ex]%
}
\def\@highlightref#1{$\overline{\underline{\hbox{#1}}}$}
\def\@ling@putperiod #1{\@ling@pp#1}
\def\@ling@pp#1{\ifx#1\char \else .\fi #1}
% ----- specialexno=reference -----
\def\@baregetref#1{\csname lingtag@#1\endcsname}
\def\@baregetfullref #1{%
   \@baregetfullref@a #1\@nil
   \expandafter\expandafter
      \expandafter\@ling@putperiod\csname lingtag@#1\endcsname
}
\def\@baregetfullref@a #1.#2\@nil{\@baregetref{#1}}
\define@key{ling}{specialexno}{%
   \@specialexnotrue
   \@setupreadtags
   {\let\getfullref=\@baregetfullref
   \let\getref=\@baregetref
   \expandafter\xdef\csname ling@specialexno\endcsname{#1}}%
}
% ----- opening the tag file -----
\newif\if@g@thertags
\@g@thertagsfalse
\newwrite\ling@tagsfile
\def\write@tags{\write\ling@tagsfile}
\def\tagfilesuffix#1{\edef\@tagfilesuffix{#1}}
\def\@tagfilesuffix{-tags}
\def\gathertags{%
   \@setupreadtags
   \@g@thertagstrue
   \immediate\openout\ling@tagsfile=\jobname\@tagfilesuffix\relax
}
% ----- reading the tag file and defining the tags it encodes -----
\newread\ling@tagsin
\def\@fd@f#1 #2 {%
   \expandafter\ifx\csname lingtag@#1\endcsname \relax
      \expandafter\xdef\csname lingtag@#1\endcsname{#2}\fi
}
\newif\if@readtags
\@readtagstrue
\def\@setupreadtags{\if@readtags
   \do@readtags \global\@readtagsfalse \fi}
\def\do@readtags{%
   \immediate\openin\ling@tagsin=\jobname\@tagfilesuffix\relax
   \ifeof\ling@tagsin \else
      \closein\ling@tagsin
      {\catcode`@=11 \input \jobname\@tagfilesuffix\relax}%
   \fi
}
% ----- tagging sections, adapt to your needs -----
% If \tagsec is used with section macros that do not define
% counters \secno,\subsecno,\subsubsecno, and \subsubsubsecno,
% then \currsec must be redefined to whatever is appropriate.
%\def\chapscurrsec{\ifnum\chapno>0 \the\chapno
%   \ifnum\secno>0 .\the\secno
%   \ifnum\subsecno>0 .\the\subsecno
%   \ifnum\subsubsecno>0 .\the\subsubsecno \fi\fi\fi\fi}
%\def\nochapscurrsec{\ifnum\secno>0 .\the\secno
%   \ifnum\subsecno>0 .\the\subsecno
%   \ifnum\subsubsecno>0 .\the\subsubsecno \fi\fi\fi}
%\let\currsec\nochapscurrsec
%\def\deftagsec#1{\deftag\currsec{#1}}
\resetatcatcode

